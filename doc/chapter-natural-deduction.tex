\chapter{Natural Deduction (in construction)}


%%%%%%%%%%%%%% New chapter 
\subsection{Contexts as sets}

\begin{exercise}

Prove the following lemma:
\inputsnippets{FolExamples/MPdiag}

\end{exercise}


\section{Derived rules and natural deduction}


 The library 
 \href{../theories/html/hydras.Ackermann.folLogic.html}{Ackermann.folLogic} contains many derived rules which allow the user to build proofs in a natural deduction style (with introduction and elimination rules).

\subsection{Example}

For instance, let us prove Peirce's rule.
 (in  ~\href{../theories/html/hydras.MoreAck.FolExamples.html}{MoreAck.FolExamples} ).

\subsubsection{Prelude}

%\inputsnippets{folPeirce/prelude}


\subsection{Proof of Peirce's law}


\inputsnippets{FolExamples/PeirceProof}

Let us try to do an implication introduction.

\inputsnippets{FolExamples/step1}

Now, we may use the law of excluded middle with the formula $P$. The only non-trivial case is about $\sim P$.

\inputsnippets{FolExamples/step2}

The rest of the proof is composed of basic proof steps, 
and bookkeeping steps (about \texttt{Ensembles.In}).

\inputsnippets{FolExamples/step3, FolExamples/step4}

\index{ackermann}{Exercises}
\begin{exercise}
Prove, using the rules described in 
 \href{../theories/html/hydras.Ackermann.folLogic.html}{Ackermann.folLogic}, the famous \emph{drinkers theorem}:

$$\exists\,x,\; (D(x)\Longrightarrow \forall\,y,\; D(y))$$
\end{exercise}

where $D$ (for ``drinks'') is some predicate symbol of arity $1$.


