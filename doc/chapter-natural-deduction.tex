\chapter{Natural Deduction (in construction)}


%%%%%%%%%%%%%% New chapter 
\section{Contexts as sets}

Let us look again at the proof scripts shown in~\vref{proof:PrfEx2} and \vref{proof:PrfEx3}.
\begin{itemize}
\item The statement of \texttt{PrfEx2} contains a sequence
of hypotheses with two occurrences of \texttt{A}. Moreover, the order in which the 4 hypotheses are listed is determined by the type of the constructor \texttt{MP}  (please look at
Figure~\vref{fig:prf-def}). It would be better to replace this
precise list of hypotheses with ``any list whose elements
belong to the set $\{A, A \arrow B; A \arrow B \arrow C\}$''.

\item In the proof of \texttt{PrfEx3} the \coq user would certainly ask ``How do we apply implication's introduction rule?''.
\end{itemize}

The answer to both questions in~\cite{Goedel}  is the definition of a proof system, derived from \texttt{Prf}, which considers
\emph{sets of hypotheses} (called \emph{systems} in ~\cite{Goedel}) instead of
\emph{list of hypotheses}, thus making abstraction of
the repetition and order of appearance of hypotheses in the context.

The new system is simply defined as below (for  a given language $L$)\,\footnote{In some shown snippets, arguments like $L$ may be or not be implicit (depending on the section they are extracted from). Please look at the \coq source.}.


\inputsnippets{folProof/SysPrfDef}


\subsubsection{Notations}

In the text, we may use the abbreviation $\Gamma \sdash A$ for
(\texttt{SysPrf $L$ $\Gamma$ $A$}) and
$\sdash A$ for
(\texttt{SysPrf $L$ Empty\_set $A$}) 



We may also omit obvious braces in some set expressions:


\begin{itemize}

\item $\Gamma, \Delta$ for $\Gamma\cup\Delta$,
\item $A, B, C$ for $\{A, B, C\}$,
\item $\Gamma, A,\dots B$ for $\Gamma\cup\{A \dots B\}$
\item \emph{etc.}
\end{itemize}

\subsubsection{Examples}


In the following script, we use \texttt{PrfEx2}
as a witness for proving a set-based version of the 
original proof term, namely $A; A \arrow B; A \arrow B \arrow C \sdash C$.



\inputsnippets{FolExamples/SysPrfEx2}


\begin{exercise}

Prove the following lemma (without the handy lemmas from
\href{../theories/html/hydras.Ackermann.folLogic.html}{Ackermann.folLogic} and their corollaries!).

\inputsnippets{FolExamples/MPSys}

\end{exercise}

\subsection{Using properties of sets}

The following three lemmas, from 
\href{../theories/html/hydras.Ackermann.folLogic.html}{Ackermann.folLogic} are direct consequences of \texttt{SysPrf}'s definition. 

\inputsnippets{folLogic/Axm}

\inputsnippets{folLogic/SysExtend}

The \emph{rule of implication elimination} is derived from \texttt{Prf}'s 
\emph{modus ponens} \texttt{MP}.  Since all elements of the fringe 
\texttt{x} [resp. \texttt{x1}] of the proof tree \texttt{px} [resp. \texttt{px1}] belong to T, so are the elements of the fringe \texttt{x++x1} of
(\texttt{MP \dots{}  px px1}).

\inputsnippets{folLogic/impE}

\section{The Deduction theorem}

The deduction theorem (proved in \href{../theories/html/hydras.Ackermann.Deduction.html}{Ackermann.Deduction} is a handy 
tool for proving a proposition $A\arrow B$ by pushing 
the hypothesis $A$ into the context (it corresponds to 
implication introduction in \coq).


\inputsnippets{Deduction/deductionThm}

\subsection{Sketch of proof}

We advise the reader to replay this proof on h.er.is computer
 in order to better understand its structure, which we will only comment briefly.

Let us assume the hypothesis \texttt{H: $T, g \sdash f$},
meaning that there exists some list $F$ whose elements 
belong to $T \cup \{g\}$, and a proof-tree $t$ of type
$F \vdash h$.

The heart of the proof is an induction on $t$ proving 
$F \cap T \sdash (g\arrow f)$ \footnote{Please forgive the implicit coercion from lists to sets!}. 

Please note that the case $S=\texttt{nil}$ is common to 
many constructors of \texttt{Prf}, thus the proof script starts
with a study of this particular case, simply applied 11 times in the rest of the proof. 

 \index{ackermann}{Projects}

\begin{project}[**]
  The type \texttt{SysPrf} has sort \texttt{Prop}, which prevent us from
extracting the underlying proof tree and its fringe from an hypothesis 
of the form $X\sdash f$. 
It would be nice (\emph{e.g.} for a better understanding of the proof of the deduction theorem) to be able to compute the  proof-tree built
by the proof of the deduction lemma. 
On a fresh branch of the project, please change the definition of
\texttt{SysPrf} and fix the errors this change could cause in the rest of the files. 
\emph{If ensuring compatibility with all the \texttt{Goedel} project is too long and/or difficult, you may just make the changes in separate modules with an ``informative'' \texttt{SysPrf} and limit the compatibility study  to the contents of basic modules like
\texttt{Deduction}, \texttt{folLogic}, etc.
Perhaps you will have to consider other implementation of finite sets of formulas (\emph{e.g.} lists).
}

\end{project}




\section{Derived rules and natural deduction}


 The library 
 \href{../theories/html/hydras.Ackermann.folLogic.html}{Ackermann.folLogic} contains many derived rules which allow the user to build proofs in a natural deduction style (with introduction and elimination rules).

%\subsection{Applications of the Deduction Theorem}

The \emph{rule of implication introduction} is a trivial application of the deduction theorem.

\inputsnippets{folLogic/impI}

The following lemma (corresponding to the \texttt{CP} constructor of type \texttt{Prf}) is proven with the help of \texttt{impE}, \texttt{impI}, 
\texttt{sysWeaken} and \texttt{CP}.  

\inputsnippets{folLogic/contradiction}

We let the reader consult the proof of the following lemmas, or, much better, re-prove them as exercises.

\inputsnippets{folLogic/nnE}

\inputsnippets{ folLogic/nnI}

\inputsnippets{folLogic/cp1}

\inputsnippets{folLogic/cp2}

\subsection{Rules for derived connectives and quantifiers}

Let us keep in mind that the derived connectives: \texttt{orH}, \texttt{andH}, etc. and the existential quantifier \texttt{existH}  are defined
in terms of \texttt{impH}, \texttt{notH}, \texttt{forallH}.
By unfolding these definitions, we prove easily a few natural deduction rules for the derived symbols. 
For instance, the law of excluded middle for $f$: $ T\sdash \sim f \vee f$
for any $T$,  is just an abbreviation of $T \sdash \sim\sim f \arrow f$.

\inputsnippets{folLogic/noMiddle}


\inputsnippets{folLogic/orI1}

\inputsnippets{folLogic/orI2}


\subsection{Example: proof of Peirce's law}

For instance, let us prove Peirce's rule, \emph{i.e.}
$\sdash ((A \arrow B) \arrow A) \arrow A$ for any formulas $A$ and $B$.

 The proof in \coq is available in  ~\href{../theories/html/hydras.MoreAck.FolExamples.html}{MoreAck.FolExamples}.

\inputsnippets{FolExamples/PeirceProof}

Let us start with  an implication introduction.

\inputsnippets{FolExamples/step1}

Now, we may eliminate the disjunction 
in the instance $\sim A \vee A$ of the law of excluded middle.
The only non-trivial case is about $\sim A$.

\inputsnippets{FolExamples/step2}

The rest of the proof is composed of basic proof steps, 
and bookkeeping steps (about \texttt{Ensembles.In}).

\inputsnippets{FolExamples/step3, FolExamples/step4}

\index{ackermann}{Exercises}
\begin{exercise}
Prove, using the rules described in 
 \href{../theories/html/hydras.Ackermann.folLogic.html}{Ackermann.folLogic}, the famous \emph{drinkers theorem}:

$$\exists\,x,\; (D(x)\Longrightarrow \forall\,y,\; D(y))$$
\end{exercise}

where $D$ (for ``drinks'') is some predicate symbol of arity $1$.


