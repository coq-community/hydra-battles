\chapter{Natural Deduction (in construction)}


%%%%%%%%%%%%%% New chapter 
\section{Contexts as sets}

Let us look again at the proof scripts shown in~\vref{proof:PrfEx2} and \vref{proof:PrfEx3}.
\begin{itemize}
\item The statement of \texttt{PrfEx2} contains a sequence
of hypotheses with two occurrences of \texttt{A}. Moreover, the order in which the 4 hypotheses are listed is determined by the type of the constructor \texttt{MP}  (please look at
Figure~\vref{fig:prf-def}). It would be better to replace this
precise list of hypotheses with ``any list whose elements
belong to the set $\{A, A \arrow B; A \arrow B \arrow C\}$''.

\item In the proof of \texttt{PrfEx3} the \coq user would certainly ask ``How do we apply implication's introduction rule?''.
\end{itemize}

The answer to both questions in~\cite{Goedel}  is the definition of a proof system, derived from \texttt{Prf}, which considers
\emph{sets of hypotheses} (called \emph{systems} in ~\cite{Goedel}) instead of
\emph{list of hypotheses}, thus making abstraction of
repetition and order of appearance of hypotheses in the context.

The new system is simply defined as below:
\inputsnippets{folProof/SysPrfDef}


\subsubsection{Notations}


\begin{itemize}
\item  $\Gamma \sdash A$ for
  (\texttt{SysPrf $L$ $\Gamma$ $A$}).
\item $\Gamma, \Delta$ for $\Gamma\cup\Delta$
\item $\Gamma, A,\dots B$ for $\Gamma\cup\{A \dots B\}$
\end{itemize}

\subsubsection{Examples}


In the following script, we use \texttt{PrfEx2}
as a witness for proving a set-based version of the 
original proof term.



$$A \sststile{\mathrm{S}}{} A$$



\inputsnippets{folExamples/SysPrfEx2}


\begin{exercise}

Prove the following lemma:
\inputsnippets{FolExamples/MPSys}

\end{exercise}

\subsection{Using properties of sets}

The following three lemmas, from 
\href{../theories/html/hydras.Ackermann.folLogic.html}{Ackermann.folLogic} are direct consequences of \texttt{SysPrf}'s definition. 

\inputsnippets{folLogic/Axm}

\inputsnippets{folLogic/SysExtend}




\section{The deduction theorem}

\inputsnippets{Deduction/deductionThm}



\section{Derived rules and natural deduction}


 The library 
 \href{../theories/html/hydras.Ackermann.folLogic.html}{Ackermann.folLogic} contains many derived rules which allow the user to build proofs in a natural deduction style (with introduction and elimination rules).

\subsection{Example}

For instance, let us prove Peirce's rule.
 (in  ~\href{../theories/html/hydras.MoreAck.FolExamples.html}{MoreAck.FolExamples} ).

\subsubsection{Prelude}

%\inputsnippets{folPeirce/prelude}


\subsection{Proof of Peirce's law}


\inputsnippets{FolExamples/PeirceProof}

Let us try to do an implication introduction.

\inputsnippets{FolExamples/step1}

Now, we may use the law of excluded middle with the formula $P$. The only non-trivial case is about $\sim P$.

\inputsnippets{FolExamples/step2}

The rest of the proof is composed of basic proof steps, 
and bookkeeping steps (about \texttt{Ensembles.In}).

\inputsnippets{FolExamples/step3, FolExamples/step4}

\index{ackermann}{Exercises}
\begin{exercise}
Prove, using the rules described in 
 \href{../theories/html/hydras.Ackermann.folLogic.html}{Ackermann.folLogic}, the famous \emph{drinkers theorem}:

$$\exists\,x,\; (D(x)\Longrightarrow \forall\,y,\; D(y))$$
\end{exercise}

where $D$ (for ``drinks'') is some predicate symbol of arity $1$.


