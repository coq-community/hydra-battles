\documentclass[a4paper]{book}
\usepackage{alltt}
\usepackage{makeidx}
\usepackage{hyperref}
\usepackage{url}
\usepackage{mdframed}
\usepackage{amsmath}
\usepackage{varioref}
\usepackage{synttree, proof, centernot}
\usepackage[firstpage]{draftwatermark}
\usepackage{verbatim}
\usepackage[note]{marginote}
\usepackage{amsmath}
\usepackage{amsfonts}
\usepackage{mathtools}
\usepackage{dsfont}
\usepackage{threeparttable}
\setcounter{tocdepth}{1}
\definecolor{termcolor}{rgb}{0.1,0.1,0.9}
\definecolor{prooftermcolor}{rgb}{0.3,0.1,1.0}
\definecolor{metavarcolor}{rgb}{0.5,0.0,1.0}
\definecolor{darkgreen}{rgb}{0.1,0.7,0.1}
\definecolor{answercolor}{rgb}{.8,.15,.08}
\definecolor{sourcecolor}{rgb}{.07,.1,.7}
\definecolor{normalcolor}{rgb}{0.0,0.0,0.0}
\definecolor{exbluecolor}{rgb}{0.1,0.1,0.9}
\definecolor{dontlookcolor}{rgb}{0.5,0.5,0.5}
\definecolor{termcolor}{rgb}{0.05,0.05,0.4}
\definecolor{lookcolor}{rgb}{0.9,0.1,0.0}
\definecolor{darklookcolor}{rgb}{0.5,0.1,0.0}
\definecolor{prooftermcolor}{rgb}{0.3,0.1,1.0}
\definecolor{typecolor}{rgb}{1.0,0.4,0.0}
\definecolor{taccolor}{rgb}{0.1,0.10,0.0}
\definecolor{pink}{rgb}{0.8,0.6,0.6}
\definecolor{darkmagenta}{rgb}{0.4,0.0,0.6}
\definecolor{darkblue}{rgb}{0.0,0.0,0.6}


%\newcommand{\todo}{\textbf{To do}}




%%% MMCG symbols

\newcommand{\idot}[1]{\mbox{$\bullet_{#1}$}}
\newcommand{\baredot}{\mbox{$\bullet$}}
\newcommand{\iback}[1]{\mbox{$\backslash_{#1}$}}
\newcommand{\back}{\mbox{$\backslash$}}
\newcommand{\islash}[1]{\mbox{$/_{#1}$}}
\newcommand{\icomma}[3]{\mbox{$(#1\,,\,#2)_{#3}$}}
%\newcommand{\icomma}[3]{\mbox{$#1\,\mathbin{\circ}_{#3}\,#2$}}
%\newcommand{\comma}{\mbox{$\mathbin{\circ}$}}
\newcommand{\comma}{\mbox{,}}

\newcommand{\slashe}{\mbox{\textbf{$/_{\textbf{E}}$}}}
\newcommand{\slashi}{\mbox{\textbf{$/_{\textbf{I}}$}}}
\newcommand{\backe}{\mbox{\textbf{$\backslash_{\textbf{E}}$}}}
\newcommand{\backi}{\mbox{\textbf{$\backslash_{\textbf{I}}$}}}

\newcommand{\dote}{\mbox{\textbf{$\baredot_{\textbf{E}}$}}}
\newcommand{\doti}{\mbox{\textbf{$\baredot_{\textbf{I}}$}}}
\newcommand{\diame}{\mbox{\textbf{$\Diamond_{\textbf{E}}$}}}
\newcommand{\diami}{\mbox{\textbf{$\Diamond_{\textbf{I}}$}}}
\newcommand{\struct}{\mbox{\textbf{struct}}}
\newcommand{\axrule}{\mbox{\textbf{Ax}}}
\newcommand{\boxe}{\mbox{\textbf{$\Box_{\textbf{E}}$}}}
\newcommand{\boxi}{\mbox{\textbf{$\Box_{\textbf{I}}$}}}
\newcommand{\AssD}{\textbf{L$_\Diamond$}}
\newcommand{\ComD}{\textbf{P$_\Diamond$}}

\newcommand{\marginok}[1]{\marginpar{\raggedright OK:#1}}
\newcommand{\tab}{{\null\hskip1cm}}
\newcommand{\Ltac}{\mbox{{$\cal L$}tac}}
%\newcommand{\coq}{\mbox{{Coq}}}
\newcommand{\compcert}{\mbox{{CompCert}}}

\newcommand{\pcoq}{\mbox{{Pcoq}}}
\newcommand{\grail}{\mbox{{Grail}}}
\newcommand{\lcf}{\mbox{{LCF}}}
\newcommand{\hol}{\mbox{{HOL}}}
\newcommand{\pvs}{\mbox{{PVS}}}
\newcommand{\icharate}{\mbox{{Icharate}}}
\newcommand{\isabelle}{\mbox{{Isabelle}}}
%\newcommand{\coq}{\mbox{{Coq}}}
\newcommand{\prolog}{\mbox{{Prolog}}}
\newcommand{\goalbar}{\tt{}{\color{black}------------------------------------}\it}
\newcommand{\gallina}{Gallina\xspace}
\newcommand{\joker}{\texttt{\_}}
\newcommand{\eprime}{\(\e^{\prime}\)}
\newcommand{\Ztype}{\textbf{Z}}
\newcommand{\propsort}{\textbf{Prop}}
\newcommand{\setsort}{\textbf{Set}}
\newcommand{\typesort}{\textbf{Type}}
\newcommand{\ocaml}{\mbox{{OCaml}}}
\newcommand{\haskell}{\mbox{{Haskell}}}
\newcommand{\why}{\mbox{{Why}}}
\newcommand{\Pascal}{\mbox{{Pascal}}}

\newcommand{\ml}{\mbox{{ML}}}

\newcommand{\scheme}{\mbox{{Scheme}}}
\newcommand{\lisp}{\mbox{{Lisp}}}

\newcommand{\implarrow}{\mbox{$\Rightarrow$}}
\newcommand{\metavar}[1]{?#1}
\newcommand{\notincoq}[1]{#1}
\newcommand{\coqscope}[1]{\%#1}
\newcommand{\arrow}{\mbox{$\rightarrow$}}
\newcommand{\fleche}{\arrow}
\newcommand{\funarrow}{\mbox{$\Rightarrow$}}
\newcommand{\ltacarrow}{\funarrow}
 \newcommand{\coqand}{\mbox{\(\wedge\)}}
 \newcommand{\coqor}{\mbox{\(\vee\)}}
 \newcommand{\coqnot}{\mbox{\(\neg\)}}
\newcommand{\hide}[1]{}
\newcommand{\hidedots}[1]{...}
\newcommand{\sig}[3]{\texttt{\{}#1\texttt{:}#2 \texttt{|} #3\texttt{\}}}
\renewcommand{\neg}{\mbox{$\sim$}}

%%% Operateurs, etc.
\newcommand{\impl}{\mbox{$\rightarrow$}}
\newcommand{\appli}[2]{\mbox{\tt{#1 #2}}}
\newcommand{\applis}[1]{\mbox{\texttt{#1}}}
\newcommand{\abst}[3]{\mbox{\tt{fun #1:#2 \funarrow #3}}}
\newcommand{\coqle}{\mbox{$\leq$}}
\newcommand{\coqge}{\mbox{$\geq$}}
\newcommand{\coqdiff}{\mbox{$\neq$}}
\newcommand{\coqiff}{\mbox{$\leftrightarrow$}}
\newcommand{\prodsym}{\mbox{\(\forall\,\)}}
\newcommand{\exsym}{\mbox{\(\exists\,\)}}
\newcommand{\included}{\mbox{\(\subseteq\)}}

\newcommand{\substsign}{/}
\newcommand{\subst}[3]{\mbox{#1\{#2\substsign{}#3\}}}
\newcommand{\anoabst}[2]{\mbox{\tt[#1]#2}}
\newcommand{\letin}[3]{\mbox{\tt let #1:=#2 in #3}}
\newcommand{\prodep}[3]{\mbox{\tt \(\forall\,\)#1:#2,$\,$#3}}
\newcommand{\prodplus}[2]{\mbox{\tt\(\forall\,\)$\,$#1,$\,$#2}}
\newcommand{\dom}[1]{\textrm{dom}(#1)} % domaine d'un contexte (log function)
\newcommand{\norm}[1]{\textrm{n}(#1)} % forme normale (log function)
\newcommand{\coqZ}[1]{\mbox{\tt{`#1`}}}
\newcommand{\coqnat}[1]{\mbox{\tt{#1}}}
\newcommand{\coqcart}[2]{\mbox{\tt{#1*#2}}}
\newcommand{\alphacong}{\mbox{$\,\cong_{\alpha}\,$}} % alpha-congruence
\newcommand{\betareduc}{\mbox{$\,\rightsquigarrow_{\!\beta}$}\,} % beta reduction
%\newcommand{\betastar}{\mbox{$\,\Rightarrow_{\!\beta}^{*}\,$}} % beta reduction
\newcommand{\deltareduc}{\mbox{$\,\rightsquigarrow_{\!\delta}$}\,} % delta reduction
\newcommand{\dbreduc}{\mbox{$\,\rightsquigarrow_{\!\delta\beta}$}\,} % delta,beta reduction
\newcommand{\ireduc}{\mbox{$\,\rightsquigarrow_{\!\iota}$}\,} % delta,beta reduction

\newcommand{\slam}[2] {\mbox{$\lambda_#1\;#2$}}
\newcommand{\srho}[2] {\mbox{$\rho_#1\;#2$}}
\newcommand{\Vforall}[2] {\mbox{$\forall\,#1\,\in\,V(#2)$}}
\newcommand{\Eforall}[2] {\mbox{$\forall\,#1\,\in\,E(#2)$}}
\newcommand{\Vforone}[2] {\mbox{$\exists!\,#1\,\in\,V(#2)$}}
\newcommand{\Eforone}[2] {\mbox{$\exists!\,#1\,\in\,E(#2)$}}
\newcommand{\Vforsome}[2] {\mbox{$\exists\,#1\,\in\,V(#2)$}}
\newcommand{\Eforsome}[2] {\mbox{$\exists\,#1\,\in\,E(#2)$}}


% jugement de typage
\newcommand{\these}{\mbox{$\boldsymbol{\large \vdash}$}}
\newcommand{\rthese}[1]{\mbox{$\boldsymbol{\vdash_{#1}}$}}
\newcommand{\replace}[2]{\mbox{$#1[#2]$}}
\newcommand{\msubst}[2]{\mbox{$#1\{#2\}$}}
\newcommand{\disj}{\mbox{$\backslash/$}}
\newcommand{\conj}{\mbox{$/\backslash$}}
\newcommand{\deriv}[2]{\mbox{$#1\;\these\;#2$}}
\newcommand{\smalljuge}[3]{\mbox{$#1 \these #2 \boldsymbol{:} #3 $}}
%\newcommand{\juge}[3]{\mbox{$#1 \these #2 \boldsymbol{:} #3 $}}
\newcommand{\goal}[3]{\mbox{$#1,#2 \these^{\!\!\!?} #3  $}}
\newcommand{\sgoal}[2]{\mbox{$#1\,\these^{\!\!\!\!?}\, #2 $}}
\newcommand{\sequent}[2]{\mbox{$#1\;\these\; #2 $}}

\newcommand{\reduc}[5]{\mbox{$#1,#2 \these #3 \rhd_{#4}#5 $}}
\newcommand{\convert}[5]{\mbox{$#1,#2 \these #3 =_{#4}#5 $}}
\newcommand{\convorder}[5]{\mbox{$#1,#2 \these #3\leq _{#4}#5 $}}
\newcommand{\wouff}[2]{\mbox{$\emph{WF}(#1)[#2]$}}




\newcommand{\type}{\boldsymbol{:}}

% jugement absolu

%\newcommand{\ajuge}[2]{\mbox{$ \boldsymbol{\vdash} #1 : #2 $}}
\newcommand{\ajuge}[2]{\mbox{$\these #1 \boldsymbol{:} #2 $}}







\newcommand{\coqsimple}[1]{\mbox{\textbf{#1}}}
\newcommand{\cons}{\mbox{\tt\,::\,}}
\newcommand{\nil}{\mbox{\tt\,[\,]\,}}


%%% Colors 


\definecolor{mintedbgcolor}{rgb}{0.95,0.95,1.0}
\definecolor{badcoqcolor}{rgb}{0.8,0.8,0.8}
\definecolor{altcoqcolor}{rgb}{0.7,1.0,0.8}
\definecolor{mathcolor}{rgb}{0.95,0.90,0.85}
\definecolor{vertfluo}{rgb}{0.623, 0.94, 0.886}
\definecolor{lightred}{rgb}{1.0, 0.74, 0.7}
\definecolor{lightgray}{rgb}{0.7, 0.7, 0.7}
\definecolor{lookcolor}{rgb}{0.9,0.2,0.0}
\definecolor{darkred}{rgb}{.3,.0,.0}
\definecolor{sourcecolor}{rgb}{.07,.1,.7}
\definecolor{cyan}{rgb}{.0,1.0,1.0}


\newunicodechar{𝟙}{\ensuremath{\mathds{1}}}
\newunicodechar{ℤ}{\ensuremath{\mathds{Z}}}




\newcommand{\rounds}{\mbox{\,\texttt{-+->}\,}}
\newcommand{\round}{\mbox{\,\texttt{-1->}\,}}
\newcommand{\rplus}[1]{\mbox{$\,\underset{#1}{\longrightarrow}\,$}}
\newcommand{\canonseq}[2]{\mbox{$\{#1\}(#2)$}}
\newcommand{\showmath}[1]{\mathcolor{\mbox{$#1$}}}
\newcommand{\bigmath}[1]{\textcolor{blue}{\mbox{\[#1\]}}}
\newcommand{\myemph}[1]{\textcolor{lookcolor}{\,\it{#1}\,}}

%%% Coq and libraries

\newcommand{\coq}{Coq\xspace}
\newcommand{\community}{Coq-community\xspace}
\newcommand{\gaia}{Gaia\xspace}
\newcommand{\alectr}{Alectryon\xspace}
\newcommand{\equations}{Equations\xspace}
\newcommand{\Hydras}{Hydras \& Co$\text.$\xspace}
\newcommand{\HydrasLib}{Hydra-battles\xspace}
\newcommand{\gaiaHydras}{Gaia-hydras\xspace}
\newcommand{\ssreflect}{SSReflect\xspace}
\newcommand{\stdpp}{Stdpp\xspace}
\newcommand{\mathcomp}{MathComp\xspace}
\newcommand{\additions}{Addition-chains\xspace}
%%% Gaia sign

\newcommand{\mycircled}[2][none]{%
 \tikz[baseline=(a.base)]\node[draw,circle,inner sep=1pt, outer sep=0pt,fill=#1](a){\ensuremath{\scriptsize #2}\strut};
 }

{\theorembodyfont{\upshape}
 \newtheorem{exercise}{Exercise}[chapter]
 \newtheorem{project}{Project}[chapter]
\newtheorem{remark}{Remark}[chapter]
}

\newtheorem{theorem}{Theorem}[chapter]
\newtheorem{proposition}{Proposition}[chapter]

\newtheorem{lemma}{Lemma}[chapter]
\newtheorem{conjecture}{Conjecture}[chapter]
\newtheorem{definition}{Definition}[chapter]
\newtheorem{todo}{To do}[chapter]

\newmdenv[linecolor=red]{mathframe}

\newcommand{\mathcolor}[1]{\colorbox{mathcolor}{\mbox{#1}}}

%% ajoute des parenthèses légères aux expressions de la forme "f x y"

\newcommand{\kscite}[1]{\texttt{#1} {of} KS~\cite{KS81}}
\newcommand{\kpcite}[1]{\texttt{#1} {of} KP~\cite{KP82}}
\newcommand{\slcite}[1]{\texttt{#1} {of} TT~\cite{Sladek07thetermite}}

%% KS eaters
\newcommand{\gnaw}[2]{\mbox{$\{#1\}\langle #2 \rangle $}}

%\input{setup-latex}

\DefineVerbatimEnvironment%
{Coqsrc}{Verbatim}
{fontsize=\small, frame=single,rulecolor=\color{blue},fillcolor=\color{blue!05}}


\DefineVerbatimEnvironment%
{Coqanswer}{Verbatim}
{fontsize=\small, frame=single,fontshape=it,rulecolor=\color{red},fillcolor=\color{red!05}}

\DefineVerbatimEnvironment%
{Coqbad}{Verbatim}
{fontsize=\small, frame=single,rulecolor=\color{black},fillcolor=\color{black!05}}


\DefineVerbatimEnvironment%
{Coqalt}{Verbatim}
{fontsize=\small, frame=single,rulecolor=\color{green},fillcolor=\color{green!05}}

\setcounter{secnumdepth}{5}
\DeclarePairedDelimiter{\floor}{\lfloor}{\rfloor}
\DeclarePairedDelimiter{\ceil}{\lceil}{\rceil}




\author{Pierre Castéran\\ LaBRI, Univ. Bordeaux, CNRS UMR 5800, Bordeaux-INP\\with contributions by \'Evelyne Contejean,  Florian Hatat and Pascal Manoury}
\date{\today}
\title{Hydra battles in Coq}
\makeindex
\begin{document}
\maketitle




\cleardoublepage




% \mbox{~}
% \clearpage
% \newpage
% \mbox{~}
\vspace{10cm}
\begin{quote} 
{ \Large {\it
``I start from one point and go as far as possible.'' } 

John Coltrane.}
\end{quote}


% \begin{verse}
% Hydra hydra hydram \\

% Hydrae hydrae hydra \\

% Hydrae hydrae hydras \\

% Hydrarum hydris hydris\\
% \end{verse}


\tableofcontents
 

%-------------------------------------------------------------------

\chapter{Introduction}

  

\vspace{16pt}



Proof assistants are excellent tools for exploring the structure of mathematical proofs,
studying  which hypotheses are really needed, and which proof patterns are useful and/or
necessary. Since the development of a theory is represented as a bunch of computer files,
everyone is able to read the proofs with an arbitrary level of detail, or to play with the theory by writing alternate proofs or definitions.


Among all the theorems proved with the help of proof assistants like \coq{}, \isabelle{}, \hol{}, etc.,
several statements and proofs  share some interesting features:
\begin{itemize}
\item Their statements are easy to understand, even by non-mathematicians
\item Their proof requires some non-trivial mathematical tools
\item Their mechanization on computer presents some methodological interest.
\end{itemize}


This is obviously the case of the four-color theorem~\cite{fourcolors}  and the Kepler conjecture~\cite{flyspeck2015}. We do not mention impressive works like the proof of the odd-order theorem ~\cite{oddorderthm}, since understanding its statement requires a quite good mathematical culture.




Hydra games (a.k.a. \emph{Hydra battles}) appeared in an article published in 1982 by two mathematicians:
L. Kirby and J. Paris~\cite{KP82}: \emph{Accessible Independence Results for Peano Arithmetic}. 
Although the mathematical contents of this 
paper are quite advanced, the rules of hydra battles are very easy to understand. There are now several sites on Internet where you can find tutorials on hydra games, together with simulators you can play with. See, for instance, the page written by Andrej Bauer~\cite{bauer2008}.



Hydra battles, as well as Goodstein Sequences~\cite{goodstein_1944, KP82}
are a nice way to present complex termination problems.
The article by Kirby and Paris presents a proof of termination
based on ordinal numbers, as well as a proof that this termination is not
provable in Peano arithmetic. In the book dedicated to 
J.P. ~Jouannaud \cite{HommageJPJ}, N.~Dershowitz and G.~Moser  give a thorough survey on this topic~\cite{Dershowitz2007}.



Here, we present a development  for the \coq{} proof assistant, after the work of Kirby and Paris. This formalization contains the following main parts:

\begin{itemize}
\item Representation in \coq{} of hydras and hydra battles
\item A proof that every battle is finite and won by Hercules. This proof is based on a \emph{variant} which maps any hydra to an ordinal strictly less than $\epsilon_0$ and is strictly decreasing along any battle.

\item Using a combinatorial toolkit designed by J.~Ketonen and R.~Solovay~\cite{KS81}, we prove that, for any ordinal $\mu<\epsilon_0$, there exists no such variant mapping any hydra to an ordinal stricly less than $\mu$. Thus, the complexity of $\epsilon_0$ is really needed in the previous proof.

\item We prove a relation between the length of a ``classic''  class of  battles \footnote{This class is also called \emph{standard} in this document (text and proofs). The \emph{replication factor} of the hydra is exactly $i$ at the $i$-th round of the battle (see Sect~\vref{sect:replication-def}).}
and the Wainer-Hardy hierarchy of ``rapidly growing functions'' $H_\alpha$~\cite{Wainer1970}. The considered class of battles, which we call \emph{standard}  is the most considered one in the scientific  litterature(including popularization).
\end{itemize}


Simply put, this document tries to combines the scientific interest of two articles~\cite{KP82, KS81} and a book~\cite{schutte} with the playful activity of truly proving theorems.
We hope  that such a work, besides exploring a nice piece of discrete maths, 
will show how \coq{} and its standard library are well fitted to help us to understand some non-trivial mathematical developments, and also to experiment the constructive parts of  the proof through functional programming.

 We also hope to provide a little clarification on infinity (both potential and actual) through the notions of function, computation, limit,
 types and proofs.



\section{Remarks}

\subsection{Difference from Kirby and Paris's work}
In~\cite{KP82}, Kirby and Paris showed  that there is no proof of termination of all hydra battles in Peano Arithmetic (PA).
Since we are used to writing proofs in higher order logic, the restriction to PA was quite unnatural for us. So we chosed to prove another statement without any reference to PA, by considering a class of proofs indexed by ordinal numbers upto $\epsilon_0$.

\subsection{Trust in our proofs?}
\label{sect:trust-in-proofs}

Unlike mathematical literature, where definitions and proofs are spread over many articles and books,
the whole proof is now inside your computer. It is composed of the \texttt{.v} files you downloaded and 
parts of \coq's standard library. Thus, there is no ambiguity in our definitions and the premises of the theorems. Furthermore, you will be able to navigate through the development, using your favourite text editor or IDE, and some commands like \texttt{Search}, \texttt{Locate},  etc.



\subsection{Assumed redundancy}

It may often happen that several definitions of a given concept, or several proofs of a given theorem are possible. If all the versions present some interest, we will make them available, since each one may be of some methodological 
interest (by illustrating some tactic of proof pattern, for instance).
We may use \coq's module system to make several proofs of a given theorem co-exist in our libraries (see also Sect~\vref{sect:alt-proofs}).
After some discussions of the pros and cons of each solution, we develop only one of them, leaving the others  as exercises or projects (i.e., big or difficult exercises).
In order to discuss which assumptions are really needed for proving a theorem, we will also present 
several aborted proofs.
Of course, do not hesitate to contribute nice proofs or alternative definitions !

It may also happen that some proof looks to be useless, because the proven theorem is a trivial consequence of another (proven too) result.
For instance, let us consider the three following statements:
\begin{enumerate}
\item There is no measure into $\mathbb{N}$ for proving the termination of all hydra battles (Sect~\vref{omega-case}).
\item There is no measure into $[0,\omega^2)$ for proving the termination of all hydra battle (Sect~\vref{omega2-case}).
\item There is no measure into $[0,\mu)$ for proving the termination of all hydra battles, for any $\mu<\epsilon_0$ (Sect\vref{sec:free-battles-case}).
\end{enumerate}

Obviously, the third theorem implies the second one, which implies the first one. So, theoretically, a library would contain only a proof of $(3)$ and remarks for $(2)$ and $(1)$. But we found it interesting to make all the three proofs available, allowing the reader to compare their common structure and notice their technical differences.
In particular, the proof of $(3)$ uses several non-trivial combinatorial properties of ordinal numbers up to $\epsilon_0$~\cite{KS81}, whilst $(1)$ and $(2)$ use simple properties of $\mathbb{N}$ and $\mathbb{N}^2$.


\subsection{About Logic}

Most of the proofs we present are \emph{constructive}. Whenever possible, we provide the user with an associated function, which she or he can apply in \gallina{} or \ocaml{} in order to get a ``concrete'' feeling of the meaning of the considered theorem.
For instance, in Chapter~\vref{chap:ketonen}, the notion of \emph{limit ordinal} is
made more ``concrete'' thanks to a function \texttt{canon} which computes every item of a sequence which converges on a given limit ordinal $\alpha$. This simply typed function allows the user/reader to make her/his own experimentations.
For instance, one can very easily compute the $42$-nd item of a sequence which converges towards $\omega^{\omega^\omega}$.


 
Except in the \texttt{Schutte} library, dedicated to an axiomatic presentation of the set of countable ordinal numbers, all our development is axiom-free, and respects the rules of intuitionistic logic. Note that we also use the \texttt{Equations} plug-in~\cite{sozeau:hal-01671777} in the definitition of  several rapidly growing hierarchy of functions, in Chap.~\ref{chap:alpha-large}. This plug-in imports several known-as-harmless  axioms.

% \begin{Coqsrc}
% FunctionalExtensionality.functional_extensionality_dep : 
% forall (A : Type) (B : A -> Type) (f g : forall x : A, B x),
% (forall x : A, f x = g x) -> f = g
% \end{Coqsrc}

\index{Coq!Commands!Print Assumptions}

At any place of our development, you may use the  \texttt{Print Assumptions {\it ident}} command in order to verify on which axiom the theorem {\it ident} may depend. The following example is extracted from 
Library~\href{../src/html/hydras.Epsilon0.F_alpha.html}{hydras.Epsilon0.F\_alpha}, where we use the \texttt{coq-equations} plug-in (see Sect.~\vref{sect:wainer}).

\begin{Coqsrc}
About F_zero_eqn.
\end{Coqsrc}

\begin{Coqanswer}
F_zero_eqn : forall i : nat, F_ Zero i = S i
\end{Coqanswer}

\begin{Coqsrc}
Print Assumptions F_zero_eqn. 
\end{Coqsrc}

\begin{Coqanswer}
Axioms:
FunctionalExtensionality.functional_extensionality_dep
  : forall (A : Type) (B : A -> Type) (f g : forall x : A, B x),
    (forall x : A, f x = g x) -> f = g
Eqdep.Eq_rect_eq.eq_rect_eq
  : forall (U : Type) (p : U) (Q : U -> Type) (x : Q p) (h : p = p),
    x = eq_rect p Q x p h
\end{Coqanswer}



\subsection{Main references}

In our development, we adapt the definitions and prove many theorems which
we found in the following articles. 
\begin{itemize}
\item ``Accessible independence results for Peano arithmetic''  by Laurie Kirby and Jeff Paris~\cite{KP82}
\item ''Rapidly growing Ramsey Functions'' by Jussi Ketonen and Robert Solovay~\cite{KS81}
\item ``The Termite and the Tower'', by Will Sladek~\cite{Sladek07thetermite}
\item Chapter V of ``Proof Theory'' by Kurt Schütte~\cite{schutte}
\end{itemize}







\paragraph*{Warning:}

This document is \emph{not} an introductory text for \coq{}, and there are many aspects of this proof assistant that are not covered. 
 The reader should already have some basic experience with the \coq{} system. The Reference Manual and several tutorials are available on \coq{} page~\cite{Coq}.  First chapters of textbooks like \emph{Interactive Theorem Proving and Program Development}~\cite{BC04}, \emph{Software Foundations}~\cite{SF} or  \emph{Certified Programming with Dependent Types} ~\cite{chlipalacpdt2011} will give you the right background. 


\subsection{State of the development}
The \coq{} scripts herein are in constant development since our contribution~\cite{CantorContrib} on  notations for the ordinals $\epsilon_0$ and $\Gamma_0$.
We added new material : axiomatic definition of countable ordinals after Schütte~\cite{schutte}, combinatorial aspects of $\epsilon_0$, after Ketonen and Solovay~\cite{KS81} and Kirby and Paris~\cite{KP82}, recent \coq{} technology: type classes, equations, etc.

We are now working in order to make clumsy proofs more readable, simplify definitions, and ``factorize'' proofs as much as possible. 
Many possible improvements are suggested as ``todo''s or ``projects'' in this text.

\subsection{Contributions}
Many thanks to \'Evelyne Contejean,  Florian Hatat and Pascal Manoury  for their contribution. \'Evelyne contributed libraries on the recursive path ordering (\emph{rpo}) for proving the well-foundedness of our representation of $\epsilon_0$ and $\Gamma_0$.
Florian Hatat proved many useful lemmas on countable sets, which we used in our adaptation of Schütte's formalization of countable ordinals. Pascal Manoury is integrating the ordinal $\omega^\omega$ into our hierarchy of ordinal notations.



\label{sec:orgheadline2}

Any form of contribution  is welcome: correction of errors (typos and more serious mistakes), improvement of
Coq scripts, proposition of inclusion of new chapters, and generally any
comment or proposition that would help us. The text contains several \emph{projects} which, when completed, may improve the present work.
Please do not hesitate to bring your contribution!



\subsection{Typographical conventions}

Quotations from our  \coq{} source are displayed as follows:


  \begin{Coqsrc}
 Require Import Arith.

 Definition square (n:nat) := n * n.

 Lemma square_double : exists n:nat, n + n = square n.
 Proof.
    exists 2. 
  \end{Coqsrc}

Answers from \coq{} (including sub-goals, error messages, etc.) are displayed in slanted style
with a different background color.



 \begin{Coqanswer}
 1 subgoal, subgoal 1 (ID 5)
  
  ============================
   2 + 2 = square 2
   
 \end{Coqanswer}

 \begin{Coqsrc}
   reflexivity.
Qed.
 \end{Coqsrc}


\subsection{Active links}
The  links which appear in this pdf  document lead are of three possible kinds of destination:
\begin{itemize}
\item Local links to the document itself,
\item External links, mainly to \coq's page,
\item Local links to pages generated by \texttt{coqdoc}. According to the current makefile,
  we assume that the page generated from a library \texttt{XXX/YYY.v} is stored as
the relative address \texttt{../src/html/hydras.XXX.YYY.html} (from the location of the pdf).
Thus,  active links towards our \coq{} modules may be incorrect if you got this \texttt{pdf} document otherwise than by compiling the distribution available in
\url{https://framagit.org/casteran/hydra-battles-in-coq}.

\end{itemize}

\subsection{Alternative or bad definitions}
\label{sect:alt-proofs}
Finally, we decided to include definitions or lemma statements, as well as tactics,  that lead to
dead-ends or too complex developments, with the following coloring.
Bad definitions 
 and encapsulation in modules called \texttt{Bad}, \texttt{Bad1}, etc.


\begin{Coqbad}
Module Bad.

Definition double (n:nat)  := n + 2.
 
Lemma lt_double : forall n:nat, n < double  n.
Proof.
   unfold double; omega.
Qed.

End Bad.
\end{Coqbad}

Likewise, alternative, but still unexplored definitions will be presented in modules
\texttt{Alt}, \texttt{Alt1}, etc. Using these definitions is left as an implicit exercise.


\begin{Coqalt}
Module Alt.

  Definition double (n : nat) := 2 * n.

End Alt.
\end{Coqalt}

\begin{Coqsrc}
Lemma alt_double_ok n : Nat.double n = Alt.double n.
Proof.
  unfold Alt.double, Nat.double.
  omega.
Qed.
\end{Coqsrc}

\section{Acknowledgements}
\label{sec:orgheadline5}
    Many thanks to David Ilcinkas, Sylvain Salvati, Alan Schmitt and Théo Zimmerman for their help on the elaboration of this document, and to the
 members of the \emph{Formal Methods} team at laBRI for their helpful comments 
on an oral presentation of this work. 

Many thanks also to the Coq development team, Yves Bertot, and members of the \emph{Coq Club} for interesting discussions about the \coq{} system and the Calculus of Inductive Constructions.

The author of the present document wishes to express his gratitude to the late Patrick Dehornoy, whose talk  was determinant for our desire to work on this topic.

I owe my interest in discrete mathematics and their relation to formal proofs and functional programming  to Srecko Brlek.  Equally, there is W. H. Burge's book ``\emph{Recursive Programming Techniques}'' ~\cite{burge} which was a great  source of inspiration.



% \subsubsection{Links to the Coq source}







%-----------------------------------------------------------


\chapter{Hydras and Hydra Games}
\label{sec:orgheadline91}

This chapter is dedicated to the representation of hydras and rules of the hydra game in \coq's specification language:\gallina. 

Technically, a \emph{hydra} is just a finite ordered tree, each node of which 
has any number of sons. Note that, contrary to the computer science tradition, we will show the hydras 
with the heads up and the foot (i.e., the root of the tree) down.
Fig.~\ref{fig:Hy} represents such  a hydra, which will be referred to as \texttt{Hy} in our examples (please look at the 
module~\href{../src/html/hydras.Hydra.Hydra_Examples.html}{Hydra.Hydra\_Examples}). 

\begin{figure}[h]
\centering
\begin{tikzpicture}[very thick, scale=0.6]

\node (foot) at (2,0) {$\bullet$};
\node (N1) at (2,2) {$\bullet$};
\node (N2) at (2,4) {$\bullet$};
\node (N3) at (2,6) {$\bullet$};
\node (H0) at (0,2) {$\Smiley[2][vertfluo]$};
\node (H1) at (0,8) {$\Smiley[2][vertfluo]$};
\node (H2) at (4,8) {$\Smiley[2][vertfluo]$};
\node (H4) at (4,2) {$\Smiley[2][vertfluo]$};
\node (H5) at (4,4) {$\Smiley[2][vertfluo]$};
\draw (foot) -- (N1)[very thick] ;
\draw (N1) -- (N2);
\draw (N2) -- (N3);
\draw (N3) to [bend left= 10]  (H1) ;
\draw (N3) to [bend right= 16] (H2);
\draw (foot) to [bend left= 10]  (H0) ;
\draw (foot) to [bend right = 10] (H4) ;
\draw (N1) to [bend right= 16] (H5);
\end{tikzpicture}
\caption{The hydra Hy \label{fig:Hy}}
\end{figure}



We use a specific vocabulary for talking about hydras. Table~\ref{tab:hyd2tree} shows the correspondance between our terminology and the usual vocabulary for trees in computer science.


\begin{figure}[h]
  \centering
  \begin{tabular}{ll}
Hydras & Finite rooted trees\\
\hline
foot & root\\
head & leaf\\
node & node\\
segment  & (directed) edge \\
sub-hydra & subtree\\
daughter & immediate subtree\\
\end{tabular}
  \caption{Translation from hydras to trees}
  \label{tab:hyd2tree}
\end{figure}


The hydra \texttt{Hy} has a \emph{foot} (below), five \emph{heads}, and eight \emph{segments}. 
We leave it to the reader to define various parameters such as the height, the size, the highest arity (number of sons of a node) of a hydra. In our example, these parameters have the respective values : $4$, $9$ and $3$.




\subsection{The rules of the game}

\label{sec:orgheadline44}
\label{sect:replication-def}

A \emph{hydra battle} is a fight between Hercules and the Hydra. 
More formally, a  battle is a sequence of \emph{rounds}.
At each round:
\begin{itemize}
\item If the hydra is composed of just one head, the battle is finished
and  Hercules is the winner.
\item Otherwise, Hercules chops off \emph{one} head of the hydra,

\begin{itemize}
\item If the head is at distance 1 from the foot, the head is just lost by the hydra, with no more reaction.
\item Otherwise, let us denote by \(r\) the node that was at distance \(2\) from 
the removed head in the direction of the foot,  and consider the  sub-hydra \(h'\) of \(h\), whose  root is \(r\) \footnote{$h'$ will be called ``the wounded part of the hydra'' in the subsequent text. In Figures~\vref{fig:Hy2} and ~\vref{fig:Hy4}, this sub-hydra  is displayed in red.}. Let $n$ be some natural number.
Then $h'$ is replaced by  $n+1$ of copies of \(h'\) which share the same root $r$.
 The \emph{replication factor} $n$ may be different (and generally is)   at each round of the fight.
It may be chosen by the hydra, according to its strategy, or imposed by some 
particular rule. In many presentations of hydra battles, this number is increased by $1$ at each round. In the following presentation, we will also consider battles where the hydra is free to choose its ~replication factor at each round of the battle\footnote{Let us recall that, if the chopped-off head was at distance 1 from the foot, the replication factor is meaningless.}.
\end{itemize}
\end{itemize}



Note that the description given in~\cite{KP82} of the replication process in hydra battles is also  semi-formal. 

\label{original-rules}

\begin{quote}
  ``From the node that used to be attached to the head which was just chopped off, traverse one 
segment towards the root until the next node is reached. From this node sprout $n$ replicas of 
that part of the hydra (after decapitation) which is ``above'' the segment just traversed, i.e., those 
nodes and segments from which, in order to reach the root, this segment would have to be 
traversed. If the head just chopped off had the root of its nodes, no new head is grown. ''
\end{quote}

Moreover, we note that this description is in \emph{imperative} terms. In order to build a formal  study of the properties of hydra battles, we prefer to use a mathematical vocabulary, i.e., graphs, relations, functions, etc.
Thus, the replication process will be represented as a binary relation on a data type \texttt{Hydra},
linking the state of the hydra \emph{before} and \emph{after} the transformation.
A battle will thus be represented as a sequence of terms of type \texttt{Hydra}, respecting the rules of the game.





\subsection{Example}
Let us start a battle between Hercules and the hydra \texttt{Hy} of Fig.~\ref{fig:Hy}.

At the first round, Hercules choses to chop off the rightmost head of \texttt{Hy}.
Since this head is near the floor, the hydra loses this head. Let us call 
 \texttt{Hy'} the resulting state of the hydra, represented in Fig.~\vref{fig:Hy-prime}.

Next, assume Hercules choses to chop off one of the two highest heads of \texttt{Hy'}, for instance the rightmost one. Fig.~\vref{fig:Hy2} represents the rotten neck in dashed lines, and the part that will be replicated in red. Assume also that the hydra decides to add 4 copies of the red part\footnote{In other words, the replication factor at this round is equal to $4$.}. We obtain a new state \texttt{Hy''} depicted in Fig.~\ref{fig:Hy3}.



\begin{figure}[h]
\centering
\begin{tikzpicture}[very thick, scale=0.6]

\node (foot) at (2,0) {$\bullet$};
\node (N1) at (2,2) {$\bullet$};
\node (N2) at (2,4) {$\bullet$};
\node (N3) at (2,6) {$\bullet$};
\node (H0) at (0,2) {$\Smiley[2][vertfluo]$};
\node (H1) at (0,8) {$\Smiley[2][vertfluo]$};
\node (H2) at (4,8) {$\Smiley[2][vertfluo]$};
\node (H5) at (4,4) {$\Smiley[2][vertfluo]$};
\node (H4) at (6,0) {$\Xey[2][lightgray]$};
\draw (foot) -- (N1)[very thick] ;
\draw (N1) -- (N2);
\draw (N2) -- (N3) ;
\draw (N3) to [bend left= 10]  (H1) ;
\draw (N3) to [bend right= 16] (H2);
\draw (foot) to [bend left= 10]  (H0) ;
\draw (N1) to [bend right= 16] (H5);
\end{tikzpicture}

\caption{Hy': the state  of Hy after one round \label{fig:Hy-prime}}
\end{figure}


\begin{figure}[hp]
\centering
\begin{tikzpicture}[very thick, scale=0.5]

\node (foot) at (2,0) {$\bullet$};
\node (N1) at (2,2) {$\bullet$};
\node (N2) at (2,4)  {{\color{lightred}$\bullet$}};
\node (N3) at (2,6) {{\color{lightred}$\bullet$}};
\node (H0) at (0,2) {$\Smiley[2][vertfluo]$};
\node (H1) at (0,8) {$\Sey[2][lightred]$};
\node (H2) at (5,0) {$\Xey[2][lightgray]$};
\node (H5) at (4,4) {$\Smiley[2][vertfluo]$};
\node (ex) at (5,8) {};
\draw (foot) -- (N1)[very thick] ;
\draw (N1) -- (N2);
\draw  (N2) -- (N3)[draw=lightred];
\draw (N3) to   [bend left= 10](H1) [draw=lightred];
\draw [dashed] (N3) to [bend left= 10](ex);
\draw (foot) to [bend left= 10]  (H0) ;
\draw (N1) to [bend right= 16] (H5);
\end{tikzpicture}
\caption{A second beheading}
\label{fig:Hy2}
\end{figure}

\begin{figure}[hp]
\centering
\begin{tikzpicture}[very thick, scale=0.6]

\node (foot) at (2,0) {$\bullet$};
\node (N1) at (2,2) {$\bullet$};
\node (N2) at (2,4) {$\bullet$};
\node (N3) at (2,6) {{\color{lightred}$\bullet$}};
\node (H1) at (0,8) {$\Smiley[2][vertfluo]$};
\node (H11) at (2,8) {$\Smiley[2][vertfluo]$};
\node (H12) at (4,8) {$\Smiley[2][vertfluo]$};
\node (H13) at (6,8) {$\Smiley[2][vertfluo]$};
\node (H14) at (8,8) {$\Smiley[2][vertfluo]$};

\node (N3) at (1,6) {$\bullet$};
\node (N31) at (2,6) {$\bullet$};
\node (N32) at (3,6) {$\bullet$};
\node (N33) at (4,6) {$\bullet$};
\node (N34) at (5,6) {$\bullet$};

\node (H0) at (0,2) {$\Smiley[2][vertfluo]$};
\node (H5) at (4,4) {$\Smiley[2][vertfluo]$};
\draw (foot) -- (N1)[very thick] ;
\draw (N1) -- (N2);
\draw (N2) -- (N3);
\draw (N2) -- (N31);
\draw (N2) -- (N32);
\draw (N2) -- (N33);
\draw (N2) -- (N34);
\draw (N3) to   [bend left= 10](H1) ;
\draw (N31) to   [bend left= 10](H11) ;
\draw (N32) to   [bend left= 10](H12) ;
\draw (N33) to   [bend left= 10](H13) ;
\draw (N34) to   [bend left= 10](H14) ;
\draw (foot) to [bend left= 10]  (H0) ;
\draw (N1) to [bend left= 10]  (H5) ;
\end{tikzpicture}
\caption{Hy'', the state of Hy after two rounds \label{fig:Hy3}}
\end{figure}

Figs.~\ref{fig:Hy4} and~\vref{fig:Hy5} represent a possible third round of the battle, with a replication factor equal to $2$. Let us call \texttt{Hy'''} the state of the hydra after that third round.

\begin{figure}[hp]
\centering
\begin{tikzpicture}[very thick, scale=0.6]

\node (foot) at (2,0)  {{\color{lightred}$\bullet$}};
\node (N1) at (2,2) {{\color{lightred}$\bullet$}};
\node (N2) at (2,4) {{\color{lightred}$\bullet$}};
\node (N3) at (2,6) {{\color{lightred}$\bullet$}};
\node (exN4) at (4,4) {};
\node (H1) at (0,8) {$\Sey[2][lightred]$};
\node (H11) at (2,8) {$\Sey[2][lightred]$};
\node (H12) at (4,8) {$\Sey[2][lightred]$};
\node (H13) at (6,8) {$\Sey[2][lightred]$};
\node (H14) at (8,8) {$\Sey[2][lightred]$};

\node (N3) at (1,6) {{\color{lightred}$\bullet$}};
\node (N31) at (2,6) {{\color{lightred}$\bullet$}};
\node (N32) at (3,6) {{\color{lightred}$\bullet$}};
\node (N33) at (4,6) {{\color{lightred}$\bullet$}};
\node (N34) at (5,6) {{\color{lightred}$\bullet$}};

\node (H0) at (0,2) {$\Smiley[2][vertfluo]$};
\node (H5) at (4,0) {$\Xey[2][lightgray]$};
\draw (foot) -- (N1)[very thick,draw=lightred] ;
\draw (N1) -- (N2)[draw=lightred];
\draw (N2) -- (N3)[draw=lightred];
\draw (N2) -- (N31)[draw=lightred];
\draw (N2) -- (N32)[draw=lightred];
\draw (N2) -- (N33)[draw=lightred];
\draw (N2) -- (N34)[draw=lightred];
\draw (N3) to   [bend left= 10](H1) [draw=lightred];
\draw (N31) to   [bend left= 10](H11) [draw=lightred];
\draw (N32) to   [bend left= 10](H12) [draw=lightred];
\draw (N33) to   [bend left= 10](H13) [draw=lightred];
\draw (N34) to   [bend left= 10](H14) [draw=lightred];
\draw (foot) to [bend left= 10]  (H0) ;
\draw [dashed] (N1) to  [bend left= 10](exN4);
\end{tikzpicture}
\caption{A third beheading (wounded part in red) \label{fig:Hy4}}
\end{figure}

\begin{figure}[hp]
\centering
\begin{tikzpicture}[very thick, scale=0.4]

\node (foot) at (10,0) {$\bullet$};


\node (N1) at (2,2) {$\bullet$};
\node (N2) at (2,4) {$\bullet$};
\node (N3) at (2,6) {{\color{lightred}$\bullet$}};
\node (H1) at (0,8) {$\Smiley[1][vertfluo]$};
\node (H11) at (2,8) {$\Smiley[1][vertfluo]$};
\node (H12) at (4,8) {$\Smiley[1][vertfluo]$};
\node (H13) at (6,8) {$\Smiley[1][vertfluo]$};
\node (H14) at (8,8) {$\Smiley[1][vertfluo]$};

\node (N3) at (1,6) {$\bullet$};
\node (N31) at (2,6) {$\bullet$};
\node (N32) at (3,6) {$\bullet$};
\node (N33) at (4,6) {$\bullet$};
\node (N34) at (5,6) {$\bullet$};

\node (H0) at (-3,3) {$\Smiley[1][vertfluo]$};

\draw (foot) to [bend left=10] (N1)[very thick] ;
\draw (N1) -- (N2);
\draw (N2) -- (N3);
\draw (N2) -- (N31);
\draw (N2) -- (N32);
\draw (N2) -- (N33);
\draw (N2) -- (N34);
\draw (N3) to   [bend left= 10](H1) ;
\draw (N31) to   [bend left= 10](H11) ;
\draw (N32) to   [bend left= 10](H12) ;
\draw (N33) to   [bend left= 10](H13) ;
\draw (N34) to   [bend left= 10](H14) ;
\draw (foot) to [bend left = 15]  (H0) ;


% second copy 
\node (N01) at (12,2) {$\bullet$};
\node (N02) at (12,4) {$\bullet$};
\node (N03) at (12,6) {{\color{lightred}$\bullet$}};
\node (H001) at (10,8) {$\Smiley[1][vertfluo]$};
\node (H0011) at (12,8) {$\Smiley[1][vertfluo]$};
\node (H0012) at (14,8) {$\Smiley[1][vertfluo]$};
\node (H0013) at (16,8) {$\Smiley[1][vertfluo]$};
\node (H0014) at (18,8) {$\Smiley[1][vertfluo]$};

\node (N03) at (11,6) {$\bullet$};
\node (N031) at (12,6) {$\bullet$};
\node (N032) at (13,6) {$\bullet$};
\node (N033) at (14,6) {$\bullet$};
\node (N034) at (15,6) {$\bullet$};

\draw (foot) -- (N01)[very thick] ;
\draw (N01) -- (N02);
\draw (N02) -- (N03);
\draw (N02) -- (N031);
\draw (N02) -- (N032);
\draw (N02) -- (N033);
\draw (N02) -- (N034);
\draw (N03) to   [bend left= 10](H001) ;
\draw (N031) to   [bend left= 10](H0011) ;
\draw (N032) to   [bend left= 10](H0012) ;
\draw (N033) to   [bend left= 10](H0013) ;
\draw (N034) to   [bend left= 10](H0014) ;

% third copy 
\node (N001) at (22,2) {$\bullet$};
\node (N002) at (22,4) {$\bullet$};
\node (N003) at (22,6) {{\color{lightred}$\bullet$}};
\node (H001) at (20,8) {$\Smiley[1][vertfluo]$};
\node (H0011) at (22,8) {$\Smiley[1][vertfluo]$};
\node (H0012) at (24,8) {$\Smiley[1][vertfluo]$};
\node (H0013) at (26,8) {$\Smiley[1][vertfluo]$};
\node (H0014) at (28,8) {$\Smiley[1][vertfluo]$};

\node (N003) at (21,6) {$\bullet$};
\node (N0031) at (22,6) {$\bullet$};
\node (N0032) at (23,6) {$\bullet$};
\node (N0033) at (24,6) {$\bullet$};
\node (N0034) at (25,6) {$\bullet$};

\draw (foot) -- (N001)[very thick] ;
\draw (N001) -- (N002);
\draw (N002) -- (N003);
\draw (N002) -- (N0031);
\draw (N002) -- (N0032);
\draw (N002) -- (N0033);
\draw (N002) -- (N0034);
\draw (N003) to   [bend left= 10](H001) ;
\draw (N0031) to   [bend left= 10](H0011) ;
\draw (N0032) to   [bend left= 10](H0012) ;
\draw (N0033) to   [bend left= 10](H0013) ;
\draw (N0034) to   [bend left= 10](H0014) ;
\end{tikzpicture}
\caption{The configuration Hy''' of Hy \label{fig:Hy5}}
\end{figure}
\FloatBarrier

We leave it to the reader  to guess the following  rounds of the battle \dots
 % Please keep in mind that, in this 
% the hydra is free to chose any number of replications at each time, whereas
% Hercules chops only one head per round.

% Let us precise that, in this game, Hercules wins if the hydra is eventually reduced 
% to a single head. 
% We know from~\cite{KP82} that, whichever the initial configuration of the
% hydra, and the strategies of both players, Hercules eventually wins. The 
% aforementionned paper shows also that there do not exist any \emph{simple} proof of this result.


\section{Hydras and their representation in \emph{Coq}}
\label{sec:orgheadline48}


In order to describe trees where each node can have an arbitrary (but finite) number of sons, it usual to define a type where each node carries a \emph{forest}, \emph{i.e} a list of trees
(see for instance Chapter 14, pages 400-406 of \cite{BC04}).

For this purpose, we define two mutual \emph{ad-hoc}  inductive types, where \texttt{Hydra} is the main type, and \texttt{Hydrae} is a helper for describing finite sequences of hydra.
\label{types:Hydra}
\label{types:Hydrae}

\vspace{4pt}
\emph{From Module~\href{../src/html/hydras.Hydra.Hydra_Definitions.html\#Hydra}{Hydra.Hydra\_Definitions}}

\begin{Coqsrc}
Inductive Hydra : Set :=
| node :  Hydrae -> Hydra
with Hydrae : Set :=
| hnil : Hydrae
| hcons : Hydra -> Hydrae -> Hydrae.
\end{Coqsrc}

%\index{To do}

\begin{todo}
Look for an existing library on trees with nodes of arbitrary arity, in order to replace  this ad-hoc type with something more generic.
\end{todo}


\index{Projects}

\begin{project}[**]

 Another very similar representation could use the \texttt{list} type family instead of the specific 
type \texttt{Hydrae}:


\begin{Coqalt}
Module Alt.

Inductive Hydra: Set :=
  hnode (daughters : list Hydra).

End Alt.
\end{Coqalt}

Using this representation, re-define all the constructions of this chapter.
You will probably have to use patterns described for instance in~\cite{BC04} or the archives of the Coq-club~\cite{Coq}.

  
\end{project}


\index{Projects}

\begin{project}
The type \texttt{Hydra} above describes hydras as \emph{plane trees}, i.e., as drawn on a sheet of paper or computer screen. Thus, hydras are \emph{oriented},
and it is appropriate to consider a \emph{leftmost} or \emph{rightmost} head of
the beast. It could be interesting to consider another representation, in which
every non-leaf node has a \emph{multi-set} -- not an ordered list -- of daughters.
\end{project}

\subsubsection{Abbreviations}

We provide several notations for ``patterns'' which occur often in our developments. 

\emph{From Module~\href{../src/html/hydras.Hydra.Hydra_Definitions.html\#head}{Hydra.Hydra\_Definitions}}

\begin{Coqsrc}
(** heads *)
Notation head := (node hnil).
 
(** nodes  with 1, 2 or 3 daughters *)
Notation hyd1 h := (node (hcons h hnil)).
Notation hyd2 h h' := (node (hcons h (hcons h' hnil))).
Notation hyd3 h h' h'' := 
                   (node (hcons h (hcons h' (hcons h'' hnil)))).
\end{Coqsrc}


For instance, the hydra \texttt{Hy}  of Figure~\vref{fig:Hy} is defined in \emph{Gallina} as follows:

\vspace{4mm}
\emph{From Module~\href{../src/html/hydras.Hydra.Hydra_Examples.html\#Hy}{Hydra.Hydra\_Examples}}

\begin{Coqsrc}
Example Hy := hyd3 head
                   (hyd2
                      (hyd1 
                         (hyd2 head head))
                      head) 
                   head.
\end{Coqsrc}



Hydras quite frequently contain  multiple  copies of the same pattern. The following functions
will help us to describe and reason about replications in hydra battles.

\vspace{4pt}

\emph{From Module~\href{../src/html/hydras.Hydra.Hydra_Definitions.html\#hcons_mult}{Hydra.Hydra\_Definitions}}

\begin{Coqsrc}
Fixpoint hcons_mult (h:Hydra)(n:nat)(s:Hydrae):Hydrae :=
  match n with 
  | O => s
  | S p => hcons h (hcons_mult h p s)
  end.

(** hydra with n copies of the same daughter *)

Definition hyd_mult h n :=
  node (hcons_mult h n hnil).
\end{Coqsrc}

\vspace{4mm}



For instance, the hydra $Hy''$ of Fig~\vref{fig:Hy3}  can be defined in \coq{} as follows:

\vspace{4pt}

\emph{From Module~\href{../src/html/hydras.Hydra.Hydra_Examples.html}{Hydra.Hydra\_Examples}}

\begin{Coqsrc}
Example Hy'' := 
     hyd2 head
          (hyd2 (hyd_mult (hyd1 head) 5)
                head).
\end{Coqsrc}




\subsubsection{Recursive functions on type \texttt{Hydra}}
\label{sec:orgheadline41}
\label{sec:hsize-def}




When defining a recursive function over the type \texttt{Hydra}, one has to consider the three constructors 
\texttt{node}, \texttt{hnil} and \texttt{hcons} of the mutually inductive types \texttt{Hydra} and \texttt{Hydrae}. 
Let us define for instance the function that  computes the number of nodes of any hydra:

\vspace{4pt}
\emph{From Module~\href{../src/html/hydras.Hydra.Hydra_Definitions.html}{Hydra.Hydra\_Definitions}}


\begin{Coqsrc}
Fixpoint hsize (h:Hydra) : nat :=
  match h with node l => S (lhsize l)
  end
with lhsize l : nat :=
  match l with hnil => 0
            | hcons h hs => hsize h + lhsize hs 
  end.

 Compute hsize Hy.
\end{Coqsrc}

\begin{Coqanswer}
 = 9
     : nat 
\end{Coqanswer}


Likewise, the \emph{height} (maximum distance between the foot and a head) 
is defined by mutual recursion:

\begin{Coqsrc}
Fixpoint height  (h:Hydra) : nat :=
  match h with node l => lheight l
  end
with lheight l : nat :=
  match l with 
  | hnil => 0
  | hcons h hs => Max.max (S (height h)) (lheight hs)
  end.
\end{Coqsrc}

\begin{Coqsrc}
Compute height Hy.
\end{Coqsrc}

\begin{Coqanswer}
 = 4
     : nat  
\end{Coqanswer}

\index{Exercises}

\begin{exercise}
Define a function \texttt{max\_degree: Hydra $\arrow$ nat} which  returns the highest degree of a node in any hydra. For instance, the evaluation of the term \texttt{(max\_degree Hy)} should return $3$.
\end{exercise}

\subsection{Induction principles for hydras}
\label{sec:orgheadline42}


In this section, we show how induction principles are used to prove properties on the type 
\texttt{Hydra}. Let us consider for instance the following statement:
\begin{quote}
  `` The height of any hydra is strictly less than its size. ''
\end{quote}



\subsubsection{A failed attempt}

One may try to use the default tactic of proof by induction, that corresponds to an application of the automatically  generated  induction principle for  type \texttt{Hydra}:

\begin{Coqanswer}
Hydra_ind :
forall P : Hydra -> Prop,
(forall h : Hydrae, P (node h)) -> forall h : Hydra, P h
\end{Coqanswer}

Ler us start a simple proof by induction.

\vspace{4pt}
\emph{From Module~\href{../src/html/hydras.Hydra.Hydra_Examples.html}{Hydra.Hydra\_Examples}}

\begin{Coqbad}
Module Bad.

Lemma height_lt_size (h:Hydra) :
  height h <= hsize h.
Proof.
  induction h as [s].
\end{Coqbad}

\begin{Coqanswer}
1 subgoal, subgoal 1 (ID 11)
  
  s : Hydrae
  ============================
   height (node s) <= hsize (node s)
\end{Coqanswer}

We might be tempted to do an induction on the sequence \texttt{s}:

% \begin{Coqbad}
% induction  s as [ | h s'].
%   -   cbn; auto with arith.
%   - 
% \end{Coqbad}


\begin{Coqanswer}
 1 focused subgoal
(unfocused: 0), subgoal 1 (ID 19)
  
  h : Hydra
  s' : Hydrae
  IHs' : height (node s') <= hsize (node s')
  ============================
   height (node (hcons h s')) <= hsize (node (hcons h s'))

\end{Coqanswer}

Note that the displayed subgoal does not contain any assumption on \texttt{h}, thus there is no way to 
infer any property about the height and size of the hydra \texttt{(hcons h t)}.

\begin{Coqbad}
Abort.

End Bad.
\end{Coqbad}

\subsubsection{A principle for  mutual induction}
In order to get an appropriate induction scheme for the types 
\texttt{Hydra} and \texttt{Hydrae}, we can use  \coq{}'s  command \texttt{Scheme}.


\index{Coq!Techniques!Mutual induction}
\index{Coq!Commands!Scheme}

\begin{Coqsrc}
Scheme Hydra_rect2 := Induction for Hydra Sort Type
with Hydrae_rect2 := Induction for Hydrae Sort Type.
\end{Coqsrc}  


\begin{Coqsrc}
Check Hydra_rect2.
\end{Coqsrc}


\begin{Coqanswer}
Hydra_rect2
 : forall (P : Hydra -> Type) (P0 : Hydrae -> Type),
   (forall h : Hydrae, P0 h -> P (node h)) ->
    P0 hnil ->
    (forall h : Hydra, P h -> 
             forall h0 : Hydrae, P0 h0 -> P0 (hcons h h0)) ->
    forall h : Hydra, P h
\end{Coqanswer}  




\subsubsection{A correct proof}

Let us now use \texttt{Hydra\_rect2} for proving that the height of any hydra is strictly less than its size.
Using this scheme requires an auxiliary predicate, called \texttt{P0} in \texttt{Hydra\_rect2}'s statement. 
Let us begin by defining an ad-hoc version of \texttt{List.Forall}.

\vspace{4pt}
\emph{From Module~\href{../src/html/hydras.Hydra.Hydra_Examples.html}{Hydra.Hydra\_Examples}}

\begin{Coqsrc}
(** All elements of s satisfy P *)

Fixpoint h_forall (P: Hydra -> Prop) (s: Hydrae) :=
  match s with
      hnil => True
    | hcons h s' => P h /\ h_forall P s'
  end.
 \end{Coqsrc}

 \begin{Coqsrc}
Lemma height_lt_size (h:Hydra) :
 height h < hsize h.
Proof.
  induction h using Hydra_rect2  with 
  (P0 :=  h_forall (fun h =>  height h < hsize h)).
 \end{Coqsrc}

\begin{enumerate}
\item The first subgoal is as follows:

\begin{Coqanswer}

  h: Hydrae
  IHh : h_forall (fun h : Hydra => height h < hsize h) h
  ============================
   height (node s) < hsize (node s) 

\end{Coqanswer}

This goal is easily solvable, using some arithmetic. We let the reader look at the source.

\item The second subgoal is trivial:

\begin{Coqanswer}

  ============================
    h_forall (fun h : Hydra => height h < hsize h) hnil

\end{Coqanswer}

\begin{Coqsrc}
  reflexivity.
\end{Coqsrc}

\item Finally, the last subgoal is also easy to solve:



\begin{Coqanswer}
  h : Hydra
  h0 : Hydrae
  IHh : height h < hsize h
  IHh0 : h_forall (fun h : Hydra => height h < hsize h) h0
  ============================
   h_forall (fun h1 : Hydra => height h1 < hsize h1) 
                 (hcons h h0)
\end{Coqanswer}  



\begin{Coqsrc}
 split;auto. 
Qed.   
\end{Coqsrc}

\end{enumerate}

\index{Exercises}

\begin{exercise}
It happens very often that, in the proof of  a proposition of the form 
(\texttt{$\forall\,$ h:Hydra, $P$ h}), the predicate \texttt{P0}
is  (\texttt{h\_forall $P$}).  Design a tactic for induction on hydras that frees the user from binding explicitly \texttt{P0},  and solves trivial subgoals. Apply it for writing  a shorter proof of \texttt{height\_lt\_size}.
\end{exercise}
 


\section{Relational description of hydra battles}


In this section, we represent the rules of hydra battles as a binary relation associated with
a \emph{round}, i.e., an interaction composed of the two following actions:
\begin{enumerate}
\item Hercules chops off one head of the hydra
\item Then, the  hydra replicates the wounded part (if the head is at distance $\geq 2$ from the foot).
\end{enumerate}
The relation associated with each round of the battle is \emph{parameterized}  by the \emph{expected} replication  factor (irrelevant if the chopped head is at distance 1 from the foot,
but present for consistency's sake).

In our description,  we will apply the following naming convention: if $h$ represents the configuration of the hydra before a round, then the configuration of $h$ after this round will be called $h'$.
 Thus, we are going to define a proposition  (\texttt{round\_n $n\;h\;h'$})  whose intended meaning will be `` the hydra $h$  is transformed into $h'$  in a single round of a battle, with the expected replication factor $n$ ''.


Since the replication of parts of the hydra depends on the distance of the chopped head from  the foot, we  decompose our description into two main  cases, under the form of a bunch of [mutually] inductive predicates over the types \texttt{Hydra} and \texttt{Hydrae}.

The mutually exclusive cases we consider are the following:
\begin{itemize}
\item \textbf{R1}: The chopped off head was at distance 1 from the foot.
\item \textbf{R2}: The chopped off head was at a distance greater than or equal to  $2$ from the foot.
\end{itemize}



\subsection{Chopping off a head at distance 1 from the foot (relation  R1)}

If Hercules chops off a head near the floor, there is no replication at all. We use an auxiliary 
predicate \texttt{S0}, associated with the removing of one head from a sequence of hydras.


\vspace{4pt}\emph{From Module\href{../src/html/hydras.Hydra.Hydra_Definitions.html}{Hydra.Hydra\_Definitions}}

\begin{Coqsrc}
Inductive S0 :  relation Hydrae :=
| S0_first : forall s, S0  (hcons head s) s
| S0_rest : forall  h s s',  S0  s s' ->
                             S0  (hcons h s) (hcons h s').

Inductive R1  :  Hydra -> Hydra -> Prop :=
| R1_intro : forall s s', S0 s s' -> R1 (node s) (node s').
\end{Coqsrc}

\subsubsection{Example}
\label{sec:orgheadline45}

Let us represent in \coq{}   the transformation of the hydra of Fig.~\vref{fig:Hy} into
the configuration represented in Fig.~\ref{fig:Hy-prime}.

\vspace{4pt}
\emph{From Module~\href{../src/html/hydras.Hydra.Hydra_Examples.html}{Hydra.Hydra\_Examples}}


\begin{Coqsrc}
Example Hy_1 : R1 Hy Hy'.
Proof. 
  split; right; right; left.
Qed.
\end{Coqsrc}


\subsection{Chopping of a head at distance \texorpdfstring{$\geq 2$}{>= 2} from the foot (relation R2) }


Let us now consider beheadings  where the chopped-off head is at distance greater than or equal to $2$ from the foot. All the following relations are parameterized by the replication factor  $n$.

 Let $s$ be a sequence of hydras. 
The proposition (\texttt{S1 n s s'}) holds if $s'$ is obtained by replacing some element $h$ of $s$ by 
$n+1$ copies of $h'$, where  the proposition (\texttt{R1 h h'}) holds, in other words, \texttt{h'} is just \texttt{h}, without the chopped-off  head. \texttt{S1} is an inductive relation with two constructors that allow us to choose the position in $s'$ of the wounded sub-hydra $h$.

\vspace{4pt}
\emph{From Module~\href{../src/html/hydras.Hydra.Hydra_Definitions.html\#S1}{Hydra.Hydra\_Definitions}}

\begin{Coqsrc}
Inductive S1 (n:nat)  : Hydrae -> Hydrae -> Prop :=
| S1_first : forall s h h' ,   R1 h h' -> 
                  S1 n (hcons h s) (hcons_mult h' (S n) s)
| S1_next : forall h s s',  S1 n s s' ->
                   S1 n (hcons h s) (hcons h s').
\end{Coqsrc}


The rest of the definition is structured as two mutually inductive relations on hydras and sequences of hydras. The first constructor of \texttt{R2} describes the case where the chopped head is exactly at height $2$. The others constructors allow us to consider beheadings at height strictly greater than $2$.


\vspace{4pt}
\emph{From Module~\href{../src/html/hydras.Hydra.Hydra_Definitions.html\#R2}{Hydra.Hydra\_Definitions}}

\begin{Coqsrc}
Inductive R2 (n:nat)  :  Hydra -> Hydra -> Prop :=
| R2_intro : forall s s', S1 n s s' -> R2 n (node s) (node s')
| R2_intro_2 : forall s s', S2 n s s' -> R2 n (node s) (node s')

with S2 (n:nat) :  Hydrae -> Hydrae -> Prop :=
|  S2_first : forall h h' s ,
               R2 n  h h'  -> 
               S2  n (hcons h s) (hcons h'  s)
|  S2_next  : forall h   r r',
               S2 n   r r' ->
               S2 n (hcons h r) (hcons h r').                  
\end{Coqsrc}


\subsubsection{Example}
Let us prove the transformation of \texttt{Hy'} into \texttt{Hy''} (see Fig.~\vref{fig:Hy3}). We use an experimental set of tactics for specifying the place where the 
interaction between Hercules and the hydra holds. 


\vspace{4pt}\emph{From Module~\href{../src/html/hydras.Hydra.Hydra_Examples.html}{Hydra.Hydra\_Examples}}. 

\begin{Coqsrc}
Example R2_example:  R2 4 Hy' Hy''.
Proof.
    (** move to 2nd sub-hydra (0-based indices) *) R2_up 1. 
    (** move to first sub-hydra *)  R2_up 0.
    (** we're at distance 2 from the to-be-chopped-off head 
        let's go to the first daughter, 
        then chop-off the leftmost head *)  r2_d2  0 0. 
Qed.
\end{Coqsrc}

The reader is encouraged to look at all the successive subgoals of this example.
\emph{Please consider also exercise~\vref{exo:interactive-battle}.}


\subsection{Relation associated with a round}

We combine the two cases above into a single relation.
First,  we define the  relation \texttt{(round\_n n h h')} where \texttt{n} is the expected number of  replications (irrelevant in the case of an \texttt{R1}-transformation).

\vspace{4pt}
\emph{From Module~\href{../src/html/hydras.Hydra.Hydra_Definitions.html\#round_n}{Hydra.Hydra\_Definitions}}

\index{Predicates!round\_n}

\begin{Coqsrc}
Definition round_n n h h' := R1 h h' \/ R2 n h h'.  
\end{Coqsrc}

By abstraction over \texttt{n}, we define a \emph{round} (small step) of a battle:

\index{Predicates!round}
\label{sect:infix-round}
\begin{Coqsrc}
Definition round h h' := exists n,  round_n n h h'.

Infix "-1->" := round (at level 60).
\end{Coqsrc}

\index{Projects}

\begin{project}
Give a direct translation of Kirby and Paris's description of hydra battles (quoted on page~\pageref{original-rules}) and prove that our relational description is consistent with theirs.
\end{project}


\subsection{Rounds and battles}


Using library \href{https://coq.inria.fr/distrib/current/stdlib/Coq.Relations.Relation_Operators.html}{Relations.Relation\_Operators}, we define \texttt{round\_plus},  the transitive closure of \texttt{round}, and \texttt{round\_star},  the reflexive and transitive closure of \texttt{round}.

\label{sect:infix-rounds} 

\begin{Coqsrc}
Definition round_plus := clos_trans_1n Hydra round.
Infix "-+->" := rounds (at level 60).

Definition round_star h h' := h = h' \/ round_plus h h'.
Infix "-*->" := round_star (at level 60).
\end{Coqsrc}


\index{Exercises}

\begin{exercise}
Prove the following lemma:

\begin{Coqsrc}
Lemma rounds_height : forall h h', 
   h -+-> h' -> height h' <= height h.  
\end{Coqsrc}
  
\end{exercise}

\begin{remark}
\label{remark:transitive-closure}
\coq's library \href{https://coq.inria.fr/distrib/current/stdlib/Coq.Relations.Relation_Operators.html}{Coq.Relations.Relation\_Operators} 
contains three logically equivalent definitions of the transitive closure of a binary relation. This equivalence is proved in 
\href{https://coq.inria.fr/distrib/current/stdlib/Coq.Relations.Operators_Properties.html}{Coq.Relations.Operators\_Properties} . 

Why three definitions for a single mathematical concept?
Each definition generates an associated induction principle. 
 According to the form of statement one would like to prove, there is a ``best choice'':

\begin{itemize}
\item For proving $\forall y, x\,R^+\,y \;\arrow\; P\,y$, prefer 
\texttt{clos\_trans\_n1}
\item For proving $\forall x,\,x\,R^+\,y \;\arrow\; P\,x$, prefer \texttt{clos\_trans\_1n}
\item For proving $\forall x\,y, \,x\,R^+\,y \;\arrow\;P\,x\,y$,  
prefer \texttt{clos\_trans},
\end{itemize}
But there is no ``wrong choice'' at all: the equivalence lemmas in \linebreak 
\href{https://coq.inria.fr/distrib/current/stdlib/Coq.Relations.Operators_Properties.html}{Coq.Relations.Operators\_Properties} 
 allow the user
to convert any one of the three closures into another one before applying the corresponding elimination tactic.
The same remark also holds for reflexive and transitive closures. 
\end{remark}

\index{Exercises}

\begin{exercise}
Define a restriction of \coqsimple{round},  where Hercules always chops off
the leftmost among the lowest heads.

Prove that, if $h$ is not a simple head, then there exists a unique $h'$ such that \texttt{h}  is transformed into \texttt{ h'} in one round, according to this restriction.


\end{exercise}

\index{Exercises}

\begin{exercise}[Interactive battles]
\label{exo:interactive-battle}
Given a hydra \texttt{h}, the specification of a hydra battle for \texttt{h} is the type 
\Verb@{h':Hydra | h -*-> h'}@. In order to avoid long sequences of \texttt{split}, \texttt{left}, and 
\texttt{right}, design a set of dedicated tactics for the interactive building of a battle.
Your tactics will have the following functionalities:
\begin{itemize}
\item  Chose to stop a battle, or continue
\item Chose an expected number of replications
\item Navigate in a hydra, looking for a head to chop off.
\end{itemize}

Use your tactics for simulating a small part of a hydra battle, for instance the rounds which lead from
\texttt{Hy} to \texttt{Hy'''}  (Fig.~\vref{fig:Hy5}).

\textbf{Hints:} 
\begin{itemize}

\item Please keep in mind that the last  configuration of your interactively built battle is known only at the end of the battle. Thus, you will have to create and solve subgoals with existential variables. For that purpose, the tactic \texttt{eexists}, applied to the 
goal \Verb@{h':Hydra | h -*-> h'}@ generates the subgoal \Verb|h -*-> ?h'|.
\item You may use Gérard Huet's \emph{zipper} data structure~\cite{zipper} for writing tactics associated with Hercule's  interactive search of a head to chop off.
\end{itemize}






\end{exercise}




\subsection{Classes of battles}
\label{sect:battle-classes}

In some presentations of hydra battles, e.g.~\cite{KP82, bauer2008}, the transformation associated with the $i$-th round may depend on $i$. For instance, in these articles, the replication factor at the $i$-th round is equal to $i$. In other examples, one can allow the hydra to apply any replication factor at any time. In order to be the most general as possible, we define the type of predicates which relate the state of the hydra before and after the $i$-th round of a battle.

\vspace{4pt}
\emph{From Module~\href{../src/html/hydras.Hydra.Hydra_Definitions.html}{Hydra.Hydra\_Definitions}}
\index{Coq!Type classes}
\label{types:Battle}

\begin{Coqsrc}
Definition dep_round_t := nat -> Hydra -> Hydra -> Prop.

Class Battle :=  {battle_r : dep_round_t;
                  battle_inclusion : forall i h h',
                      battle_r i h h' -> round h h'}.

\end{Coqsrc}

The most general class of battles is \texttt{free}, which allows the hydra to chose any replication factor at every step:

\vspace{4pt}
\emph{From Module~\href{../src/html/hydras.Hydra.Hydra_Definitions.html\#free}{Hydra.Hydra\_Definitions}}

\begin{Coqsrc}
Program Instance free : Battle :=
  (Build_Battle ( fun _  h h' => round h h') _).
\end{Coqsrc}

We chosed to call \emph{standard} the kind of battles which appear  most often in the litterature and correspond to an arithmetic progression of the replication factor : $0,1,2,3, \dots$

\vspace{4pt}
\emph{From Module~\href{../src/html/hydras.Hydra.Hydra_Definitions.html\#standard}{Hydra.Hydra\_Definitions}}

\begin{Coqsrc}
Program Instance standard : Battle := (Build_Battle round_n _).
Next Obligation.
  now exists i.  
Defined.
\end{Coqsrc}


\subsection{Big steps}

Let $B$ be any instance of class \texttt{Battle}. It is easy to define inductively the relation between the $i$-th and the $j$-th steps of a battle of type $B$.

\vspace{4pt}
\emph{From Module~\href{../src/html/hydras.Hydra.Hydra_Definitions.html\#fight}{Hydra.Hydra\_Definitions}}

\begin{Coqsrc}
Inductive battle (B:Battle) : nat -> Hydra -> nat -> Hydra -> Prop :=

| battle_1 : forall i h  h', battle_r   B i  h h' -> 
                            battle B i h (S i) h'
| battle_n : forall i h  j h' h'',  battle_r  B i h h''  ->
                                   battle B (S i) h'' j h'  ->
                                   battle B i h j h'.
\end{Coqsrc}

% \begin{remark}
%  The class \texttt{free} is strongly related with the transitive closure  \texttt{round\_plus}, as expressed by the following lemmas.

% \vspace{4pt}
% \emph{From Module~\href{../src/html/hydras.Hydra.Hydra_Lemmas.html}{Hydra.Hydra\_Lemmas}}

%  \begin{Coqsrc}
%  Lemma battle_free_equiv1 : forall i j h h',  
%              battle free i h j h' ->   h -+-> h'.
 
%  Lemma battle_free_equiv2 : forall h h',
%      h -+-> h' ->
%     forall i, exists j,  battle free i h j h'.
%  \end{Coqsrc}

% \end{remark}



\section{A long battle}
\label{sect:big-battle}


In this section we consider a simple example of battle, starting with a small hydra,
shown on figure~\vref{fig:hinit}, with a simple strategy for both players:

\begin{itemize}
\item Hercules chops off always the rightmost head of the hydra.
\item The battle is standard: at the round number $i$, the expected replication is $i$.
\end{itemize}



\begin{figure}[h]
  \centering
  \begin{tikzpicture}[thick, scale=0.30]
 \node (foot) at (6,0) {$\bullet$};
\node (n1) at  (3,3) {$\bullet$};
\node (h1) at  (1,6) {$\Smiley[1][green]$};
\node (h2) at  (3,6) {$\Smiley[1][green]$};
\node (h3) at  (6,6) {$\Smiley[1][green]$};
\node (h4) at  (6,6) {$\Smiley[1][green]$};
\node (h5) at  (6,3) {$\Smiley[1][green]$};
\node (h6) at  (9,3) {$\Smiley[1][green]$};
\draw (foot) -- (n1);
\draw (n1) to   [bend left=20] (h1);
\draw (n1) to   (h2);
\draw (n1) to   [bend right=20] (h3);
\draw (foot) -- (h5);
\draw (foot) to  [bend right=20] (h6);
\end{tikzpicture}

  \caption{The hydra hinit}
  \label{fig:hinit}
\end{figure}


\begin{Coqsrc}
Definition hinit := hyd3 (hyd_mult head 3)  head head.  
\end{Coqsrc}



The lemma we would like to prove is ``The considered battle lasts exactly $N$ rounds'',
with $N$ being a natural number we gave to guess.

But the  battle is so long that no \emph{test} can give us an estimation of its length, and we do need the expressive power of logic to compute this length. However, in order to  guess this length, we made some experiments, computing with \gallina{}, \coq{}'s  functional programming language.
Thus, we can consider this development as a collaboration of proof with computation.

In the following lines, we try to show faithfully how we found the value of the number $N$.

The complete proof is in file \url{../src/html/hydras.Hydra.BigBattle.html}. 

\subsection{The beginning of hostilities}
During the two first rounds, our hydra loses its two rightmost heads. Thus just before the third round, it looks like in figure~\vref{fig:hinit-plus2}.


\begin{figure}[h]
  \centering
  \begin{tikzpicture}[thick, scale=0.30]
 \node (foot) at (3,0) {$\bullet$};
\node (n1) at  (3,3) {$\bullet$};
\node (h1) at  (1,6) {$\Smiley[1][green]$};
\node (h2) at  (3,6) {$\Smiley[1][green]$};
\node (h3) at  (6,6) {$\Smiley[1][green]$};
\node (h4) at  (6,6) {$\Smiley[1][green]$};
\draw (foot) -- (n1);
\draw (n1) to   [bend left=20] (h1);
\draw (n1) to   (h2);
\draw (n1) to   [bend right=20] (h3);
\end{tikzpicture}

  \caption{The hydra (hyd1 h3)}
  \label{fig:hinit-plus2}
\end{figure}

The following lemma  is a formal description of these first rounds, in terms of the
\texttt{battle} predicate.

\begin{Coqsrc}
Lemma L_0_2 : battle standard 0 hinit 2 (hyd1 h3).   
\end{Coqsrc}


\subsection{Looking for regularities}


A first study with pencil and paper suggested us that, after three rounds, the hydra always looks like in figure~\vref{fig:hinit-plusn} (with a variable number of 
subtrees of height 1 or 0).
Thus, we introduce handy notations.

\begin{Coqsrc}
Notation h3 := (hyd_mult head 3).
Notation h2 := (hyd_mult head 2).
Notation h1 := (hyd1 head).

Definition hyd a b c := 
  node (hcons_mult h2  a
             (hcons_mult h1  b
                         (hcons_mult head c hnil))).
\end{Coqsrc}


For instance Fig~\vref{fig:hinit-plusn} shows the hydra (\texttt{hyd 3 4 2}). The hydra (\texttt{hyd 0 0 0})  is the ``final'' hydra of any terminating battle, {i.e.},
a tree whith exactly one node and no edge.


\begin{figure}[h]
  \centering
  \begin{tikzpicture}[thick, scale=0.30]
 \node (foot) at (15,0) {$\bullet$};
\node (a) at  (3,4) {$\bullet$};
\node (b) at  (6,4) {$\bullet$};
\node (c) at  (9,4) {$\bullet$};
\node (d) at  (13,4) {$\bullet$};
\node (e) at  (16,4) {$\bullet$};
\node (f) at  (19,4) {$\bullet$};
\node (g) at  (22,4) {$\bullet$};
\node (h) at  (25,4) {$\Smiley[1][green]$};
\node (i) at  (28,4) {$\Smiley[1][green]$};
\node (aa) at  (2.5,8) {$\Smiley[1][green]$};
\node (ab) at  (3.5,8) {$\Smiley[1][green]$};
\node (ba) at  (5.5,8) {$\Smiley[1][green]$};
\node (bb) at  (6.5,8) {$\Smiley[1][green]$};
\node (ca) at  (8.5,8) {$\Smiley[1][green]$};
\node (cb) at  (9.5,8) {$\Smiley[1][green]$};
\node (da) at  (13,8) {$\Smiley[1][green]$};
\node (ea) at  (16,8) {$\Smiley[1][green]$};
\node (fa) at  (19,8) {$\Smiley[1][green]$};
\node (ga) at  (22,8) {$\Smiley[1][green]$};
\draw (foot) -- (a);
\draw (foot) -- (b);
\draw (foot) -- (c);
\draw (foot) -- (d);
\draw (foot) -- (e);
\draw (foot) -- (f);
\draw (foot) -- (g);
\draw (foot) -- (h);
\draw (foot) -- (i);
\draw (a) -- (aa);
\draw (a) -- (ab);
\draw (b) -- (ba);
\draw (b) -- (bb);
\draw (c) -- (ca);
\draw (c) -- (cb);
\draw (d) -- (da);
\draw (e) -- (ea);
\draw (f) -- (fa);
\draw (g) -- (ga);
% \node (a) at  (3,4) {$\bullet$};
% \node (h1) at  (1,6) 
% \node (h2) at  (3,6) {$\Smiley[1][green]$};
% \node (h3) at  (6,6) {$\Smiley[1][green]$};
% \node (h4) at  (6,6) {$\Smiley[1][green]$};
% \draw (foot) -- (n1);
% \draw (n1) to   [bend left=20] (h1);
% \draw (n1) to   (h2);
% \draw (n1) to   [bend right=20] (h3);
\end{tikzpicture}

  \caption{The hydra (hyd 3 4 2)}
  \label{fig:hinit-plusn}
\end{figure}


With these notations, we get a formal description of the first three rounds.


\begin{Coqsrc}
Lemma L_2_3 : battle standard 2 (hyd1 h3)  3 (hyd 3 0 0).

Lemma L_0_3 : battle standard 0 hinit 3 (hyd 3 0 0).
\end{Coqsrc}


\subsection{Computing \dots}
In order to study \emph{experimentally} the different  configurations of the  battle, we will use a simple datatype for representing the states as tuples composed of
the round number, and the respective number of daughters  \texttt{h2}, \texttt{h1}, and heads
of the current hydra.




\begin{Coqsrc}
 Record state : Type :=
    mks {round: nat ; n2 : nat ; n1 : nat ; nh : nat}.
\end{Coqsrc}

The following function returns the next configurarion of the game. 
Note that this function is defined only for making experiments and is not  ``certified''.  Formal proofs about our battle will only start with the lemma
\texttt{lemma:step-battle}, page~\pageref{lemma:step-battle}.


\begin{Coqsrc}
Definition next (s : state) :=
  match s with
  | mks round a b (S c) => mks (S round) a b c
  | mks round a (S b) 0 => mks (S round) a b (S round)
  | mks round (S a) 0 0 => mks (S round) a (S round) 0
  | _ => s
  end.
\end{Coqsrc}

We can make bigger steps through iterations of \texttt{next}.
The functional \texttt{iterate}, similar to Standard Library's \texttt{Nat.iter},
is defined and studied in~\href{../src/html/hydras.Prelude.Iterates.html\#iterate}{Prelude.Iterates}.

\label{Functions:iterate}

\begin{Coqsrc}
Fixpoint iterate {A:Type}(f : A -> A) (n: nat)(x:A) :=
  match n with
  | 0 => x
  | S p => f (iterate  f p x)
  end.
\end{Coqsrc}



The following function computes the state of the battle at the $n$-th round.


\begin{Coqsrc}
Definition test n := iterate next (n-3) (mks 3 3 0 0).

Compute test 3.
   (**
     = {| round := 3; n2 := 3; n1 := 0; nh := 0 |}
     : state
    *)

Compute test 4.
 (*
  = {| round := 4; n2 := 2; n1 := 4; nh := 0 |}
     : state
 *)

Compute test 5.
 (*
   = {| round := 5; n2 := 2; n1 := 3; nh := 5 |}
     : state
 *)

Compute test 2000.
(*
  = {| round := 2000; n2 := 1; n1 := 90; nh := 1102 |}
     : state
*)
\end{Coqsrc}


The battle we are studying seems to be awfully long. Let us concentrate our
tests on some particular events : the states where $\texttt{nh}=0$ 

From the value of \texttt{test 5},  it is obvious that at the 10-th round, the counter \texttt{nh} will be
equal to zero.

 \begin{Coqsrc}
 Compute test 10.
(*

    = {| round := 10; n2 := 2; n1 := 3; nh := 0 |}
     : state
*)
 \end{Coqsrc}

Thus, $ (1 + 11)$ rounds later, the \texttt{n1} field will be equal to $2$, and 
\texttt{nh}  will equal to $0$. 


\begin{Coqsrc}
Compute test 22.
(*

 = {| round := 22; n2 := 2; n1 := 2; nh := 0 |}
     : state
*)
\end{Coqsrc}


\begin{Coqsrc}
 Compute test 46.
(*

 = {| round := 46; n2 := 2; n1 := 1; nh := 0 |}
     : state

*)

Compute test 94.

(*

 = {| round := 94; n2 := 2; n1 := 0; nh := 0 |}
     : state

*) 
\end{Coqsrc}


Next round, we decrement \texttt{n2} and set \texttt{n1} to $95$.


\begin{Coqsrc}
 Compute test 95.

(*

  = {| round := 95; n2 := 1; n1 := 95; nh := 0 |}
     : state

*)
\end{Coqsrc}

We now have some intuition of the sequence.
It looks like the next ``\texttt{nh}=0'' event will happen at the $192=2(95+1)$-th round, then at the $2(192+1)$-th round, etc.


\begin{Coqsrc}
Definition doubleS (n : nat) := 2 * (S n).

Compute test (doubleS 95).

(**
 = {| round := 192; n2 := 1; n1 := 94; nh := 0 |}
     : state
 *)


Compute test (iterate doubleS 2 95).

(*
  = {| round := 386; n2 := 1; n1 := 93; nh := 0 |}
     : state
*)
\end{Coqsrc}

\subsection{Proving \dots}
We are now able to reason about the sequence of transitions defined by our hydra battle. Instead of using the data-type \texttt{state} we study the relationship
between different configurations of the battle.

Let us define a binary relation associated with every round of the battle.
In the following definition \texttt{i} is associated with the round number (or date, if we consider a discrete time), and \texttt{a}, \texttt{b}, \texttt{c} respectively associated with the number of \texttt{h2}, \texttt{h1} and heads connected to the hydra's foot.

\begin{Coqsrc}
Inductive one_step (i: nat) :
  nat -> nat -> nat -> nat -> nat -> nat -> Prop :=
| step1: forall a b c, one_step i a b (S c) a b c
| step2:  forall a b, one_step i a (S b) 0 a b (S i)
| step3: forall a, one_step i (S a) 0 0 a (S i) 0.
\end{Coqsrc}

The relation between \texttt{one\_step} and the rules of hydra battle is asserted by the following lemma. 

\label{lemma:step-battle}

\begin{Coqsrc}
Lemma step_battle : forall i a b c a' b' c', 
   one_step i a b c a' b' c' ->
   battle standard i (hyd  a b c)  (S i) (hyd a' b' c').
\end{Coqsrc}

Next, we define ``big steps'' as the transitive closure of \texttt{one\_step},
and reachability (from the initial configuration of figure~\ref{fig:hinit} at time $0$).



\begin{Coqsrc}
 Inductive steps : nat -> nat -> nat -> nat ->
                  nat -> nat -> nat -> nat -> Prop :=
| steps1 : forall i a b c a' b' c',
    one_step i a b c a' b' c' -> steps i a b c (S i) a' b' c'
| steps_S : forall i a b c j a' b' c' k a'' b'' c'',
    steps i a b c j a' b' c' ->
    steps j a' b' c' k a'' b'' c'' ->
    steps i a b c k  a'' b'' c''.

Definition reachable (i a b c : nat) : Prop :=
  steps 3 3 0 0 i a b c.
\end{Coqsrc}


The following lemma establishes a relation between \texttt{steps} and the predicate \texttt{battle}.

\begin{Coqsrc}
 Lemma steps_battle : forall i a b c j a' b' c', 
   steps i a b c j a' b' c' ->
   battle standard i (hyd  a b c)   j  (hyd a' b' c').
\end{Coqsrc}

Thus, any result about \texttt{steps} will be applicable to standard battles.
Using the predicate \texttt{steps} our study of the length of the considered battle
can  be decomposed into three parts:

\begin{enumerate}
\item  Characterization of regularities of some events
\item Study of the beginning of the battle
\item Computing the exact length of the battle.
\end{enumerate}

First, we prove that, if at round $i$ the hydra is equal to
(\texttt{hyd a (S b) 0}), then it will be equal to (\texttt{hyd a b 0}) at the $2(i+1)$-th round.  

\begin{Coqsrc}
Lemma LS : forall c a b i,  steps i a b (S c) (i + S c) a b 0.
Proof.
  induction c.
 -   intros;  replace (i + 1) with (S i).
     + repeat constructor.
     + ring.
 -  intros; eapply  steps_S.
   +   eleft;   apply rule1.
   +   replace (i + S (S c)) with (S i + S c) by ring;  apply IHc.
Qed.
\end{Coqsrc}

\begin{Coqsrc}
Lemma doubleS_law : forall  a b i, steps i a (S b) 0 (doubleS i) a b 0.
Proof.
  intros;  eapply steps_S.
  +   eleft;   apply step2.
  +   unfold doubleS; replace (2 * S i) with (S i + S i) by ring; 
        apply LS.
Qed.
\end{Coqsrc}


\begin{Coqsrc}
Lemma reachable_S  : forall i a b, reachable i a (S b) 0 ->
                                   reachable (doubleS i) a b 0.
Proof.
  intros; right with  (1 := H); apply doubleS_law.
Qed.
\end{Coqsrc}

From now on, the lemma \texttt{reachable\_S} allows us to watch larger steps of 
the battle.


\begin{Coqsrc}
 Lemma L4 : reachable 4 2 4 0.
Proof.
  left; constructor.
Qed.
\end{Coqsrc}

\begin{Coqsrc}
Lemma L10 : reachable 10 2 3 0.
Proof.
  change 10 with (doubleS 4).
  apply reachable_S, L4.
Qed.
\end{Coqsrc}

\begin{Coqsrc}
Lemma L22 : reachable 22 2 2 0.
Proof.
  change 22 with (doubleS 10).
  apply reachable_S, L10.
Qed.
\end{Coqsrc}

\begin{Coqsrc}
Lemma L46 : reachable 46 2 1 0.
Proof.
  change 46 with (doubleS 22); apply  reachable_S, L22.
Qed.
\end{Coqsrc}

\begin{Coqsrc}
Lemma L94 : reachable 94 2 0 0.
Proof.
  change 94 with (doubleS 46); apply reachable_S, L46.
Qed.
\end{Coqsrc}

\begin{Coqsrc}
Lemma L95 : reachable 95 1 95 0.
Proof.
  eapply steps_S.
  -  eexact L94.
  -  repeat constructor.
Qed.
\end{Coqsrc}

\subsection{Giant steps}

We are now able to make bigger steps in the simulation of the battle.
First, we iterate the lemma \texttt{reachable\_S}.

\begin{Coqsrc}
Lemma Bigstep : forall b i a , reachable i a b 0 ->
                               reachable (iterate doubleS b i) a 0 0.
 Proof.
  induction b.
  -  trivial.
  -  intros;  simpl;   apply reachable_S in H.
     rewrite <- iterate_comm; now apply IHb.
 Qed.
\end{Coqsrc}

Applying lemmas \texttt{BigStep} and \texttt{L95} we make a first jump.


\begin{Coqsrc}
 Definition M := (iterate doubleS 95 95).

Lemma L2_95 : reachable M 1 0 0.
Proof.
  apply Bigstep,  L95.
Qed.
\end{Coqsrc}

Figure~\ref{fig:HM}  represents the hydra at the $M$-th round.
At the $(M+1)$-th round, it will look like in fig~\ref{fig:HM-plus1}.





\begin{figure}[htb]
\centering
\begin{tikzpicture}[very thick, scale=0.5]
\node (foot) at (2,0) {$\bullet$};
\node (N1) at (2,2) {$\bullet$};
\node (N2) at (3,4) {$\Smiley[2][green]$};
\node (N3) at (1,4) {$\Smiley[2][green]$};
\draw (foot) -- (N1);
\draw (N1) to [bend right =15] (N2);
\draw (N1) to  [bend left=15](N3);
\end{tikzpicture}
\caption{\label{fig:HM}}
The state of the hydra after $M$ rounds.
% The hydra \texttt{h} of the proof that \(\omega^2\) is too small for proving Hercules' victory

\end{figure}


\begin{figure}[htb]
\centering
\begin{tikzpicture}[very thick, scale=0.5]
\node (foot) at (10,0) {$\bullet$};
\node (N1) at (0,5) {$\bullet$};
\node (N12) at (0,8) {$\Smiley[2][green]$};
\node (N2) at (2,5) {$\bullet$};
\node (N22) at (2,8) {$\Smiley[2][green]$};
\node (N3) at (4,5) {$\bullet$};
\node (N32) at (4,8) {$\Smiley[2][green]$};
\node (N4) at (6,5) {$\bullet$};
\node (N42) at (6,8) {$\Smiley[2][green]$};

\node (Ndots) at (12,8) {\Huge $\dots$};
\node (Ndots2) at (12,5) {\Huge $\dots$};

\node (N8) at (18,5) {$\bullet$};
\node (N82) at (18,8) {$\Smiley[2][green]$};
\node (N9) at (20,5) {$\bullet$};
\node (N92) at (20,8) {$\Smiley[2][green]$};


\draw (foot) -- (N1);
\draw (foot) -- (N2);
\draw (foot) -- (N3);
\draw (foot) -- (N4);
\draw (foot) -- (N8);
\draw (foot) -- (N9);
\draw (N1) to  (N12);
\draw (N2) to  (N22);
\draw (N3) to  (N32);
\draw (N4) to  (N42);
\draw (N8) to  (N82);
\draw (N9) to  (N92);
\end{tikzpicture}
\caption{\label{fig:HM-plus1}}
The state of the hydra after $M+1$ rounds (with $M+1$ heads). 

\end{figure}

\begin{Coqsrc}
Lemma L2_95_S : reachable (S M) 0 (S M) 0.
Proof.
  eright.
  - apply L2_95.
  -  left; constructor 3.
Qed.
\end{Coqsrc}


Then, applying once more the lemma \texttt{BigStep}, we get the exact time when
Hercules wins!


\begin{Coqsrc}
Definition N :=   iterate doubleS (S M) (S M).

Theorem   SuperbigStep : reachable N  0 0 0 .
Proof.
  apply Bigstep, L2_95_S.
Qed.
\end{Coqsrc}

We are now able to prove formally that the considered battle is 
composed of $N$ steps.

\begin{Coqsrc}
Lemma Almost_done :
  battle standard 3 (hyd 3 0 0) N (hyd 0 0 0).
Proof. 
  apply steps_battle, SuperbigStep.
Qed.

Theorem Done :
  battle standard 0 hinit N head.
Proof.
  eapply battle_trans.
  -   apply Almost_done.
  -  apply L_0_3.
Qed.
\end{Coqsrc}



\subsection{A minoration lemma}

Now, we would like to get an intuition of  how big the number $N$ is.
For that purpose, we use a minoration of the function \texttt{doubleS} by the
function (\texttt{fun n => 2 * n}).

\begin{Coqsrc}
Definition exp2 n := iterate (fun n => 2 * n) n 1.
\end{Coqsrc}
Using some facts (proven in \url{../src/html/BigBattle.html}),we get several  minorations.

\begin{Coqsrc}
Lemma minoration_0 : forall n,  2 * n <= doubleS n.

Lemma minoration_1 : forall n x, exp2 n * x <= iterate doubleS n x.

Lemma minoration_2 : exp2 95 * 95 <= M.

Lemma minoration_3 : exp2 (S M) * S M <= N.

Lemma minoration : exp2 (exp2 95 * 95) <= N.
\end{Coqsrc}


The number $N$ is greater than or  equal to $2^{2^{95}\times 95}.$ If we wrote $N$ in base $10$, $N$ would require at least $10^{30}$ digits!


\section{Generic properties}


The example we just studied shows that the termination of any battle may take a very long time. If we want to study hydra battles in general, we have to consider 
any hydra and any strategy, both for Hercules and the hydra itself. So, we first  give some definitions, generally borrowed from transition systems vocabulary (see~\cite{tel_2000} for instance).


\subsection{Liveliness}


Let $B$ be an instance of \texttt{Battle}. We say that $B$ is \emph{alive} if
for any configuration $(i,h)$, where $h$ is not a head, there exists a further step in class $B$.


\vspace{4pt}
\emph{From Module~\href{../src/html/hydras.Hydra.Hydra_Definitions.html\#Alive}{Hydra.Hydra\_Definitions}}

\begin{Coqsrc}
Definition Alive (B : Battle) :=
  forall i h, 
     h <> head -> {h' : Hydra |  B i h h'}.
\end{Coqsrc}

The theorems \texttt{Alive\_free} and \texttt{Alive\_standard} of the module 
\url{../src/html/hydras.Hydra.Hydra_Theorems.html} show that the classes \texttt{free} and \texttt{standard} satisfy this property.

\begin{Coqsrc}
Theorem Alive_free: Alive free.

Theorem Alive_standard: Alive standard.  
\end{Coqsrc}

Both theorems are proved with the help of the  following strongly specified function:

\vspace{4pt}
\emph{From Module~\href{../src/html/hydras.Hydra.Hydra_Lemmas.html\#next_round_dec}{Hydra.Hydra\_Lemmas}}

\begin{Coqsrc}
Definition  next_round_dec n :
 forall h , (h = head) + {h' : Hydra & {R1 h h'} + {R2  n h  h'}}.
\end{Coqsrc}


\subsection{Termination}

The termination of all battles is naturally expressed by the predicate \texttt{well\_founded} defined in the module \href{https://coq.inria.fr/distrib/current/stdlib/Coq.Init.Wf.html}{Coq.Init.Wf} 
 of the Standard Library.

\index{Predicates!Termination}

\begin{Coqsrc}
Definition Termination :=  well_founded (transp _ round).
\end{Coqsrc}


Let $B$ be an instance of class \texttt{Battle}. A \emph{variant} for $B$ consists
in a well-founded relation $<$  on some type \texttt{A}, and a function
(also called a \emph{measure}) \texttt{m:Hydra->A} such that for any successive steps $(i,h)$ and $(1+i,h')$  of a battle in $B$, the inequality $m(h')<m(h)$ holds.


\vspace{4pt}
\emph{From Module~\href{../src/html/hydras.Hydra.Hydra_Definitions.html\#Hvariant}{Hydra.Hydra\_Definitions}}

\index{Coq!Type classes}
\label{sect:hvariant-def}

\begin{Coqsrc}
Class Hvariant {A:Type}{Lt:relation A}(Wf: well_founded Lt)(B : Battle)
  (m: Hydra -> A):   Prop :=
  {variant_decr :forall i h h',
      h <> head ->
      battle_r  B i  h h' -> Lt (m h') (m h)}.
\end{Coqsrc}  

\index{Exercises}

\begin{exercise}
 Prove that, if there is an instance of (\texttt{Hvariant Lt wf\_Lt $B$ $m$}), then there exists no infinite battle in  $B$.
\end{exercise}




\subsection{A  small proof of impossibility}
\index{Maths!Proofs of impossibility}

\label{omega-case}

When one wants to prove a termination theorem with the help of a variant, 
one has to consider first a well-founded set $(A,<)$, then a strictly decreasing measure on this set.  The following two lemmas show that if  the order structure $(A,<)$ is too simple, it is useless to look for a convenient measure, which simply no exists. Such kind of result is useful, because it saves you time and effort.


The best known well-founded order is the natural order on the set $\mathbb{N}$ of natural numbers (the type \texttt{nat} of Standard library). It would be interesting to look for some measure $m:\texttt{nat}\arrow\texttt{nat}$ and prove it is a variant.

Unfortunately, we can prove that 
\emph{no} instance of class (\texttt{WfVariant round Peano.lt $m$}) can be built, where
$m$ is \emph{any} function of type \texttt{Hydra $\arrow$ nat}.


Let us present the main steps of that proof, the script of which  is in the module ~\href{../src/html/hydras.Hydra.Omega_Small.html}{Hydra/Omega\_Small.v} \footnote{ The name of this file means ``the ordinal $\omega$ is too small for proving the termination of [free] hydra battles ''. In effect, the elements of $\omega$, considered as a set, are just the natural numbers (see next chapter for more details)}.

%\subsubsection{Preliminaries}


Let us assume there exists some variant $m$ from \texttt{Hydra} into \texttt{nat} for proving
    the  termination of all hydra battles.

\begin{Coqsrc}
Section Impossibility_Proof.
 Variable m : Hydra -> nat.
 Hypothesis Hvar : Hvariant lt_wf free m.
\end{Coqsrc}

We define an injection from the type \texttt{nat} into \texttt{Hydra}.
For any natural number $i$, $\iota(i)$ is the hydra composed of a foot and
$i+1$ heads at height $1$. For instance, Fig.~\ref{fig:flower} represents the hydra $\iota(3)$.

\begin{figure}[htb]
\centering
\begin{tikzpicture}[very thick, scale=0.5]
\node (foot) at (4,0) {$\bullet$};
\node (N1) at (2,2) {$\Smiley[2][green]$};
\node (N2) at (4,2) {$\Smiley[2][green]$};
\node (N3) at (6,2) {$\Smiley[2][green]$};
\node (N4) at (8,2) {$\Smiley[2][green]$};
\draw (foot) to [bend left =25] (N1);
\draw (foot) to [bend left =15] (N2);
\draw (foot) to [bend right =15] (N3);
\draw (foot) to [bend right =25] (N4);
\end{tikzpicture}
\caption{\label{fig:flower}
The hydra $\iota(3)$}
\end{figure}

  \begin{Coqsrc}
  Let iota (i: nat) := hyd_mult head (S i).    
  \end{Coqsrc}

Let us consider now some hydra \texttt{big\_h} out of the range of the injection $\iota$ (see Fig.~\vref{fig:h-omega-omega}).

\begin{figure}[htb]
\centering
\begin{tikzpicture}[very thick, scale=0.5]
\node (foot) at (2,0) {$\bullet$};
\node (N1) at (2,2) {$\bullet$};
\node (N2) at (2,4) {$\Smiley[2][green]$};
\draw (foot) -- (N1);
\draw (N1) to  (N2);
\end{tikzpicture}
\caption{\label{fig:h-omega-omega}}
 The hydra \texttt{big\_h}.
\end{figure}

 \begin{Coqsrc}
  Let big_h := hyd1 (hyd1 head).
 \end{Coqsrc}

 Using the functions $m$ and $\iota$, we define a second hydra \texttt{small\_h}, and show
 there is a one-round battle that transforms \texttt{big\_h} into \texttt{small\_h}. Please note that,
due to the hypothesis \texttt{Hvar}, we are interested in the termination of \emph{free} battles. 
There is no problem to consider a round with (\texttt{m big\_h}) as the replication factor.

  \begin{Coqsrc}
 Let small_h := iota (m big_h).
   
 Fact big_to_small : big_h -1-> small_h.
 Proof.
      exists (m big_h); right; repeat constructor.     
 Qed.
    \end{Coqsrc}
 
But, by hypothesis, $m$ is a variant. Hence, we infer the following inequality.


 \begin{Coqsrc}
Lemma m_lt : m small_h < m big_h.
 \end{Coqsrc}

In order to get a contradiction, it suffices to  prove the inequality
\texttt{m big\_h <= m small\_h}, i.e.,  \texttt{m big\_h <= m (iota (m big\_h))}.

More generally, we prove the following lemma: 

\begin{Coqsrc}
Lemma m_ge : forall i:nat, i <= m (iota i).
\end{Coqsrc}

Intuitively, it means that, from any hydra of the form (\texttt{iota $i$}), the battle will 
take (at least) $i$ rounds. Thus the associated measure cannot be less than $i$.
Technically, we prove this lemma by Peano induction on $i$.

\begin{itemize}
\item The base case $i=0$ is trivial
\item Otherwise, let $i$ be any natural number and assume  the inequality
  $i \leq m(\iota(i))$.
  \begin{enumerate}
  \item  But the hydra $\iota(S(i))$ can be transformed in one round into
    $\iota(i)$ (by losing its righmost head, for instance)
  \item Since $m$ is a variant, we have $m(\iota(i)) < m(\iota(S(i)))$,
    hence  $i< m(\iota(S(i)))$, which implies  $S(i)\leq  m(\iota(S(i)))$.
  \end{enumerate}
\end{itemize}

 Then our proof is almost finished.
 
   \begin{Coqsrc}
Theorem Contradiction : False.
Proof.
 apply (Nat.lt_irrefl (m big_h));
   apply  Lt.le_lt_trans with (m small_h).
  - apply m_ge.
  - apply m_lt.
Qed. 

End Impossibility_Proof.
\end{Coqsrc}

\index{Exercises}

\begin{exercise}
Prove that there exists no variant $m$ from \texttt{Hydra} into \texttt{nat} for proving
    the  termination of all \emph{standard} battles.
\end{exercise}






\subsubsection{Conclusion}

In order to build a variant for proving the termination of all hydra battles, we need to consider order structures more complex than the usual order on type \texttt{nat}. 
The notion of \emph{ordinal number} provides a catalogue of well-founded order types.
For a reasonably large bunch of ordinal numbers, \emph{ordinal notations} are data-types which allow the \coq{} user to define functions, to compute and prove some properties, for instance by reflection.

So, the next chapters will present ordinal numbers and ordinal notations, and we will
look for a variant complex enough for proving the termination of all battles, but not too complex if possible.



%--------------------------------------------------------------

\chapter{Introduction to ordinal numbers and ordinal notations}


The proof of termination of all hydra battles presented in~\cite{KP82} is based
on \emph{ordinal numbers}.
From a mathematical point of view, an ordinal is a representant of an equivalence class for isomorphims of  totally ordered well-founded sets.

More intuitively, let us have a look at Figure~\ref{fig:ordinal-sequence}. It presents a small sequence of ordinal numbers, which extends the sequence of natural numbers. 




\begin{figure}[h]
  \centering
\fbox{\Large
  \begin{minipage}{1.0\linewidth}
  \begin{align*}
     &\textcolor{blue}{0},\,1,2,3,4,5,6,7,8,9,10,11,12,13,14,15,16,17,\ldots\\
&\textcolor{red}{\omega},\,\omega+1,\omega+2,\omega+3,\ldots\\
&\textcolor{red}{\omega\times 2},\,\omega\times 2+1,\ldots, \textcolor{red}{\omega\times 3},\,\omega\times 3+1,\ldots, \textcolor{red}{\omega\times 4},\ldots,\\
&\textcolor{red}{\omega^2},\ldots, \textcolor{red}{\omega^2\times 42},\ldots,\textcolor{red}{\omega^3},\ldots, \textcolor{red}{\omega^4},\omega^4+1,\ldots,\\
&\textcolor{red}{\omega^\omega},\ldots, \textcolor{red}{\omega^\omega+\omega^7\times 8},\ldots,\textcolor{red}{\omega^\omega\times 2},\omega^\omega\times 2+1, \ldots,\\
&\textcolor{red}{\omega^{\omega^\omega}},\ldots, \textcolor{red}{\omega^{\omega^\omega}+\omega^\omega\times 42+ \omega^{55}+\omega}, \ldots, \textcolor{red}{\omega^{\omega^{\omega+1}}}, \omega^{\omega^{\omega+1}}+1,\dots\\
& \textcolor{red}{\epsilon_0 (= \omega^{\epsilon_0)}}, \epsilon_0+1, \epsilon_0+2, \epsilon_0+3, \ldots,\\
& \textcolor{red}{\epsilon_1}, \ldots, \textcolor{red}{\epsilon_2}, \ldots, \textcolor{red}{\epsilon_\omega},\ldots \\
& \textcolor{red}{\Gamma_0}, \Gamma_0+1, \Gamma_0+2, \Gamma_0+3,\ldots, \textcolor{red}{\Gamma_0+\omega}, \ldots,\\
&\ldots
  \end{align*}   
  \end{minipage}}
 
 
  \caption{A short overview of the sequence of ordinal numbers}
  \label{fig:ordinal-sequence}
\end{figure}


Let us comment some features of this figure:

\begin{itemize}
\item The ordinals are listed in a strictly increasing order. 
\item Dots : ``$\ldots$'' stand for an infinite sequence of ordinals, not shown for lack of space. For instance, the ordinal $42$ is not shown in the first line, but it exists, between $17$ and $\omega$.
\item Each ordinal printed in black is the immediate successor of another ordinal. We call it a \emph{successor} ordinal. For instance, $12$ is the successor of $11$, and $\omega^4+1$ the successor of $\omega^4$.
\item Ordinals (displayed in red)  that  follow immediately dots are called \emph{limit ordinals}. With respect to the order induced by this sequence, any limit ordinal $\alpha$ is the least upper bound of  the set $\mathbb{O}_\alpha$ of all ordinals strictly less than $\alpha$.
\item
For instance $\omega$ is the least upper bound of the set of all finite ordinals (in the first line). It is also the first limit ordinal, and the first infinite ordinal, in the sense that 
the set $\mathbb{O}_\omega$ is infinite.
\item The ordinal $\epsilon_0$ is the first number which is equal to its own exponential of base $\omega$. It plays an important role in proof theory, and is particularly studied in chapters~\ref{chap:T1} to \ref{chap:alpha-large}.
\item Any ordinal is  either the ordinal \textcolor{blue}{$0$},
a successor ordinal, or a \textcolor{red}{limit ordinal}.
\end{itemize}




\section{The mathematical point of view}

\subsection{Well-ordered sets}
Let us start with some definitions.
A  \emph{well-ordered set} is a set provided with a binary relation $<$ which has the following properties.
\begin{description}
\item[irreflexivity] : $\forall x\in A, x\not< x$
\item[transitivity] : $\forall x\,y\,z\in A, x<y \Rightarrow y<z \Rightarrow x<z$
\item[trichotomy]: $\forall x\,y\in A, x<y \vee x = y \vee y < x$
\item[well foundedness]: $<$ is well-founded (every element of $A$ is accessible)\footnote{In classical mathematics, we would say that there is no infinite sequence $a_1>a_2> \dots a_n> a_{n+1}\dots$ in $A$. Please refer to any documenation on \coq{} for having more details on well-foundedness and accessibility.}.

\end{description}

The best known examples of well-ordered sets are the set $\mathbb{N}$ of natural numbers (with the usual order $<$), as well as any finite segment $[0,i)=\{j\in\mathbb{N}\,|\,j<i\}$.
The disjoint union of two copies of $\mathbb{N}$, \emph{i.e.} the set $\{0,1\}\times\mathbb{N}$ is also well-ordered,
with respect to the order below:

\begin{align*}
(i,j) < (i,k) & \;\textit{if} \; j < k\\
(0,k) < (1,l) & \;\textit{for\,any}\;k \;\textit{and} \; l
\end{align*}

\subsection{Ordinal numbers}

\index{Maths!Ordinal numbers}

Let $(A,<_A)$ and $(B,<_B)$ two well-ordered sets. $A$ and $B$ are said to have \emph{the same order type} if 
there exists a strictly monotonous bijection $b$ from $A$ to $B$, \emph{i.e.} which verifies the proposition
$\forall x\,y\in A, x <_A y \Rightarrow b(x) <_B  b(y)$.

Having the same order type is an equivalence relation between well-ordered sets. Ordinal numbers (in short \emph{ordinals}) are descriptions (\emph{names}) of the equivalence classes.
For instance, the order type of $(\mathbb{N},<)$ is associated with the ordinal called  $\omega$, and the order we considered on 
the disjoint union of $\mathbb{N}$ and itself is named $\omega+\omega$.

In a set-theoretic framework, one can consider any ordinal $\alpha$ as a well-ordered set, whose  elements are just the ordinals strictly less than $\alpha$, \emph{i.e.} the \emph{segment} $\mathbb{O}_\alpha=[0, \alpha)$. So, one can speak about \emph{finite}, \emph{infinite}, \emph{countable}, etc., ordinals. Nevertheless, since we work within type theory, 
we do not identify ordinals as sets of ordinals, but the correspondance between ordinals and sets of ordinals is the function that maps $\alpha$ to $\mathbb{O}_\alpha$.
For instance $\mathbb{O}_\omega=\mathbb{N}$, and $\mathbb{O}_7=\{0,1,2,3,4,5,6\}$.


We cannot cite all the litterature published on ordinals since Cantor's book 
\cite{cantorbook}, and 
leave it to the reader to explore the bibliography. 


\section{Ordinal numbers in Coq}

Two kinds of representation of ordinals are defined herein.

\begin{itemize}
\item A ``mathematical'' representation of the set of countable ordinal numbers, afer Kurt Schütte~\cite{schutte}. This representation uses several (hopefully harmless) axioms. We use it as a reference for proving the correctness of ordinal notations.
\item A family of \emph{ordinal notations}, \emph{i.e.} data types used to represent segments $[0,\mu)$, where $\mu$ is some countable ordinal. Each ordinal notation is defined inside the Calculus of Inductive Constructions (without axioms). Many functions are defined, allowing proofs by computation. Note that proofs of 
correctness of a given ordinal notation with respect to Schütte's model obviously use axioms.
Please execute the \texttt{Print Assumptions} command in case of doubt.
\end{itemize}

\section{Countable ordinals}

Chapter~\ref{chap:schutte} of this document presents an adaptation to \coq{} of an axiomatization in classical logic of the set of countable ordinals by K. Schütte~\cite{schutte}. 
That formalization is quite complex, technical and unshamedly non-constructive,  so we put its description  in the last chapter of this document. 

Please note that Schütte considers the (uncountable) set $\mathbb{O}$ of all countable ordinals. This set is well ordered (which is one of Schütte's axioms), and associates to any ordinal $\alpha$ the segment $\mathbb{O}_\alpha$ of all ordinals strictly less than $\alpha$.

In our adaptation to \coq{}, we declare a type \texttt{Ord}, a binary relation \texttt{lt} (with infix notation \texttt{"\_<\_"}, and assume Schütte's axiom. In Chapter~\ref{chap:schutte},
we derive some interesting properties of countable ordinals from these axioms.
It is interesting to compare proofs of a given property (for instance the associativity of addition) in the computatuinal framework of some ordinal notation, and in the axiomatic model of Schütte.


\section{Ordinal notations}


Fortunately, the ordinals we need for  studying hydra battles are much simpler than Schütte's, and can be represented as quite simple data types in \gallina. So, we will use \emph{ordinal notations} (also called \emph{[ordinal] notation systems}). 

Let $\alpha$ be some (countable) ordinal; 
in \coq{} terms, we call ordinal notation for $\alpha$ a structure composed 
of:
\begin{itemize}
\item A data type $A$ for representing all ordinals strictly below $\alpha$,
\item A well founded order $<$ on $A$, 
\item A correct function for comparing two ordinals. Note  that the reflexive closure of $<$ is thus a \emph{total order}.
\end{itemize}


Such a structure can be proved correct relatively to another ordinal notation or
to Schütte's model.




\subsubsection*{Ordered types}

The library ~\href{https://coq.inria.fr/distrib/current/stdlib/Coq.Classes.RelationClasses.html}{%
Coq.Classes.RelationClasses} contains some definitions and facts about binary relations, among them strict orders.

\index{Coq!Type classes}

\begin{Coqsrc}
Variable A: Type.

  Class StrictOrder (R : relation A) : Prop := {
    StrictOrder_Irreflexive :> Irreflexive R ;
    StrictOrder_Transitive :> Transitive R }.
\end{Coqsrc}



\subsection{A class for ordinal notations}

The following class definition, parameterized with a type $A$, a binary relation \texttt{lt} on $A$, specifies that \texttt{lt} is a well-founded strict order. The reflexive closure of \texttt{lt}, (called \texttt{le}, for ``less  or equal than'') is a total decidable order, implemented through a comparison function \texttt{compare}.  The correctness of this function is expressed through Stdlib's type 
\texttt{Datatypes.CompareSpec}.


\begin{Coqsrc}
Inductive CompareSpec (Peq Plt Pgt : Prop) : comparison -> Prop :=
    CompEq : Peq -> CompareSpec Peq Plt Pgt Eq
  | CompLt : Plt -> CompareSpec Peq Plt Pgt Lt
  | CompGt : Pgt -> CompareSpec Peq Plt Pgt Gt
\end{Coqsrc}

\vspace{4pt}
\noindent\emph{From Library~\href{../src/html/hydras.OrdinalNotations.Definitions.html}{OrdinalNotations.Definitions}}

\index{Coq!Type classes}
\label{types:ON}

\begin{Coqsrc}
Class ON {A:Type}(lt: relation A)
      (compare : A -> A -> comparison)  :=
  {
  sto :> StrictOrder lt;
  wf : well_founded lt;
  compare_correct :
    forall alpha beta:A,
      CompareSpec (alpha=beta) (lt alpha beta) (lt beta alpha)
                  (compare alpha beta);
  }.
\end{Coqsrc}

The following definitions allow us to make implicit several guessable arguments.
\label{sect:on-lt-notation}
\label{sect:on-le-notation}
\begin{Coqsrc}
Definition on_t  {A:Type}{lt: relation A}
            {compare : A -> A -> comparison}
            {on : ON lt compare} := A.

Definition ON_compare {A:Type}{lt: relation A}
            {compare : A -> A -> comparison}
            {on : ON lt compare} := compare.

Definition ON_lt {A:Type}{lt: relation A}
           {compare : A -> A -> comparison}
           {on : ON lt compare} := lt.
Infix "o<" := ON_lt : ON_scope.

Definition ON_le  {A:Type}{lt: relation A}
           {compare : A -> A -> comparison}
           {on : ON lt compare} :=
  clos_refl _ ON_lt.

Infix "o<=" := ON_le : ON_scope.
\end{Coqsrc}


\begin{remark}
The infix notations \texttt{o<} and \texttt{o<=} were defined in order to make apparent the distinction between the various notation scopes that may co-exist in a same statement. So the infix \texttt{<} and \texttt{<=} are reserved to the natural numbers. In the mathematical formulas, we still use $<$ and $\leq$ for comparing ordinals.
\end{remark}


\subsection{Ordinal notations and measures for proving termination}
\label{sect:measure-ON}

The following lemma (together with the type class mechnism) allows us to use simply  measures towards an ordinal notation. It is just an application of  the libraries \texttt{Coq.Wellfounded.Inverse\_Image}
and  \texttt{Coq.Wellfounded.Inclusion}. 

\begin{Coqsrc}
Definition measure_lt {A:Type}{lt: relation A}
            {compare : A -> A -> comparison}
            {on : ON lt compare}
            {B : Type} (m : B -> A) : relation B :=
             fun x y => on_lt (m x) (m y).
            
Lemma wf_measure  {A:Type}(lt: relation A)
            {compare : A -> A -> comparison}
            {on : ON lt compare}
            {B : Type}
            (m : B -> A):  well_founded (measure_lt m). 
\end{Coqsrc}

A simple example of application is given in Sect.~\vref{sect:merge-example}.


\section{Examples of ordinal notations}


\subsection{The ordinal \texorpdfstring{$\omega$}{omega}}

The simplest example of ordinal notation is built over the type \texttt{nat} of \coq's standard library. We have only to apply already proven lemmas about Peano numbers.

\vspace{4pt}
\noindent\emph{From Library~\href{../src/html/hydras.OrdinalNotations/ON_Omega.html}{OrdinalNotations.ON\_Omega}}


\begin{Coqsrc}
Global Instance Omega : ON  Peano.lt Nat.compare.
Proof.
 split.
 - apply Nat.lt_strorder.
 - apply Wf_nat.lt_wf.
 - apply Nat.compare_spec.
Qed.
\end{Coqsrc}


\subsection{Sum of two ordinal notations}

Let \texttt{NA} and \texttt{NB} be two ordinal notations, on the respective types \texttt{A} and \texttt{B}.

 We consider a new strict order
on the disjoint sum of the associated types, by putting all elements of \texttt{A} before the elements of \texttt{B} (thanks to Standard Library's relation operator \texttt{le\_AsB}).



\emph{From Library~\href{https://coq.inria.fr/distrib/current/stdlib/Coq.Relations.Relation_Operators.html}{Relations.Relation\_Operators}}.

\begin{Coqanswer}
Inductive
le_AsB (A B : Type) (leA : A -> A -> Prop) (leB : B -> B -> Prop)
  : A + B -> A + B -> Prop :=
| le_aa : forall x y : A, leA x y -> le_AsB A B leA leB (inl x) (inl y)
| le_ab : forall (x : A) (y : B), le_AsB A B leA leB (inl x) (inr y)
| le_bb : forall x y : B, leB x y -> le_AsB A B leA leB (inr x) (inr y)
\end{Coqanswer}


\vspace{4pt}
\noindent\emph{From Library~\href{../src/html/hydras.OrdinalNotations/ON_plus.html}{OrdinalNotations.ON\_plus}}


\begin{Coqsrc}
Section Defs.

  Context `(ltA: relation A)
          (compareA : A -> A -> comparison)
          (NA: ON ltA compareA).
  Context `(ltB: relation B)
          (compareB : B -> B -> comparison)
          (NB: ON ltB compareB).


Definition t := (A + B)%type.
Arguments inl  {A B} _.
Arguments inr  {A B} _.

Definition lt : relation t := le_AsB _ _ ltA ltB.
\end{Coqsrc}

Before building an instance of \texttt{ON}, we have to define a comparison function.


\begin{Coqsrc}
Definition compare (alpha beta: t) : comparison :=
   match alpha, beta with
     inl _, inr _ => Lt
   | inl a, inl a' => compareA a a'
   | inr b, inr b' => compareB b b'
   | inr _, inl _ => Gt
  end.

Lemma compare_correct alpha beta :
    CompareSpec (alpha = beta) (lt alpha beta) (lt beta alpha)
                            (compare alpha beta).
\end{Coqsrc}

The Lemma \texttt{Wellfounded.Disjoint\_Union.wf\_disjoint\_sum} of Standard Library
helps us to prove that our order \texttt{lt} is well-founded.


\begin{Coqsrc}
Global Instance ON_plus : ON lt compare.
Proof.
  split.
  - apply lt_strorder.
  -  apply lt_wf.
  - apply compare_correct.
Qed.
\end{Coqsrc}






\subsection{The ordinal \texorpdfstring{$\omega+\omega$}{omega + omega}}

The ordinal $\omega+\omega$ (also known as $\omega\times 2$) may be represented as the concatenation 
of two copies of $\omega$ (Figure~\ref{fig:omega-plus-omega}).

\begin{figure}[h]
   \centering
   \begin{tikzpicture}[very thick, scale=0.5]
\begin{scope}[color=blue]
\node(A0) at (2,0)[label=below:$0$]{$\bullet$};
\node(A1) at (3,0)[label=below:$1$]{$\bullet$};
\node(A2) at (4,0)[label=below:$2$]{$\bullet$};
\node (Adots) at (6,0) {$\ldots$};
\node(An) at (8,0)[label=below:$n$]{$\bullet$};
\node(A2) at (10,0)[label=below:$n+1$]{$\bullet$};
\node (Adots1) at (12,0) {$\ldots$};
\end{scope}
\begin{scope}[color=red]
\node(B0) at (14,0)[label=below:$0$,label=above:\textcolor{red}{$\omega$}]{$\bullet$};
\node(B1) at (16,0)[label=below:$1$, label=above:$\omega+1$]{$\bullet$};
\node(B2) at (18,0)[label=below:$2$,label=above:$\omega+2$]{$\bullet$};
\node (Bdots) at (20,0) {$\ldots$};
\node (Bn) at (22,0) [label=below:$p$, label=above:$\omega+p$]{$\bullet$};
\node (Bdots2) at (24,0) {$\ldots$};
\end{scope}
\end{tikzpicture}
   \caption{\textcolor{blue}{$\omega+{\color{red}\omega}$}}
   \label{fig:omega-plus-omega}
 \end{figure}

We can define this notation in \coq{} as an instance of \texttt{ON\_plus}.


\vspace{4pt}
\noindent\emph{From Module~\href{../src/html/hydras.OrdinalNotations.ON_Omega_plus_omega.html}{OrdinalNotations.ON\_Omega\_plus\_omega}}

\begin{Coqsrc}
Definition Omega_plus_Omega := ON_plus Omega Omega.

Existing Instance Omega_plus_Omega.
Definition t := @ON_plus.t nat nat.
\end{Coqsrc}

\begin{Coqsrc}
Example ex1 : inl 7 o< inr 0.
Proof. constructor. Qed.
\end{Coqsrc}

We can now define abbreviations. For instance, the finite ordinals are represented by terms built with  the constructor \texttt{inl}, and the first infinite ordinal $\omega$ by the term \texttt{(inr 0)}.

\begin{Coqsrc}
Definition fin (i:nat) : t := inl i.
Coercion fin : nat >-> t.

Notation "'omega'" := (inr  0:t).
\end{Coqsrc}

\begin{Coqsrc}
Example ex1' : fin 7 o< omega.
Proof. constructor. Qed.

Lemma lt_omega alpha : 
     alpha o< omega <-> exists n:nat,  alpha = fin n.
(* ... *)
\end{Coqsrc}


% \label{warning:coercions}
% \index{Coq!Techniques!Coercions} 
% \begin{remark}
% Beware of coercions and notation scopes!
% Let us consider the following goal:

% \begin{Coqsrc}
%  Goal (6 o< 8).
%  auto with arith.
% \end{Coqsrc}


% \begin{Coqanswer}
% 1 subgoal (ID 9)
  
%   ============================
%   6 o< 8
% \end{Coqanswer}

% Please keep in mind that the current notation scope interprets the infix \texttt{``<''} as the predicate \texttt{Omega\_plus\_omega.lt} and not \texttt{Nat.lt}. More,  the coercion mechanism converts the terms \texttt{6:nat} [resp. \texttt{8:nat} ]
% into \texttt{inl 6} [resp. \texttt{inl 8}].  So, the initial goal is correctly interpreted by \coq{}, but not as an inequality between two natural numbers.


% \begin{Coqsrc}
% Set Printing All.
% \end{Coqsrc}

% \begin{Coqanswer}
% 1 subgoal (ID 337)
  
%   ============================
%   @on_lt nat Peano.lt Nat.compare Omega (S (S (S (S (S (S O))))))
%     (S (S (S (S (S (S (S (S O))))))))
% \end{Coqanswer}


% Anyway, the initial goal is provable, using \texttt{le\_AsB}'s first constructor.

% \begin{Coqsrc}
%   constructor; auto with arith.
% Qed.
% \end{Coqsrc}

% \end{remark}
%

\section{Limits and Successors}

Let us look again at our implementation of $\omega+\omega$. We can distinguish between the three kinds of ordinals seen in Fig~\ref{fig:ordinal-sequence}:

\begin{itemize}
\item The least ordinal, \texttt{(inl 0)}, also written \texttt{(fin 0)}.
\item The limit ordinal $\omega$.
\item The successor ordinals, either of the form \texttt{(inl (S $i$))} or \texttt{(inr (S $i$))}
\end{itemize}

\subsection{Definitions}
It would be interesting to specify at the most generic level, what is a zero, a successor or a limit ordinal. Let $<$ be a strict order on a type $A$.

\begin{itemize}
\item A \emph{least} element is a minorant (in the large sense) of the full set  on $A$,
\item $y$ is a \emph{successor} of $x$ if $x<y$ and there is no element between $x$ and $y$. We will also say that $x$ is a \emph{predecessor} of $y$.
\item $x$ is a \emph{limit} if $x$ is not a least element, and for any $y$ such that $yo<x$,
 there exists some $z$ such that $y<z<x$.
\end{itemize}


The following definitions are in Library \href{../src/html/hydras.Prelude.MoreOrders.html}{Prelude.MoreOrders}.

\begin{Coqsrc}
Section A_given.
  Variables (A : Type)  (lt: relation A).
  
Local Infix "<" := lt.
Local Infix "<=" := (clos_refl _ lt).

Definition Least {sto : StrictOrder lt} (x : A):=
  forall y,  x <= y.

Definition Successor {sto : StrictOrder lt} (y x : A):=
  x < y /\ (forall z,  x < z ->  z <  y -> False).

Definition Limit {sto : StrictOrder lt}  (x:A)  :=
  (exists w:A,  w < x) /\
  (forall y:A, y < x -> exists z:A, y < z /\ z < x).
\end{Coqsrc}

\index{Exercises}
\begin{exercise}
Prove, that, in any ordinal notation system, that every ordinal has at most one predecessor, and at most one successor. 
\end{exercise}

\index{Exercises}
\begin{exercise}
Prove, that, in any ordinal notation system, that if $\beta$ is a successor of $\alpha$,
then for any $\gamma$, $\gamma<\beta$ implies 
$\gamma\leq\alpha$.
\end{exercise}




\subsection{Limits and successors in \texorpdfstring{$\omega+\omega$}{omega+omega}}

Using the definitions above, we can prove the following lemma:

\vspace{4pt}

\noindent\emph{From Module~\href{../src/html/hydras.OrdinalNotations.ON_Omega_plus_omega.html}{OrdinalNotations.ON\_Omega\_plus\_omega}}

\begin{Coqsrc}
Lemma limit_iff (alpha : t) : Limit alpha <-> alpha = omega.
\end{Coqsrc}

Regarding successors, let us introduce the following definition:

\begin{Coqsrc}
Definition succ (alpha : t) :=
  match alpha with
    inl n => inl (S n)
  | inr n => inr (S n)
  end.
\end{Coqsrc}


\begin{Coqsrc}
Lemma Successor_correct alpha beta : Successor beta alpha <->
                                     beta = succ alpha.
\end{Coqsrc}

We can also check whether an ordinal is a successor by a simple pattern matching:

\begin{Coqsrc}
Definition succb (alpha: t) : bool := match alpha with
                                 | inr (S  _) | inl (S _) => true
                                 | _ => false
                                 end.

Lemma succb_correct (alpha : t) :
    succb alpha <->  exists beta: t, alpha = succ beta.
\end{Coqsrc}


Finally, the nature of any ordinal is decidable(inside this notation system) :

\noindent\emph{From Module~\href{../src/html/hydras.OrdinalNotations.Generic.html}{OrdinalNotations.Generic}}
\begin{Coqsrc}
  Definition ZeroLimitSucc_dec {A:Type}{lt: relation A}
           {compare : A -> A -> comparison}
           {on : ON lt compare} :=
  forall alpha,
    {Least alpha} +
    {Limit alpha} +
    {beta: A | Successor alpha beta}.
(* ... *)
\end{Coqsrc}

\noindent\emph{From Module~\href{../src/html/hydras.OrdinalNotations.ON_Omega_plus_omega.html}{OrdinalNotations.ON\_Omega\_plus\_omega}}

\begin{Coqsrc}
Definition Zero_limit_succ_dec : ZeroLimitSucc_dec.
\end{Coqsrc}


\section{The ordinal \texorpdfstring{$\omega^2$}{omega^2}}

The ordinal $\omega^2$ (also called $\varphi_0(2)$, see Chap.~\ref{chap:schutte}), is represented by the lexicographically ordered cartesian product $\mathbb{N}\times\mathbb{N}$. 

\vspace{4pt}
\noindent\emph{From Module~\href{../src/html/hydras.OrdinalNotations.ON_Omega2.html}{ON\_Omega2}}

\begin{Coqsrc}
Definition t := (nat * nat)%type.

Definition lt : relation t := lexico Peano.lt Peano.lt.
Definition le := clos_refl _ lt.

Infix "o<" := lt : o2_scope.
Infix "o<=" := le : o2_scope.
\end{Coqsrc}

It is easy to represent the finite ordinals, successors and limits.

\begin{Coqsrc}
Definition zero: t := (0,0).

Definition fin (n:nat) : t := (0, n).

Notation "'F'" := fin : o2_scope.

Coercion fin : nat >-> t.

Definition succ (alpha : t) := (fst alpha, S (snd alpha)).

Notation "'omega'" := (1,0) : o2_scope.

Definition limitb (alpha : t) := 
 match alpha with
 |  (S _, 0) => true
 | _ => false
end.
\end{Coqsrc}


In order to build an ordinal notation out of the type \texttt{t}, we have to define a comparison function, and prove that the order \texttt{lt} is well-founded.



\begin{Coqsrc}
Instance lt_strorder : StrictOrder lt.
(* ... *)

Definition compare (alpha beta: t) : comparison :=
  match Nat.compare (fst alpha) (fst beta) with
    Eq => Nat.compare (snd alpha) (snd beta)
  | c => c
  end.

Lemma compare_correct alpha beta :
    CompareSpec (alpha = beta) (lt alpha beta) (lt beta alpha)
                (compare alpha beta).

Lemma lt_wf : well_founded lt.

Instance Omega2: ON lt compare.
Proof.
  split.
  - apply lt_strorder.
  - apply lt_wf.
  - apply compare_correct.
Qed.
\end{Coqsrc}


\subsection{Arithmetic of \texorpdfstring{$\omega^2$}{omega^2}} 

\subsubsection{Addition}

We can define on \texttt{ON\_omega2} an addition which extends the addition on \texttt{nat}.

\begin{Coqsrc}
Definition  plus (alpha beta : t) : t :=
  match alpha,beta with
  | (0, b), (0, b') => (0, b + b')
  | (0,0), y  => y
  | x, (0,0)  => x
  | (0, b), (S n', b') => (S n', b')
  | (S n, b), (S n', b') => (S n + S n', b')
  | (S n, b), (0, b') => (S n, b + b')
   end.

Infix "+" := plus : o2_scope.
\end{Coqsrc}

Please note that this operation is not commutative:

\begin{Coqsrc}
Example non_commutativity_of_plus :  omega + 3 <> 3 + omega.
Proof.
  cbn.
\end{Coqsrc}

\begin{Coqanswer}
1 subgoal (ID 237)
  
  ============================
(1, 3) <> omega.
\end{Coqanswer}

\begin{Coqsrc}
 discriminate.
Qed.
\end{Coqsrc}


\subsubsection{Multiplication}

The restriction of ordinal multiplication to the segment $[0,\omega^2)$ is not a total function.
For instance $\omega\times\omega= \omega^2$ is outside the set of represented values.
Nevertheless, we can define two operations mixing natural numbers and ordinals.

\begin{Coqsrc}
(** multiplication of an ordinal by a natural number *)

Definition mult_fin  (alpha : t) (p : nat): t :=
  match alpha, p with
 |  (0,0), _  => zero
 |  _, 0 => zero
 |  (0, n), p => (0, n * p)
 |  ( n, b),  n' => ( n *  n', b)
               end.
Infix "*" := mult_fin : o2_scope.

(** multiplication of  a natural number by an ordinal *)

Definition fin_mult (n:nat)(beta : t) : t :=
  match n, beta with
 |  0, _  => zero
 |  _, (0,0) => zero
 |   n , (0,n') => (0, (n*n')%nat)
 |  n, (n',p') => (n', (n * p')%nat)
 end.

Example e1 : fin_mult 3 (omega * 7 + 15) = omega * 7 + 45.
Proof. reflexivity. Qed.

Example e2 :  omega * 2 + 2 + omega * 3 + 10 = omega * 5 + 10.
Proof. reflexivity. Qed.
\end{Coqsrc}

Multiplication with a finite ordinal and addition are related through the following lemma:

\begin{Coqsrc}
Lemma unique_decomposition alpha : 
    exists! i j: nat,  alpha = omega * i + j.
\end{Coqsrc}

\subsection{A proof of termination using \texorpdfstring{$\omega^2$}{omega^2}} 
\label{sect:merge-example}.

Using the lemma of Sect.~\vref{sect:measure-ON}, we can define easily a total function which merges two lists.

\index{Coq!Commands!Function}

\begin{Coqsrc}
(* adapted from Pascal Manoury et al. *)


Require Import Coq.Program.Wf List.
Require Import FunInd Recdef.

Section Merge.

  Variable A: Type.

  Local Definition m (p : list A * list A) :=
    omega * length (fst p) + length (snd p).

  Function  merge  (ltb: A -> A -> bool)
          (xys: list A * list A)
          {wf (measure_lt m) xys} :
    list A :=
    match xys with
      (nil, ys) => ys
    | (xs, nil) => xs
    | (x :: xs, y :: ys) =>
      if ltb x y then x :: merge  ltb (xs, (y :: ys))
      else y :: merge  ltb ((x :: xs), ys)
    end.

  - intros; unfold m, measure_lt; cbn; destruct xs0; simpl; left; lia.
  - intros; unfold m, measure_lt; cbn; destruct ys0; simpl; right; lia.
   - auto.
  Defined.

End Merge.

Goal forall l, merge nat Nat.leb (nil, l) = l.
  intro; now rewrite merge_equation.
Qed.
\end{Coqsrc}
 

\subsection{Yet another  proof of impossibility}
\label{omega2-case}

In Sect.~\vref{omega-case}, we proved that there exists no variant towards \texttt{nat}
(\emph{i.e.} the ordinal $\omega$) for proving the termination of all hydra battles.
We  prove now that  the ordinal $\omega^2$ is also insufficient for this purpose. 

The proof we are going to develop has exactly the same structure as in Section~\ref{omega-case}.
 Nevertheless, the proof of technical  lemmas is a little more complex, due to 
 the structure of the lexicographic order on $\mathbb{N}\times\mathbb{N}$. 
Consider for instance that there exists an infinite number of ordinals  between
$\omega$ and $\omega\times 2$.



The detailed  proof script is in the file \url{../src/html/hydras.Hydra.Omega2_Small.html}.

\subsubsection{Preliminaries}
Let us assume there is a variant from \texttt{Hydra} into $\omega^2$  for proving the   termination of all hydra battles.

\vspace{4pt}
\emph{From Module~\href{../src/html/hydras.Hydra.Omega2_Small.html}{Hydra.Omega2\_Small}}


\begin{Coqsrc}
  Section Impossibility_Proof.
  
 Variable m : Hydra -> ON_Omega2.t.
  
 Context (Hvar : Hvariant lt_wf free m).
\end{Coqsrc}


Let us follow the same pattern as in Sect.~\ref{omega-case}.
First, we define an injection from type \texttt{t} into \texttt{Hydra}, by
 associating to  each ordinal $\omega\times i+ j = (i,j)$ the hydra with $i$ branches of length $2$ and
$j$ branches of length $1$.

%% revenir ici

\vspace{4pt}
\emph{From Module ~\href{../src/html/hydras.Hydra.Omega2_Small.html\#iota}{Hydra.Omega2\_Small}}

\begin{Coqsrc}
Let iota (p: ON_Omega2.t) := 
    node (hcons_mult (hyd1 head) (fst p)
                     (hcons_mult head (snd p) hnil)).
  \end{Coqsrc}

For instance, Figure~\vref{fig:essai2} shows the hydra associated to the ordinal 
$(3,5)$, a.k.a. $\omega\times 3 + 5$.

\begin{figure}[htb]
\centering
\begin{tikzpicture}[very thick, scale=0.4]
\node (foot) at (6,0) {$\bullet$};
\node (N1) at (1,3) {$\bullet$};
\node (N2) at (3,3) {$\bullet$};
\node (N3) at (5,3) {$\bullet$};
\node (N4) at (8,3) {$\Smiley[2][green]$};
\node (N5) at (11,3) {$\Smiley[2][green]$};
\node (N6) at (14,3) {$\Smiley[2][green]$};
\node (N7) at (17,3){$\Smiley[2][green]$};
\node (N8) at (20,3){$\Smiley[2][green]$};
\node  (N9) at (0,5) {$\Smiley[2][green]$};
\node (N10) at (2,5) {$\Smiley[2][green]$};
\node (N11) at (4,5) {$\Smiley[2][green]$};
\draw (foot) to [bend left=10] (N1);
\draw (foot) -- (N2);
\draw (foot) -- (N3);
\draw (foot) -- (N4);
\draw (foot) -- (N5);
\draw (foot) -- (N6);
\draw (foot) to [bend right=10] (N7);
\draw (foot) to [bend right=15] (N8);
\draw (N1) to [bend left=10] (N9);
\draw (N2) -- (N10);
\draw (N3) -- (N11);
\end{tikzpicture}
\caption{\label{fig:essai2}
The hydra $\iota(\omega\times 3+5)$}
\end{figure}




Like in Sect.~\ref{omega-case}, we build a hydra out of the range of \texttt{iota} (represented in Fig.~\vref{fig:h-omega2-small}).

\begin{figure}[htb]
\centering
\begin{tikzpicture}[very thick, scale=0.5]
\node (foot) at (2,0) {$\bullet$};
\node (N1) at (2,2) {$\bullet$};
\node (N2) at (3,4) {$\Smiley[2][green]$};
\node (N3) at (1,4) {$\Smiley[2][green]$};
\draw (foot) -- (N1);
\draw (N1) to [bend right =15] (N2);
\draw (N1) to  [bend left=15](N3);
\end{tikzpicture}
\caption{\label{fig:h-omega2-small}}
 The hydra \texttt{big\_h}.
\end{figure}


\begin{Coqsrc}
   Let big_h := hyd1 (hyd2 head head).  
\end{Coqsrc}
 
 In a second step, we build a ``smaller'' hydra.
 
\begin{Coqsrc}
   Let small_h := iota (m big_h).
\end{Coqsrc}

Like in Sect.~\ref{omega-case}, we prove the double inequality \texttt{m big\_h o<= m small\_h o< m big\_h}, which is impossible.

\subsubsection{Proof of the inequality \texttt{m small\_h o< m big\_h}}

In order to prove the inequality  \texttt{m\_lt: m small\_h o< m big\_h}, it suffices to
build a battle transforming \texttt{big\_h} into \texttt{small\_h}.

First we prove that \texttt{small\_h} is reachable from \texttt{big\_h} in one or two steps. Let us decompose \texttt{m big\_h} as $(i,j)$.
If $j=0$, then one round suffices to transform \texttt{big\_h} into $\iota(i,j)$.
If $j>0$, then a first round transforms \texttt{big\_h} into $\iota(i+1,0)$ and a second round into $\iota(i,j)$. So, we have the following result.

\begin{Coqsrc}
Lemma big_to_small: big_h -+-> small_h.
\end{Coqsrc}

Since $m$ is a variant, we infer the following inequality:

\begin{Coqsrc}
Corollary m_lt : m small_h o< m big_h.
\end{Coqsrc}


\subsubsection{Proof of the inequality \texttt{m big\_h o<= m small\_h} }


The proof of the inequality \texttt{m big\_h o<= m small\_h} is quite more complex than in Sect~\ref{omega-case}.  If we consider any ordinal $\alpha=(i,j)$, where $i>0$, there exists an infinite number of
ordinals stricly less than $\alpha$, and there exists an infinite number of battles that start from
$\iota(\alpha)$. Indeed, at any configuration $\iota(k,0)$, where $k>0$, the hydra can freely choose any replication number. Intuitively, the measure of such a hydra must be large enough for taking into account
all the possible battles issued from that hydra.
Let us now give more technical details.

\begin{itemize}
\item The proof of the lemma \texttt{m\_ge : m big\_h  o<= m small\_h} uses well-founded induction on \texttt{big\_h}.

\item For any pair $p$, we have to distinguish between three cases, according to the value of $p$'s components.
  \begin{itemize}
  \item $p=(0,0)$
  \item $p=(i,0)$, where $i>0$\,: $p$ corresponds to a limit ordinal
  \item $p=(i,j)$, where $j>0$\,: $p$ is the successor of $(i,j-1)$.
  \end{itemize}
\end{itemize}


Let us define the notion of elementary ``step'' of decreasing sequences in
\texttt{t}


\begin{Coqsrc}
Inductive step : t -> t -> Prop :=
| succ_step : forall i j,  step (i, S j) (i, j)
| limit_step : forall i j, step (S i, 0) (i, j).
\end{Coqsrc}

The following lemma establishes a correspondance between the relation
\texttt{step} and hydra battles.

\begin{Coqsrc}
Lemma step_to_battle : forall p q, step p q -> iota p -+-> iota q.
\end{Coqsrc}

\index{Maths!Transfinite induction}

Thus, starting from any inequality $q < p$ on type \texttt{t}, we can build 
by \emph{transfinite induction} (\emph{i.e.} well-founded) over \texttt{p} a battle 
that transforms the hydra $\iota(p)$ into $\iota(q)$.

\vspace{4pt}
\emph{From Module~\href{../src/html/hydras.Hydra.Omega2_Small.html\#m_ge}{Hydra.Omega2\_Small}}

\begin{Coqsrc}
Lemma m_ge : forall p : t,   p o<= m (iota p).
Proof.
  unfold small_h; pattern (m big_h) .   
   apply  well_founded_induction with (R := lt) (1:= lt_wf).
  intro p ; pattern p;
    apply  well_founded_induction with 
               (R := lt2) (1:= wf_lexico lt_wf lt_wf);
     intros (i,j) IHij. 
\end{Coqsrc}

\begin{Coqanswer}
  i, j : nat
  IHij : forall y : t, y o< (i, j) -> y o<= m (iota y)
  ============================
  (i, j) o<= m (iota (i, j)) 
\end{Coqanswer}


Then we have  three cases to consider, according to the values of $i$ and $j$.
\begin{itemize}
\item If $p=(0,0)$ then obviously, $\iota(p)\geq p = (0,0)$
\item If  $p=(i+1,0)$ for some $i\in\mathbb{N}$, we
 remark  that $p$ is strictly greater than any pair $ (i, j)$, where $j$ 
is any natural number.

Applying the battle rules, for any $j$, we have $\iota(i+1,j)  {\round} \iota(i, j) $, thus $m(\iota(p)) > m(\iota(i,j)$ since  $m$ is assumed to be a variant.

Applying the induction hypothesis, we get the inequality
 $ m(\iota(i,j)) \geq (i,j)$ for any $j$. 

Thus, $m(\iota(p)) > (i,j)$ for any $j$.
Applying the lemma \texttt{limit\_is\_lub}, we get  the inequality
$m(\iota(i+1,0))\geq (i+1,0)$

\item If $p=(i,j+1)$ with $j\in\mathbb{N}$, we have  $\iota(p)  {\round} \iota(i, j) $,
hence $m(\iota(p))> m(\iota(i,j)) \geq (i,j)$, thus $m(\iota(p))\geq (i,j+1)=p$

\end{itemize}

\begin{Coqsrc}
  (* ... *)
Qed.
\end{Coqsrc}

\subsubsection{End of the proof}

Since $<$ is a strict order (irreflexive  and transitive) on \texttt{nat*nat}, we infer that there is no
variant for termination on the lexicographic square of $(\mathbb{N},<)$.

\begin{enumerate}
\item From \texttt{m\_lt}, we infer the strict inequality 
\texttt{m small\_h o< m big\_h}
\item from \texttt{m\_ge}, we get \texttt{m big\_h o<= m (iota (m big\_h)) = m small\_h} 
\end{enumerate}


\vspace{4pt}
\emph{From Module~\href{../src/html/hydras.Hydra.Omega2_Small.html}{Hydra.Omega2\_Small}}

\begin{Coqsrc}
Theorem Impossible : False.
Proof.
  destruct (StrictOrder_Irreflexive  (m big_h)).
  apply le2_lt2_trans with (m small_h).
  -  unfold small_h;  apply m_ge.
  -  apply m_lt. 
Qed. 

End Impossibility_Proof.
\end{Coqsrc}

\index{Exercises}

\begin{exercise}
Prove that there exists no variant $m$ from \texttt{Hydra} into $\omega^2$ for proving
    the  termination of all \emph{standard} battles.
\end{exercise}



\begin{remark}
In Chapter~\ref{ks-chapter}, we  prove a generalization of the impossibility lemmas of
Sect.~\ref{omega-case} and this section, with the same proof structure, but with much more 
complex technical details.
 \end{remark}

\index{Exercises}
\begin{exercise}

\label{sec:orgheadline63}
Write \emph{direct} proofs ({i.e.},  without applying the result and tools of Chap.~\ref{ks-chapter}) that the following data structures  are too simple for defining a variant for any hydra battle.

\begin{itemize}
\item  $\omega^n$ : the set of all $n$-uples of natural numbers, ordered  by 
  lexicographic ordering
\item  $\omega^\omega$: the set of all decreasing sequences (with respect to $\le$)  of natural numbers, ordered by lexicographic ordering on lists.

For instance, the following inequality holds:
\[\langle 4,3,3,3,3,3,3,2,2,2 \rangle\,<\,\langle 4,4,2 \rangle\]
\end{itemize}

  
\end{exercise}


\section{A notation for finite ordinals}


Let $n$ be some natural number. The segment associated with $n$ is the interval 
$[0,n)\,=\,\{0,1,\dots,n-1\}$. 
One may represent the ordinal $n$ by a sigma type.


\vspace{4pt}
\noindent\emph{From Module~\href{../src/html/hydras.OrdinalNotations.ON_Finite.html}{OrdinalNotations.ON\_Finite}}

\label{def: Finite-ord-type}
\begin{Coqsrc}
Coercion is_true: bool >-> Sortclass.

Definition t (n:nat) := {i:nat | Nat.ltb i n}.
\end{Coqsrc}

The order on type (\texttt{t $n$}) is defined through the projection on \texttt{nat}.


\begin{Coqsrc}
Definition lt {n:nat} : relation (t n) :=
  fun alpha beta => Nat.ltb ( proj1_sig alpha) (proj1_sig beta).
\end{Coqsrc}

For instance, let us build two elements of the segment $[0, 7)$, \emph{i.e.} two
inhabitants of   type (\texttt{t 7}), and prove a simple  inequality (see Fig.~\ref{fig:O7}).

\begin{figure}[h]
\centering
\begin{tikzpicture}[very thick, scale=0.6]

\node (N0) at (0,0) {$\bullet$};
\node (i0) at (0,1) {$0$};
\node (N1) at (2,0) {$\bullet$};
\node (i1) at (2,1) {$1$};
\node (N2) at (4,0) {$\bullet$};
\node (i2) at (4,1) {$2$};
\node (N3) at (6,0) {$\bullet$};
\node (i3) at (6,1) {$3$};
\node (N4) at (8,0) {$\bullet$};
\node (i4) at (8,1) {$4$};
\node (N5) at (10,0) {$\bullet$};
\node (i5) at (10,1) {$5$};
\node (N6) at (12,0) {$\bullet$};
\node (i6) at (12,1) {$6$};
\node(alpha1) at (4,-1) {$\alpha_1$};
\node(alpha2) at (10,-1) {$\beta_1$};
\end{tikzpicture}

\caption{The segment $\mathbb{O}_7$\label{fig:O7}}
\end{figure}
  
\index{Coq!Commands!Program}

\begin{Coqsrc}
Program Example alpha1 : t 7 := 2.

Program Example beta1 : t 7 := 5.

Example i1 : lt  alpha1 beta1.
Proof. now compute. Qed.
\end{Coqsrc}




Note that the type (\texttt{t 0}) is empty, and that \texttt{Program} generates an obligation
for verifying that the constraint $i<n$ is fullfilled when one tries to build the $i$-th ordinal inside the type (\texttt{t $n$}).

\begin{Coqsrc}
Lemma t0_empty (alpha: t 0): False.
Proof.
  destruct alpha.
  destruct x; cbn in i; discriminate.
Qed.


Program Definition bad : t 10 := 10.
Next Obligation.
  compute.
\end{Coqsrc}

\begin{Coqanswer}
1 subgoal (ID 162)
  
  ============================
  false = true
\end{Coqanswer}

\begin{Coqsrc}
Abort.
\end{Coqsrc}

Note also that attempting to compare a term  of type (\texttt{t $n$}) with a term of
type (\texttt{t $p$})  leads to an error if $n$ and $p$ are not convertible.

\begin{Coqsrc}

Program Example gamma1 : t 8 := 7.

Fail Goal lt alpha1 gamma1.
\end{Coqsrc}

\begin{Coqanswer}
 The command has indeed failed with message:
The term "gamma1" has type "t 8" while it is expected to have type "t 7".
\end{Coqanswer}


In order to build an instance of \texttt{OrdinalNotation}, we define a comparison function, by delegation to standard library's  \texttt{Nat.compare}, and prove its correction.

\begin{Coqsrc}
Definition compare {n:nat} (alpha beta : t n) :=
  Nat.compare (proj1_sig alpha) (proj1_sig beta).

Lemma compare_correct {n} (alpha beta : t n) :
  CompareSpec (alpha = beta) (lt alpha beta) (lt beta alpha)
              (compare alpha beta).
\end{Coqsrc}

\begin{remark}
 The proof of \texttt{compare\_correct} uses a well-known pattern of \coq{}.
Let us consider  the following subgoal.

\begin{Coqanswer}
 1 subgoal (ID 110)
  
  n, x0 : nat
  i, i0 : x0 <? S n
  ============================
  exist (fun i1 : nat => i1 <=? n) x0 i =
  exist (fun i1 : nat => i1 <=? n) x0 i0
\end{Coqanswer}

Applying the tactic \texttt{f\_equal} generates a simpler subgoal.

\begin{Coqanswer}
1 subgoal (ID 112)
  
  n, x0 : nat
  i, i0 : x0 <? S n
  ============================
  i = i0
\end{Coqanswer}

We have now to prove that there exists at most one  proof of (\texttt{Nat.ltb x0 (S n)}). This is not obvious, but  a consequence of the following lemma of library 
\href{https://coq.inria.fr/distrib/current/stdlib/Coq.Logic.Eqdep_dec.html}{Coq.Logic.Eqdep\_dec}.

\index{Coq!Techniques!Unicity of equality proofs}
\label{sect:eq-proof-unicity}

\begin{Coqanswer}
eq_proofs_unicity_on :
forall (A : Type) (x : A),
(forall y : A, x = y \/ x <> y) -> 
forall (y : A) (p1 p2 : x = y), p1 = p2
\end{Coqanswer}

Thus unicity of proofs of \texttt{Nat.ltb x0 (S n)}  comes from the decidability of
equality on type \texttt{bool}.
This is why we used the boolean function \texttt{Nat.ltb} instead of the inductive predicate \texttt{Nat.lt} in the definition of type \texttt{t $n$} (see page~\pageref{def: Finite-ord-type}).
For more information about this pattern, please look at the numerous mailing lists and 
FAQs on \coq{}).



\end{remark}


Applying lemmas of the libraries \texttt{Coq.Wellfounded.Inverse\_Image}, \linebreak
 \texttt{Coq.Wellfounded.Inclusion}, and \texttt{Coq.Arith.Wf\_nat}, we prove that our
relation \texttt{lt} is well founded.

\begin{Coqsrc}
Lemma lt_wf (n:nat) : well_founded (@lt n).
\end{Coqsrc}

Now we can build our instance of \texttt{OrdinalNotation}.

\begin{Coqsrc}
Global Instance sto n : StrictOrder (@lt n).

Global Instance FinOrd (n:nat) : OrdinalNotation (sto n) compare.
Proof.
  split.
  - apply compare_correct.
  - apply lt_wf.
Qed.
\end{Coqsrc}

\begin{remark}
It is important to keep in mind  that the integer $n$ is not an ``element'' of \texttt{FinOrd $n$}. In set-theoretic presentations of ordinals, the set associated with the ordinal $n$ is $\{0,1,\dots,n-1\}$. 
In our formalization, the interpretation of an ordinal as a set is realized by the following definition
(in ~\href{../src/html/hydras.OrdinalNotations.Generic.html}{OrdinalNotations.Generic}).

\begin{Coqsrc}
Definition bigO `{nA : @OrdinalNotation A ltA stoA compareA}
           (a: A) : Ensemble A :=
  fun x: A => ltA x a.
\end{Coqsrc}
\end{remark}


\begin{remark}
 There is no interesting arihmetic on finite ordinals, since functions like successor, addition, etc.,  cannot be represented in \coq{} as \emph{total} functions.
\end{remark}

\begin{remark}
Finite ordinals are also formalized in MathComp~\cite{SSR}.  See also Adam Chlipala's \emph{CPDT}~\cite{chlipalacpdt2011} for a thorough study of the use of dependent types.  
\end{remark}



%%%

\section{Comparing two ordinal notations}

It is sometimes useful to compare two ordinal notations with respect to expressive power
(the segment of ordinals  they represent). 

The following class specifies a strict inclusion of segments. The notation \texttt{OA} describes a segment $[0,\alpha($, and \texttt{OB} is a larger segment (which contains a notation for $\alpha$, whilst $\alpha$ is not represented in \texttt{OA}). We require also  that the comparison functions of the two notation systems are compatible.

If \texttt{OB} is presumed to be correct, the we can consider that \texttt{OA} ``inherits'' its correcteness from the bigger notation system \texttt{OB}.

\index{Coq!Type classes}
\label{types:SubON}

\begin{Coqsrc}
Class  SubON 
       `(OA : @ON A ltA  compareA)
       `(OB : @ON B ltB  compareB)
       (alpha :  B)
       (iota : A -> B):=
  {
  SubON_compare: forall x y : A,  compareB (iota x) (iota y) =
                                 compareA x y;
  SubON_incl : forall x, ltB (iota x) alpha;
  SubON_onto : forall y, ltB y alpha  -> exists x:A, iota x = y}.
\end{Coqsrc}

For instance, we prove that \texttt{Omega} is a sub-notation of
\texttt{Omega\_plus\_Omega} (with $\omega$ as the first ``new'' ordinal, and \texttt{fin} as the injection).

\begin{Coqsrc}
Instance Incl : SubON Omega Omega_plus_Omega omega fin.
\end{Coqsrc}



We can also show that, if $i<j$, then the segment $[0,i)$ is a ``sub-segment'' of
$[0,j)$. Since the terms  ($t\;i$) and ($t\;j$) are not convertible, we consider a ``cast'' 
function $\iota$ from ($t\;i$) into ($t\;j$), and prove that this function is  a monotonous bijection  from ($t\;i$) to
the segment $[0,i)$ of ($t\;j$).




 \begin{figure}[h]
   \centering
   \begin{tikzpicture}[very thick, scale=0.6]
\begin{scope}[color=blue]
\node (A) at (0,0) {$A$};
\node(A0) at (2,0)[label=below:$0$]{$\bullet$};
\node(A1) at (3,0)[label=below:$1$]{$\bullet$};
\node(A2) at (4,0)[label=below:$2$]{$\bullet$};
\node (Adots) at (6,0) {$\ldots$};
\end{scope}
\begin{scope}[color=red]
\node (B) at (0,2) {$B$};
\node(B0) at (2,2)[label=above:$0$]{$\bullet$};
\node(B1) at (3,2)[label=above:$1$]{$\bullet$};
\node(B2) at (4,2)[label=above:$2$]{$\bullet$};
\node (Bdots) at (6,2) {$\ldots$};
\node (b) at (8,2) [label=above:$b$]{$\bullet$};
\node (bsucc) at (9,2) [label=above:$b+1$]{$\bullet$};
\node (Bdots2) at (10,2) {$\ldots$};
\end{scope}
\begin{scope}[color=red!50!blue]
\draw [->,thin] (A0) -- node [auto] {$\iota$} (B0);
\draw [->,thin] (A1) -- node [auto] {$\iota$} (B1);
\draw [->,thin] (A2) -- node [auto] {$\iota$} (B2);
\draw [->,thin] (Adots) -- node [auto] {$\iota$} (Bdots);
\end{scope}
\end{tikzpicture}
   \caption{\textcolor{blue}{$A$} is a sub-segment  of \textcolor{red}{$B$}}
   \label{fig:subsegment}
 \end{figure}




\index{Coq!Commands!Program}

We are now able to build an instance of \texttt{SubON}. 

\begin{Coqsrc}
Section Inclusion_ij.

  Variables i j : nat.
  Hypothesis Hij : (i < j)%nat.

   Remark Ltb_ij : Nat.ltb i j.
   Program Definition iota_ij  (alpha: t i) : t j :=  alpha.
 
   Let b : t j := exist _ i Ltb_ij.
   
   Global Instance F_incl_ij  : SubON  (FinOrd i) (FinOrd j) b iota_ij.
  (* ... *)

  End Inclusion_ij.

\end{Coqsrc}
         


\index{Projects}
\begin{project}
Prove that \texttt{Omega\_plus\_Omega} cannot be a sub-notation of \texttt{Omega}.
\end{project}

\index{Projects}
\begin{project}
Adapt the definition of \texttt{Hvariant} (Sect.~\ref{sect:hvariant-def}) in order to
have an ordinal notation as argument. Prove that if $O_A$ is a sub-notation of $O_B$, then any variant defined on  $O_A$ can be automatically transformed into 
a variant on $O_B$.
\end{project}




\section{Comparing an ordinal notation with Schütte's model}

Finally, it may be interesting to compare an ordinal notation with the more theoretical model from Schütte (well, at least with our formalization of that model). This would be a relative proof of correctenss of the considered  ordinal  notation.

The following class specifies that a notation \texttt{OA} describes a segment $[0,\alpha)$,
where $\alpha$ is a countable ordinal \emph{à la}  Schütte.

\index{Coq!Type classes}
\label{types:ON-for}

\begin{Coqsrc}
Class ON_correct `(alpha : Schutte_basics.Ord)
     `(OA : @ON A ltA  compareA)
      (iota : A -> Schutte_basics.Ord) :=
  { ON_correct_inj : forall a, Schutte_basics.lt (iota a) alpha;
    ON_correct_onto : forall beta, Schutte_basics.lt beta alpha ->
                                exists b, iota b = beta;
    On_compare_spec : forall a b:A,
        match compareA a b with
          Datatypes.Lt => Schutte_basics.lt (iota a) (iota b)
        | Datatypes.Eq => iota a = iota b
        | Datatypes.Gt => Schutte_basics.lt (iota b) (iota a)
        end}.
\end{Coqsrc}



For instance, the following theorem tells that \texttt{Epsilon0}, our notation system for the segment $[0,\epsilon0)$ is a correct implementation of the theoretically defined  ordinal $\epsilon_0$
(see chapter~\ref{chap:schutte} for more details).


\begin{Coqsrc}
Instance Epsilon0_correct :
  ON_correct epsilon0 Epsilon0  (fun alpha => inject (cnf alpha)).
\end{Coqsrc}

\index{Projects}

\begin{project}
  When you have read Chapter~\ref{chap:schutte}, prove that the sum of two ordinal notations \texttt{ON\_plus} implements the addition of ordinals.
\end{project}





\section{Isomorphism of ordinal notations}


In some cases we want to show that two notation systems describe the same segment (for instance $[0,3+\omega)$ and $[0,\omega($\;). For this purpose, one may prove that the two notation systems are order-isomorphic.

\index{Coq!Type classes}
\label{types:ON-iso} 
\begin{Coqsrc}
Class  ON_Iso 
       `(OA : @ON A ltA compareA)
       `(OB : @ON B ltB  compareB)
       (f : A -> B)
       (g : B -> A):=
  {
  iso_compare: forall x y : A, 
      compareB (f x) (f y) = compareA x y;
  iso_inv1 : forall a, g (f a)= a;
  iso_inv2 : forall b, f (g b) = b}.
\end{Coqsrc}

\index{Exercises}

\begin{exercise}
Let $i$ be some natural number. Prove that the notation systems 
\texttt{Omega} and \texttt{ON\_plus (OrdFin $i$) Omega} are isomorphic.

{\it \textbf{Note:} This property reflects the equality $i+\omega=\omega$, that we will prove in larger notation systems, as well as in Schütte's model.}

This exercise is partially solved for $i=3$ (in ~\href{../src/html/hydras.OrdinalNotations.Example_3PlusOmega.html}{OrdinalNotations.Example\_3PlusOmega}).

\end{exercise}

\index{Projects}
\label{exo:ON-mult}
\begin{project}
Define in \coq{} the product of two ordinal notations $N_A$ and $N_B$.
If $A$ [resp. $B$] is the underlying type of $N_A$ [resp. $N_B$], the
product \texttt{ON\_mult $N_A$ $N_B$} is implemented over the cartesian product $B\times A$ (with the lexicographic ordering).

For instance, the
elements of the product \texttt{ON\_mult Omega (FinOrd 3)} are ordered as follows.
\[(0,0),(0,1),(0,2),(0,3),(0,4),\dots,{\color{red}(1,0),} (1,1),(1,2),\dots, {\color{red}(2,0)},(2,1),(2,2),\dots\]

Note that the elements of  \texttt{ON\_mult (FinOrd 3) Omega} are differently ordered (without limit ordinals):
\[(0,0),(1,0),(2,0),(0,1),(1,1),(2,1),(0,2),(1,2),(2,2),(0,3),\dots\]
(no limit ordinals).

Prove that \texttt{ON\_mult (FinOrd $i$) Omega} is isomorphic to
\texttt{Omega}  whilst
\texttt{Omega}  is a sub-notation of \texttt{ON\_mult Omega (FinOrd $i$)},
for any strictly positive $i$.
\end{project}

\index{Projects}
\begin{project}
\label{project:succ-limit-dec}
Add to the class \texttt{ON} the requirement that for any $\alpha$ it is decidable whether $\alpha$ is $0$, a successor or a limit ordinal.


\textbf{Hint:}   Beware of the instances associated with sum and product of notations!
  You may consider additional fields 
to make the sum and product of notations ``compositional''.

\end{project}

\index{Projects}
\begin{project}
\label{project:on-setoid}
Reconsider the  class \texttt{ON}, with an equivalence instead of Leibniz equality.
\end{project}





%%%% ICI ICI

\section{Other ordinal notations}

\index{Projects}

\begin{project}
The directory \texttt{src/OmegaOmega} contains an ad-hoc formalization of $\omega^\omega$, contributed by Pascal Manoury. Every ordinal $\alpha$ is represented by a list $l$ whose elements are the coefficients of $\omega$ in  the Cantor normal form of $\alpha$ (in reverse order). For instance, the ordinal 
$\omega^{8}\times 5 + \omega^{6}\times 8 + \omega^2\times 10 + \omega + 7$ is represented by the list \texttt{[5;0;8;0;0.0;10,1,7]}. 


 Develop this representation and compare it with the other ordinal notations.



\end{project}

\index{Projects}

\begin{project}
Let $N_A$ be a notation system for ordinals strictly less than $\alpha$, 
with the strict order $(A,<_A)$. Please build the notation system
\texttt{ON\_Expl $N_A$}, on the type of multisets of elements of $A$
(or, if preferred, the type of non-increasing finite sequences on $A$,
provided with the lexicographic ordering on lists).

For instance, let us take $N_A=\texttt{Omega}$, and take $\alpha=\langle 4,4,2,1,0\rangle$,
 $\beta=\langle 4,3,3,3,3,3,2\rangle$, and $\gamma=\langle 5\rangle$. Then $\beta<\alpha<\gamma$. 

In contrast the list $\langle5,6,3,3\rangle$ is not non-increasing (\emph{i.e.} sorted w.r.t. $\geq$), so it is not to be considered.

Note that if the notation $N_A$ implements the ordinal 
$\alpha$,  the new notation $\omega^{N_A}$ must implement the ordinal $\varphi_0(\alpha)$, a.k.a. $\omega^\alpha$ (see chapter~\ref{chap:schutte})

\end{project}



\begin{remark}
 The set of ordinal terms in Cantor normal form (see Chap.~\ref{chap:T1}) and 
in Veblen normal form (see 
\href{../src/html/hydras.Gamma0.Gamma0.html}{Gamma0.Gamma0}) are shown to be ordinal notation systems, but there is a lot of work to be done in order to unify ad-hoc  definitions and proofs which were written before the definition of the \texttt{ON} type class.
\end{remark}










%------------------------------------------------------------------------

\chapter[A proof of termination, using epsilon0]{A proof of termination, using ordinals below \texorpdfstring{$\epsilon_0$}{epsilon0}}

\label{cnf-math-def}
\label{chap:proof-term}
\label{chap:T1}

In this chapter, we adapt to \coq{} the well-known~\cite{KP82}  proof that Hercules eventually wins every battle, whichever the strategy  of each player.
In other words, we present  a formal and self contained proof of termination  of all [free] hydra battles.
First, we take from Manolios and Vroon~\cite{Manolios2005} a representation of the ordinal $\epsilon_0$ as terms in Cantor normal form. Then, we define a variant for hydra battles as a measure that maps any hydra to some ordinal strictly less than $\epsilon_0$.



\section{The ordinal \texorpdfstring{\(\epsilon_0\)}{epsilon0}}
\label{sec:epsilon0-intro}

\subsection{Cantor normal form}
\index{Maths!Cantor normal form}

The ordinal \(\epsilon_0\) is the least ordinal number that satisfies 
the equation \(\alpha = \omega^\alpha\), where \(\omega\) is 
the least infinite ordinal. Thus, we can consider \(\epsilon_0\) as an
\emph{infinite} \(\omega\)-tower.
Nevertheless, 
any ordinal strictly less that \(\epsilon_0\) 
can be finitely represented by a unique  \emph{Cantor normal form}, 
that is, an expression  which is either  the ordinal \(0\) or 
a sum  \(\omega^{\alpha_1} \times n_1 + \omega^{\alpha_2} \times n_2 + 
  \dots + \omega^{\alpha_p} \times n_p\) where all the \(\alpha_i\) 
are ordinals in Cantor  normal form, \(\alpha_1 > \alpha_2 > \alpha_p\), 
and all the \(n_i\) are positive integers.

An example of Cantor normal form is displayed in Fig \ref{fig:cnf-example}:
Note that  any ordinal of
the form \(\omega^0 \times i + 0\) is just written \(i\).

\begin{figure}[htb]
\centering
\begin{tikzpicture}[scale=2, every node/.style={transform shape}]
\node[color=blue]{$\omega^{(\omega^\omega\,+\, \omega^2 \times 8 \,+\, \omega)}+ \omega^\omega + \omega^4+ 6$};
\end{tikzpicture}
\caption{\label{fig:cnf-example}
An ordinal in Cantor normal form}
\end{figure}




In the rest of this section, we define an inductive type for representing in \texttt{Coq}
all the ordinals strictly  less than  \(\epsilon_0\), then extend some arithmetic operations
to this type, and finally prove that our representation fits well with 
the expected mathematical properties: the order we define is a well order, 
and the decomposition into Cantor normal form  is consistent 
with the implementation of the arithmetic operations of exponentiation of base \(\omega\) 
and addition.

\paragraph*{Remark}
\label{sec:orgheadline65}
Unless explicitly mentionned, the term ``ordinal" will be used instead of
``ordinal strictly less than \(\epsilon_0\)" (except in Chapter~\ref{chap:schutte} where it stands for ``countable ordinal'').



\subsection{A data type for  ordinals in Cantor normal form}
\label{sec:orgheadline72}
\label{sec:T1-inductive-def}



% Our user contribution~\cite{CantorContrib} represents 
% the set of ordinals strictly less than $\epsilon_0$ in Cantor normal form as in~\cite{Manolios2005}, and also the set
% of ordinals strictly  less than $\Gamma_0$ in Veblen normal form.


    Let us define an inductive type whose 
constructors are respectively associated
with the ways to build Cantor normal forms:

\begin{itemize}
\item the ordinal \(0\)
\item the construction \((\alpha,\, n,\,\beta)  \mapsto \omega^\alpha \times (n + 1)+ \beta \quad (n\in\mathbb{N})\)
\end{itemize}


\vspace{4pt}
\noindent\emph{From Module~\href{../src/html/hydras.Epsilon0.T1.html\#T1}{Epsilon0.T1}}

\label{types:T1}
\index{Constants!zero:T1}


\begin{Coqsrc}
Inductive T1 : Set  :=
| zero : T1
| ocons : T1 -> nat -> T1 -> T1.
\end{Coqsrc}



\subsubsection*{Remark}
\label{sec:orgheadline66}


The name \texttt{T1} we gave to this data-type  is proper to this development and refers
to a hierarchy of ordinal notations. For instance, in~\cite{CantorContrib}, the following type is used to represent ordinals strictly less than \(\Gamma_0\),  in Veblen normal form (see also~\cite{schutte}).

\label{types:T2}
\emph{Please look also at the library
\href{../src/html/hydras.Gamma0.T2.html}{Gamma0.T2.html}}.


\begin{Coqsrc}
Inductive T2 : Set :=
  zero : T2
| gcons : T2 -> T2 -> nat -> T2 -> T2.
\end{Coqsrc}

\subsubsection{Example}

\label{alpha0-def}
For instance, the ordinal  $\omega^\omega+\omega^3\times 5+2$ is represented by the following term:

\begin{Coqsrc}
Example alpha_0 : T1 :=
  ocons (ocons (ocons zero 0 zero)
               0
               zero)
        0
       (ocons (ocons zero 2 zero)
              4
              (ocons zero 1 zero)).
\end{Coqsrc}


\begin{figure}[htb]
\centering
\begin{tikzpicture}[very thick, scale=0.5, level 1/.style={sibling distance=6cm},
level 2/.style={sibling distance=35mm},  
level 3/.style={sibling distance=17mm}]
\node  {ocons}
  child {  node {ocons}
            child { node {ocons} child {node {zero}} child {node{0}} child{node{zero}}}
         child {node {0}}
         child {node {zero}}}
    child {node {0}}
   child {node {ocons} 
 child { node {ocons} child {node {zero}} child {node{2}} child{node{zero}}}
  child {node {4}}
         child {node {ocons} child {node {zero}} child {node{1}} child{node{zero}}}};

\end{tikzpicture}

\caption{The tree-like representation of the ordinal $\omega^\omega+\omega^3\times 5 +2$\label{fig:cnf-tree}}

\end{figure}



\paragraph{Remark}
For simplicity's sake, we chosed to forbid  expressions of the form $\omega^\alpha\times 0 + \beta$. Thus, the contruction (\texttt{ocons $\alpha$ $n$ $\beta$}) is intented to represent the
ordinal $\omega^\alpha\times(n+1)+\beta$ and not $\omega^\alpha\times n+\beta$.
In a future version, we should replace  the type \texttt{nat} with \texttt{positive} in \texttt{T1}'s 
definition. But this replacement would take a lot of time \dots{}

\subsection{Abbreviations}

Some abbreviations may help to write more concisely complex ordinal terms.

\subsubsection{Finite ordinals}
\label{sec:orgheadline67}

For representing finite ordinals, \emph{i.e.} natural numbers, we first introduce a notation for terms of the form $n+1$, then define a coercion from type \texttt{nat} into \texttt{T1}.
\label{sect:notation-FS}

\begin{Coqsrc}
Notation "'FS' n" :=
     (ocons zero n zero) (at level 10) : t1_scope.
\end{Coqsrc}

\label{sect:notation-F}

\begin{Coqsrc}
Definition fin (n:nat) : T1 := 
    match n with 0 => zero | S p => FS p end. 

Coercion fin  : nat >-> T1.

Example ten : T1 := 10.   
\end{Coqsrc}

% \index{Coq!Techniques!Coercions}
% \index{Functions!Coercions@Coercions (from nat to ordinal types)}
% \begin{remark}
% Please refer to the remark~\pageref{warning:coercions} about the use of coercions.
% % The use of coercions like \texttt{fin} allow us to be close to the mathematical tradition where natural numbers are ordinals too.
% % Nevertheless, it may happen that a goal like \texttt{3 < 5} could be 
% % interpreted as \texttt{(lt (fin 3) (fin 5))},  depending on the current notation scope.  
% % When this misinterpretation happens, tactics like \texttt{auto with arith}, \texttt{lia} do not work!
% % Thus, it is useful to write \texttt{(3 < 5)\%nat}  an inequality between two natural numbers. 
% \end{remark}


\subsubsection{The ordinal \(\omega\)}
\label{sec:orgheadline68}

  Since \(\omega\)'s Cantor normal form is
i.e. \(\omega^{\omega^0}\times 1+ 0\), we can define the following abbreviation:

\label{sect:omega-notation2}
\begin{Coqsrc}
Notation omega := (ocons (ocons zero 0 zero) 0 zero): t1_scope.
\end{Coqsrc}

Note that \texttt{omega} is not an identifier, thus any tactic like \texttt{unfold omega} would fail.


\subsubsection{The ordinal \(\omega^\alpha\), a.k.a. \(\varphi_0(\alpha)\)}
\label{sec:orgheadline70}

We provide also a notation for ordinals of the form $\omega^\alpha$.

\label{sect:notation-phi0}

\index{Notations!phi0@phi0 (exponential of base omega)}

\begin{Coqsrc}
Notation "'phi0' alpha" := (ocons alpha 0 zero) (at level 29) : t1_scope.
\end{Coqsrc}

\index{Maths!Additive principal ordinals}

\begin{remark}
\label{sec:orgheadline69}
The name \(\varphi_0\)
   comes from ordinal numbers theory. In~\cite{schutte}, Schütte defines 
$\varphi_0$  as the ordering (\emph{i.e.} enumerating) function of the set  of \emph{additive principal ordinals} \emph{i.e.} strictly positive ordinals $\alpha$ that verify $\forall \beta<\alpha, \beta+\alpha=\alpha$. For Schütte,  $\omega^\alpha$ is just a notation for $\varphi_0(\alpha)$.  See also Chapter~\ref{chap:schutte} of this document.
\end{remark}



  
\subsubsection{The hierarchy of \(\omega\)-towers:}
\label{sec:orgheadline71}

The ordinal $\epsilon_0$, although not represented by a finite term in Cantor normal form, is approximed by the sequence of $\omega$-towers (see also Sect~\vref{sect:epsilon0-as-limit} ).

\vspace{4pt}
\emph{From Module~\href{../src/html/hydras.Epsilon0.T1.html}{Epsilon0.T1}}

\begin{Coqsrc}
Fixpoint omega_tower (height:nat) : T1 := 
 match height with 
 | 0 =>  1 
 | S h => phi0 (omega_tower h)
 end.
\end{Coqsrc}

For instance, Figure~\ref{fig:tower7} represents  the ordinal returned by the
 evaluation of the term \texttt{omega\_tower 7}.

\begin{figure}[htb]
\centering
\begin{tikzpicture}[scale=2, every node/.style={transform shape}]
\node[color=blue]{$\omega^{{{\omega}^{{{\omega}}^{{{\omega}}^{{\omega^{{\omega}^{\omega}}}}}}}}$};
\end{tikzpicture}
\caption{\label{fig:tower7}
The $\omega$-tower of height 7}
\end{figure}



\subsection{Comparison between ordinal terms}
\label{sec:orgheadline73}


% Our formalisation of Cantor Normal Form will take two steps:
% 1 Definition of a strict order \texttt{o<} on the type \texttt{T1}, 
% 2 Using \texttt{o<} for characterizing terms in normal form.

In order to compare two terms of type \texttt{T1}, we define a recursive function \texttt{compare} that maps two ordinals $\alpha$ and $\beta$ to a value of type \texttt{comparison}. This type is defined in \coq's standard library 
\texttt{Init.Datatypes} and
contains three constructors:  \texttt{Lt} (less than), \texttt{Eq} (equal), and
\texttt{Gt} (greater than).


\vspace{4pt}
\emph{From Module~\href{../src/html/hydras.Epsilon0.T1.html\#compare}{Epsilon0.T1}}


\begin{Coqsrc}
Fixpoint compare (alpha alpha':T1):comparison :=
  match alpha, alpha' with
    zero, zero => Eq
  | zero, ocons a' n' b' => Lt
  | _   , zero => Gt
  | (ocons a n b),(ocons a' n' b') =>
      (match compare a a' with 
          | Lt => Lt
          | Gt => Gt
          | Eq => (match lt_eq_lt_dec n n'
                   with
                       inleft  (left _) => Lt
                     | inright _ => Gt
                     |   _ => compare b b'
                   end)
       end)
  end.
\end{Coqsrc}
 
It is now easy to define the boolean predicate \texttt{lt\_b $\alpha$ $\beta$}: 
`` $\alpha$ is strictly less than $\beta$ ''. By coercion to sort \texttt{Prop} we define also the predicate \texttt{lt}.

\vspace{4pt}
\emph{From Module~\href{../src/html/hydras.Epsilon0.T1.html}{Epsilon0.T1}}


\begin{Coqsrc}
Definition lt_b alpha beta : bool :=
  match compare alpha beta with
      Lt => true
    | _ => false
  end.

Definition lt alpha beta : Prop := lt_b alpha beta.
\end{Coqsrc}

\label{Predicates:lt-T1}
Please note that this definition of \texttt{lt} makes it easy to write proofs by reflection, as shown by the following exampgles.

\begin{Coqsrc}
Example E1 : lt (ocons omega 56 zero) (tower 3).
Proof. reflexivity. Qed.

Example E2 : ~ lt (tower 3) (tower 3).
Proof.  discriminate.  Qed.
\end{Coqsrc}

The following lemmas establish relations between \texttt{compare}, 
the predicate \texttt{lt} and Leibniz equality \texttt{eq}.

\vspace{4pt}
\emph{From Module~\href{../src/html/hydras.Epsilon0.T1.html\#compare_refl}{Epsilon0.T1}}


\begin{Coqsrc}
Lemma compare_refl : forall alpha, compare alpha alpha =  Eq.
\end{Coqsrc}

\begin{Coqsrc}
Lemma compare_reflect : forall alpha beta,
    match compare alpha beta with
    |   Lt => lt alpha  beta
    |   Eq => alpha = beta
    |   Gt => lt beta  alpha
    end.
\end{Coqsrc}

We prove also that the relation \texttt{lt} is a strict total order.

\vspace{4pt}
\emph{From Module~\href{../src/html/hydras.Epsilon0.T1.html\#lt_irrefl}{Epsilon0.T1}}

  
\begin{Coqsrc}
Theorem lt_irrefl (alpha: T1):  ~ lt alpha alpha.

Theorem lt_trans (alpha beta gamma : T1) :
  lt alpha  beta -> lt beta gamma -> lt alpha gamma.

Definition lt_eq_lt_dec  :
   forall alpha beta : T1, 
          {lt alpha  beta} + {alpha = beta} + {lt beta alpha}.
\end{Coqsrc}


Note that the order \texttt{lt} is not reflected 
in the structure (size and/or height) of the terms of \texttt{T1}. 
For instance the ordinal of Fig \ref{fig:cnf-example} is strictly less
than the structurally simpler \(\omega^{\omega^\omega}\times 2\).

\subsubsection{A predicate for characterizing normal forms}
\label{sect:t1-nf}

\label{sec:orgheadline74}
\label{sec:orgheadline75}
Our data-type \texttt{T1} allows us to write expressions that
are not properly in Cantor normal form as specified in Section \ref{sec:epsilon0-intro}.
For instance, consider the following term of type  \texttt{T1}. 

\begin{Coqbad}
Example bad_term  : T1 := ocons 1 1 (ocons omega 2 zero).
\end{Coqbad}

This term would have been written \(\omega^1\times 2 + \omega^\omega \times 3\) in the usual mathematical notation. We note that the exponents of $\omega$ are not in the right (strictly decreasing) order.

With the help of the order \texttt{lt} on \texttt{T1}, we are now able to characterize
the set of all well-formed ordinal terms:


\vspace{4pt}
\emph{From Module~\href{../src/html/hydras.Epsilon0.T1.html\#nf_b}{Epsilon0.T1}}

\label{Predicates:nf-T1}

\begin{Coqsrc}
Fixpoint nf_b (alpha : T1) : bool :=
  match alpha with
    | zero => true
    | ocons a n zero => nf_b a
    | ocons a n ((ocons a' n' b') as b) =>
      (nf_b a && nf_b b && lt_b a' a)%bool
  end. 

Definition nf alpha: Prop := nf_b alpha.
\end{Coqsrc}

\begin{Coqsrc}
 Compute nf_b alpha_0.
\end{Coqsrc}

\begin{Coqanswer}
   = true 
     : bool
\end{Coqanswer}

\begin{Coqsrc}
 Compute nf_b bad_term.
\end{Coqsrc}

\begin{Coqanswer}
   = false 
     : bool
\end{Coqanswer}



\subsection{Making normality implicit}
  We would like to get rid of terms of type \texttt{T1} which are not in Cantor normal form.
A simple way to do this is to consider statements of the form 
\texttt{forall alpha: T1, nf alpha -> $P$ alpha}, where $P$ is a predicate over type \texttt{T1}, like in the following lemma \footnote{Ordinal addition is formally defined a little later (page~\ref{sect:infix-plus-T1})}.

\begin{Coqsrc}
Lemma plus_is_zero alpha beta :
  nf alpha -> nf beta ->
  alpha + beta  = zero -> alpha = zero /\  beta = zero.
\end{Coqsrc}

But this style leads to clumsy statements, and generates too many sub-goals in interactive proofs (although often solved with \texttt{auto} or \texttt{eauto}).

One may encapsulate conditions of the form \texttt{(nf $\alpha$)} in
the most used predicates. For instance, we introduce the restriction of \texttt{lt} to terms in normal form, and provide a handy notation for this restriction.

\vspace{4pt}
\emph{From Module~\href{../src/html/hydras.Prelude.Restriction.html}{Ordinals.Prelude.Restriction}}

\begin{Coqsrc}
Definition restrict {A:Type}(E: Ensemble A)(R: relation A) :=
    fun a b => E a /\ R a b /\ E b.
 \end{Coqsrc}

 
\vspace{4pt}
\emph{From Module~\href{../src/html/hydras.Epsilon0.T1.html\#LT}{Epsilon0.T1}}

\begin{Coqsrc}
Definition LT := restrict nf lt.
Infix "t1<" := LT : t1_scope.

Definition LE := restrict nf le.
Infix "t1<=" := LE : t1_scope.
\end{Coqsrc}


\label{Predicates:LT-T1}
 

For instance, in the following lemma, the condition that $\alpha$ is in normal form is included in the condition $\alpha< 1$.

\begin{Coqsrc}
Lemma LT_one : forall alpha, alpha t1< one -> alpha = zero.
\end{Coqsrc}

  
\subsubsection{A sigma type for \texorpdfstring{$\epsilon_0$}{epsilon0}}

As we noticed in Sect.~\ref{sect:t1-nf}, the type \texttt{T1} is not a correct ordinal notation, since it contains terms that are not in Cantor normal form. In certain contexts (for instance in Sections~\ref{sect:L-equations}, \ref{sect:hardy},
and \ref{sect:wainer}),  we need to define total recursive functions on well-formed ordinal terms less  than $\epsilon_0$, using the \texttt{Equations} plug-in~\cite{sozeau:hal-01671777}.
 In order to define a type whose inhabitants represent just ordinals, we build a type gathering a term of type \texttt{T1} and a proof that this term is in normal form.
 

\label{sect:E0-def}
\label{types:E0}

\emph{From Module~\href{../src/html/hydras.Epsilon0.E0.html}{Epsilon0.E0}}

\index{Coq!Type classes}

\begin{Coqsrc}
Class E0 : Type := t1_2o {cnf : T1; cnf_ok : nf cnf}.
\end{Coqsrc}

Many constructs : types, predicates, functions, notations, etc., on type \texttt{T1} are adapted to \texttt{E0}.

First, we declare a notation scope for \texttt{E0}.

\begin{Coqsrc}
Declare Scope E0_scope.
Delimit Scope E0_scope with e0.
Open Scope E0_scope.
\end{Coqsrc}

Then we redefine the predicates of comparison.

\label{Predicates:Lt-E0}

\begin{Coqsrc}
Definition Lt (alpha beta : E0) := T1.LT (@cnf alpha) (@cnf beta).
Definition Le (alpha beta : E0) := T1.LE (@cnf alpha) (@cnf beta).

Infix "o<" := Lt : E0_scope.
Infix "o<=" := Le : E0_scope.
\end{Coqsrc}
  

Equality in \texttt{E0} is just Leibniz equality. Note that, since \texttt{nf} is
defined by a Boolean function, for  any term $\alpha:\texttt{T1}$, there exists at most one proof of \texttt{nf $\alpha$}, thus two ordinals of type \texttt{E0} are
equal if and only iff their projection to \texttt{T1} are equal (see also Sect.~\vref{sect:eq-proof-unicity}).

\index{Coq!Techniques!Unicity of equality proofs}

\begin{Coqsrc}
Require Import Logic.Eqdep_dec.

Lemma nf_proof_unicity :
  forall (alpha:T1) (H H': nf alpha), H = H'.

Lemma E0_eq_iff alpha beta : alpha = beta <-> cnf alpha = cnf beta.
\end{Coqsrc}

\index{Exercises}
\begin{exercise}
In earlier versions of this development, the predicate \texttt{nf} was defined  inductively, with various constructors describing all possible cases.
\begin{enumerate}
\item Please give such a definition, in a dedicated module.
\item Prove the logical equivalence between your definition and ours.
\item Define a variant of the type \texttt{E0} (with your definition of \texttt{nf}).
\item Can you still prove a lemma like \texttt{E0\_eq\_iff} ? With the help of an axiom from some module of the standard library ?
\end{enumerate}
\end{exercise}
For upgrading constants and fonctions of \texttt{T1}, we have to prove that 
the term they build is in normal form.
For instance, let us represent the ordinals $0$ and $\omega$ as instances of the class \texttt{E0}.

\label{sect:omega-T1}
\index{Constants!zero:T1}

\begin{Coqsrc}
Instance Zero : E0.
Proof.
  now exists T1.zero.
Defined.

Instance _Omega : E0.
Proof.  now exists omega%t1. 
Defined.

Notation "'omega'"  := _Omega : E0_scope.
\end{Coqsrc}



\subsection{Syntactic definition of limit and successor ordinals}

Pattern matching and structural recursion allow us to define the notions of successor and limit ordinal with the help of  boolean functions on type \texttt{T1}. 

 \vspace{4pt}
\emph{From Module~\href{../src/html/hydras.Epsilon0.T1.html\#succb}{Epsilon0.T1}}

\begin{Coqsrc}
  Fixpoint succb alpha :=
  match alpha with
      zero => false
    | ocons zero _ _ => true
    | ocons alpha n beta => succb beta
  end.

Fixpoint limitb alpha :=
  match alpha with
      zero => false
    | ocons zero _ _ => false
    | ocons alpha n zero => true
    | ocons alpha n beta => limitb beta
  end.
\end{Coqsrc}



\begin{Coqsrc}
  Compute limitb omega.
\end{Coqsrc}

\begin{Coqanswer}
  = true
     : bool
\end{Coqanswer}

\begin{Coqsrc}
Compute succb 42.
\end{Coqsrc}

\begin{Coqanswer}
  = true
     : bool
   \end{Coqanswer}

The correctness of these definitions with respect to the mathematical notions of
limit and successor ordinals is established through several lemmas. For instance,
Lemma \texttt{canonS\_limit}, page~\pageref{lemma:canonS-limit}, shows that
if $\alpha$ is (syntactically) a limit ordinal, then it is the least upper bound of
a strictly increasing sequence of ordinals.


   The following function is very useful in constructions by cases (proofs and function definitions).
   
\begin{Coqsrc}
Definition zero_succ_limit (alpha: T1) :
    {succb alpha} + {limitb alpha} +  {alpha=zero}.
    (* ... *)
Defined.
\end{Coqsrc}



\subsection{Arithmetic on \texorpdfstring{$\epsilon_0$}{epsilon0}}
\subsubsection{Successor}

The successor of any ordinal $\alpha< \epsilon_0$ is defined by structural 
recursion on its Cantor normal form.

\label{Functions:succ-T1}

\vspace{4pt}
\emph{From Module~\href{../src/html/hydras.Epsilon0.T1.html\#succ}{Epsilon0.T1}}

\begin{Coqsrc}
Fixpoint succ (alpha:T1) : T1 :=
  match alpha with 
   | zero => 1
   | ocons zero n _ => ocons zero (S n) zero
   | ocons beta n gamma => ocons beta n (succ gamma)
 end.
\end{Coqsrc}


The following lemma establishes the connection between the  function
\texttt{succ} and the Boolean predicate \texttt{succb}.


\begin{Coqsrc}
 Lemma succb_iff alpha (Halpha : nf alpha) :
  succb alpha <-> exists beta : T1, nf beta /\ alpha = succ  beta.
\end{Coqsrc}

\index{Exercises}
 \begin{exercise}
Prove in \coq{} that for any ordinal $\alpha<\epsilon_0$, $\alpha$ is a limit if 
and only if for all $\beta<\alpha$, the interval $[\beta,\alpha)$ is infinite.
 \end{exercise}


\subsubsection{Addition and multiplication}

Ordinal addition and multiplication are also defined by structural recursion over the type \texttt{T1}. Please note that they use the \texttt{compare} function on some subterms of their arguments.

\label{sect:infix-plus-T1}

\begin{Coqsrc}
Fixpoint plus (alpha beta : T1) : T1 :=
  match alpha,beta with
 |  zero, y  => y
 |  x, zero  => x
 |  ocons a n b, ocons a' n' b' =>
    (match compare a a' with
     | Lt => ocons a' n' b'
     | Gt => (ocons a n (plus b (ocons a' n' b')))
     | Eq  => (ocons a (S(n+n')) b')
     end)
  end
where "alpha + beta" := (plus alpha beta) : t1_scope.
\end{Coqsrc}

\begin{Coqsrc}
Fixpoint mult (alpha beta : T1) :T1 :=
  match alpha,beta with
 |  zero, y  => zero
 |  x, zero => zero
 |  ocons zero n _, ocons zero n' _ => 
                 ocons zero (Peano.pred((S n) * (S n'))) zero
 |  ocons a n b, ocons zero n' b' =>  
                 ocons a (Peano.pred((S n) * (S n'))) b
 |  ocons a n b, ocons a' n' b' =>
     ocons (a + a') n' ((ocons a n b) * b')
 end
where  "alpha * beta" := (mult alpha beta) : t1_scope.
\end{Coqsrc}


\subsubsection{Examples}

The following examples are instances of \emph{proofs by computation}. Please note that  addition and multiplication on \texttt{T1}
are not commutative. Moreover,  both operations fail to be strictly monotonous in their first argument.


\begin{Coqsrc}
Example e2 : 6 + omega = omega.
Proof. reflexivity. Qed.

Example e'2 : omega t1< omega + 6.
Proof. now compute. Qed.

Example e''2 : 6 * omega = omega.
Proof. reflexivity. Qed.

Example e'''2 : omega t1< omega * 6.
Proof. now compute. Qed.
\end{Coqsrc}

\begin{Coqsrc}
Lemma plus_not_monotonous_l : exists alpha beta gamma : T1,
    alpha t1< beta /\ alpha + gamma = beta + gamma.
Proof.
  exists 3, 5, omega;  now  compute.
Qed.

Lemma mult_not_monotonous :  exists alpha beta gamma : T1,
      alpha t1< beta /\ alpha * gamma = beta * gamma.
Proof.
  exists 3, 5, omega; now compute.
Qed.
\end{Coqsrc}


\subsection{Pretty printing Ordinals in Cantor normal Form}

Let us consider again the ordinal $\alpha_0$ defined in section~\vref{alpha0-def}
If we ask \coq{} to print its  normal form, we get a hardly readable term of type \texttt{T1}.

\begin{Coqsrc}
Compute alpha_0.
\end{Coqsrc}

\begin{Coqanswer}
  = ocons omega 0 (ocons (FS 2) 4 (FS 1))
     : T1
\end{Coqanswer}

The following data type defines a abstract syntax for more readable ordinals in Cantor normal form:

\label{types:ppT1}
\index{Functions!pp@ pp (pretty printing terms in Cantor normal form)}

\begin{Coqsrc}
Inductive ppT1 : Set :=
    P_fin : nat -> ppT1
  | P_add : ppT1 -> ppT1 -> ppT1
  | P_mult : ppT1 -> nat -> ppT1
  | P_exp : ppT1 -> ppT1 -> ppT1
  | P_omega : ppT1
\end{Coqsrc}

The function \texttt{pp: T1 -> ppT1} converts any closed term of type \texttt{T1} into a human-readable expression. For instance, let us convert the term \texttt{alpha\_0}.

\begin{Coqsrc}
Compute pp alpha_0.
\end{Coqsrc}

\begin{Coqanswer}
     = (omega ^ omega + omega ^ 3 * 5 + 2)%pT1
     : ppT1
\end{Coqanswer}

\subsubsection{Arithmetic on type \texttt{E0}}

 We define an addition in type \texttt{E0}, using the function{T1.plus}
and its properties.

\begin{Coqsrc}
Search   (nf (_ + _)%t1).
\end{Coqsrc}

\begin{Coqanswer}
plus_nf: forall a : T1, nf a -> forall b : T1, nf b -> nf (a + b)
\end{Coqanswer}

\begin{Coqsrc}
Instance plus (alpha beta : E0) : E0.
Proof.
  refine (@mkord (T1.plus (@cnf alpha) (@cnf beta))_ );
    apply plus_nf; apply cnf_ok.
Defined.

Infix "+" := plus : E0_scope.

Check omega + omega.
\end{Coqsrc}

\begin{Coqanswer}
omega + omega
     : E0
\end{Coqanswer}

\index{Exercises}
 \begin{exercise}
Let $\alpha$ be an ordinal. We say that $\alpha$ is \emph{infinite} iff the segment $[0,\alpha)$ is an infinite set. 
\begin{enumerate}
\item Adapt this definition to the type \texttt{E0}.
\item Prove that \emph{being infinite} is decidable on type \texttt{E0}.
\item Prove that $\alpha$ is infinite if and only if for all finite ordinal $i$,
$i+\alpha=\alpha$.
\end{enumerate}
 \end{exercise}

\section{Well-foundedness and transfinite induction}

\index{Maths!Transfinite induction}

\subsection{About  well-foundedness}
\label{sec:orgheadline82}
   In order to use \texttt{T1} for proving termination results,
we need to prove that  our order \texttt{<} is well-founded. Then we will get \emph{transfinite induction} for free.


The proof of well-foundedness of the strict order $<$ on Cantor normal forms is already 
available in the Cantor contribution by Castéran and Contejean~\cite{CantorContrib}. That proof relies on a library on recursive path orderings written by
E. Contejean. We present here  a direct proof of the same result, which does not require any knowledge on r.p.o.s.

\index{Exercises}

\begin{exercise}
Prove that the \emph{total} order \texttt{lt} on \texttt{T1} is not well-founded. 
\textbf{Hint:}  You will have to build a counter-example with terms of type \texttt{T1}
which are not in Cantor normal form.
\end{exercise}

% \subsubsection{The total order \texttt{lt} on \texttt{T1} is \emph{not} well-founded}
% \label{sec:orgheadline76}

% Let us recall that the data type \texttt{T1} contains too many inhabitants, including
% terms which are not in Cantor normal form. Thus, the following result is not 
% very surprising.

% \begin{Coqsrc}
% Section lt_not_well_founded.
  
%   (* let us build the sequence of terms :
%         omega + omega + .... + omega ^ 2   *)
%   Let f := (fix f (i:nat): T1 :=
%             match i with 0 => phi0 2
%                        | S i => ocons 1 1 (f  i)
%             end).

 
%  Lemma  f_decreases : forall i, f (S i) <  f i.
%  Proof.
%   induction i; compute; auto with T1.
%  Qed.

%  Theorem lt_not_wf : ~  well_founded lt.
%  Proof. 
%    intro wf; case (not_decreasing _ lt);auto.
%    exists f; apply f_decreases.
%  Qed.

% End lt_not_well_founded.
% \end{Coqsrc}

% Thus, we have 

\subsubsection{A first attempt}
\label{sec:orgheadline77}
\index{Coq!Techniques!Well founded relations}

It is natural to try to prove by structural induction over \texttt{T1} 
that every term in normal form is accessible through \texttt{LT}.

Unfortunately, it won't work. Let us consider some well-formed term
 $\alpha=\texttt{ocons $\beta\;n\;\gamma$}$, and assume that \(\beta\) and \(\gamma\) are accessible
 through \texttt{LT}. For proving the accessibility of $\alpha$, we have to consider
any well formed term \(\delta\) such that \(\delta<\alpha\). 
But nothing guarantees that \(\delta\)  is strictly  less than \(\beta\) nor \(\gamma\), and we cannot use the induction hypotheses on   \(\beta\) nor \(\gamma\).

\begin{Coqbad}
Section First_attempt.

 Lemma wf_LT : forall alpha,  nf alpha -> Acc LT alpha. 
 Proof.
  induction alpha as [| beta IHbeta n gamma IHgamma].
  - split.
    inversion 1.
    destruct H2 as [H3 _];not_neg H3.
  -  split; intros delta Hdelta.
\end{Coqbad}

\begin{Coqanswer}
1 subgoal (ID 560)
  
  beta : T1
  n : nat
  gamma : T1
  IHbeta : nf beta -> Acc LT beta
  IHgamma : nf gamma -> Acc LT gamma
  H : nf (ocons beta n gamma)
  delta : T1
  Hdelta : delta t1< ocons beta n gamma
  ============================
  Acc LT delta
 \end{Coqanswer}

\begin{Coqbad}
  Abort.
\end{Coqbad}

The problem comes from the hypothesis \texttt{Hdelta}. It does not prevent  \(\delta\) to be bigger that \(\beta\) or
\(\gamma\);
for instance \(\delta\) may be of the form
\texttt{ocons $\beta'$ $p'$  $\gamma'$},
where  \(\beta' \leq  \beta\) and  \(p' < n\).
Thus, the induction hypotheses \texttt{IHbeta} and \texttt{IHgamma}  are useless for finishing our proof.

\subsubsection{Using a stronger inductive predicate.}
\label{sec:orgheadline78}
  Instead of trying to prove directly that any ordinal term \(\alpha\) in normal form is accessible
through \texttt{LT}, we propose to show first that any well formed 
term of the form \(\omega^\alpha\times(n+1)+\beta\) is accessible (which is a stronger result).

\begin{Coqsrc}
 Let Acc_strong (alpha:T1) :=
      forall n beta, 
        nf (ocons alpha n beta) -> Acc LT (ocons alpha  n beta).
\end{Coqsrc}

The following lemma is an application of the strict inequality 
\showmath{\alpha < \omega ^\alpha}. If \showmath{\omega^\alpha} is accessible,
then \showmath{\alpha} is \emph{a fortiori} accessible.

\begin{Coqsrc}
 Lemma Acc_strong_stronger : forall alpha, 
     nf alpha -> Acc_strong  alpha -> Acc LT  alpha.
 Proof.
  intros alpha H H0; apply acc_imp with (phi0 alpha).
  - repeat split; trivial.
    + now apply lt_a_phi0_a.
  -  apply H0;  now apply single_nf.
Qed.
\end{Coqsrc}

Thus, it remains to prove that every ordinal strictly less than \showmath{\epsilon_0} 
is strongly accessible.

% \subsubsection{Structure of the proof of well-foundedness of \texttt{LT}}

\label{sec:orgheadline81}
\label{proof-wf-epsilon0}
\paragraph{A helper}
\label{sec:orgheadline79}

First, we prove that, for  any \texttt{LT}-accessible term \showmath{\alpha}, \showmath{\alpha} is 
strongly accessible too (\emph{i.e.} any well formed
term (\texttt{ocons $\alpha$ $n$ $\beta$})  is accessible).

\begin{Coqsrc}
Lemma Acc_implies_Acc_strong : 
   forall alpha, Acc LT  alpha -> Acc_strong alpha.
\end{Coqsrc}


The proof is structured as an induction on \showmath{\alpha}'s accessibility. Let us consider
an accessible term $\alpha$.



\begin{Coqanswer}
  subgoal 1 

  alpha : T1
  Aalpha : forall y : T1,  y t1< alpha -> Acc LT y
  IHalpha : forall y : T1,
       LT y alpha ->
       forall (n : nat) (beta : T1),
       nf (ocons y n beta) -> Acc LT (ocons y n beta)
  ============================
   forall (n : nat) (beta : T1),
   nf (ocons alpha n beta) -> Acc LT (ocons alpha n beta)
\end{Coqanswer}

Let \texttt{n:nat} and \texttt{beta:T1} such that \texttt{ocons alpha n beta} is in normal form. 
We prove first that \texttt{beta} is accessible,  which allows us to by well-founded induction on \texttt{beta}, 
and natural induction on \texttt{n}, that (\texttt{ocons alpha n beta}) is accessible.
The proof, quite long, can be consulted in \url{../src/html/hydras.Epsilon0.T1.html} 

\paragraph{Accessibility of any well-formed ordinal term}
\label{sec:orgheadline80}

Our goal is still to prove accessibility of any well formed ordinal term.
Thanks to our previous lemmas, we are almost done.

\begin{Coqsrc}
(* A (last) structural induction *)

Theorem nf_Acc : forall alpha, nf alpha -> Acc LT  alpha.
Proof.
 induction alpha.
-  intro; apply Acc_zero.
 -  intros; eapply Acc_implies_Acc_strong;auto.
    apply IHalpha1;eauto.
    apply nf_inv1 in H; auto. 
Defined.

Corollary T1_wf : well_founded LT.
\end{Coqsrc}

And we have now our transfinite recursor:
\index{Maths!Transfinite induction}

\begin{Coqsrc}

Definition transfinite_recursor :
 forall (P:T1 -> Type),
   (forall x:T1, 
     (forall y:T1, nf x -> nf y ->  lt y  x -> P y) -> P x) ->
    forall alpha:T1, P alpha.
Proof.
 intros; apply well_founded_induction_type with LT.
 -  exact T1_wf;auto.
 - intros. apply X. intros; apply X0. repeat split;auto. 
Defined.
\end{Coqsrc}

The following tactic starts a proof by  transfinite induction on any ordinal \mathcolor{$\alpha<\epsilon_0$}.

\begin{Coqsrc}
Ltac transfinite_induction alpha :=
  pattern alpha; apply transfinite_recursor;[ | try assumption].
\end{Coqsrc}


\begin{remark}
The alternate proof of well-foundedness using \'Evelyne Contejean's work on recursive path ordering is available in the library \href{../src/html/hydras.Epsilon0.Epsilon0rpo.html}{Epsilon0rpo}.
 \end{remark}


\subsection{An ordinal notation for  \texorpdfstring{$\epsilon_0$}{epsilon0}}

We build an instance of \texttt{ON}, and prove its correction w.r.t. Schutte's model.

\begin{Coqsrc}
Instance Epsilon0 : ON Lt compare.  
(* ... *)
\end{Coqsrc}


\emph{From Module~\href{../src/html/hydras.Schutte.Schutte.Correctness_E0.html}{Schutte.Schutte.Correctness\_E0}}

\begin{Coqsrc}
Instance Epsilon0_correct :
  ON_correct epsilon0 Epsilon0 (fun alpha => inject (cnf alpha)).
\end{Coqsrc}

\index{Projects}
\begin{project}
 \emph{This exercise is a continuation of Project~\vref{exo:ON-mult}.}
Define an ordinal notation \texttt{Omega2} for $\omega^2=\omega\times\omega$ (using
\texttt{ON\_mult}). 

Prove that \texttt{Omega2} is a sub-notation of \texttt{Epsilon0}.

Define on \texttt{Omega2} an addition compatible with the addition on \texttt{Epsilon0}.

\textbf{Hint}. You may use the following definition (in 
\href{../src/html/hydras.OrdinalNotations.Definitions.html}{OrdinalNotations.Definitions}).

\begin{Coqsrc}
Definition SubON_same_op  `{OA : @ON A ltA  compareA}
       `{OB : @ON B ltB  compareB}
       {iota : A -> B} 
       {alpha: B}
       {_ : SubON OA OB alpha iota}
       (f : A -> A -> A)
       (g : B -> B -> B)
  :=
  forall x y,  iota (f x y) = g (iota x) (iota y).
\end{Coqsrc}

\end{project}

\index{Projects}
\begin{project}
The class \texttt{ON} of ordinal notations has been defined long after this 
chapter, and is not used yet in the development of the type \texttt{E0}. 
A better integration of both notions should simplify the development on ordinals in Cantor normal form. This integration is planned for the future versions.

\end{project}


\section{A variant for hydra battles}

\subsection{Natural sum (a.k.a. Hessenberg's  sum)}
\label{sec:orgheadline87}
\label{hydra-variant}

Natural sum (Hessenberg's  sum) is a commutative and monotonous version of
addition. It is used as an auxiliary operation  for defining variants
for hydra battles, where Hercules is allowed to chop off any  head of the hydra.

In the litterature, the natural sum of ordinals \(\alpha\) and \(\beta\) 
is often denoted by \(\alpha \# \beta\)  or  \(\alpha \oplus  \beta\).
Thus we called \texttt{oplus} the associated \emph{Coq} function.

\subsubsection{Definition of \texttt{oplus}}
\label{sec:orgheadline84}
\index{Functions!oplus @ oplus (Hessenberg commutative sum)}

The definition of \texttt{oplus} is recursive in both of its 
arguments, which makes a structural recursive definition a little 
complex.
We used the same pattern as for the \texttt{merge} function on lists of library
\texttt{Coq.Sorting.Mergesort}.

\begin{enumerate}
\item Define a nested recursive function, using the \texttt{Fix} 
    construct

\item Build a principle of induction dedicated to \texttt{oplus}

\item Establish equations associated to each case of the definition.
\end{enumerate}

\paragraph{Nested recursive definition}
\label{sec:orgheadline83}

The following definition is composed of 
\begin{itemize}
\item A main function \texttt{oplus}, structurally recursive in its 
first argument \texttt{alpha}
\item An auxiliary function \texttt{oplus\_aux} within the scope of \texttt{alpha},
structurally recursive in its argument \texttt{beta};  \texttt{oplus\_aux beta} 
   is supposed to compute  \texttt{oplus alpha beta}.
\end{itemize}
  
\vspace{4pt}
\emph{From Module~\href{../src/html/hydras.Epsilon0.Hessenberg.html\#oplus}{Epsilon0.Hessenberg}}

\label{sect:infix-oplus}

\begin{Coqsrc}
Fixpoint oplus (alpha beta : T1) : T1 :=
  let fix oplus_aux beta {struct beta} :=
      match alpha, beta with
        | zero, _ => beta
        | _,  zero => alpha
        | ocons a1 n1 b1, ocons a2 n2 b2 =>
          match compare a1 a2 with
            |  Gt => ocons a1 n1 (oplus b1 beta)
            |  Lt => ocons a2 n2 (oplus_aux b2)
            |  Eq => ocons a1 (S (n1 + n2)%nat) (oplus b1 b2)
          end
      end
  in oplus_aux beta.

Infix "o+" := oplus  (at level 50, left associativity).
\end{Coqsrc}


The reader will note that each recursive call of the functions
\texttt{oplus} and \texttt{oplus\_aux} satisfies \emph{Coq}'s constraint
on recursive definitions. The function \texttt{oplus} is recursively called on a sub-term of its first argument,
and \texttt{oplus\_aux} on a sub-term of its unique argument.
Thus, \texttt{oplus}'s definition is accepted by \coq{} as a structurally recursive function.

\subsubsection{Rewriting lemmas}
\label{sec:orgheadline86}

\emph{Coq}'s constraints on recursive definitions result in 
the quite  complex form of \texttt{oplus}'s definition.
Proofs of properties of this function can be simpler if we
 \emph{derive} rewriting lemmas that will help to simplify 
expressions of the form (\texttt{oplus $a$ $ b$}).

A first set of lemmas correspond to the various cases of \texttt{oplus}'s 
definition. They can be proved almost immediately, using \texttt{cbn} 
and \texttt{reflexivity} tactics.



\begin{Coqsrc}
Lemma oplus_alpha_0 (alpha : T1) : alpha o+ zero = alpha.
Proof.
  destruct alpha; reflexivity.
Qed.

Lemma oplus_0_beta (beta : T1): zero o+ beta = beta.
Proof.
  destruct beta; reflexivity.
Qed.
\end{Coqsrc}


% \subsubsection{A hand-made induction principle}
% \label{sec:orgheadline85}

% \index{Coq!Commands!Functional Scheme}

% \emph{Coq} contains a command  \texttt{Functional Scheme} that 
% generates induction principles which correspond to recursive functions.
% Unfortunately, the current version ( \texttt{8.11.0} ) doesn't work on \texttt{oplus},
% probably because of the inner \texttt{Fix}.

% \begin{Coqsrc}
% Functional Scheme oplus_ind := Induction for oplus Sort Prop.
% \end{Coqsrc}

% \begin{Coqanswer}
% Error: Anomaly "todo." Please report at http://coq.inria.fr/bugs/.
% \end{Coqanswer}


% Fortunately, it's a good exercise for a semi-experienced user, to write
% her/him-self induction principles similar to the ones returned by
% \texttt{Functional Scheme}.

% \begin{itemize}
% \item First, we chose to write a version for sort \texttt{Type}, since versions
% for sorts \texttt{Prop} and \texttt{Set} can be easily derived from
% the former one. According to \emph{Coq}'s naming politics, we will call our 
% principle \texttt{oplus\_rect}

% \item The conclusion of \texttt{oplus\_rect} will be (\texttt{$P$ a b (oplus a b)}),
% where $P$ is an arbitrary function of type 
% \texttt{T1 -> T1 -> T1 -> Type}

% \item The premises of \texttt{oplus\_rect} will describe how to build an induction 
% on the graph of \texttt{oplus}.
% \end{itemize}

% We are now ready to state and prove \texttt{oplus\_rect}, and the reader
% will note that the statement is longer than the proof script itself,
% which is a standard proof by induction, simplification and case-analysis 
% that follows  \texttt{oplus}'s definition.

% We associate also a tactic to the application of \texttt{oplus\_rect}.

% \begin{Coqsrc}
%  Lemma oplus_rect:
%       forall P: T1 -> T1 -> T1 -> Type, 
%         (forall a:T1, P zero a a) ->
%         (forall a: T1, P a zero a) ->
%         (forall a1 n1 b1 a2 n2 b2 o,
%            compare a1 a2 = Gt ->
%            P b1 (ocons a2 n2 b2) o ->
%            P (ocons a1 n1 b1) (ocons a2 n2 b2)
%              (ocons a1 n1 o)) ->
%         (forall a1 n1 b1 a2 n2 b2 o,
%            compare a1 a2 = Lt ->
%            P (ocons a1 n1 b1) b2 o ->
%            P (ocons a1 n1 b1) (ocons a2 n2 b2) 
%            (ocons a2 n2 o)) ->
%         (forall a1 n1 b1 a2 n2 b2 o,
%            compare a1 a2 = Eq ->
%            P b1 b2 o ->
%           P (ocons a1 n1 b1) (ocons a2 n2 b2)
%             (ocons a1 (S (n1 + n2)%nat) o)) ->
%          forall a b, P a b (oplus a b).
% Proof with auto.
%    induction a.
%    -    intro; simpl; destruct b;auto.
%    -   induction b.
%        + apply X0.
%        + case_eq (compare a1 b1).
%          * intro Comp; unfold oplus; rewrite Comp.
%            cbn; apply X3 ...
%          * intro Comp; cbn; rewrite Comp; apply X2...
%          * intro Comp; cbn; rewrite Comp ...
%  Defined.


% Ltac oplus_induction a b:= pattern (oplus a b); apply oplus_rect.
% \end{Coqsrc}

% \index{Exercises}

% \begin{exercise}
% The induction principle \texttt{oplus\_rect} is still unused in our development. 
% Please build some nice examples of application.
% \end{exercise}

\index{Projects}
\begin{project}
Compare \texttt{oplus}'s definition (with inner fixpoint) with other possibilities
(\texttt{coq-equations}, \texttt{Function}, etc.).
\end{project}
\subsection{More theorems on Hessenberg's sum}

We need to prove some properties of $\oplus$, particularly about 
its relation with the order $<$ on \texttt{T1}.

\subsubsection{Boundedness}
If $\alpha$ and $\beta$ are both strictly  less than  $\omega^\gamma$, then so is their natural sum
$\alpha \oplus \beta$. This result can be proved by structural induction on $\gamma$.


\begin{Coqsrc}
Lemma lt_phi0_oplus : forall gamma alpha beta,
                        lt_phi0 alpha gamma ->
                        lt_phi0 beta gamma ->
                        lt_phi0 (alpha o+ beta) gamma.

Proof with auto.
  induction gamma; destruct alpha, beta.  
(* ... *)
\end{Coqsrc}

\subsubsection{Commutativity, associativity}

We prove  the commutativity of $\oplus$ in two steps. 

First, we prove by transfinite induction on $\alpha$ that the restriction of $\oplus$ to the
interval $[0..\alpha)$ is commutative.

\index{Maths!Transfinite induction}

\begin{Coqsrc}
Lemma oplus_comm_0 : forall alpha, nf alpha ->
     forall a b,  nf a -> nf b ->
                  lt a alpha ->
                  lt b alpha ->
                  a o+ b = b o+ a.
 Proof with eauto with T1.
    intros alpha Halph; transfinite_induction alpha.
(* rest of proof omitted *)  
\end{Coqsrc}

Then, we infer  $\oplus$'s commutativity for any pair of ordinals:
Let $\alpha$ and $\beta$ be two ordinals strictly less than $\epsilon_0$. Both ordinals $\alpha$ and $\beta$ are
strictly less than $\textrm{max}(\alpha,\beta)+1$.
    Thus, we have just to apply the lemma \coqsimple{oplus\_comm\_0}.

\begin{Coqsrc}
  Lemma oplus_comm : forall alpha beta, 
      nf alpha -> nf beta ->
      alpha o+ beta =  beta o+ alpha.
  Proof with eauto with T1.
    intros alpha beta Halpha Hbeta;
    apply oplus_comm_0 with (succ (max alpha beta)) ...  
  (* ... *)
\end{Coqsrc}

The associativity of Hessenberg's sum is proved the same way.


\begin{Coqsrc}
 Lemma oplus_assoc_0 :
    forall alpha,
      nf alpha ->
      forall a b c,  nf a -> nf b -> nf c ->
                      lt a alpha ->
                      lt b alpha -> lt c alpha ->
                      a o+ (b o+ c) = (a o+ b) o+ c.
  Proof with eauto with T1.
    intros alpha Halpha.
    transfinite_induction alpha.
    (* ... *)
\end{Coqsrc}


\begin{Coqsrc}
 Lemma oplus_assoc : forall alpha beta gamma,
                        nf alpha -> nf beta -> nf gamma ->
                                    alpha o+ (beta o+ gamma) =
                                    alpha o+ beta o+ gamma.
 Proof with eauto with T1.
    intros;
    apply oplus_assoc_0 with (succ (max alpha (max beta gamma))) ...
    (* ... *)   
\end{Coqsrc}


\subsubsection{Monotonicity}

At last, we prove that $\oplus$ is strictly monotonous in both of its arguments.

\begin{Coqsrc}
Lemma oplus_strict_mono_LT_l (alpha beta gamma : T1) :
  nf gamma   -> alpha  t1< beta ->
  alpha o+ gamma  t1< beta o+ gamma.

Lemma oplus_strict_mono_LT_r (alpha beta gamma : T1) :
  nf alpha -> beta t1< gamma ->
  alpha o+ beta t1< alpha o+ gamma.
\end{Coqsrc}

\index{Projects}

\begin{project}
The library \texttt{Hessenberg} looks too long (proof scripts and compilation).
Please try to make it simpler and more efficient!
Thanks!
\end{project}

\subsection{A measure for hydra-battle termination}

\label{sec:hydra-measure}

Let us define a measure from type \texttt{Hydra} into \texttt{T1}.


\vspace{4pt}
\emph{From Module~\href{../src/html/hydras.Hydra.Hydra_Termination.html\#m}{Hydra.Hydra\_Termination}}

\begin{Coqsrc}
Fixpoint m (h:Hydra) : T1 :=
  match h with head => zero
             | node hs => ms hs
end 
with ms (s:Hydrae) :  T1 :=
  match s with  hnil => zero
              | hcons h s' => phi0 (m h) o+  ms s'
 end.  
\end{Coqsrc}

First, we prove that the measure $m(h)$  of any hydra $h$ is a well-formed ordinal term of type \texttt{T1}.

\begin{Coqsrc}
Lemma m_nf : forall h, nf (m h).
Proof.
 intro h; elim h using Hydra_rect2 
            with (P0 := fun s =>  nf (ms s)).
 (* ... *)

Lemma ms_nf : forall s, nf (ms s).
Proof with auto with T1.
(* ... *)
\end{Coqsrc}

For proving the termination of all hydra battles, we have to prove that
\texttt{m} is a variant. First, a few technical lemmas follow the decomposition of \texttt{round} into several relations. Then the lemma \texttt{round\_decr} gathers all the cases.

\begin{Coqsrc}
Lemma S0_decr :
  forall s s', S0  s s' -> ms s' t1< ms s.
\end{Coqsrc}

\begin{Coqsrc}
Lemma R1_decr : forall h h',
                  R1 h h' -> m h' t1< m h.
\end{Coqsrc}

\begin{Coqsrc}
Lemma S1_decr n:
  forall s s', S1 n s s' -> ms s' t1<  ms s.
\end{Coqsrc}

\begin{Coqsrc}
Lemma R2_decr n : forall h h', R2 n h h' -> m h'  t1< m h.
\end{Coqsrc}


\begin{Coqsrc}
Lemma round_decr : forall h h', h -1-> h' -> m h' t1< m h.
Proof.
   destruct 1 as [n [H | H]].
   -  now apply R1_decr.
   -  now apply R2_decr with n.
Qed.
\end{Coqsrc}

Finally, we prove the termination of all (free) battles.

\label{thm:every-battle-terminates}

\begin{Coqsrc}
Global Instance HVariant : Hvariant lt_wf free var.
Proof.
 split; intros; eapply round_decr; eauto.
Qed.

Theorem every_battle_terminates: Termination.
Proof. 
  red; apply Inclusion.wf_incl with 
         (R2 := fun h h' =>  m h t1< m h').
   red; intros;  now apply round_decr.
   apply Inverse_Image.wf_inverse_image, T1_wf.
Qed.
\end{Coqsrc}


\section*{Conclusion}

Let us recall three results we have proved so far.
\begin{itemize}
\item There exists a strictly decreasing variant which maps \texttt{Hydra} into 
the segment $[0,\epsilon_0)$ for proving the termination of any hydra battle
\item There exists \emph{no} such variant from \texttt{Hydra} into 
$[0,\omega^2)$, \emph{a fortiori} into $[0,\omega)$.
\end{itemize}

So, a  natural question is `` Does there exist any strictly decreasing variant mapping
type \texttt{Hydra} into some interval $[0,\alpha[$ (where $\alpha <\epsilon_0$) for proving the termination of all hydra battles''. The next chapter is dedicated to a formal proof that there exists no such $\alpha$, even if we consider a restriction to the set of ``standard'' battles.






%\include{epsilon0}


%\include{impossibility-proofs}




%-------------------------------------------------------------------

\chapter[The Ketonen-Solovay machinery]{Strolling inside \texorpdfstring{$\epsilon_0$}{epsilon0}: The Ketonen-Solovay machinery\label{ks-chapter}}
\label{chap:ketonen}
\index{Maths!Ketonen-Solovay machinery}

\section{Introduction}
The reader may think that our proof of termination in the previous  chapter requires a lot of mathematical tools and may be too  complex. So, the question is ``is there  any  simpler proof'' ?

In their article~\cite{KP82}, Kirby and Paris show that this result cannot be proved in Peano arithmetic. Their proof uses some knowledge about model theory and non-standard models of Peano arithmetic. In this chapter, we focus on a specific class of proofs of termination of hydra battles: construction of some variant mapping the type \texttt{Hydra} into a given initial  segment of ordinals. Our proof relies only on the Calculus of Inductive Constructions and is a natural complement of the results proven in the previous chapter.

\begin{itemize}
\item There is no variant mapping the type \texttt{Hydra} into the interval $[0,\omega^2)$ (section ~\vref{omega2-case}), and a fortiori 
$[0,\omega)$ (section ~\vref{omega-case}).

\item There exists a variant which maps the type \texttt{Hydra} into the
interval $[0,\epsilon_0)$ (theorem \texttt{every\_battle\_terminates}, in section~\vref{thm:every-battle-terminates}).
\end{itemize}


Thus, a very natural question is the following one:
\begin{quote}
  `` Is there  any variant from
\texttt{Hydra} into some interval $[0,\mu)$, where $\mu<\epsilon_0$, for proving the termination of all hydra battles ?''
\end{quote}

We prove in \coq{} the following result:

\begin{quote}
There is no variant for proving the termination of all hydra battles
from \texttt{Hydra} into the interval $[0..\mu)$, where
$\mu< \epsilon_0$.
The same impossibility holds even if we consider only standard battles (with the successive replication factors $0,1,2,\dots,t,t+1,\dots$).
\end{quote}

Our proofs are  constructive and require no axioms: they are  closed terms of the CIC, and are mainly composed on function definitions and proofs of properties of these functions. 
They  share much theoretical material with Kirby and Paris', although they do not use any knowledge about Peano arithmetic nor model  theory.  The combinatorial arguments we use and implement
come from 
 an article by J.~Ketonen and R.~Solovay~\cite{KS81}, already  cited in the work
 by L.~Kirby et J.~Paris.% on the termination of Goodstein sequences and hydra battles~\cite{KP82}.
 Section $2$ of this article: ''A hierarchy of probably recursive functions'', contains a systematic study of \emph{canonical sequences}, which are closely related to
rounds of hydra battles. 
Nevertheless, they have the same global structure as the simple proofs described in
sections~\vref{omega-case} and \vref{omega2-case}. 
We invite the reader to compare the three proofs step by step, lemma by lemma.

\section{Canonical Sequences}
\label{ketonen-solovay-sect}
\index{Maths!Canonical sequences}

Canonical sequences are functions that associate an ordinal $\canonseq{\alpha}{i}$ to every ordinal $\alpha<\epsilon_0$ and positive integer $i$. They satisfy several nice properties:

\index{Maths!Transfinite induction}
\begin{itemize}
\item If $\alpha\not=0$, then $\canonseq{\alpha}{i}<\alpha$. Thus canonical sequences can be used for proofs by transfinite induction or function definition by transfinite recursion
\item If $\lambda$ is a limit ordinal, then $\lambda$ is the least upper bound of the set 
$\{\canonseq{\lambda}{i}\;|\,i\in\mathbb{N}_1\}$


\item If $\beta<\alpha<\epsilon_0$, then there is a ``path'' from $\alpha$ to $\beta$, \emph{i.e.} a
sequence $\alpha_0=\alpha, \alpha_1, \dots, \alpha_n=\beta$, where for every $k<n$, there exists some $i_k$ such that $\alpha_{k+1}=\canonseq{\alpha_k}{i_k}$
\item Canonical sequences correspond tightly to rounds of hydra battles: if $\alpha\not=0$,
then $\iota(\alpha)$ is transformed into $\iota(\canonseq{\alpha}{i+1})$ in one round with
the replication factor $i$ (Lemma \href{../src/html/hydras.Hydra.O2H.html\#canonS_iota_i}{Hydra.O2H.canonS\_iota\_i}).
\item From the two previous properties, we infer that whenever $\beta<\alpha<\epsilon_0$, there exists a (free) battle from $\iota(\alpha)$ to $\iota(\beta)$.
\end{itemize}

\begin{remark}
  In~\cite{KS81}, canonical sequences are defined for any ordinal $\alpha <\epsilon_0$,
by stating that if $\alpha$ is a successor ordinal $\beta+1$,  the sequence associated with 
$\alpha$ is simply the constant sequence whose terms are equal to $\beta$.
Likewise, the canonical sequence of $0$ maps any natural number to $0$.

This convention allows us to make total the function that maps any ordinal $\alpha$ and natural number $i$ to the ordinal $\canonseq{\alpha}{i}$.

\end{remark}


First, let us recall how canonical sequences are defined in~\cite{KS81}. For efficiency's sake, we decided not to implement directly K.\&S's definitions, but to define in \gallina{} simply typed structurally recursive functions which share the abstract properties which are used in the mathematical proofs\footnote{With a small difference: the $0$-th term of the canonical sequence is not the same in our development as in~\cite{KS81}.}.





\subsubsection{Mathematical definition of canonical sequences} 

In~\cite{KS81} the definition of $\canonseq{\alpha}{i}$ is based on the following remark:
\begin{quote}
Any non-zero ordinal $\alpha$ can be decomposed in a unique way as the product
$\omega^\beta\times (\gamma+1)$.
\end{quote}

Thus the $\canonseq{\alpha}{i}$\,s are defined in terms of this decomposition:
\begin{definition}[Canonical sequences: mathematical definition]
\label{def:canonseq-math}
  
\end{definition}
\begin{mathframe}
  \begin{itemize}
\item Let $\lambda<\epsilon_0$ be a limit ordinal 

\begin{itemize}
\item If $\lambda=\omega^{\alpha+1}\times (\beta+1)$, then 
$\canonseq{\lambda}{i}= \omega^{\alpha+1}\times\beta +  \omega^\alpha \times i$
\item If $\lambda=\omega^{\gamma}\times (\beta+1)$, where $\gamma<\lambda$ is a limit ordinal, then 
$\canonseq{\lambda}{i}=\omega^{\gamma}\times \beta + \omega^{\canonseq{\gamma}{i}}$
\end{itemize}

\item For successor ordinals, we have $\canonseq{\alpha+1}{i}= \alpha$ 

\item Finally, $\canonseq{0}{i}= \alpha$.
\end{itemize}
\end{mathframe}

\subsubsection{Canonical sequences in Coq}

Our definition may look more complex than the mathematical one, but
uses plain structural recursion over the type \coqsimple{T1}. Thus, tactics like
\coqsimple{cbn}, \coqsimple{simpl}, \coqsimple{compute}, etc., are applicable. For simplicity's sake, 
we use an auxiliary function \texttt{canonS} of type \texttt{T1 -> nat  -> T1} such that
(\texttt{canonS  $\alpha$ $i$}) is equal to $\canonseq{\alpha}{i+1}$.

\vspace{4pt}
\emph{From Module~\href{../src/html/hydras.Epsilon0.Canon.html\#canonS}{Epsilon0.Canon}}

\label{Functions:canonS}
\label{Functions:canon}
\begin{Coqsrc}
Fixpoint canonS alpha (i:nat) :=
  match alpha with
      zero => zero
    | ocons zero 0 zero => zero
    | ocons zero (S k) zero => FS k
    | ocons gamma 0 zero =>
      match pred gamma with
          Some gamma' => ocons gamma' i zero
        | None => ocons (canonS gamma i) 0 zero
      end
    | ocons gamma (S n) zero =>
       match pred gamma with
           Some gamma' => ocons gamma n (ocons gamma' i zero)
         | None => ocons gamma n (ocons (canonS gamma i) 0 zero)
       end
    | ocons alpha n beta => ocons alpha n (canonS beta i)
  end.
\end{Coqsrc}


The following function computes $\canonseq{\alpha}{i}$, except for the case $i=0$, where it simply returns $0$\;\footnote{This restriction did not prevent us from proving all the main theorems of~\cite{KS81, KP82}. Nevertheless, in a future version of this development, we may define $\canonseq{\alpha}{0}$ exactly as 
in~\cite{KS81}. But we are afraid this would  be done at the cost of making some proofs much more complex.}.

\begin{Coqsrc}
 Definition canon alpha i := 
   match i with 0 => zero | S j => canonS alpha j end. 
\end{Coqsrc}

For instance \coq's computing facilities allow us to verify the equalities\linebreak 
\mathcolor{$\canonseq{\omega^\omega}{3} = \omega^3$} and
\mathcolor{$\canonseq{\omega^\omega*3}{42} = \omega^\omega*2 + \omega^{42}$}.


\begin{Coqsrc}
Compute (canon (omega ^ omega) 3).
\end{Coqsrc}

\begin{Coqanswer}
  = phi0 (FS 2) : T1
\end{Coqanswer}

\begin{Coqsrc}
Example canon3 :  canon (omega ^ omega) 3 = omega ^ 3.
Proof. reflexivity. Qed.
\end{Coqsrc}


\begin{Coqsrc}
Compute pp (canon (omega ^ omega * 3) 42).  
\end{Coqsrc}

\begin{Coqanswer}
    = (omega ^ omega * 2 + omega ^ 42)%pT1
     : ppT1
\end{Coqanswer}

\index{Projects}
\begin{project}
Many lemmas presented in this chapter were stated and proved before the introduction of 
the type class \texttt{ON} of ordinal notations, and in particular its  instance \texttt{Epsilon0}.
Thus definitions and lemmas refer to the type \texttt{T1} of possibly not well-formed terms.
This should be fixed in  a future version.
\end{project}


\subsection{Basic properties of canonical sequences}

We did not  try to prove that our definition really implements Ketonen and Solovay's  \cite{KS81}'s canonical sequences. The most important is that we were able to prove the 
abstract properties  of canonical sequences that are really used in our proof. The complete proofs are in the module
~\href{../src/html/hydras.Epsilon0.Canon.html}{Epsilon0.Canon}


Proving the equality $\canonseq{\alpha+1}{i}=\alpha$ is not 
as simple as suggested by the equations of definition~\ref{def:canonseq-math}\,.
Nevertheless, we can prove it by  plain structural induction on $\alpha$.

\begin{Coqsrc}
Lemma canonS_succ i alpha :
  nf alpha ->  canonS (succ alpha) i = alpha.
Proof.
 induction alpha.
 (* ... *)
\end{Coqsrc}

\subsubsection{Canonical sequences and the order $<$}

\index{Maths!Transfinite induction}

We prove by transfinite induction over $\alpha$ that $\canonseq{\alpha}{i+1}$ is an ordinal strictly less than $\alpha$ (assuming $\alpha\not=0$). This property allows us to use the function \texttt{canonS} and its derivates in function definitions by transfinite recursion.

\label{lemma:canonS_LT}
\begin{Coqsrc}
Lemma canonS_LT i alpha :
  nf alpha -> alpha <> zero -> canonS alpha i t1<  alpha.
\end{Coqsrc}


\subsubsection{Limit ordinals are really limits}
The following theorem states that any limit ordinal $\lambda<\epsilon_0$ 
is the limit of the sequence \showmath{\canonseq{\lambda}{i}\;(1\le i)}.


\vspace{4pt}
\emph{From Module~\href{../src/html/hydras.Epsilon0.Canon.html\#canonS_limit_strong}{Epsilon0.Canon}}


\begin{Coqsrc}
Lemma canonS_limit_strong (lambda : T1) : 
     nf lambda ->
     limitb lambda  ->
     forall beta, beta t1< lambda ->
                  {i:nat | beta t1< canonS lambda i}.

Proof.
  transfinite_induction_LT lambda.
  (* ... *)
Defined.
\end{Coqsrc}

\label{lemma:canonS-limit}


Note the use of \coq's \texttt{sig} type in the theorem's statement, which
relates the boolean function \texttt{limitb} defined on the \texttt{T1} data-type with a constructive view of the limit of a sequence: for any $\beta<\lambda$, we can compute an item of the canonical sequence of $\lambda$ which is greater than $\beta$.
We can also state directly that $\lambda$ is a (strict) least upper bound of the elements of its canonical sequence.


\begin{Coqsrc}
Lemma canonS_limit_lub (lambda : T1) :
  nf lambda -> limitb lambda  ->
  strict_lub (fun i => canonS lambda i) lambda.
\end{Coqsrc}

\index{Exercises}

\begin{exercise}\label{exo:simply-typed-canonseq}
Instead of using the \texttt{sig} type, define a simply typed function that, given two ordinals $\alpha$ and $\beta$, returns a natural number $i$ such that, if $\alpha$ is a limit ordinal and $\beta<\alpha$, then $\beta< \canonseq{\alpha}{i+1}$. Of course, you will have to prove the correctness of your function.
\end{exercise}






\section{Accessibility inside \texorpdfstring{$\epsilon_0$}{epsilon0} : paths}
\index{Maths!Accessibility inside epsilon0}
\label{sect:pathes-intro}

Let us consider a kind of accessibility problem inside $\epsilon_0$: given two ordinals $\alpha$ and $\beta$, where $\beta<\alpha<\epsilon_0$, find a \emph{path} consisting of a finite sequence $\gamma_0=\alpha,\dots,\gamma_l=\beta$,
where, for every $i<l$, $\gamma_i \not= 0$ \footnote{This condition allows us to ignore paths which end by a lot of useless $0$s.} and there exists some strictly positive integer $s_i$
such that $\gamma_{i+1}=\canonseq{\gamma}{s_i}$,

Let $s$ be the sequence $\langle s_0,s_1,\dots, s_{l-1} \rangle$. We describe the
existence of such a path with the notation $\alpha\xrightarrow [s]{}\beta$.


For instance, we have $\omega*2 \xrightarrow[2,2,2,4,5]{}3$, through the 
path $\langle\omega\times 2, \omega+2,\omega+1,\omega,4,3\rangle$.


\begin{remark}
  

Note that, given $\alpha$ and $\beta$, where $\beta < \alpha$, the sequence $s$ which leads from $\alpha$ to $\beta$ is not unique.

Indeed, if $\alpha$ is a limit ordinal, the first element of $s$ can be any integer $i$ such that $\beta<\canonseq{\alpha}{i}$, and if $\alpha$ is a successor ordinal,
then the sequence $s$ can start with any positive integer.


For instance, we have also 
$\omega*2 \xrightarrow[3,4,5,6]{}\omega$. 
Likewise,
$\omega*2 \xrightarrow[1,2,1,4]{} 0$ and
$\omega*2 \xrightarrow[3,3,3,3,3,3,3,3]{} 0$.
\end{remark}

\subsection{Formal definition}

\label{path-to-definition}

In \coq{}, the notion of path can be simply defined as an inductive predicate 
parameterized by the destination $\beta$.

\vspace{4pt}
\emph{From Module~\href{../src/html/hydras.Epsilon0.Paths.html}{Epsilon0.Paths}}

\index{Predicates!path\_to}
\label{sect:path-to-def}

\begin{Coqsrc}
(Definition transition_S i : relation T1 :=
  fun alpha beta =>  alpha <> zero /\ beta = canonS alpha i.

Definition transition i : relation T1 :=
  match i with 0 => fun _ _ => False | S j => transition_S j end.

Inductive path_to (beta: T1) : list nat -> T1 -> Prop :=
  path_to_1 : forall (i:nat) alpha , 
    i <> 0 ->
    transition i alpha beta ->
    path_to beta (i::nil) alpha
| path_to_cons : forall i alpha s gamma,
    i <> 0 ->
    transition i alpha gamma ->
    path_to beta  s gamma ->
    path_to beta  (i::s) alpha.
\end{Coqsrc}

\begin{remark}
The definition above is parameterized with the \emph{destination} of the path and indexed by the origin, hence the name \texttt{path\_to}. The rationale behind this choice is a personal preference of the developer  for the kind of eliminators generated by \coq{} in this case. The symmetric option could have been also considered (see also Remark~\vref{remark:transitive-closure}).
\end{remark}



\begin{remark}
In the present version of our library, we use a variant \texttt{path\_toS} of
\texttt{path\_to}, where the proposition
(\texttt{path\_toS $\beta$ $s$ $\alpha$}) is equivalent to
(\texttt{path\_to $\beta$ (shift $s$) $\alpha$}). This variant is scheduled to be deprecated.
\end{remark}

\index{Exercises}

\begin{exercise}
Write a tactic for solving goals of the form (\texttt{path\_to $\beta$ $s$ $\alpha$})
where $\alpha$, $\beta$ and $s$ are closed terms. 
You should solve automatically the following goals:

\begin{Coqsrc}
 path_to omega (2::2::2::nil) (omega * 2).

 path_to omega (3::4::5::6::nil) (omega * 2).

 path_to zero (interval 3 14) (omega * 2).

 path_to zero (repeat 3 8) (omega * 2).
\end{Coqsrc}

\end{exercise}



\subsection{Existence of a path}

\index{Maths!Transfinite induction}

By transfinite induction on $\alpha$, we prove that for any $\beta<\alpha$, 
one can build a path from $\alpha$ to $\beta$ (in other terms, $\beta$ is accessible from $\alpha$).

\begin{Coqsrc}
Lemma LT_path_to (alpha beta : T1) :
  beta t1< alpha -> {s : list nat | path_to beta s alpha}.
\end{Coqsrc}

\index{Exercises}

\begin{exercise}[continuation of exercise~\vref{exo:simply-typed-canonseq}]
Define a simply typed function for computing a path from $\alpha$ to $\beta$.
\end{exercise}


\noindent 
From the lemma \texttt{canonS\_LT}~\vref{lemma:canonS_LT}, we can convert any path into an inequality on ordinals (by induction on paths).


\begin{Coqsrc}
Lemma path_to_LT beta s alpha :
  path_to beta s alpha -> nf alpha -> beta t1< alpha.
\end{Coqsrc}

\subsection{Paths and hydra battles}
\label{KS-o2h}

In order to apply our knowledge about  ordinal numbers less than $\epsilon_0$ to the study of hydra battles, we define an injection
from the interval $[0,\epsilon_0)$ into the type \texttt{Hydra}.

\vspace{4pt}

\emph{From Module~\href{../src/html/hydras.Hydra.O2H.html}{Hydra.O2H}}


\begin{Coqsrc}
Fixpoint iota (alpha : T1) : Hydra :=
  match alpha with
  | T1.zero => head
  | ocons gamma n beta => 
         node (hcons_mult (iota gamma) (S n) (iotas beta))
  end 
with iotas (alpha : T1) :  Hydrae :=
       match alpha with
       | T1.zero => hnil
       | ocons alpha0 n beta  => 
           hcons_mult (iota alpha0) (S n) (iotas beta)
       end.
\end{Coqsrc}  




For instance Fig.~\ref{fig:iota-example} shows the image by $\iota$ of the ordinal  \textcolor{black}{$\omega^{\omega+2}+\omega^\omega \times 2 + \omega + 1$}

  \begin{figure}[htb]
\centering
\begin{tikzpicture}[very thick, scale=0.3]
\node (foot) at (10,0) {$\bullet$};
\node (N1) at (2,2) {$\bullet$};
\node (N2) at (10,2) {$\bullet$};
\node (N22) at (7,2) {$\bullet$};
\node (N3) at (14,2) {$\bullet$};
\node (N4) at (18,2) {$\Smiley[2][green]$};
\node (N5) at (0,4) {$\bullet$};
\node (N6) at (2,5) {$\Smiley[2][green]$};
\node (N7) at (4,6) {$\Smiley[2][green]$};
\node (N88) at (7,4) {$\bullet$};
\node (N8) at (10,4) {$\bullet$};
\node (N9) at (14,6) {$\Smiley[2][green]$};
\node (N10) at (0,8) {$\Smiley[2][green]$};
\node (N11) at (10,7) {$\Smiley[2][green]$};
\node (N111) at (7,7) {$\Smiley[2][green]$};
\draw (foot) to [bend left=10] (N1);
\draw (foot) -- (N2);
\draw (foot) -- (N22);
\draw (foot) -- (N3);
\draw (foot) -- (N4);
\draw (N1) to  (N5);
\draw (N1) to   [bend left=10] (N6);
\draw (N1) to   [bend right=20] (N7);
\draw (N2) to  (N8);
\draw (N22) to  (N88);
\draw (N8) to  (N11);
\draw (N88) to  (N111);
\draw (N3) to  (N9);
\draw (N5) to  (N10);
\end{tikzpicture}
\caption{The hydra $\iota(\omega^{\omega+2}+\omega^\omega \times 2 + \omega + 1$) \label{fig:iota-example}}

\end{figure}


The following lemma (proved in ~\href{../src/html/hydras.Hydra.O2H.html}{Hydra.O2H.v}) maps  canonical sequences to rounds of hydra battles.


\label{lemma:canonS-iota}

\begin{Coqsrc}
Lemma canonS_iota i alpha :
    nf alpha -> alpha <> 0 ->
    iota alpha -1-> iota (canonS alpha i).
\end{Coqsrc}
                

The next step of our development extends this relationship to
the order $<$ on $[0,\epsilon_0)$ on one side, and hydra battles on the other side.


\begin{Coqsrc}
Lemma path_to_battle alpha s beta :
  path_to  beta  s alpha -> nf alpha ->
  iota alpha -+-> iota beta.
\end{Coqsrc}

As a corollary, we are now able to transform any inequality $\beta<\alpha<\epsilon_0$ into a (free) battle.

\begin{Coqsrc}
Lemma LT_to_battle alpha beta :
    beta t1< alpha ->  iota alpha -+-> iota beta.
\end{Coqsrc}

\section{A  proof of impossibility}

We now have  the tools for proving that  there exists no variant bounded by some $\mu<\epsilon_0$ for proving the termination   of all battles. The proof we are going to show is a proof by contradiction. It  can
 be considered as a generalization of the
proofs described in  sections~\vref{omega-case} and \vref{omega2-case}.



In the module \href{../src/html/hydras.Hydra.Epsilon0_Needed_Generic.html}{Hydra.Epsilon0\_Needed\_Generic}, we assume there exists some variant $m$ bounded by some ordinal $\mu<\epsilon_0$. This part of the development is parameterized by some class $B$ of battles, which will be instantiated later to \texttt{free} or \texttt{standard}.


\index{Coq!Type classes}

\begin{Coqsrc}
Class BoundedVariant (B:Battle) :=
  {
    mu:T1 ;
    m: Hydra -> T1;
    mu_nf: nf mu;
    Hvar: Hvariant T1_wf B m;
    m_bounded: forall h, m h t1< mu
  }.
\end{Coqsrc}

Let us assume there exists such a variant:

\begin{Coqsrc}
Section Bounded.
  Context (B: Battle) (Hy : BoundedVariant B).

  Hypothesis m_decrease : forall  i h h',
        round_n i h h' -> m h' t1< m h.
\end{Coqsrc}

\label{remark:m-decrease}
\begin{remark}
  The hypothesis \texttt{m\_decrease} is not provable  in general, but is satisfied by
the  \texttt{free} and \texttt{standard} kinds of battles. This trick allows to 
``factorize'' our proofs  of impossibility.
\end{remark}

\index{Maths!Transfinite induction}

First, we prove that $m(\iota(\alpha))$ is always greater than or equal to $\alpha$, by  transfinite induction over $\alpha$.

\begin{Coqsrc}
Lemma m_ge_0 alpha:  nf alpha -> alpha t1<= m (iota alpha).
\end{Coqsrc}


\begin{itemize}
\item If $\alpha=0$, the inequality trivially holds
\item If $\alpha$ is the successor of  some ordinal $\beta$, the inequality $\beta \leq m(\iota(\beta))$ holds (by induction hypothesis). But the hydra $\iota(\alpha)$ is transformed in one round into 
$\iota(\beta)$, thus $m(\iota(\beta))<m(\iota(\alpha))$. Hence $\beta<m(\iota(\alpha))$, which implies $\alpha \leq m(\iota(\alpha))$
\item If $\alpha$ is a limit ordinal, then $\alpha$ is the least upper bound of the set
of all  the $\canonseq{\alpha}{i}$.  Thus, we have just to prove that $\canonseq{\alpha}{i}< m(\iota(\alpha))$ for any $i$. 
\begin{itemize}
\item Let $i$ be some natural number.
By the induction hypothesis, we have $\canonseq{\alpha}{i} \leq m(\iota(\canonseq{\alpha}{i}))$. But the hydra $\iota(\alpha)$ is transformed into $\iota(\canonseq{\alpha}{i})$ in one round, thus $m(\iota(\canonseq{\alpha}{i})) < m(\iota(\alpha))$, by our hypothesis \texttt{m\_decrease}.
\end{itemize}
\end{itemize}

Please note that the impossibility proofs of 
sections~\vref{omega-case} and \vref{omega2-case} contain a similar lemma, also called \texttt{m\_ge}.
We are now able to build a counter-example.

\begin{Coqsrc}
  Definition big_h := iota mu.
  Definition beta_h := m big_h.
  Definition small_h := iota beta_h.
\end{Coqsrc}
  
From Lemma \texttt{m\_ge\_0} we infer the following inequality :

\begin{Coqsrc}
    Corollary m_ge_generic : m big_h t1<= m small_h.
 \end{Coqsrc}

The (big) rest of the proof is dedicated to prove formally the converse inequality 
\texttt{m small\_h t1< m big\_h}. 



\subsection{The case of free battles}
\label{sec:free-battles-case}
Let us now consider that $B$ is instantiated to \texttt{free} (which means that we are considering proofs of termination of \emph{all} battles). The following lemmas are proved in Module~\href{../src/html/hydras.Hydra.Epsilon0_Needed_Free.html}{Hydra.Epsilon0\_Needed\_Free}.
The case $B=\texttt{standard}$ is studied in section~\vref{std-case}.



\begin{Coqsrc}
Section Impossibility_Proof.

  Context (Var : BoundedVariant free ).
  \end{Coqsrc}


\begin{enumerate}
\item The following lemma is an application of \texttt{m\_ge\_generic}, since \texttt{free}
satisfies trivially the hypothesis \texttt{m\_decrease} (see page~\pageref{remark:m-decrease}).

\begin{Coqsrc}
Lemma m_ge : m big_h t1<= m small_h.
  Proof.
    apply m_ge_generic.
   (* ... *)
\end{Coqsrc}

\item From the hypothesis \texttt{m\_bounded}, we have \texttt{m big\_h t1< mu}
\item By Lemma \texttt{LT\_to\_battle}, we get a (free) battle from
\texttt{big\_h = iota mu} to \texttt{small\_h = iota (m big\_h)}.

\begin{Coqsrc}
  Lemma  big_to_small : big_h  -+-> small_h.
\end{Coqsrc}
\item From the hypotheses on $m$, we infer:

\begin{Coqsrc}
Lemma m_lt : m small_h t1< m big_h.
\end{Coqsrc}


\item From lemmas \texttt{m\_ge} and \texttt{m\_lt}, and the irreflexivity of $<$, we get a contradiction. 

  \begin{Coqsrc}
Theorem Impossibility_free : False.

End Impossibility_Proof.
\end{Coqsrc}


\end{enumerate}

We have now proved there exists no bounded variant for the class of free battles.

 
\begin{Coqsrc}
Check Impossibility_free.
\end{Coqsrc}

\begin{Coqanswer}
  Impossibility_free
     : BoundedVariant free -> False
\end{Coqanswer}
%%% ICI
  



\section{The case of standard battles}
\label{sec:standard-intro}\label{std-case}
One may wonder if our theorem holds also in the framework of standard battles. Unfortunately, its proof relies on the lemma \texttt{LT\_to\_round\_plus} of
Module~\href{../src/html/hydras.Hydra.O2H.html}{Hydra.O2H}.

\begin{Coqsrc}
Lemma LT_to_round_plus alpha beta :
    beta t1< alpha ->  iota alpha -+-> iota beta.
\end{Coqsrc}

This lemma builds a battle out of any inequality $\beta<\alpha$. 
It is a straightforward application of \texttt{LT\_path\_to} of
Module~\href{../src/html/hydras.Epsilon0.Paths.html}{Epsilon0.Paths}:

\begin{Coqsrc}
Lemma LT_path_to (alpha beta : T1) :
  beta t1< alpha -> {s : list nat | path_to beta s alpha}.
\end{Coqsrc}

The sequence $s$, used to build the sequence of replication factors of the battle depends on 
$\beta$, so we cannot be sure that the generated battle is a genuine standard battle.


The solution of this issue comes  once again from Ketonen and Solovay's article~\cite{KS81}. Instead of considering plain paths, i.e. sequences 
$\alpha_0=\alpha,\alpha_1,\dots,\alpha_k=\beta$ where $\alpha_{j+1}$ is equal
to $\canonseq{\alpha_j}{i_j}$ where $i_j$ is \emph{any} natural number, 
we consider various constraints on these sequences.
In particular, a path is called \emph{standard} if $i_{j+1} = i_j + 1$ for every $j<k$.
It  corresponds to a ``segment'' of some standard battles. 
Please note that the vocabulary on paths is ours, but all the concepts come really from~\cite{KS81}.

In \coq{}, standard paths can be defined as follows.

\vspace{4pt}
\emph{From Module~\href{../src/html/hydras.Epsilon0.KS.html}{Epsilon0.KS}}

\begin{Coqsrc}
(**  standard path from (i, alpha) to (j, beta) *)

Inductive standard_pathR(j:nat)( beta:T1):  nat -> T1 -> Prop :=
  std_1 : forall i alpha, 
       beta = canon alpha i -> j = S i ->
       standard_pathR j beta i  alpha
| std_S : forall i alpha, 
      standard_pathR j beta (S i) (canon alpha i)  ->
      standard_pathR j beta i alpha.

Definition standard_path  i alpha j beta := 
   standard_pathR j beta i alpha.
\end{Coqsrc}

In the mathematical text and figures, we shall use the notation 
$\alpha \xrightarrow[i,j]{}\beta$ for the proposition 
(\texttt{standard\_path $i$ $\alpha$ $j$ $\beta$}).
In~\cite{KS81} the notation is
$\alpha \xrightarrow[i]{*}\beta$
for 
the proposition  $\exists j, i<j \wedge \alpha \xrightarrow[i,j]{} \beta$.



Our goal is now  to transform any inequality $\beta<\alpha<\epsilon_0$ into a standard path $\alpha \xrightarrow[i,j]{} \beta$ for some $i$ and $j$, then into a standard battle
from $\iota(\alpha+i)$ to $\iota(\beta)$. 
Following~\cite{KS81}, we proceed in two stages:
\begin{enumerate}
\item we simulate plain (free) paths from $\alpha$ to $\beta$ with
paths made of steps $(\gamma,\canonseq{\gamma}{n})$, \emph{with the same $n$ all along the path}
\item we simulate any such path by a standard path.
\end{enumerate}



\subsection{Paths with constant index}

First of all, paths with a constant index 
enjoy nice properties. They are defined as paths where all the $i_j$ are equal to the same natural number $i$, for some $i>0$. 


Like in~\cite{KS81}, we shall use the notation $\alpha \xrightarrow[i]{} \beta$ for denoting such a path, also called an $i$-path.

\begin{Coqsrc}
Definition const_pathS i :=
    clos_trans_1n T1 (fun alpha beta => beta = canonS alpha i).

Definition const_path i alpha beta :=
  match i with
    0 => False
  | S j => const_pathS j alpha beta
end.
\end{Coqsrc}

% Paths with a given index can be effectively computed.
% Given $i$, $\alpha$ and $l$, the following function returns the ordinal $\beta$ such that there exists a path 
% $\alpha \xrightarrow [i+1] {} \beta$ of length $l$. 

% \begin{Coqsrc}
% Fixpoint const_funS (i:nat)(alpha : T1)(l:nat):  T1  :=
%   match l
%   with
%   | 0 => alpha
%   | S m => const_funS i (canonS i alpha) m
%   end.
% \end{Coqsrc}

% The following computations show  applications of \texttt{constS\_fun} to the 
% ordinal $\omega^\omega$, with various values of $i$ and $l$.

% \begin{Coqsrc}
% Compute  (const_funS 2 (omega ^omega)  55).
% \end{Coqsrc}

% \begin{Coqanswer}
%   = zero
%      : T1 
% \end{Coqanswer}

% \begin{Coqsrc}
% Compute pp (const_funS 2 (omega ^omega) 15).
% \end{Coqsrc}

%   \begin{Coqanswer}
%  = (omega ^ 2 * 2)%pT1
%      : ppT1   
%   \end{Coqanswer}


% \begin{Coqsrc}
% Compute pp (const_funS 4 (omega^omega)  100).
% \end{Coqsrc}

% \begin{Coqanswer}
% = (omega ^ 4 * 4 + omega ^ 3 * 4 + omega ^ 2 + omega * 4 + 4)%pT1
%      : ppT1
% \end{Coqanswer}




A most interesting property of $i$-paths is that we can ``upgrade'' their index, as stated by K.\&S.'s Corollary 12.

\index{Maths!Transfinite induction}

\begin{Coqsrc}
Corollary Cor12 (alpha : T1) :  nf alpha ->
         forall beta i n, beta  t1< alpha  ->
                (i < n)%nat ->
                 const_pathS i alpha beta ->
                 const_pathS n alpha beta.
Proof.
  transfinite_induction_lt alpha.
  (* (long) proof skipped *)
\end{Coqsrc}

We  also use a version of \texttt{Cor12} with large inequalities.


\begin{Coqsrc}
Corollary Cor12' (alpha : T1) :  nf alpha ->
         forall beta i n, (beta t1< alpha)%t1 ->
               (i <= n)%nat ->
               const_pathS i alpha beta ->
               const_pathS n alpha beta.
\end{Coqsrc}


\subsubsection{Sketch of proof of \texttt{Cor12}}
\index{Maths!Transfinite induction}

We prove this lemma by transfinite induction on $\alpha$.
Let us consider a path $\alpha \xrightarrow [i]{} \beta$ $(i>0)$. Its first step is
the pair $(\alpha,\canonseq{\alpha}{i})$, We have $\canonseq{\alpha}{i}<\alpha$ and
$\canonseq{\alpha}{i} \xrightarrow [i]{} \beta$. 
Let $n$ be any natural number such that $n>i$.
By the induction hypothesis, there exists a path $\canonseq{\alpha}{n} \xrightarrow[i]{} \beta$.
\begin{itemize}
\item  If $\alpha$ is a successor ordinal $\gamma+1$, then $\canonseq{\alpha}{n} =
\canonseq{\alpha}{i}=\gamma$. Thus we have a path 
$\alpha  \xrightarrow [n]{}  \gamma \xrightarrow [n]{} \beta$
\item If $\alpha$ is a limit ordinal, we apply the following theorem (numbered \texttt{2.4} in Ketonen and Solovay's article). 

%   \begin{theorem}
% Let $\lambda$ be a limit ordinal, then for any pair of indices $0<i<j$, there is a path $\canonseq{\lambda}{j} \xrightarrow[1]{} \canonseq{\lambda}{i}$.    
%   \end{theorem}


  \begin{Coqsrc}
Theorem Theorem_2_4 (lambda : T1) :
   nf lambda ->
   limitb lambda  ->
   forall i j, (i < j)%nat ->
               const_pathS 0 (canonS lambda j)
                             (canonS lambda i). 
  \end{Coqsrc}

 We build the following paths :

 \begin{enumerate}
 \item $\alpha \xrightarrow[n]{} \canonseq{\alpha}{n}$
 \item $\canonseq{\alpha}{n} \xrightarrow[1]{} \canonseq{\alpha}{i}$ (by \texttt{Theorem\_2\_4}),
\item $\canonseq{\alpha}{n} \xrightarrow[n]{} \canonseq{\alpha}{i}$ (applying the induction hypothesis to the preceding path);
\item $\canonseq{\alpha}{i} \xrightarrow[n]{} \beta$ (applying the induction hypothesis)\item $\alpha \xrightarrow[n]{} \beta$ (by composition of 1, 3, and 4).


 \end{enumerate}


\end{itemize}





\begin{remark}
 \texttt{Cor12} ``casts'' $i$-paths into $n$-paths for any $n>i$.
But the obtained $n$-path can be much longer than the original $i$-path.
The following exercise will give an idea of this increase. 
\end{remark}

\index{Exercises}
\begin{exercise}
  Prove that  the length of the $i+1$-path from
  $\omega^\omega$ to $\omega^i$ is $1 + (i+1)^{(i+1)}$, for any $i$. Note that the $i$-path from
  $\omega^\omega$ to $\omega^i$ is only one step long.
 \end{exercise}


Why is \texttt{Cor12} so useful? 
Let us  consider two ordinals  $\beta<\alpha<\epsilon_0$. By induction on $\alpha$,
we decompose any inequality $\beta<\alpha$ into $\beta < \canonseq{\alpha}{i}< \alpha$, where $i$ is some integer. Applying collorary \texttt{Cor12'} we build a $n$-path from $\beta$ to $\alpha$,
where $n$ is the maximum of the indices $i$ met in the induction.

 Lemma 1, Section 2.6 of~\cite{KS81} is naturally expressed in terms of \coq's
\verb@sig@ construct.

\label{lemma:L-2_6-1}
\index{Coq!Techniques!Sigma types}

\begin{Coqsrc}
Lemma Lemma2_6_1 (alpha : T1) :  
  nf alpha -> forall beta,  beta t1< alpha  ->
  {n:nat | const_pathS n alpha beta}.
Proof.
  transfinite_induction alpha.
  (* ... *)
\end{Coqsrc}



Intuitively, lemma   \texttt{L2\_6\_1}  shows that if $\beta<\alpha<\epsilon_0$, then there exists  a battle from $\iota(\alpha)$ to $\iota(\beta)$ where the replication factor is constant, although large enough. 







\subsection{Casting paths with constant index into standard paths}

%%% traduire la v.f.  (voir %%% A traduire %%%% )

The article~\cite{KS81} contains 
the following lemma, the proof of which is quite complex, which allows to simulate $i$-paths by $[i+1,j]$-paths, where $j$ is large enough.


\begin{Coqsrc}
(* Lemma 1 page 300 of [KS] *)

Lemma constant_to_standard_path 
  (alpha beta : T1) (i : nat):
  nf alpha -> const_pathS i alpha beta -> zero  t1< alpha ->
  {l:nat | standard_path (S i) alpha j beta}.
\end{Coqsrc}

 

\subsubsection{Sketch of proof of \texttt{constant\_to\_standard\_path}}

Our proof follows the proof by Ketonen and Solovay, including its organization as a sequence of lemma.  Since it is a non-trivial proof, we will comment its main steps below.

\subsubsection*{Préliminaries}


Please note that, given an ordinal $\alpha:\texttt{T1}$, and two natural numbers $i$ and $l$, there exists at most a standard path $\alpha \xrightarrow [i,i+l]{*} \beta$.
The following function computes $\beta$ from $\alpha$, $i$ and $l$.

\begin{Coqsrc}
Fixpoint standard_gnaw (i:nat)(alpha : T1)(l:nat):  T1  :=
  match l with
  | 0 => alpha
  | S m => standard_gnaw (S i) (canon alpha i) m
  end.
\end{Coqsrc}

\begin{Coqsrc}
  Compute standard_gnaw 2 omega 15.
(*   = zero
     : T1 *)
Compute pp (standard_gnaw 2 (omega^omega)  10).
(*
= (omega + 7)%pT1
     : ppT1
*)
Compute pp (standard_gnaw 4 (omega^omega)  100).
(*
 = (omega ^ 3 * 4 + omega ^ 2 * 5 + omega * 3 + 39)%pT1
     : ppT1 *)
\end{Coqsrc}

\index{Maths!Transfinite induction}

By transfinite induction over  $\alpha$, we prove that the ordinal $0$ is reachable from any ordinal $\alpha<\epsilon_0$ by some standard path.


\begin{Coqsrc}
Lemma standard_path_to_zero :
  forall  alpha i, nf alpha ->
                   {j: nat | standard_path (S i) alpha j zero}.
\end{Coqsrc}

\paragraph*{}
Noq, let us consider two ordinals  $\beta<\alpha<\epsilon_0$.  Let $p$  be some $(n+1)$-path from $\alpha$ to $\beta$.

\begin{Coqsrc}
 Section Constant_to_standard_Proof.

  Variables (alpha beta: T1) (n : nat).
  Hypotheses (Halpha: nf alpha) (Hpos : zero t1<  beta)
             (p : const_pathS n alpha  beta).
\end{Coqsrc}

Applying \texttt{standard\_path\_to\_zero}, $0$ is reachable from $\alpha$ by some standard path  (see figure~\vref{fig:belle-preuve-1}).

\begin{figure}[h]
  \centering
 
\begin{tikzpicture}[very thick, scale=0.25]
\node (alpha) at (0,0) {$\alpha$};
    \node (beta) at (32, 0){$\beta$};
  

  \draw[->, very thick,blue] (alpha)-- node [below]{$n+1$} node [above] {$+$} (beta);

  \node (alpha1) at (5,5) {};
  \node (alpha2) at (13,5) {};
    \node (alpha3) at (20,5) {};
  \node (alphalast) at (35,5) {};
  \node (zero) at (45,0) {$0$};
  \draw [->, dashed,very thick,blue] (alpha)-- node [below, rotate=40]{$n+1$}  (alpha1);
  \draw [->, dashed,very thick,blue] (alpha1)-- node [below]{$n+2$}  (alpha2);
   \draw [->, dashed,very thick,blue] (alpha2)-- node [below]{$n+3$}  (alpha3);
  
  \node (dots) at (24,5) {$\dots$};
  \draw [->, dashed, very thick,blue] (alphalast)-- node [below, rotate=-26]{$n+p+1$}  (zero);

\end{tikzpicture}
\caption{A nice proof (1)}
  \label{fig:belle-preuve-1}
\end{figure}


\paragraph*{}




Since comparison on \texttt{T1} is decidable, one can compute the last step $\gamma$ of the standard path from $(\alpha,n+1)$  such that $\beta\leq \gamma$.
Let $l$ be the length of the path from $\alpha$ to $\gamma$.  
This step of the proof is illustrated in figure~\vref{fig:belle-preuve-2}.



\begin{figure}[h]
  \centering
 

\begin{tikzpicture}[very thick, scale=0.25]
\node (alpha) at (0,0) {$\alpha$};
    \node (beta) at (32, 0){$\beta$};
  



  \node (alpha1) at (5,5) {};
  \node (alpha2) at (13,5) {};
  \node (dots) at (17,5) {$\ldots$};
    \node (alpha3) at (20,5) {};
    \node (gamma) at (24,0) {$\gamma$};
    \node (delta) at (38,0) {$\delta$};
    \draw [->, dashed,very thick,blue] (alpha)-- node [below,rotate=35]{$n+1$}  (alpha1);
  \draw [->, dashed,very thick,blue] (alpha1)-- node [below]{$n+2$}  (alpha2);
   \draw [->, dashed,very thick,blue] (alpha3)-- node [below,rotate = -48]{\tiny $n+l$}  (gamma);
   \draw  [->, dashed, blue] (gamma) to    [bend left=80] node [below]{$n+l+1$} (delta);
   \draw[->, very thick,blue] (alpha) to [bend right=34] node [below]{$n+1$} node [above] {$+$} (beta);
   \draw[thick] (alpha)--  (gamma);
   \draw[thick] (gamma)--  node [above] {$\geq$} (beta);
    \draw[thick] (beta)--  node [above] {$>$} (delta);
\end{tikzpicture}

\caption{A nice proof (2)}
  \label{fig:belle-preuve-2}
\end{figure}

\paragraph*{}

\begin{itemize}
\item If $\beta=\gamma$, its OK! We have got a standard path
from  
$\alpha$ to $\beta$ with successive indices  $n+1, n+2, \dots, n+l+1$

\item Otherwise,  $\beta < \gamma$.  Let us consider  $\delta=\canonseq{\gamma}{n+l+1}$.
By applying several times lemma \texttt{Cor12},  one converts  every path of Fig~\ref{fig:belle-preuve-2} into
 a $n+l+1$-path  (see figure~\ref{fig:belle-preuve-3}).


But $\gamma$ is on the $n+l+1$-path from $\alpha$ to $\beta$.
As shown by figure~\vref{fig:fin-belle-preuve}, the ordinal $\delta$, reachable from
$\gamma$ in one single step,  must be greater than or equal to $\beta$, which contradicts our  hypothesis $\beta < \gamma$.


\begin{figure}[h]
  \centering
  
\begin{tikzpicture}[very thick, scale=0.25]
\node (alpha) at (0,0) {$\alpha$};
    \node (beta) at (32, 0){$\beta$};
    \node (alpha1) at (5,5) {};
  \node (alpha2) at (13,5) {};
  \node (dots) at (17,5) {$\ldots$};
    \node (alpha3) at (20,5) {};
    \node (gamma) at (24,0) {$\gamma$};
    \node (delta) at (38,0) {$\delta$};
     \draw [->, dashed,very thick,blue] (alpha)-- node [below, rotate = 40] {\tiny $n+l+1$}  node [above, rotate = 40]{\tiny $+$}  (alpha1);
  \draw [->, dashed,very thick,blue] (alpha1)-- node [below]{\tiny $n+l+1$} node [above]{\tiny $+$} (alpha2);
   \draw [->, dashed,very thick,blue] (alpha3)-- node [below, rotate = -48]{\tiny $n+l+1$} node [above, rotate = -36]{\tiny $+$}  (gamma);
   \draw  [->, dashed, blue] (gamma) to    [bend left=80] node [below]{\tiny $n+l+1$} node [above]{\color{red} $1$} (delta);
   \draw[->, very thick,blue] (alpha) to [bend right=34] node [below]{\small $n+l+1$} node [above] {\tiny $+$} (beta);
    \draw[thick] (gamma)--   node [above]{\color{red} $>$}(beta);
   \draw[thick] (alpha)--  (gamma);
  
    \draw[thick] (beta)--  node [above] {$>$} (delta);

  
  
\end{tikzpicture}

\caption{A nice proof (3)}
  \label{fig:belle-preuve-3}
\end{figure}


\begin{figure}[h]
  \centering
\begin{tikzpicture}[very thick, scale=0.25]
\node (alpha) at (0,0) {$\alpha$};
    \node (beta) at (32, 0){$\beta$};
  
  \node (alpha1) at (5,5) {};
  \node (alpha2) at (13,5) {};
  \node (dots) at (15,5) {$\ldots$};
    \node (alpha3) at (18,5) {};
    \node (gamma) at (24,0) {$\gamma$};
    \node (delta) at (42,0) {$\delta$};
    \draw [->, dashed,very thick,blue] (alpha)-- node [below, rotate = 40] {\tiny $n+l+1$}  node [above, rotate = 40]{\tiny $+$}  (alpha1);
  \draw [->, dashed,very thick,blue] (alpha1)-- node [below]{\tiny $n+l+1$} node [above]{\tiny $+$} (alpha2);
   \draw [->, dashed,very thick,blue] (alpha3)-- node [below, rotate = -36]{\tiny $n+l+1$} node [above, rotate = -36]{\tiny $+$}  (gamma);
   \draw  [->, dashed, blue] (gamma) to    [bend left=80] node [below]{\small $n+l+1$} node [above]{\color{red} $1$} (delta);
   \draw[->, very thick,blue] (alpha) to [bend right=34] node [below]{\small $n+l+1$} node [above] {\tiny $+$} (beta);
   \draw[thick] (alpha)--  (gamma);
  
    \draw[thick] (gamma)--  node [below]{\tiny $n+l+1$} node [above]{\color{red} $+$}(beta);
    \draw[thick] (beta)--  node [above] {$>$} (delta);

\end{tikzpicture}

\caption{A nice proof (4)}
  \label{fig:fin-belle-preuve}
\end{figure}


\end{itemize}
 The only possible case is  thus $\beta=\gamma$, so we have got a standard path  from $\alpha$ to $\beta$.


\begin{Coqsrc}
 Lemma constant_to_standard_0 : 
    {l : nat | standard_fun (S n) alpha l = beta}.
 (* ... *)

End Constant_to_standard_Proof.
\end{Coqsrc}

Here is the full statement of the conversion from constant to standard paths.

\begin{Coqsrc}
Lemma constant_to_standard_path 
  (alpha beta : T1) (i : nat):
  nf alpha -> const_pathS i alpha beta -> zero  t1< alpha ->
  {j:nat | standard_path (S i) alpha j beta}.
\end{Coqsrc}




Applying \texttt{Lemma2\_6\_1} and \texttt{constant\_to\_standard\_path}, we get the following corollary.

\begin{Coqsrc}
Corollary  LT_to_standard_path  (alpha beta : T1) :
  beta t1< alpha ->
  {n : nat & {j:nat | standard_path (S n) alpha j beta}}.
\end{Coqsrc}


\subsection{Back to hydras}
\label{sec:standard-battles-cases}
We are now able to complete our proof that there exists no bounded variant for proving the termination of standard hydra battles. This proof can
be consulted in the module 
\url{../src/html/hydras.Hydra.Epsilon0_Needed_Std.html}.
Please note that it has the same global structure as in section\ref{sec:free-battles-case} 
% ICI !
Applying the  lemmas  \texttt{Lemma2\_6\_1} of the module 
\href{../src/html/hydras.Epsilon0.Paths.html\#Lemma2_6_1}%
{\texttt{Lemma2\_6\_1}}   and 
\href{../src/html/hydras.Epsilon0.Paths.html\#constant_to_standard_path}%
{\texttt{constant\_to\_standard\_path}},
we can convert any inequality $\beta<\alpha<\epsilon_0$ into a standard path from
$\alpha$ to  $\beta$, then into a fragment of a standard battle from 
$\iota(\alpha)$ to $\iota(\beta)$.


\vspace{4pt}
\emph{From Module~\href{../src/html/hydras.Hydra.Epsilon0_Needed_Std.html\#LT_to_standard_battle}{Hydra.Epsilon0\_Needed\_Std}}

\begin{Coqsrc}
Lemma LT_to_standard_battle :
    forall alpha beta,
      beta t1< alpha ->
      exists n i,  battle standard  n (iota alpha) i (iota beta).
\end{Coqsrc}


Next, please consider the following context:

\begin{Coqsrc}
Section Impossibility_Proof.
 
  Context (Var : BoundedVariant standard).
 \end{Coqsrc}

In the same way as for free battles, we import a large inequality 
from 
the module \href{../src/html/hydras.Hydra.Epsilon0_Needed_Generic.html}{Hydra.Epsilon0\_Needed\_Generic}.


\begin{Coqsrc}
 Lemma m_ge : m big_h t1<= m small_h.
\end{Coqsrc}

\paragraph*{} If remains to prove the following strict inequality, in order to have a contradiction.

\begin{Coqsrc}
Lemma m_lt : m small_h  t1< m big_h.
\end{Coqsrc}


\paragraph*{Sketch of proof:} Let us recall that $\texttt{big\_h} = \iota(\mu)$
 and $\texttt{small\_h} = \iota (m (\texttt{big\_h}))$.

Since $m(\texttt{big\_h})< \mu$, there exists a standard path from $\mu$ to
$m(\texttt{big\_h})$, hence a   standard battle from $\iota(\mu)$  to
$\iota(m(\texttt{big\_h}))$,  i.e. from \texttt{big\_h} to \texttt{small\_h}.

Since $m$ is assumed to be a variant for standard battles, we get the inequality  $m(\texttt{small\_h}) < m(\texttt{big\_h})$.



\subsection{Remarks}

We are grateful to 
 J. Ketonen and R. Solovay  for the high quality of their explanations and proof details.
Our proof follows tightly the sequence of lemmas in their article, with a focus on 
constructive aspects.
Roughly steaking, our implementation \emph{builds}, out of a hypothetic 
  variant $m$, bounded by some ordinal $\mu<\epsilon_0$, a hydra \texttt{big\_h} which verifies the impossible inequality  $m(\texttt{big\_h})< m(\texttt{big\_h})$.



On may ask whether the preceding results are not too restrictive, since they 
refer to a particular data type \texttt{T1}.
In fact, our representation of ordinals strictly less than 
 $\epsilon_0$ is faithful to their mathematical definition, at least 
Kurt Schütte's~\cite{schutte}, as proved in Chapter~\vref{chap:schutte}.
(please see also \href{../src/html/hydras.Schutte.Correctness_E0.html}{the module \texttt{Ordinals.Schutte.Correctness\_E0}}).

Thus, we can infer that our theorems can be applied to any well order.

\index{Projects}
\begin{project}
Study a possible modification of the definition of a variant  (for  standard battles).

\begin{itemize}
\item The variant is assumed to be strictly decreasing \emph{on configurations 
reachable from some initial configuration where the replication factor is equal to $0$}
\item The variant may depend on the number of the current round.
\end{itemize}

In other words, its type should be \texttt{nat -> Hydra -> T1}, and it must 
verify the inequality $m\, (S\,i)\, h' < m\,i\, h$ whenever the configuration 
$(i,h)$ is reachable from some initial configuration $(0,h_0)$
and \texttt{h} is transformed into \texttt{h'} in the considered round.
Can we still prove the theorems of section~\ref{std-case} with this new definition?

\end{project}



%---------------------------------------------------------------------
\chapter{Large sets and rapidly growing functions}\label{chap:alpha-large}


%\section{Introduction}

In this chapter, we try to feel how long a standard battle can be.
To be precise, for any ordinal $\alpha<\epsilon_0$ and any positive integer $k$,
we give a minoration of the number of steps of a standard battle which
starts with the hydra $\iota(\alpha)$ and the replication factor $k$.

We express this number in terms of the Hardy hierarchy of fast-growing 
functions~\cite{BW85, Wainer1970, KS81, Promel2013}.
 From the \coq{} user's point of view, such  functions are  very 
attractive:  they are defined as functions  in \gallina{}, and we can apply them \emph{in theory}, but they are so complex that you will never be able to look at the result of the computation.
 Thus, our knowledge on these functions must rely on \emph{proofs}. In our development, we use often the rewriting rules generated by \coq's \texttt{Equations} plug-in.


\section{Definitions}

%\subsection{Definition}

\begin{definition}
Let $0<\alpha<\epsilon_0$ be any ordinal, and $s$ a sequence of strictly positive natural numbers. 
We say that $s$ is \emph{minimally $\alpha$-large} (in short:
\emph{$\alpha$-mlarge}) if $s$ if $s$ is $\alpha$-large 
 and every strict prefix of $s$ leads to a non-zero ordinal (\emph{cf} Sect.~\vref{sect:path-to-def}).

\index{Maths!Large sequences}
\index{Maths!Minimal large sequences}

\end{definition}

\vspace{4pt}

\noindent
\emph{From Module~ \href{../src/html/hydras.Epsilon0.Large_Sets.html\#mlarge}{Epsilon0.Large\_Sets}}


\index{Predicates!mlarge @ mlarge (minimal large sequences)}
\begin{Coqsrc}
Definition mlarge alpha (s:list nat) := path_to zero s alpha.
\end{Coqsrc}



\begin{remark}
  Ketonen and Solovay~\cite{KS81} consider  large finite \emph{sets} of natural numbers,  but they are mainly used as sequences. Thus, we chosed to represent them explicitely as (sorted) lists. 

They also consider large (but not minimally large) sets. They can be defined in
\coq{} as follows:


\noindent
\emph{From Module~\href{../src/html/hydras.Epsilon0.Paths.html\#gnaw}{Epsilon0.Paths}}

\begin{Coqsrc}
Fixpoint gnaw (alpha : T1) (s: list nat) :=
  match s with
    | nil => alpha
    | (0::s') => gnaw  alpha s'
    | (S i :: s')  =>  gnaw (canonS i alpha) s'
  end.

Definition large alpha (s:list nat) := gnaw alpha s = zero.
\end{Coqsrc}
\end{remark}



Let us consider two integers $k$ and $l$, such that $0<k<l$. In order to check whether the interval $[k,l]$ is minimally large for $\alpha$, it is enough to
follow from $\alpha$ the path associated with the interval $[k,l($ and verify that the last ordinal we obtain is equal to $1$.
 
\subsection{Example}

For instance the interval $[6,70]$ leads $\omega^2$ to $\omega\times 2 + 56$. Thus this interval is not $\omega^2$-mlarge.


\begin{Coqsrc}
Compute pp (gnaw (omega * omega) (interval 6 70)).
\end{Coqsrc}

\begin{Coqanswer}
 = (omega * 2 + 56)%pT1
     : ppT1
\end{Coqanswer}

Let us try another computation.

\begin{Coqsrc}
Compute (gnaw (omega * omega) (interval 6 700)).
\end{Coqsrc}

\begin{Coqanswer}
 = zero : T1
\end{Coqanswer}

We may say that the interval $[6,700]$ is $\omega^2$-large, since it leads to $0$, but nothing assures us that the condition of minimality is satisfied.

The following lemma relates minimal largeness with the function 
\texttt{gnaw}. 

\begin{Coqsrc}
Lemma mlarge_iff alpha x (s:list nat) :
  s <> nil -> ~ In 0 (x::s) ->
  mlarge alpha (x::s) <-> gnaw alpha (but_last x s) = one.
 \end{Coqsrc}


For instance, we can verify that the interval $[6,510]$ is $\omega^2$-mlarge.

 \noindent
\emph{From Module~\href{../src/html/hydras.Epsilon0.Large_Sets_Examples.html}{Epsilon0.Large\_Sets\_Examples}}
\begin{Coqsrc}
Example Ex1 : mlarge (omega * omega) (interval 6 510).
\end{Coqsrc}


\section{The length of minimal large sequences}

Now, consider any natural number $k>0$. We would like to compute
a number $l$ such that the interval $[k,l]$ is $\alpha$-mlarge. So, 
the standard battle starting with $\iota(\alpha)$ and the replication factor $k$ will end after $(l-k+1)$ steps.



First, we notice that this  number $l$ exists, since the segment $[0,\epsilon_0)$ is well-founded and $\canonseq{\alpha}{i}<\alpha$ for any $i$ and $\alpha>0$.
Moreover, it is unique:

\vspace{4pt}
\noindent
\emph{From Module~\href{../src/html/hydras.Epsilon0.Large_Sets.html}{Epsilon0.Large\_Sets}}
\begin{Coqsrc}
Lemma mlarge_unicity alpha k l l' : 
  mlarge alpha (interval (S k) l) ->
  mlarge alpha (interval (S k) l') ->
  l = l'.
\end{Coqsrc}

Thus, it seems obvious that there must exist a function, parameterized by $\alpha$ which associates to any  strictly positive integer $k$ the number $l$ such that
the interval $[k,l]$ is $\alpha$-mlarge. It would be fine to write in \gallina{} a definition like this:

\begin{Coqbad}
Function L (alpha: E0) (i:nat) :  nat := ...
\end{Coqbad}

But we do not know how to fill the dots yet \dots{}   In the next section, we will 
use \coq{} to reason  about the \emph{specification} of \texttt{L},
prove properties of any function which satisfies this specification.
In Sect.~\ref{sect:L-equations}, we use the \texttt{coq-equations} plug-in
to define a function \texttt{L\_}, and prove its correctness w.r.t. its specification.



%\subsection{Looking for a function definition}

Let $0<\alpha<\epsilon_0$ be an ordinal term. We consider the functions which associate to each stritly positive integer $k$ the number $l$, where 
the interval $[k,l)$ is $\alpha$-mlarge.

\begin{remark}
The upper bound of the considered interval has been chosen to be  $l-1$ and not $l$, in order to simplify some statements and proofs in composition lemmas associated to ordinals of the form $\alpha\times i$ and 
$\omega^\alpha\times i + \beta$.
In order to consider any ordinal below $\epsilon_0$, we consider a special case for $\alpha=0$.
\end{remark}


\subsection{Formal specification}

Our specification of the function \texttt{L} is as follows:

\begin{Coqsrc}
Inductive L_spec : T1 -> (nat -> nat) -> Prop :=
| L_spec0 : forall f, (forall k, f k = k) ->  L_spec zero f
| L_spec1 : forall alpha f,
    alpha <> zero ->
    (forall k, mlarge alpha (interval (S k) (Nat.pred (f (S k))))) ->
    L_spec alpha f.
\end{Coqsrc}


Note that, for $\alpha\not=0$, the value of $f(0)$ is not specified.
Nevertheless, the restriction of $f$ to the set of strictly positive integers is unique (up to extensionnality).

\begin{Coqsrc}
Lemma L_spec_unicity alpha f g :
  L_spec alpha f -> L_spec alpha g -> forall k, f (S k) = g (S k).
\end{Coqsrc}


\subsection{Abstract properties}



Let us now prove properties of any function $f$ (if any) which satisfies 
\texttt{L\_spec}. We are looking for properties which could be used for writing \emph{equations} and prove the correctness of the function generated by the \texttt{coq-equations} plug-in. Moreover, they will give us some examples of
$L_\alpha$ for small values of $\alpha$. 


Our exploration of the $L_\alpha$s  follows the usual scheme : transfinite induction, and proof-by-cases : zero, successors and limit ordinals.

\index{Maths!Transfinite induction}

\subsubsection{The  ordinal zero}
\label{sect:L-spec-zero}
The base case is directly a consequence of the specification.

\begin{Coqsrc}
Lemma L_zero_inv f : L_spec zero f -> forall k, f (S k) = S k.
\end{Coqsrc}

\subsubsection{Successor ordinals}
\label{sect:L-spec-succ}
Let $\beta$ be some ordinal, and assume the arithmetic function $f$ satisfies 
the specification $(\texttt{L\_spec}\;\beta)$.  Let $k$ be any natural number.
Any path from $\texttt{succ}\,\beta$ to $0$ starting at $k+1$ can be decomposed into a first step from $\texttt{succ}\,\beta$ to $\beta$, then a path from
$\beta$ at $k+2$ to $0$. 
By hypothesis the interval $[k+2, f(k+2)-1]$ is $\beta$-mlarge.
But the interval $[k+1, f(k+2)-1]$ is the concatenation of the singleton
$\{k+1\}$ and the interval $[k+2, f(k+2)-1]$.
So, the function $\lambda\,k.\,f(k+1)$ satisfies the specification $\texttt{L\_spec}\,\beta$.


Note that our decomposition of intervals works only if the intervals we consider are not empty. In order to ensure this property, we assume that $f\;k$ is always greater than $k$, which we note \texttt{S <<= f}, or \texttt{(fun\_le S f)} (defined 
in~\href{../src/html/hydras.Prelude.Iterates.html\#fun_le}{Prelude.Iterates}).

\begin{Coqsrc}
Definition fun_le f g  := forall n:nat,  f n <=  g n.
\end{Coqsrc}

It looks also natural to show that the functions we consider are strictly monotonous. The section on successor ordinals has thus the following structure.

\begin{Coqsrc}
Section succ.
   Variables (beta : T1) (f : nat -> nat).

   Hypotheses (Hbeta : nf beta)
              (f_mono : strict_mono f)
              (f_Sle : S <<= f)
              (f_ok : L_spec beta f).

   Definition L_succ := fun k => f (S k).

   Lemma L_succ_mono : strict_mono L_succ.

   Lemma L_succ_Sle : S <<= L_succ.
  
   Lemma L_succ_ok : L_spec (succ beta) L_succ.
     
End succ.

\end{Coqsrc}

\subsubsection{Limit ordinals}
\label{sect:L-spec-lim}

Let $\lambda<\epsilon_0$ be any limit ordinal. In a similar way as for successors, we decompose any path from $\lambda$ (at $k$) into a step to
$\canonseq{\lambda}{k}$, then to $0$. In the following section, we assume that there exists à correct function for $\canonseq{\lambda}{k}$, \emph{for any strictly positive $k$}.

\begin{Coqsrc}
Section lim.
  Variables (lambda : T1)
            (Hnf : nf lambda)
            (Hlim : limitb lambda)
            (f : nat -> nat -> nat)
            (H : forall k, L_spec (canonS lambda k) (f (S k))).
  
  Let L_lim k := f k (S k).

  Lemma L_lim_ok : L_spec lambda L_lim.
  
End lim.
\end{Coqsrc}

\subsection{First results}

Applying the previous lemmas on successors and limit ordinals, 
we get several correct implementations of \texttt{(L\_spec $\alpha$)} for small values of $\alpha$.

\subsubsection{Finite ordinals}

By iterating the functional \texttt{L\_succ}, we get a realization of
\texttt{(L\_spec (fin $i$))} for any natural number $i$. 

\begin{Coqsrc}
Definition L_fin i := fun k => (i + k)%nat.

Lemma L_fin_ok i : L_spec (fin i) (L_fin i).
\end{Coqsrc}

\subsubsection{The first limit ordinal \texorpdfstring{$\omega$}{omega}}

The lemmas \texttt{L\_fin\_ok} and \texttt{L\_lim\_ok}   allow us to get 
by diagonalization a correct implementation for 
\texttt{L\_spec omega}.

\begin{Coqsrc}
Definition L_omega k := S (2 * k)%nat.

Lemma L_omega_ok : L_spec omega L_omega.
\end{Coqsrc}

\subsubsection{Towards  \texorpdfstring{$\omega^2$}{omega*omega}}

We would like to get exact formulas for the ordinal $\omega^2$, a.k.a.
$\varphi_0(2)$. This ordinal is the limit of the sequence $\omega\times i\;(i \in \mathbb{N}$. Thus, we have to study ordinals of this form, then use 
our lemma on limits.

The following lemma establishes a path from $\omega\times ( i+1)$ to
$\omega \times i$.

\begin{Coqsrc}
Lemma path_to_omega_mult (i k:nat) :
  path_to (omega * i) (interval (S k) (2 * (S k))) (omega * (S i)).
\end{Coqsrc}

Let us consider a path from  $\omega\times(i+1)$ to $0$ starting at $k+1$.
A first ``big step'' will lead to $\omega\times i$ at $2(k+1)$. If $i>0$, the
next jump leads to $\omega\times(i-1)$ at $2(2(k+1))+1$, etc.


The following lemma expresses the length of the mlarge sequences associated with the finite multiples of $\omega$.


\begin{Coqsrc}
Lemma omega_mult_mlarge_0 i  : forall k,
    mlarge  (omega * (S i))
            (interval (S k)
                      (Nat.pred (iterate (fun p =>  S (2 * p)%nat)
                                         (S i)
                                         (S k)))).
\end{Coqsrc}

Thus, we infer the following result:

\emph{From Module~ \href{../src/html/hydras.Epsilon0.Large_Sets.html\#L_omega_mult}{Epsilon0.Large\_Sets}}

\begin{Coqsrc}
Definition L_omega_mult i (x:nat) :=  iterate L_omega i x.

Lemma L_omega_mult_ok (i: nat) :  L_spec (omega * i) (L_omega_mult i).
\end{Coqsrc}

For instance, let us consider the ordinal $\omega\times 8$, and a sequence 
starting at $k=5$.

\begin{Coqsrc}
Compute L_omega_mult 8 5.
\end{Coqsrc}

\begin{Coqanswer}
= 1535
     : nat
\end{Coqanswer}

More generally, we prove the equality $L_{\omega\times i}(k)=2^i\times(k+1)-1$.

\begin{Coqsrc}
Lemma L_omega_mult_eqn (i : nat) :
  forall (k : nat),  (0 < k)%nat  ->
                     L_omega_mult i k = (exp2 i * S k - 1)%nat.
\end{Coqsrc}


By diagonalization, we obtain a simple formula for $L_{\omega^2}$.

\begin{Coqsrc}
Definition L_omega_square k := iterate (fun z => S (2 * z)%nat)
                                        k
                                        (S k).

Lemma L_omega_square_eqn k :
  (0 < k)%nat ->
  L_omega_square k = (exp2 k * (k + 2) - 1)%nat.


Lemma L_omega_square_ok: L_spec (omega * omega) 
          L_omega_square.

Compute L_omega_square 8.
\end{Coqsrc}

\begin{Coqanswer}
 = 2559
     : nat
\end{Coqanswer}


%%%% ICI 


\subsubsection{Going further}
Let us consider a last example, ``computing'' $L_{\omega^3}$.
Since the canonical sequence associated with this ordinal is composed of the
$\omega^2\times i\;(i\in\mathbb{N}_1)$, we have to study this sequence.

To this end, we prove a generic lemma, which expresses $L_{\omega^\alpha\times i}$ as an iterate of $L_{\omega^\alpha}$. Note that in this lemma, we assume that the fonction associated with $\alpha$ is stritly monotonous and 
greater or equal than the successor function, and prove that $L_{\omega^\alpha\times i}$satisfies  the same properties.

\begin{Coqsrc}
Section phi0_mult.
 Variables (alpha : T1) (f : nat -> nat).
 Hypotheses (Halpha : nf alpha)
            (f_mono : strict_mono f)
            (f_Sle : S <<= f)
            (f_ok : L_spec (phi0 alpha) f).

 Definition L_phi0_mult i := iterate f i.

Lemma L_phi0_mult_ok i: 
  L_spec (ocons alpha i zero)  (L_phi0_mult (S i)).

 Lemma L_phi0_mult_smono i: strict_mono (L_phi0_mult i).

 Lemma L_phi0_mult_Sle i: S <<=  L_phi0_mult (S i).

End phi0_mult.
\end{Coqsrc}

Let us look 
at the ordinal $\omega^2\times i$, using \texttt{L\_phi0\_mult}

\begin{Coqsrc}
 Definition L_omega_square_times i :=  iterate L_omega_square i.

 Lemma L_omega_square_times_ok i : 
    L_spec (ocons 2 i zero) (L_omega_square_times (S i)).
 Proof.
  apply L_phi0_mult_ok.
  -  auto with T1.
  -  apply L_omega_square_Sle.
  -  apply L_omega_square_ok.
 Qed.
\end{Coqsrc}


We are now ready to get an exact formula for $L_{\omega^3}$. 
\begin{Coqsrc}
Definition L_omega_cube  := L_lim  L_omega_square_times .

Lemma L_omega_cube_ok : L_spec (phi0 3) L_omega_cube.
\end{Coqsrc}


The function  $L_{\omega^3}$  is just obtained by diagonalization upon $L_{\omega^2\times i}$.

\begin{Coqsrc}
Lemma L_omega_cube_eqn i : 
   L_omega_cube i = L_omega_square_times i (S i).
Proof. reflexivity. Qed.
\end{Coqsrc}

Thus, for instance, $L_{\omega^3}(3)=L_{\omega^2\times 4}(3)$.
Thus, we obtain an exact expression of this number.


\begin{Coqsrc}
Lemma L_omega_cube_3_eq:
   let N := exp2 95 in
   let P := (N * 97 - 1)%nat in
   L_omega_cube 3  =  (exp2 P * (P + 2) - 1)%nat.
\end{Coqsrc}


This number is quite big. Using \texttt{Ocaml}'s \texttt{float} arithmetic,
we can [under-]approximate it by $2^{3.8\times10^{30}}\times 3.8\times{10^{30}}$.

\begin{Coqsrc}
# let exp2 x = 2.0 ** x;;

val exp2 : float -> float = <fun>
#   exp2 95.0 *. 97.0 -. 1.0;;
- : float = 3.84256588194182037e+30
# let n = exp2 95.0 ;;
# let p = n *. 97.0 -. 1.0;;
val p : float = 3.84256588194182037e+30

Estimation :
2 ** (3.84 e+30) * 3.84 e+30.
\end{Coqsrc}


\subsection{Using \texttt{Equations}}
\label{sect:L-equations}

Note that we did not define any function $L_\alpha$ \emph{for any $\alpha<\epsilon_0$} yet. We have got no more than a collection of proved realizations of $\texttt{L\_spec}\;\alpha$ for several values of $\alpha$.

\index{Coq!Plug-ins!Equations}

Using the \texttt{coq-equations} plug-in by 
M. Sozeau~\cite{sozeau:hal-01671777}, we will now define a function \texttt{L\_} which maps any ordinal  $\alpha<\epsilon_0$ to a proven realization of 
$\texttt{L\_spec}\;\alpha$.   
\subsection{Definition}

In order to get a total function, we use our type \texttt{E0} of well-formed ordinal terms,(see Sect~\vref{sect:E0-def}).  Our definition is structured along a well-founded recursion and
a case-study (zero, limit and successor ordinals). 


\vspace{4pt}
\noindent
\emph{From Module~\href{../src/html/hydras.Epsilon0.L_alpha.html\#L_}{L\_alpha}).}

\label{Functions:L-alpha}
 
\begin{Coqsrc}
From Equations Require Import Equations.
Require Import ArithRing Lia.

Instance Olt : WellFounded Lt := Lt_wf.
Hint Resolve Olt : E0.

(** Using Coq-Equations for building a function which satisfies 
    Large_sets.L_spec *)

Equations  L_ (alpha: E0) (i:nat) :  nat  by wf  alpha Lt :=
  L_ alpha  i with E0_eq_dec alpha Zero :=
    { | left _ =>  i ;
      | right nonzero
          with Utils.dec (Limitb alpha) :=
          { | left _ =>  L_ (Canon alpha i)  (S i) ;
            | right notlimit =>  L_ (Pred alpha) (S i)}}.

Solve All Obligations with auto with E0.
\end{Coqsrc}

It is worth looking at the answer from \texttt{Equations} and check (with \texttt{About} ) all the lemmas this plug-in gives you for free. We show here only a part of \coq's anwer.


\begin{Coqanswer}
L__obligations_obligation_1 is defined
L__obligations_obligation_2 is defined
L__obligations is defined
L__clause_1 is defined
L__functional is defined
L_ is defined
...
L__equation_1 is defined
L__graph_mut is defined
L__graph_rect is recursively defined
L__graph_correct is defined
L__elim has type-checked, generating 1 obligation
L__elim is defined
FunctionalElimination_L_ is defined
FunctionalInduction_L_ is defined
\end{Coqanswer}

Sometimes, these automatically generated statements may look cryptic. 

\begin{Coqsrc}
About L__equation_1.
\end{Coqsrc}

\begin{Coqanswer}
L__equation_1 :
forall (alpha : E0) (i : nat),
L_ alpha i = L__unfold_clause_1 alpha (E0_eq_dec alpha Zero) i
\end{Coqanswer}

In most cases, it may be useful to write human-readable  paraphrases of these statements.

\begin{Coqsrc}
Lemma L_zero_eqn : forall i, L_ Zero i = i.
Proof. intro i; now rewrite L__equation_1. Qed.

Lemma L_lim_eqn alpha i : Limitb alpha -> L_ alpha i =
                                        L_ (Canon alpha i) (S i).

Lemma L_succ_eqn alpha i :  L_ (Succ alpha) i = L_  alpha (S i).

Hint Rewrite L_zero_eqn L_succ_eqn : L_rw.
\end{Coqsrc}

Using these three lemmas as rewrite rules, we can prove more properties of the functions \texttt{L\_$\alpha$}.

\begin{Coqsrc}
Lemma L_finite : forall i k :nat,  L_ i k = (i+k)%nat.  
(* Proof by induction on i, using L_zero_eqn and L_succ_eqn *)

Lemma L_omega : forall k, L_ omega%e0 k = S (2 * k)%nat.
(* Proof using L_finite and L_lim_eqn *)
\end{Coqsrc}

By  well-founded induction on $\alpha$, we prove the following lemmas:

\begin{Coqsrc}
Lemma L_ge_S alpha : alpha <> Zero -> S <<= L_ alpha.

Theorem L_correct alpha : L_spec (cnf alpha) (L_ alpha).
\end{Coqsrc}

Please note that the proof of \texttt{L\_correct} applies the lemmas proven in Sections~\ref{sect:L-spec-zero}, ~\ref{sect:L-spec-succ} and ~\ref{sect:L-spec-lim}.
Our previous study of \texttt{L\_spec} allowed us to pave the way for the definition by \texttt{Equations} and the correctness proof.



\subsubsection{Back to Hydra battles}

Theorem \texttt{battle\_length\_std } of
Module~\href{../src/html/hydras.Hydra.Hydra_Theorems.html\#battle_length_std}{Hydra.Hydra\_Theorems} relates the length of standard battles with the functions $L_\alpha$.

\begin{Coqsrc}
Theorem battle_length_std (alpha : E0)  :
  alpha <> Zero ->
  forall k, (1 <= k)%nat ->
            battle_length standard k (iota (cnf alpha))
                         (L_ alpha (S k) - k).
\end{Coqsrc}


\index{Projects}
\begin{project}
Instead of considering standard paths and battles, consider ``constant'' paths and the corresponding battles. Please use \texttt{Equations} in order to define the function that computes the length of the $k$-path which leads  from $\alpha$ to $0$.
Prove a few  exact formulas and minoration lemmas.
\end{project}

\section{The Wainer-Hardy hierarchy (functions \texorpdfstring{$H_\alpha$}{\texttt{H\_alpha}})}

\label{sect:hardy}
In order to give an idea of the complexity of the functions  $L_\alpha$s, we compare them with a better known family of functions, the so called \emph{Wainer-Hardy hierarchy} of fast growing functions,
presented for instance in~\cite{Promel2013}. 
\index{Maths!Hardy Hierarchy}

For each ordinal $\alpha$ below $\epsilon_0$, $H_\alpha$ is a 
total arithmetic function, defined  by  transfinite recursion on $\alpha$, according to three cases:

\index{Maths!Transfinite induction}

\begin{itemize}
\item If $\alpha=0$, then $H_\alpha (k)= k$ for any natural number $k$.
\item If $\alpha=\textrm{succ}(\beta)$, then 
$H_\alpha(k)=H_\beta(k+1)$ for any $k \in \mathbb{N}$
\item If $\alpha$ is a limit ordinal, then 
$H_\alpha(k) = H_{(\canonseq{\alpha}{k+1})}(k)$ for any $k\in \mathbb{N}$.
\end{itemize}

\subsection{Hardy functions in \texttt{Coq}}


We define a function \texttt{H\_} of type \texttt{E0 -> nat -> nat} by transfinite induction over the type \texttt{E0} of the well formed ordinals below $\epsilon_0$.

\vspace{4pt}
\emph{From Module~\href{../src/html/hydras.Epsilon0.H_alpha.html\#H_}{Epsilon0.H\_alpha}}


\index{Coq!Plug-ins!Equations}
\label{Functions:H-alpha}

\begin{Coqsrc}
Equations H_ (alpha: E0) (i:nat) :  nat  by wf  alpha lt :=
  H_ alpha  i with E0_eq_dec alpha Zero :=
    { | left _ =>  i ;
      | right nonzero
          with Utils.dec (Limitb alpha) :=
          { | left _ =>  H_ (Canon alpha (S i))  i ;
            | right notlimit =>  H_ (Pred alpha) (S i)}}. 

Solve All Obligations with auto with E0.
\end{Coqsrc} 
 


\begin{Coqsrc}
Lemma H_eq1 : forall i, H_ Zero i = i.
Proof.   intro i; now rewrite H__equation_1.  Qed.

Lemma H_eq2 alpha i : Is_Succ alpha ->
                      H_ alpha i = H_ (Pred alpha) (S i).

Lemma H_eq3 alpha i : Limitb alpha ->
                      H_ alpha i =  H_ (Canon alpha (S i)) i.

Lemma H_eq4  alpha i :  H_ (Succ alpha) i = H_ alpha (S i).
\end{Coqsrc}


\subsection{First  steps of the Hardy hierarchy}
Using rewrite rules from \texttt{H\_eq1} to \texttt{H\_eq4}, we can explore the functions $H_\alpha$ for some small values of $\alpha$.

\subsubsection{Finite ordinals} 

By induction on $i$, we prove a simple expression of \texttt{H\_ (Fin i)}, where 
\texttt{Fin $i$}  is the $i$-th finite ordinal.

\begin{Coqsrc}
Lemma H_Fin : forall i k: nat,  H_ (Fin i) k = (i+k)%nat.
Proof with eauto with E0.
  induction i.
  - intros; simpl OF; simpl; autorewrite with H_rw E0_rw ... 
  - intros ;simpl; autorewrite with H_rw E0_rw ... 
    rewrite IHi; lia. 
Qed.
\end{Coqsrc}

\subsubsection{Multiples of \texorpdfstring{$\omega$}{omega}}

Since the canonical sequence of $\omega$ is composed of finite ordinals, 
it is easy to get the formula associated with $H_\omega$.


\begin{Coqsrc}
Lemma H_omega : forall k, H_ Omega k = S (2 * k)%nat.
Proof with auto with E0.
  intro k; rewrite H_eq3 ...
  - replace (Canon omega (S k)) with (Fin (S k)).
    + rewrite H_Fin; lia.
    +  now autorewrite with E0_rw.
Qed.
\end{Coqsrc}


Before going further, we prove a useful rewriting lemma:

\begin{Coqsrc}
Lemma H_Plus_Fin alpha : forall i k : nat,
    H_ (alpha + i)%e0 k = H_ alpha (i + k)%nat.
(* Proof by induction on i *)
\end{Coqsrc}


Then, we get easily formulas for $H_{\omega+i}$, and $H_{\omega\times i}$ for any natural number $i$.

\begin{Coqsrc}
Lemma H_omega_double k : H_ (omega * 2)%e0 k =  (4 * k + 3)%nat.
Proof.
 rewrite H_lim_eqn; simpl Canon.
 - ochange  (CanonS  (omega * 2)%e0 k)  (omega + (S k))%e0.
  + rewrite H_Plus_Fin, H_omega;  lia.
  -  now compute.
Qed.

Lemma H_omega_3 k : H_ (omega * 3)%e0 k = (8 * k + 7)%nat.

Lemma H_omega_4 k : H_ (omega * 4)%e0 k = (16 * k + 15)%nat.

Lemma H_omega_i i  : forall k,
    H_ (omega * i)%e0 k = (exp2 i * k + Nat.pred (exp2 i))%nat.
\end{Coqsrc}

Crossing a new limit, we prove the following equality: 
$$H_{\omega^2} (k) = 2 ^ {k+1} \times (k+1) - 1$$.

\begin{Coqsrc}

Lemma H_omega_sqr : forall k,
    H_ (Phi0  2)%e0 k = (exp2 (S k ) *  S k - 1)%nat.
Proof.
  intro k; 
   rewrite H_lim_eqn; auto with E0.
  - ochange (Canon (Phi0 2) (S k)) (omega * (S k))%e0.
    +  rewrite H_omega_i; simpl (exp2 (S k)).
       *  rewrite Nat.add_pred_r.
          -- lia. 
          --   generalize (exp2_not_zero k);  lia.
    + cbn; f_equal; lia.
Qed.
\end{Coqsrc}

\subsubsection{New limits}

Our next step would be to prove an exact formula for $H_{\omega^\omega}(k)$.
Since the canonical sequence of $\omega^\omega$ is composed of all the
$\omega^i$, we first need to express $H_{\omega^i}$ for any natural number $i$.

Let $i$ and $k$ be two natural numbers. 
The ordinal $\canonseq{\omega^(i+1)}{k}$ is the product
$\omega^i \times k$, so we need also to consider ordinals of this form.

\begin{enumerate}
\item First,  we express $H_{\omega^\alpha \times (i+2)}$ in terms of
$H_{\omega^\alpha \times (i+1)}$.

\begin{Coqsrc}
Lemma H_Omega_term_1 : alpha <> Zero -> forall  k,  
    H_ (Omega_term alpha (S i)) k =
    H_ (Omega_term alpha i) (H_ (Phi0 alpha) k).
\end{Coqsrc}

\item
Then, we prove by induction on $i$ that $H_{\omega^\alpha \times (i+1)}$ is just the
$(i+1)$-th iterate of $H_{\omega^\alpha}$.


\begin{Coqsrc}
Lemma H_Omega_term (alpha : E0)  :
alpha <> Zero -> forall i k, 
  H_ (Omega_term alpha i) k = iterate  (H_ (Phi0 alpha)) (S i) k.
\end{Coqsrc}

\item In particular, we have got a formula for $H_{\omega^{i+1}}$.

\begin{Coqsrc}
Definition H_succ_fun f k := iterate f (S k) k.

Lemma H_Phi0_succ alpha  : alpha <> Zero -> forall k,
      H_ (Phi0 (Succ alpha)) k = H_succ_fun (H_ (Phi0 alpha)) k. 

Lemma H_Phi0_Si : forall i k,
      H_ (Phi0 (S i)) k = iterate H_succ_fun i (H_ omega) k. 
\end{Coqsrc}

\end{enumerate}
We get now a  formula for $H_{\omega^3}$:

\begin{Coqsrc}
Lemma H_omega_cube : forall k,
    H_ (Phi0 3)%e0 k = iterate (H_ (Phi0 2))  (S k) k.
Proof.
  intro k; rewrite <-FinS_eq, -> Fin_Succ, H_Phi0_succ; auto.
  compute; injection 1; discriminate.
Qed.
\end{Coqsrc}

\subsubsection{A numerical example}

It seems hard to capture the complexity of this function by looking only at this
``exact'' formula. 
Let us consider a simple example, the number $H_{\omega^3}(3)$.  

\begin{Coqsrc}
Section H_omega_cube_3.
  
Let f k :=   (exp2 (S k ) * (S k) - 1)%nat.

Remark R0 k :  H_ (Phi0 3)%e0 k = iterate f (S k) k.
\end{Coqsrc}

Thus, the number $H_{\omega^3}(3)$ can be written as four nested applications of $f$.

\begin{Coqsrc}
Fact F0 : H_ (Phi0 3) 3 = f (f (f (f 3))).
 rewrite R0; reflexivity. 
Qed.
\end{Coqsrc}

In order to make this statement more readable, we can introduce a local définition.

\begin{Coqsrc}
Let N := (exp2 64 * 64 - 1)%nat.
\end{Coqsrc}

This number looks quite big; let us compute an approximation in \texttt{Ocaml}:


\begin{Coqsrc}
# (2.0 ** 64.0 *. 64.0 -. 1.0);; 
\end{Coqsrc}

\begin{Coqanswer}
- : float = 1.1805916207174113e+21
\end{Coqanswer}


\begin{Coqsrc}
Fact F1 : H_ (Phi0 3) 3 = f (f N).
Proof.
 rewrite H_omega_cube_0; reflexivity. 
Qed.


Lemma F1_simpl : H_ (Phi0 3) 3 =
                 (exp2 (exp2 (S N) * S N) * (exp2 (S N) * S N) - 1)%nat.

\end{Coqsrc}


In a more classical writing, this number is displayed as follows:

{\Large
$$
H_{\omega^3}(3) =  2 ^ {(2 ^ {N + 1} \times (N+1) } )  \times  (2 ^ {N+1} \times ( N +1) ) - 1
$$
}


We leave as an exercise to determine the best approximation as possible of
 the size of this number (for instance its number of digits).  For instance, if
we do not take into account the multiplications in the formula above,
we obtain that, in base $2$, the number $H_{\omega^3}(3)$ has at least
$2^{10^{21}}$  digits. But it is still an under-approximation !


\begin{Coqsrc}
End H_omega_cube_3.
\end{Coqsrc}




Now, we have got at last an exact formula for $H_{\omega^\omega}$.

\begin{Coqsrc}
Lemma H_Phi0_omega : forall k, H_ (Phi0 omega) k =
                               iterate H_succ_fun  k (H_ omega) k.
Proof with auto with E0.
  intro k; rewrite H_lim_eqn, <- H_Phi0_Si ...
  -  rewrite CanonS_Canon, CanonS_Phi0_lim;  f_equal ...
Qed.
\end{Coqsrc}

Using extensionality of the functional \texttt{iterate}, we can get a closed formula.

\begin{Coqsrc}
Lemma H_Phi0_omega_closed_formula k :
  H_ (Phi0 omega) k =
  iterate (fun (f: nat -> nat) (l : nat) => iterate  f (S l) l)
               k
               (fun k : nat => S (2 * k)%nat)
               k.
\end{Coqsrc}




Note that this short formula contains two occurences of the functional \texttt{iterate}, the outer one is in fact a second-order iteration (on type \texttt{nat -> nat)}
and the inner one  first-order (on type \texttt{nat}). 


\subsection{Abstract properties of H-functions}
~\label{sect:H-alpha-prop} 

Since pure computation seems to be useless for dealing with expressions of the form $H_\alpha(k)$, even for small values of $\alpha$ and $k$, we need to prove theorems for comparing $H_\alpha(k)$ and $H_\beta(l)$, in terms of comparison
between $\alpha$ and $\beta$ on the one hand, $k$ and $l$ on the other hand.

But beware of non-theorems! For instance, one could believe that $H$ is monotonous in its first argument. The following proof shows this is false.

\begin{Coqsrc}
Remark H_non_mono1 :
  ~ (forall alpha beta k, (alpha o<= beta)%e0 ->
                          (H_ alpha k <= H_ beta k)%nat).
Proof.
 intros H ;specialize (H 42 omega 3).
 assert (H0 :(42 o<= omega)%e0) by (repeat split; auto).  
 apply H in H0; rewrite H_Fin, H_omega  in H0; lia.
Qed.
\end{Coqsrc}

On the contrary, the fonctions of the Hardy hierarchy have the following five properties~\cite{KS81}: for any $\alpha < \epsilon_0$,
\begin{itemize}
\item the function $H_\alpha$ is strictly monotonous :
      For all $n,p \in\mathbb{N}, n < p \Rightarrow H_\alpha(n)< H_\alpha(p)$.
\item If $\alpha \not= 0$, then for every $n$, $n<H_\alpha(n)$.
\item The function $H_\alpha$ is pointwise less or equal than $H_{\alpha+1}$

\item For any $n\geq 1$, $H_\alpha(n)<H_{\alpha+1}(n)$.
\emph{We say that $H_{\alpha+1}$ dominates $H_\alpha$ from $1$}.
\item For any $n$ and $\beta$, if $\alpha \xrightarrow[n]{} \beta$, then
$H_\beta(n)\leq H_\alpha(n)$.
\end{itemize}

\index{Maths!Transfinite induction}

In \coq{}, we follow the  proof in~\cite{KS81}. This proof is mainly a single  proof by transfinite induction on $\alpha$ of the conjonction of the five properties.
For each $\alpha$, the three cases : $\alpha=0$, $\alpha$ is a limit, and 
$\alpha$ is a successor are considered. Inside each case, the five sub-properties are proved sequentially. 


\begin{definition}
 Let $f$ zand $g$ be two arithmetic  functions; $f$ is said to \emph{dominate} $g$ if $f(p)>g(p)$ for any all sufficiently large $p$.
\end{definition}


\begin{Coqsrc}
Section Proof_of_Abstract_Properties.
  Record P (alpha:E0) : Prop :=
    mkP {
        PA : strict_mono (H_ alpha);
        PB : alpha <> Zero -> forall n,  (n < H_ alpha n)%nat;
        PC : H_ alpha <<= H_ (Succ alpha);
        PD : dominates_from 1 (H_ (Succ alpha)) (H_ alpha);
        PE : forall beta n, Canon_plus n alpha beta -> 
                            (H_ beta n <= H_ alpha n)%nat}.


Theorem P_alpha : forall alpha, P alpha.
  Proof.
    intro alpha; apply well_founded_induction with lt.
   (* rest of proof skipped *)

Section Proof_of_Abstract_Properties.
\end{Coqsrc}


\subsection{Comparison between \texttt{L\_} and \texttt{H\_} }

By well-founded induction on $\alpha$, we prove that our $L$ hierachy is ``almost'' the Hardy hierarchy (up to a small shift).

\emph{From Module~\href{../src/html/hydras.Epsilon0.L_alpha.html\#H_L_}{Epsilon0.L\_alpha}}

\begin{Coqsrc}
 Theorem H_L_ alpha: forall i:nat,  (H_ alpha i <= L_ alpha (S i))%nat.
\end{Coqsrc}
 
\subsubsection{Back to hydras}

The following theorem relates the length of (standard) battles with the Hardy hierarchy.

\emph{From Module~\href{../src/html/hydras.Epsilon0.L_alpha.html}{Epsilon0.L\_alpha}}

\begin{Coqsrc}
Theorem battle_length_std_Hardy (alpha : E0) :
  alpha <> Zero ->
  forall k , 1 <= k -> exists l: nat,  
       H_ alpha k - k <= l /\
       battle_length standard k (iota (cnf alpha)) l.    
\end{Coqsrc}



\section{The Wainer hierarchy (functions \texorpdfstring{$F_\alpha$}{F\_alpha})}
\label{sect:wainer}

\index{Maths!Wainer Hierarchy}

The Wainer hierarchy~\cite{BW85, Wainer1970, KS81}, is also a family of fast growing functions, indexed by ordinals below $\epsilon_0$, by the following equations:

\label{F_equations}
\begin{itemize}
\item $F_0(i)=i+1$
\item $F_{\beta+1}(i)= (F_\beta)^{(i+1)}(i)$, where $f^{(i)}$ is the $i$-th iterate of $f$.
\item $F_\alpha(i) = F_{\canonseq{\alpha}{i}} (i)$ if $\alpha$ is a limit ordinal.
\end{itemize}

A first attempt is to write a definition of $F_\alpha$ by equations, in the same as for $H\_alpha$.  We use the functional \texttt{iterate} defined in 
Module~\href{../src/html/hydras.Prelude.Iterates.html\#iterate}{Prelude.Iterates}.

\begin{Coqsrc}
Fixpoint iterate {A:Type}(f : A -> A) (n: nat)(x:A) :=
  match n with
  | 0 => x
  | S p => f (iterate  f p x)
  end.
\end{Coqsrc}

The following code comes from 
 \url{../src/html/hydras.Epsilon0.F_alpha.html}.


\index{Coq!Plug-ins!Equations}
\begin{Coqsrc}
Fail Equations F_ (alpha: E0) (i:nat) :  nat  by wf  alpha Lt :=
  F_ alpha  i with E0_eq_dec alpha Zero :=
    { | left _ =>  i ;
      | right nonzero
          with Utils.dec (Limitb alpha) :=
          { | left _ =>  F_ (Canon alpha i)  i ;
            | right notlimit =>  iterate (F_ (Pred alpha))  (S i) i}}.
\end{Coqsrc}

\begin{Coqanswer}
The command has indeed failed with message:
In environment
alpha : E0
notlimit : Limitb alpha = false
nonzero : alpha <> Zero
i : nat
F_ : forall x : E0, nat -> x o< alpha -> nat
The term "F_ (Pred alpha) ?x" has type "Pred alpha o< alpha -> nat"
while it is expected to have type "Pred alpha o< alpha -> Pred alpha o< alpha"
(cannot unify "nat" and "Pred alpha o< alpha").
\end{Coqanswer}


We presume that this error comes from the recursive call of \texttt{F\_} inside
an application of \texttt{iterate}. The workaround we propose is to define first 
the iteration of \texttt{F\_}  as an helper $F^*$, then to define the function $F$ as a ``iterating $F^*$ once''.

\texttt{Equations} accepts the following definition, relying on  lexicographic ordering on pairs $(\alpha,n)$.


\label{sect:F-equations}

\index{Coq!Plug-ins!Equations}
\label{Functions:F-alpha}
\index{Maths! Fast growing functions}
  
\begin{Coqsrc}
Definition call_lt (c c' : E0 * nat) :=
  lexico Lt (Peano.lt) c c'.

Lemma call_lt_wf : well_founded call_lt.
  unfold call_lt; apply Inverse_Image.wf_inverse_image,  wf_lexico.
  -  apply E0.Lt_wf.
  -  unfold Peano.lt; apply Nat.lt_wf_0. 
Qed.

Instance WF : WellFounded call_lt := call_lt_wf.

Equations  F_star (c: E0 * nat) (i:nat) :  nat by wf  c call_lt :=
    F_star (alpha, 0) i := i;
    F_star (alpha, 1) i
      with E0_eq_dec alpha Zero :=
           { | left _ => S i ;
             | right nonzero
                 with Utils.dec (Limitb alpha) :=
                 { | left _ => F_star (Canon alpha i,1) i ;
                   | right notlimit =>
                     F_star (Pred alpha, S i)  i}};
    F_star (alpha,(S (S n))) i :=
               F_star (alpha, 1) (F_star (alpha, (S n)) i).

(* Finally, F_ alpha is defined as its first iterate ! *)

Definition F_  alpha i := F_star (alpha, 1) i.
\end{Coqsrc}

It is quite easy to prove that our function \texttt{F\_} satisfies the equations on page~\pageref{sect:F-equations}.

\begin{Coqsrc}
Lemma F_zero_eqn : forall i, F_ Zero i = S i.

Lemma F_lim_eqn : forall alpha i,  Limitb alpha ->
                               F_ alpha i = F_ (Canon alpha i) i.

Lemma F_succ_eqn : forall alpha i,
    F_ (Succ alpha) i = iterate (F_ alpha) (S i) i.
\end{Coqsrc}

As for the Hardy functions, we can use these equalities as rewrite rules for
``computing'' some values of $F_\alpha(i)$, for small values of $\alpha$.

\begin{Coqsrc}
Lemma LF1 : forall n,  F_ 1 n = S (2 * n).

Lemma LF2 : forall i, (exp2 i * i < F_ 2 i)%nat.
\end{Coqsrc}


Like in Sect~\ref{sect:H-alpha-prop}, we prove by induction the following properties (see~\cite{KS81}). 

\begin{Coqsrc}
Theorem F_alpha_mono alpha : strict_mono (F_ alpha).
 
Theorem F_alpha_ge_S alpha : forall n, (n < F_ alpha n)%nat.

Theorem F_alpha_Succ_le alpha : F_ alpha <<= F_ (Succ alpha).

Theorem F_alpha_dom alpha : dominates_from 1 (F_ (Succ alpha)) (F_ alpha).

Theorem F_alpha_beta alpha : forall beta n, Canon_plus n alpha beta -> 
                                        (F_ beta n <= F_ alpha n)%nat.
\end{Coqsrc}

As a corollary, we prove the following proposition, p. 284 of~\cite{KS81}.

\begin{quote}
  If $\beta<\alpha$, $F_\alpha$ dominates $F_\beta$.
\end{quote}

\begin{Coqsrc}
Lemma Propp284 : forall alpha beta : E0, 
   beta o< alpha -> dominates (F_ alpha) (F_ beta).
\end{Coqsrc}

\index{Exercises}

\begin{exercise}
Let us quote a theorem from ~\cite{KS81} (page 297).

\begin{quote}
\begin{align*}
  H_{\omega^\alpha}(n+1) &\geq F_{\alpha}(n) \quad (n\geq 1, \alpha<\epsilon_0) \\
 F_{\alpha}(n+1) &\geq H_{\omega^\alpha}(n) \quad (n\geq 1, \alpha<\epsilon_0) 
\end{align*}
Thus $H_{\omega^\alpha}$ and $F_{\alpha}$ have essentially the same order of growth.

\end{quote}

 But, before trying to prove these facts, look at the definition of function $H$ in Ketonen and Solovay's paper ! Is it really the same as the definition we quote from Pr{\H o}mel's chapter~\cite{Promel2013},
whereas \cite{KS81} define $H_\alpha(n)$ as ``the least integer $k$ such that $[n,k]$ is $\alpha$-large''. Thus, it may be useful to adapt the statement above.



\end{exercise}

\index{Exercises}

\begin{exercise}
Prove the following result~\cite{KS81}(p. 298).

\begin{quote}
   For $n\geq 2$ and $\alpha \geq 3$, $F_\alpha(n+1)\geq 2^{F_\alpha(n)}$.
\end{quote}
\end{exercise}



%  \subsection{Gnawing ordinals}

% \begin{definition}[After~\cite{KS81}]
%   Let $S$ be a finite set of positive integers, and $\alpha$ be an ordinal strictly less than $\epsilon_0$.  Let us denote by $s=s_1,s_2, \dots, s_N$ the sequence of 
% the elements of $S$, enumerated in strictly increasing order.

% We consider the sequence of ordinals $\alpha_o=\alpha, \alpha_1=\canonseq{\alpha_0}{s_1},\dots,\alpha_{i+1}=\canonseq{\alpha_i}{s_{i+1}},\dots, 
% \alpha_{N}=\canonseq{\alpha_{N-1}}{s_N}$.
% We denote by $\gnaw{s}{\alpha}$ the last ordinal of the sequence, \emph{i.e.}  $\alpha_N$.
% \end{definition}


% The following function computes $\gnaw{s}{\alpha}$ by recursion on $s$.
% \vspace{4pt}



% For instance, let us consider the ordinal $\omega^\omega$, and try some sequence of integers. 


% \begin{Coqsrc}
% Compute pp (gnaw (omega ^ omega) (1::nil)).  
% \end{Coqsrc}


% \begin{Coqanswer}
%   = omega%pT1 : ppT1
% \end{Coqanswer}


% \begin{Coqsrc}
% Compute pp (gnaw (omega ^ omega) (1::2::nil)).  
% \end{Coqsrc}

% \begin{Coqanswer}
% = P_fin 2
%      : ppT1
% \end{Coqanswer}

% Likewise, we can verify that $\gnaw{\omega^\omega}{\langle 1,2,3 \rangle}=1$
% and $\gnaw{\omega^\omega}{\langle 1,2,3,4 \rangle}=0$.

% \begin{Coqsrc}
% Example omega_omega_1_4 : gnaw (omega ^omega) (interval 1 4) = 0.
% Proof. trivial. Qed.

% Example omega_omega_1_3 : gnaw (omega ^omega) (interval 1 3) = 1.
% Proof. trivial. Qed.
% \end{Coqsrc}

% \begin{remark}
% \label{remark:gnaw-vs-battles}
% Let us consider an hydra battle, where Hercules always chops off the rightmost head of the hydra. Let $\alpha<\epsilon_0$ be an ordinal number, and 
% $s=\langle i_1<i_2<\dots<i_N\rangle$ be  any finite sequence of positive integers.
% Then the battle initiated by the hydra $\iota(\alpha)$, with $s$ as the sequence of successive replication factors, leads to the hydra $\iota(\gnaw{s}{\alpha})$ as the final state.
% \end{remark}


% \subsection{Large sequences}

% \begin{remark}
% In their article~\cite{KS81}, Ketonen and Solovay use the appellation ``large set'' instead of ``large sequence'', but their definitions use an enumeration of the elements in increasing order. Thus, we shall use the term ``sequence'' when referring to our implementation in \coq{}, and ``set'' when referring to the statements of~\cite{KS81}.
% \end{remark}


% \begin{definition}[After\cite{KS81}]
% The sequence $s$ is said to be \emph{$\alpha$-large} if $\gnaw{\alpha}{s}=0$.  
% \end{definition}


% \begin{Coqsrc}
% Definition largeb (alpha : T1) (s: list nat) :=
%   match gnaw alpha s with
%     | zero => true
%     | _ => false
%   end.


% Definition large (alpha : T1) (s : list nat) : Prop :=
%   largeb alpha s.
% \end{Coqsrc}




% For instance, the sequence $\langle 1,2,3,4 \rangle$ is $\omega^\omega$-large but not
% $\omega^{\omega+1}$-large, since $\gnaw{\omega^{\omega+1}}{\langle 1,2,3,4\rangle}$ is equal to $\omega \times 2 + 4$.


% %\omega^\omega+\omega^2\times 3 + \omega\times 4+ 5$.


% \begin{Coqsrc}
% Compute largeb (omega^omega) (interval 1 4).  
% \end{Coqsrc}

% \begin{Coqanswer}
% = true : bool
% \end{Coqanswer}

% \begin{Coqsrc}
% Compute largeb (omega^(omega+1)) (interval 1 4).
% \end{Coqsrc}

% \begin{Coqanswer}
% = false : bool
% \end{Coqanswer}


% \begin{remark}
% A sequence $s$ is $\alpha$-large if (still considering Hercules ``righmost-head'', tactic), it leads to Hercules' victory.
% \end{remark}


% \subsection{A little game}  

% Let $\alpha<\epsilon_0$ be an ordinal, and $i$ a positive integer. We want to guess the least natural number  $j$  such that the interval $[i,i+j]$ is $\alpha$-large.
% Equivalently, we should have $\gnaw{\alpha}{[i, i+j-1]}=1$.

% The following functions takes three arguments: an ordinal $\alpha$, and two positive natural numbers  $i$ and $j$ (we assume, but not verify that $i$ and $j$ are strictly positive). It returns one of the three possible answers:

% \begin{itemize}
% \item \texttt{Ok} if $j$ is the smallest integer such that the interval $[i,i+j]$ is $\alpha$-large
% \item \texttt{Too\_far} if $[i,i+j]$ is  $\alpha$-large, but $j$ is not the smallest such positive integer
% \item \texttt{(Remaining $\beta$)} if $j$ is too small, and gnawing $\alpha$ with
%   $[i,i+j]$ is still equal to $\beta$, instead of $0$
% \end{itemize}


% \vspace{4pt}
% \noindent
% \emph{From Module~\href{../src/html/hydras.Epsilon0.Large_Sets_Demo.html}{Ordinals.Epsilon0.Large\_Sets\_Demo}}   

% \begin{Coqsrc}
% Inductive answer : Set := You_won | Too_far | Remaining (rest : ppT1).

% Definition large_set_check alpha i j :=
%   let beta := gnaw alpha (interval i (Nat.pred j))
%   in match beta with
%      | one => Ok
%      | zero => Too_far
%      |  _ => Remaining (pp (canonseq j beta))
%      end.
% \end{Coqsrc}

% \subsubsection{\texorpdfstring{$\omega$}{omega}-large intervals}


% For instance, let us consider the ordinal $\omega$ and start with $i=1$.

% \begin{Coqsrc}
% Compute large_set_check omega 1 2.
% \end{Coqsrc}

% \begin{Coqanswer}
%      = Ok
%      : answer
% \end{Coqanswer}

% Let us give greater values of $i$, still with the ordinal $\omega$.

% \begin{Coqsrc}
% Compute large_set_check omega 2 3.
% \end{Coqsrc}


% \begin{Coqanswer}
%   = Remaining 1
%      : answer
% \end{Coqanswer}

% \begin{Coqsrc}
% Compute large_set_check omega 3 3.
% \end{Coqsrc}

% \begin{Coqanswer}
%   = Remaining 3
%      : answer
% \end{Coqanswer}

% \begin{Coqsrc}
% Compute large_set_check omega 2 4.
% \end{Coqsrc}


% \begin{Coqanswer}
%   = Ok
%      : answer
% \end{Coqanswer}

% \begin{Coqsrc}
% Compute large_set_check omega 3 6.
% \end{Coqsrc}

% \begin{Coqanswer}
%   = Ok
%      : answer
% \end{Coqanswer}

% It looks like every request to compute (\texttt{large\_set\_check omega $i$ $2\times i$})  will succeed. Let us try an example. 

% \begin{Coqsrc}
% Compute large_set_check omega 49 98.
% \end{Coqsrc}

% \begin{Coqanswer}
%   = Ok
%      : answer
% \end{Coqanswer}

% \subsubsection{\texorpdfstring{$\omega^2$}{omega\^{}2}-large intervals}

% Still using \texttt{Compute} and \texttt{large\_set\_check}, we obtained the 
% following values of $j$, the least integer such that the interval $[i,j]$ is $\omega^2$ large.

% $$
% \begin{array}{|c|c|}
% \hline
%   i & j \\
% \hline 
% 1 & 4 \\
% 2 & 14 \\
% 3 & 38 \\
% 4 & 94 \\
% 5 & 222 \\
% % 6 & 510 
% \hline
% \end{array}
% $$

% \begin{exercise}
% Please give the 6-th and 7-th line of the array above.
% \end{exercise}


% \subsubsection{The limits of (pure) computation}

% Let us now play with bigger ordinals, for instance $\alpha=\omega^\omega$ or
% $\alpha=\omega^{\omega + 1}$. We notice that, even for small values of $i$, it is hard
% to guess values of $j$ such that $[i,i+j]$ is $\alpha$-large.

% \begin{Coqsrc}
% Compute large_set_check (omega ^ omega) 1 4.
% \end{Coqsrc}

% \begin{Coqanswer}
%  = Ok
%      : answer
% \end{Coqanswer}

% \begin{Coqsrc}
% Compute large_set_check (omega ^ omega) 2 38.
% \end{Coqsrc}

% \begin{Coqanswer}
%  = Ok
%      : answer
% \end{Coqanswer}




% \begin{Coqsrc}
% Compute large_set_check (omega ^ omega) 3 1000.
% \end{Coqsrc}

% \begin{Coqanswer}
%   = Remaining (omega ^ 2 * 2 + omega * 220 + 798)%pT1
%      : answer
% \end{Coqanswer}


% \begin{Coqsrc}
% Compute large_set_check (omega ^ omega) 3 1798.
% \end{Coqsrc}

% \begin{Coqanswer}
%   = Remaining (omega ^ 2 * 2 + omega * 220)%pT1
%      : answer
% \end{Coqanswer}


% \begin{Coqsrc}
% Compute large_set_check (omega ^ omega) 3 5000.
% \end{Coqsrc}

% \begin{Coqanswer}
%   = Remaining (omega ^ 2 * 2 + omega * 218 + 2198)%pT1
%      : answer
% \end{Coqanswer}







% \begin{Coqsrc}
% Compute large_set_check (omega ^ (omega + 1)) 3 5000.
% \end{Coqsrc}

% \begin{Coqanswer}
%  = Remaining
%          (omega ^ omega * 2 + omega ^ 3 * 4 + 
%           omega ^ 2 * 4 + omega * 1148 +  4222)%pT1
%      : answer  
% \end{Coqanswer}

% \begin{Coqsrc}
% Compute large_set_check (omega ^ (omega + 1)) 3 10000.
% \end{Coqsrc}


% \begin{Coqanswer}
% Warning: To avoid stack overflow, large numbers in nat are interpreted as
% applications of Init.Nat.of_uint. [abstract-large-number,numbers]

%     = Remaining
%          (omega ^ omega * 2 + omega ^ 3 * 4 + 
%           omega ^ 2 * 4 + omega * 1147 +
%           8446)%pT1
%      : answer
% \end{Coqanswer}

% Since computation is not enough, let us comme back to proofs, or, better, proofs \emph{and} computations.

% \subsection{Proving largeness}



% %%% ICI %%%

% \subsection{$n$-large and $\omega$-large intervals }

% A finite ordinal is just made out of \texttt{zero} and \texttt{succ} constructions.
% Thus any sequence of strictly positive integers of length greater than  or equal to $n$ will completely gnaw the ordinal $n$.

% Concerning the first limit ordinal $\omega$ the largeness of a sequence $s$ depends only on its first element and the length of the rest of the sequence (please keep in mind that the argument $i$ of $\canonseq{\alpha}{i}$ is meaningful only if $\alpha$ is a limit ordinal).
% The following proposition is labeled $4.2$ in~\cite{KS81}.

% \begin{proposition}
%   For $n<\omega$, a set $X$id $n$-large if and only if $|X|\geq n$.
% A finite set $X$ is $\omega$-large if $|X|>\min{X}$.
% \end{proposition}

% Rewritten in terms of list of strictly positive integers, we get the following statements:

% \begin{Coqsrc}
% Lemma large_n_iff : forall s (n:nat) , ~ In 0 s  ->
%                                   large n s  <-> (n <= List.length s)%nat.
% \end{Coqsrc}


% \begin{Coqsrc}
% Lemma large_omega_iff : forall s n,  ~In 0 (n::s) -> 
%                                             large omega (n::s) <->
%                                             (n <=  List.length s)%nat.
% \end{Coqsrc}



% % \begin{Coqsrc}
% % (* sorted list of natural numbers greater than or equal to n *)

% % Inductive sorted_ge (n: nat) : list nat -> Prop :=
% % | sorted_ge_nil : sorted_ge n nil
% % | sorted_ge_one : forall p, n<=p ->
% %                             sorted_ge n (p::nil)
% % | sorted_ge_cons: forall p q s,  n<=p -> p<q ->
% %                                  sorted_ge p (q::s) ->
% %                                  sorted_ge n (p::q::s).
% % \end{Coqsrc}



% \subsection{Mimimal large sequences}



% Let us consider \emph{minimal} large sequences, \emph{i.e.} large sequences 
% the strict prefix of which do not lead to $0$. In other words, only the last ordinal 
% defined by the sequence is null.  For this purpose, we use the predicate \texttt{path\_to} introduced page~\pageref{path-to-definition}.

% \begin{Coqsrc}
% Definition mlarge alpha s := path_to zero s alpha.
% \end{Coqsrc}







%----------------------------------------------------------------------
\chapter[Countable Ordinals (after Sch\"{u}tte)]{Kurt Schütte's axiomatic Definition of countable Ordinals}

\label{chap:schutte} 
%ON

In the present chapter, we  compare our implementation of the segment $[0,\epsilon_0)$ with a mathematical text in order to ``validate'' our constructions.
We chose as reference the axiomatic definition of the set of countable ordinals,
in chapter V of Kurt Schütte's book `` Proof Theory ''~\cite{schutte}.

\begin{remark}
\emph{In all this chapter, the word ``ordinal'' will be considered as a synonymous of
``countable ordinal''}  
\end{remark}



Schütte's definition of countable ordinals relies on the following three axioms:

There  exists a strictly ordered set , such that
\begin{enumerate}
\item  $(\mathbb{O},<)$ is well-ordered
\item Every bounded subset of $\mathbb{O}$  is countable
\item Every countable subset of $\mathbb{O}$  is bounded.
\end{enumerate}

Starting with these three axioms, Schütte re-defines the vocabulary about ordinal numbers: the null ordinal $0$, limits and successors, the addition of ordinals, the infinite ordinals $\omega$, $\epsilon_0$, $\Gamma_0$, etc.

This chapter describes an adaptation to \coq{} of Schütte's axiomatization. 
 Unlike the rest of our libraries, our library
\href{../src/html/hydras.Schutte.Schutte.html}{Ordinals.Schutte}
is not constructive, and relies on several axioms.

\begin{itemize}
\item First, please keep in mind  that the set of countable ordinals is not countable. Thus, we cannot hope to represent all countable ordinals as finite terms of an inductive type, which was possible with  the set of ordinals strictly less than $\epsilon_0$ (resp. $\Gamma_0$)
\item We tried to be as close as possible to K. Schütte's text, which uses ``classical'' mathematics : excluded middle, Hilbert's $\epsilon$ (choice) and Russel's $\iota$ (definite description) operators. Both operators allow us to write definitions close to the natural mathematical language, such as ``$\textrm{succ}$ is \emph{the} least ordinal strictly greater than $\alpha$''
\item Please note that only the library \href{../src/html/hydras.Schutte.Schutte.html}{Schutte/*.v} is ``contaminated'' by axioms, and that the rest of our libraries remain constructive.
\end{itemize}

\section{Declarations and Axioms}

Let us declare a type 
\texttt{Ord} for representing countable ordinals, and a binary relation
 \texttt{lt}. Note that, in our development, \texttt{Ord} is a type, while the \emph{set} of countable ordinals (called $\mathbb{O}$ by Schütte) 
is the full set over the type \texttt{Ord}.

\label{types:Ord} 

We use Florian Hatat's library on countable sets, written as he was a student of  \emph{\'Ecole Normale Supérieure de Lyon}. A set $A$ is countable if there is an injective function from $A$ to $\mathbb{N}$ (see 
Library \href{../src/html/hydras.Schutte.Countable.html}%
{\texttt{Schutte.Countable}}).


\vspace{6pt}

\emph{From Module\href{../src/html/hydras.Schutte.Schutte_basics.html}%
{\texttt{Schutte.Schutte\_basics}}}

\begin{Coqsrc}
Parameter Ord : Type.
Parameter lt : relation Ord.
Infix "<" := lt : schutte_scope.

Definition ordinal := Full_set Ord.
\end{Coqsrc}

Schütte's first axiom tells that \texttt{lt} is a well order on the set 
\texttt{ordinal} (The  class \texttt{WO} is defined in
Module~\href{../src/html/hydras.Schutte.Well_Orders.html}%
{\texttt{Well\_Orders.v}}).

\label{types:WO}

\begin{Coqsrc}
Variables (M:Type)
         (Lt : relation M).
  
Class WO : Type:=
    {
      Lt_trans : Transitive  Lt;
      Lt_irreflexive : forall a:M, ~ (Lt a a);
      well_order : forall (X:Ensemble M)(a:M),
          In X a ->
          exists a0:M, least_member  X a0
    }.
\end{Coqsrc}



\begin{Coqsrc}
  Axiom AX1 : WO lt.
\end{Coqsrc}

The second and third axioms say that a subset $X$ of $\mathbb{O}$ is
(strictly) bounded if and only if it is countable. 



\begin{Coqsrc}
Axiom AX2 : forall X: Ensemble Ord, 
   (exists a,  (forall y, In X y -> y < a)) ->
   countable X.

Axiom AX3 : forall X : Ensemble Ord,
              countable X -> 
              exists a,  forall y, In X y -> y < a.
\end{Coqsrc}

\texttt{AX2} and \texttt{AX3} could have been replaced by a single axiom (using the \texttt{iff} connector), but we decide to respect as most as possible the structure of Schütte's definitions.

\section{Additional  axioms}

The adaptation of Schütte's mathematical discourse to \coq{} led us to
import a few axioms from the standard library. We encourage the reader to consult \coq{}'s FAQ about the safe use of axioms
 \url{https://github.com/coq/coq/wiki/The-Logic-of-Coq#axioms}.

\subsubsection{Classical logic}

In order to work with classical logic, we import the module
\href{https://coq.inria.fr/distrib/current/stdlib/Coq.Logic.Classical.html}{Coq.Logic.Classical}  of \coq{}'s standard library, specifially the following axiom:

\begin{Coqsrc}
 Axiom classic : forall P:Prop, P \/ ~P.
\end{Coqsrc}


\subsubsection{Description operators}

In order to respect Schütte's style, we imported also the library 
\href{https://coq.inria.fr/distrib/current/stdlib/Coq.Logic.Epsilon.html}{\texttt{Coq.Logic.Epsilon}}.  The rest of this section presents a few examples of
how Hilbert's choice operator and Church's definite description allow us
 to write understandable definitions (close to the mathematical natural language).


\subsubsection{The definition of zero}

According to the  definition of a well order, every non-empty subset of \texttt{Ord} has a least element. Furthermore, this least element is unique.


\begin{Coqsrc}
Remark R : exists! z : Ord, least_member lt  ordinal z.
Proof.
  destruct inh_Ord as [a]; apply (well_order (WO:=AX1)) with a .
  split.
Qed.
\end{Coqsrc}

Assume we want to call this element  \texttt{zero}.



\begin{Coqsrc}
Definition zero : Ord.
Proof.
  Fail destruct R.
\end{Coqsrc}

\begin{Coqanswer}
The command has indeed failed with message:
Case analysis on sort Type is not allowed for inductive 
definition ex.
\end{Coqanswer}


Indeed, the basic logic of  \coq{} does not allow us to eliminate a proof of a proposition 
$\exists\,x:A,\,P(x)$ for building a term whose type lies in the sort \texttt{Type}. 
The reasons for this impossibility are explained in many documents~\cite{BC04, chlipalacpdt2011, Coq}.

Let us import the library \texttt{Coq.Logic.Epsilon}, which contains the following axiom and lemmas.


\begin{Coqsrc}
Axiom epsilon_statement:
  forall (A : Type) (P : A->Prop), inhabited A ->
    {x : A | (exists x, P x) -> P x}.
\end{Coqsrc}

Hilbert's $\epsilon$ \emph{operator} is derived from this  axiom.

\begin{Coqsrc}
  Definition epsilon (A : Type) (i:inhabited A) (P : A->Prop) : A
  := proj1_sig (epsilon_statement P i).

Lemma constructive_indefinite_description :
  forall (A : Type) (P : A->Prop),
    (exists x, P x) -> { x : A | P x }.
\end{Coqsrc}




If we consider the \emph{unique existential} quantifier $\exists!$, we obtain
Church's \emph{definite description operator}.

\begin{Coqsrc}
Definition iota (A : Type) (i:inhabited A) (P : A->Prop) : A
  := proj1_sig (iota_statement P i).
\end{Coqsrc}


\begin{Coqsrc}
 Lemma constructive_definite_description :
  forall (A : Type) (P : A->Prop),
    (exists! x, P x) -> { x : A | P x }.
\end{Coqsrc}


\begin{Coqsrc}
Definition iota_spec (A : Type) (i:inhabited A) (P : A->Prop) :
  (exists! x:A, P x) -> P (iota i P)
  := proj2_sig (iota_statement P i).
\end{Coqsrc}



Indeed, the operators \texttt{epsilon} and \texttt{iota} allowed us to make our definitions 
quite close to Schütte's text. Our libraries \href{../src/html/hydras.Schutte.MoreEpsilonIota.html}%
{\texttt{Schutte.MoreEpsilonIota}}
and
\href{../src/html/hydras.Schutte.PartialFun.html}%
{\texttt{Schutte.PartialFun}} are extensions of \texttt{Coq.logic.Epsilon} for making easier 
such definitions. See also an article in french~\cite{PCiota}. 

\index{Coq!Type classes}

\begin{Coqsrc}
Class InH (A: Type) : Prop :=
   InHWit : inhabited A.

Definition some {A:Type} {H : InH A} (P: A -> Prop) := 
   epsilon (@InHWit A H) P.

Definition the {A:Type} {H : InH A} (P: A -> Prop) := 
   iota (@InHWit A H) P.
\end{Coqsrc}

In order to use these tools,  we had to tell \coq{}  that the type \texttt{Ord} is not empty:

\begin{Coqsrc}
Axiom inh_Ord : inhabited Ord.
\end{Coqsrc}


We are now able to define \texttt{zero} as the least ordinal. For this purpose,
we define a function returning the least element of any [non-empty]  subset.


\begin{Coqsrc}
Definition the_least {M: Type} {Lt}
           {inh : InH M} {WO: WO Lt} (X: Ensemble M)  : M :=
  the  (least_member  Lt X ).
\end{Coqsrc}


\vspace{4pt}

From Module \href{../src/html/hydras.Schutte.Schutte_basics.html}%
{\texttt{~Schutte.Schutte\_basics}}

\label{Constants:zero:Ord}

\begin{Coqsrc}
Definition zero: Ord :=the_least ordinal.
\end{Coqsrc}

We want to prove now that zero is less than or equal to any ordinal number.

\begin{Coqsrc}
Lemma zero_le (alpha : Ord) :  zero <= alpha.
Proof.
  unfold zero, the_least, the; apply iota_ind.
\end{Coqsrc}

According to the use of the description operator \texttt{iota}, we have to solve  two trivial sub-goals.
\begin{enumerate}
\item Prove that there exists a unique least member of \texttt{Ord}
\item Prove that being a least member of \texttt{Ord} entails the announced inequality 
\end{enumerate}


\begin{Coqanswer}
2 subgoals (ID 155)
  
  alpha : Ord
  ============================
  exists ! x : Ord, least_member lt ordinal x

subgoal 2 (ID 156) is:
 forall a : Ord, unique (least_member lt ordinal) a -> 
                a <= alpha
\end{Coqanswer}

\begin{Coqsrc}
  -  apply the_least_unicity, Inh_ord.
  -  destruct 1 as [[_ H1] _]; apply H1; split. 
Qed.
\end{Coqsrc}

\subsubsection{Remarks on \texttt{epsilon} and \texttt{iota}}

 What would happen in case of a misuse of \texttt{epsilon} or \texttt{iota} ?
For instance, one could give a unsatisfiable specification to \texttt{epsilon} or 
a specification for \texttt{iota} that admits several realizations.

Let us consider an example:

\begin{Coqbad}
Module Bad.

 Definition bottom := the_least (Empty_set Ord).
\end{Coqbad}

\begin{Coqanswer}
 bottom is defined
\end{Coqanswer}

Since we won't be able to prove the proposition
\linebreak \Verb|{exists! a: Ord, least_member (Empty_set Ord) a|, the only properties we would be able to prove about \texttt{bottom} would be \emph{trivial} properties, 
\emph{i.e.}, satisfied by \emph{any} element of type \texttt{Ord}, like for instance
\texttt{bottom = bottom}, or \texttt{zero <= bottom}.

\begin{Coqbad}
Lemma le_zero_bottom : zero <= bottom. 
Proof. apply zero_le. Qed.

Lemma bottom_eq : bottom = bottom.
Proof. trivial. Qed.

Lemma le_bottom_zero : bottom <= zero.
Proof.
   unfold bottom, the_least, the; apply iota_ind.
\end{Coqbad}

\begin{Coqanswer}
2 subgoals (ID 413)
  
  ============================
  exists ! x : Ord, least_member lt (Empty_set Ord) x

subgoal 2 (ID 414) is:
 forall a : Ord, unique (least_member lt (Empty_set Ord)) a -> 
      a <= zero
\end{Coqanswer}

\begin{Coqbad}
Abort.
End Bad.
\end{Coqbad}


In short, using \texttt{epsilon} and \texttt{iota} in our implementation of countable ordinals after Schütte has two main advantages.


\begin{itemize}
\item It allows us to give a \emph{name} (using \texttt{Definition}) two witnesses 
of existential quantifiers (let us recall that, in classical logic, one may consider non-constructive proofs of existential statements)
\item By separating definitions from proofs of [unique] existence, one may make definitions  more concise and readable. Look for instance at 
the definitions of  \texttt{zero}, \texttt{succ}, \texttt{plus}, etc. in the rest of this chapter.
\end{itemize}
%%%% ICI ICI 

\section{The  successor function}

The definition of the function \texttt{succ:Ord -> Ord} is very concise. The successor of any ordinal $\alpha$ is the smallest ordinal strictly greater than $\alpha$.

\label{Functions:succ-sch}

\begin{Coqsrc}
Definition succ (alpha : Ord) := the_least (fun beta => alpha < beta).
\end{Coqsrc}

Using \texttt{succ}, we define the folloing predicates.

\begin{Coqsrc}
Definition is_succ (alpha:Ord) := exists beta, alpha = succ beta.

Definition is_limit (alpha:Ord) := alpha <> zero /\ ~ is_succ alpha.
\end{Coqsrc}



% \begin{remark}
% Please look at remark~\vref{warning:coercions}.
% \end{remark}

How do we prove properties of the successor function?
First, we make its specification explicit.

\begin{Coqsrc}
Definition succ_spec (alpha:Ord) :=
  least_member   lt (fun z => alpha < z).
\end{Coqsrc}

Then, we prove that our function \texttt{succ} meets this specification. 


\begin{Coqsrc}
Lemma succ_ok : forall alpha,  succ_spec alpha  (succ alpha).
Proof.
  intros; unfold succ, the_least, the;  apply iota_spec.
\end{Coqsrc}

\begin{Coqanswer}
1 subgoal (ID 172)
  
  alpha : Ord
  ============================
  exists ! x : Ord, succ_spec alpha x
\end{Coqanswer}

We have now to prove that the set of all ordinals strictly greater than $\alpha$ has a unique least element. But the singleton set $\{\alpha\}$ is countable, hence  bounded (by the axiom \texttt{AX3}). Hence; the set $\{\beta\in\mathbb{O}|\alpha < \beta\}$ is not empty
and therefore has a unique least element.

The \coq{} proof script is quite short.

\begin{Coqsrc}
  destruct (@AX3 (Singleton _ alpha)).
  - apply countable_singleton.
  -  unfold succ_spec; apply the_least_unicity;  exists x; intuition.
Qed.     
\end{Coqsrc}


We can ``uncap'' the description operator for proving properties of the
\texttt{succ} function.

\begin{Coqsrc}
Lemma lt_succ (alpha : Ord) :  alpha < succ alpha.
Proof.
  destruct  (succ_ok  alpha);  tauto.
Qed.

Hint Resolve lt_succ : schutte.

Lemma lt_succ_le (alpha beta : Ord):
  alpha < beta -> succ alpha <= beta.
Proof with eauto with schutte.
  intros  H;  pattern (succ alpha); apply the_least_ok ... 
  exists (succ alpha); red;apply lt_succ ...
Qed.
\end{Coqsrc}


\begin{Coqsrc}
Lemma lt_succ_le_2 (alpha beta : Ord):
  alpha < succ beta -> alpha <= beta.

Lemma succ_mono (alpha beta : Ord):
  alpha < beta -> succ alpha < succ beta.

Lemma succ_monoR (alpha beta : Ord) :
 succ alpha < succ beta -> alpha < beta.

Lemma lt_succ_lt (alpha beta : Ord) :
  is_limit beta ->  alpha < beta -> succ alpha < beta.
\end{Coqsrc}

\section{Finite ordinals}

Using \texttt{succ}, it is now easy to define recursively all the finite ordinals.

\label{sect:notation-F-sch}

\begin{Coqsrc}
Reserved Notation "'F' n" (at level 29) .

Fixpoint finite (i:nat) : Ord :=
  match i with 
            | 0 => zero
            | S i => succ (F i)
  end
where "'F' i" := (finite i)  : schutte_scope.

Coercion finite : nat >-> Ord.
\end{Coqsrc}

\section{The definition of \texttt{omega}}
In order to define $\omega$, the first infinite ordinal, we use an operator which
``returns'' the least upper bound (if it exists) of a subset $X\subseteq \mathbb{O}$.
For that purpose, we first use a predicate:
(\texttt{is\_lub $D$ \textit{lt} $X$ $a$}) if $a$ belongs to $D$ and is the least 
upper bound  of $X$ (with respect to \textit{lt}).


\begin{Coqsrc}
Definition is_lub (M:Type)
                  (D : Ensemble M)
                  (lt : relation M)
                  (X:Ensemble M)
                  (a:M) :=
   In _ D a  /\ upper_bound  D lt X a  /\
   (forall y, In _ D y -> upper_bound  D lt X y  -> 
                  y = a \/ lt a y).
\end{Coqsrc}


\begin{Coqsrc}
Definition sup_spec X lambda := is_lub ordinal lt X lambda.

Definition sup (X: Ensemble Ord) : Ord  := the  (sup_spec X).

Notation "'|_|' X" := (sup X) (at level 29) : schutte_scope.
\end{Coqsrc}



Then, we define the function \texttt{omega\_limit} which returns the least upper bound 
of the  (denumerable) range of any sequence \texttt{s: nat -> Ord}. 
By \texttt{AX3} this range is bounded, hence the set of its upper bounds is not empty and has a least element.


\begin{Coqsrc}
Definition omega_limit (s:nat->Ord) : Ord 
  := |_| (seq_range s).
\end{Coqsrc}

Then we define \texttt{omega} as the limit of the sequence of finite ordinals.


\label{sect:notation-omega}
\begin{Coqsrc}
Definition _omega := omega_limit finite.

Notation "'omega'" := (_omega) : schutte_scope.
\end{Coqsrc}



Among the numerous properties of the ordinal $\omega$, les us quote the following ones
(proved in Module 
\href{../src/html/hydras.Schutte.Schutte_basics.html\#finite_lt_omega}{\texttt{Schutte.Schutte\_basics}})

\begin{Coqsrc}
Lemma finite_lt_omega : forall i: nat,  i < omega.

Lemma lt_omega_finite alpha : Ord) : 
  alpha < omega ->  exists i:nat, alpha =  i.

Lemma is_limit_omega : is_limit omega.
\end{Coqsrc}


\subsection{Ordering functions and ordinal addition}

After having defined the finite ordinals and the infinite ordinal $\omega$, we  define the sum $\alpha+\beta$ of two countable ordinals.
Schütte's definition looks like the following one:

\begin{quote}
``$\alpha+\beta$ is the $\beta$-th ordinal greater than or equal to $\alpha$''
\end{quote}


The purpose of this section is to give a meaning to the construction
``the $\alpha$-th element of $X$''  where $X$ is any non-empty subset of $\mathbb{O}$.
We follow Schütte's approach, by defining the notion of \emph{ordering functions},
a way to associate a unique ordinal to each element of $X$.
Complete definitions and proofs can be found in Module
 \href{../src/html/hydras.Schutte.Ordering_Functions.html}%
{\texttt{Schutte.Ordering\_Functions}} ).

\subsection{Definitions}

A \emph{segment} is a set $A$ of ordinals such that, whenever  $\alpha\in A$ and
$\beta<\alpha$, then $\beta\in A$; a segment is  \emph{proper} if it strictly included in $\mathbb{O}$.

\begin{Coqsrc}
 Definition segment (A: Ensemble Ord) :=
  forall alpha beta, In A alpha -> beta < alpha -> In A  beta.

Definition proper_segment (A: Ensemble Ord) :=
  segment A /\ ~ Same_set A ordinal.
\end{Coqsrc}


Let  $A$ be a segment, and $B$ a subset of $\mathbb{O}$ : an \emph{ordering function for $A$ and  $B$} is a strictly increasing bijection from $A$ to $B$.
The set $B$ is said to be an \emph{ordering segment} of $A$.
Our definition in \coq{} is a direct translation of the mathematical text of~\cite{schutte}.

\index{Maths!Ordering functions}

\begin{Coqsrc}
Definition ordering_function (f : Ord -> Ord)(A B : Ensemble Ord) :=
 segment A /\
 (forall a, In A a -> In B (f a)) /\
 (forall b, B b -> exists a, In A a /\ f a = b) /\
 forall a b, In A a -> In A b -> a < b ->  f a < f b.

Definition ordering_segment (A B : Ensemble Ord) :=
  exists f : Ord -> Ord, ordering_function f A B.
\end{Coqsrc}


We are now able to associate with any subset $B$ of $\mathbb{O}$ its ordering segment and ordering function.

\begin{Coqsrc}
Definition the_ordering_segment (B : Ensemble Ord) :=
  the  (fun x => ordering_segment x B).

Definition ord  (B : Ensemble Ord) := 
  some (fun f => ordering_function f (the_ordering_segment B) B).
\end{Coqsrc}

Thus (\texttt{ord $B \;\alpha$}) is the $\alpha$-th element of $B$.
Please note that the last definition uses the epsilon-based operator \texttt{some} and
not \texttt{the}. This is due to the fact that we cannot prove the unicity (w.r.t. Leibniz' equality) of the ordering function of a given set. 
By contrast, we admit the axiom  \texttt{Extensionality\_Ensembles}, from the library 
\href{https://coq.inria.fr/distrib/current/stdlib/Coq.Sets.Ensembles.html}{Coq.Sets.Ensembles}, so we use the operator \texttt{the} in the definition of
\texttt{the\_ordering\_segment}.

One of the main theorems of
\href{../src/html/hydras.Schutte.Ordering_Functions.html\#ordering_function_ex}%
{\texttt{Ordering\_Functions}} 
associates a unique segment and a unique (up to extensionality) ordering function to every subset $B$ of $\mathbb{O}$.

\begin{Coqsrc}
About ordering_function_ex.
\end{Coqsrc}

\begin{Coqanswer}
forall B : Ensemble Ord,
 exists ! S : Ensemble Ord, 
      exists f : Ord -> Ord, ordering_function f S B
\end{Coqanswer}


\begin{Coqanswer}
ordering_function_unicity :
forall (B S1 S2 : Ensemble Ord) (f1 f2 : Ord -> Ord),
ordering_function f1 S B ->
ordering_function f2 S2 B -> 
fun_equiv f1 f2 S1 S2
\end{Coqanswer}

Thus,  our function \texttt{ord}  which enumerates the elements of $B$ is defined in a non-ambiguous way.
Let us quote the following theorems (see Library
\href{../src/html/hydras.Schutte.Ordering_Functions.html}%
{\texttt{Schutte.Ordering\_Functions}} for more details).
 

\begin{Coqsrc}
Theorem ordering_le : forall f A B,
    ordering_function f A B ->
    forall alpha, In A alpha -> alpha <= f alpha.

Th_13_5_2 :
forall (A B : Ensemble Ord) (f : Ord -> Ord),
ordering_function f A B -> Closed B -> continuous f A B
\end{Coqsrc}


\subsection{Ordinal addition}

We are now ready to define and study addition on the type \texttt{Ord}.
The following definitions and proofs can be consulted in Module
\href{../src/html/hydras.Schutte.Addition.html}%
{\texttt{Schutte.Addition.v}}.

\begin{Coqsrc}
Definition plus alpha := ord  (ge alpha).
Notation "alpha + beta " := (plus alpha beta) : schutte_scope.
\end{Coqsrc}

In other words,  $\alpha + \beta$ is the  $\beta$-th ordinal greater than or equal to $\alpha$. 
Thanks to generic properties of ordering functions, we can show the following 
properties of addition on $\mathbb{O}$. First, we prove a useful lemma:

\begin{Coqsrc}
Lemma plus_elim (alpha : Ord) :
  forall P : (Ord->Ord)->Prop,
    (forall f: Ord->Ord, 
        ordering_function f ordinal (ge alpha)-> P f) ->
    P (plus alpha).
\end{Coqsrc}


\begin{Coqsrc}
Lemma alpha_plus_zero (alpha: Ord): alpha + zero = alpha.
Proof.
 pattern  (plus alpha); apply plus_elim; eauto.
 \end{Coqsrc}

 \begin{Coqanswer}
 1 subgoal (ID 24)
  
  alpha : Ord
  ============================
  forall f : Ord -> Ord,
  ordering_function f ordinal (ge alpha) -> 
  f zero = alpha
 \end{Coqanswer}

 \begin{Coqsrc}
 (* rest of proof skipped *)
 \end{Coqsrc}

The following lemmas are proved the same way.

 \begin{Coqsrc}

Lemma zero_plus_alpha (alpha : Ord) : zero + alpha = alpha.

Lemma le_plus_l (alpha beta : Ord) : alpha <= alpha + beta.

Lemma le_plus_r (alpha beta : Ord) :  beta <= alpha + beta.

Lemma plus_mono_r (alpha beta gamma : Ord) : 
    beta < gamma -> alpha + beta < alpha + gamma.

Lemma plus_of_succ (alpha beta : Ord) :
    alpha + (succ beta) = succ (alpha + beta).

Theorem plus_assoc (alpha beta gamma : Ord) :
  alpha + (beta + gamma) = (alpha + beta) + gamma.

Lemma one_plus_omega :  1 + omega = omega.

Lemma finite_plus_infinite (n : nat) (alpha : Ord) :
  omega <= alpha -> n + alpha = alpha.
\end{Coqsrc} 


It isinteresting to compare the proof of these lemmas with the 
computational proofs of the corresponding statements in Module
\href{../src/html/hydras.Epsilon0.T1.html}%
{\texttt{Epsilon0.T1}}. 
For instance, the proof of the lemma 
\texttt{one\_plus\_omega} uses the continuity of ordering functions (applied to  \texttt{(plus 1)}) and compares the limit of the $\omega$-sequences $i_{(i \in \mathbb{N})}$ and
$(1+i)i_{(i \in \mathbb{N})}$, whereas in the library  \texttt{Epsilon0/T1}, the equality 
$1+\omega=\omega$ is just proved with \texttt{reflexivity}!



\subsubsection{Multiplication by a natural number}

The multiplication of an ordinal by a natural number is defined in terms of addition.
This operation is useful for the study of Cantor normal forms.

\begin{Coqsrc}
Fixpoint mult_Sn (alpha:Ord)(n:nat){struct n} :Ord :=
 match n with 
            | 0 => alpha
            | S p => mult_Sn alpha p + alpha
 end.

Definition mult_n alpha n :=
  match n with
      0 => zero
    | S p => mult_Sn alpha p
  end.

Notation "alpha * n" := (mult_n alpha n) : schutte_scope.
\end{Coqsrc}

\section{The exponential of basis \texorpdfstring{$\omega$}{omega}}

In this section, we define the function which maps any $\alpha\in\mathbb{O}$ to
the ordinal  $\omega^\alpha$, also written 
$\varphi_0(\alpha)$. 
It is an opportunity to apply the definitions and results of the preceding section. 
Indeed,  Schütte first defines a subset of $\mathbb{O}$: the set of additive principal ordinals, and $\varphi_0$  is just defined as the ordering function of this set.

\subsection{Additive principal ordinals}

\index{Maths!Additive principal ordinals}

\begin{definition}
A non-zero ordinal  $\alpha$ is said to be \emph{additive principal} if, for all  $\beta<\alpha$, $\beta+\alpha$ is equal to  $\alpha$.
We call \texttt{AP} the set of additive principal ordinals.

\end{definition}



\noindent\emph{From Module \href{../src/html/hydras.Schutte.AP.html}%
{\texttt{Schutte.AP}}}

\begin{Coqsrc}
Definition AP : Ensemble Ord :=
  fun alpha => 
  zero < alpha /\
  (forall beta, beta < alpha ->  beta + alpha = alpha).
\end{Coqsrc}

\subsection{The function \texttt{phi0}}

Let us call  $\varphi_0$ the ordering function of \texttt{AP}.
In the mathematical text, we shall use indifferently the notations  $\omega^\alpha$ and$\varphi_0(\alpha)$. 


\begin{Coqsrc}
Definition phi0 := ord AP.

Notation "'omega^'" := phi0 (only parsing) : schutte_scope.
\end{Coqsrc}

\subsection{Omega-towers and the ordinal \texorpdfstring{$\epsilon_0$}{epsilon0}}


Using $\varphi_0$, we can define recursively the set of finite omega-towers.


\begin{Coqsrc}
Fixpoint omega_tower (i : nat) : Ord :=
  match i with
    0 =>  1
  | S j => phi0 (omega_tower j)
  end.
\end{Coqsrc}

\label{sect:epsilon0-as-limit}
Then, the ordinal  $\epsilon_0$ is defined as the limit of the sequence of all finite towers (a kind of infinite tower).

\begin{Coqsrc}
Definition epsilon0 := omega_limit omega_tower.
\end{Coqsrc}

The rest of our library \texttt{AP} is devoted to the proof of properties of additive principal ordinals, hence of the ordering function  $\varphi0$ and the ordinal $\epsilon_0$ (which we could not express within the type \texttt{T1}).

\subsection{Properties of the set  \texttt{AP}}

The set of additive principal ordinals is not empty: it contains at least the ordinals  $1$ and  $\omega$. 

\begin{Coqsrc}
Lemma AP_one : In AP 1.

Lemma AP_omega : In AP omega.
\end{Coqsrc}

Moreover, $1$ is the least principal ordinal and $\omega$ is the second element of
\texttt{AP}.


\begin{Coqsrc}
Lemma least_AP: least_member  lt AP 1. 

Lemma omega_second_AP :
  least_member   lt 
                  (fun alpha => 1 < alpha /\ In AP alpha)
                  omega.
\end{Coqsrc}

The set  \texttt{AP} is  \emph{closed} under addition, and unbounded.

\begin{Coqsrc}
Lemma AP_plus_closed (alpha beta gamma : Ord): 
     In AP alpha -> beta < alpha -> gamma < alpha -> beta + gamma < alpha.

Theorem AP_unbounded : Unbounded AP.
\end{Coqsrc}

Finally, \texttt{AP} is (topologically) \emph{closed} and ordered by the segment of all countable ordinals.

\index{Predicates!Closed}

\begin{Coqsrc} 
Definition Closed (B : Ensemble Ord) : Prop := 
  forall M, Included M B -> Inhabited _ M -> 
                 countable M -> In B (|_| M).
\end{Coqsrc}

\begin{Coqsrc}
Theorem AP_closed : Closed AP.

Lemma AP_o_segment :  the_ordering_segment AP = ordinal.
\end{Coqsrc}

\subsubsection{Properties of the function \texorpdfstring{$\varphi_0$}{phi0}}
 
The ordering function $\varphi_0$ of the set \texttt{AP} is defined on the full set $\mathbb{O}$ and is continuous (Schütte says that this function is \emph{normal}).

\begin{Coqsrc}
Theorem normal_phi0 : normal phi0 AP.
\end{Coqsrc}

The following properties come from  the definition of $\varphi_0$ as the ordering function of \texttt{AP}. It may be interesting to compare these proofs with the computational ones described in Chapter ~\ref{chap:T1}.

\begin{Coqsrc}
Lemma AP_phi0 (alpha : Ord) : In AP (phi0 alpha).

Lemma phi0_zero : phi0 zero =  1.

Lemma phi0_mono (alpha beta : Ord) :
  alpha < beta ->  phi0 alpha < phi0 beta.

Lemma phi0_inj (alpha beta : Ord) :
    phi0 alpha = phi0 beta -> alpha = beta.

Lemma phi0_sup : forall (U: Ensemble Ord),
   Inhabited _ U ->   countable U ->  phi0 (|_| U) = |_| (image U phi0).

Lemma is_limit_phi0 (alpha : Ord) :
  zero < alpha ->  is_limit (phi0 alpha).

Lemma omega_eq : omega = phi0 1. 

Lemma phi0_le (alpha : Ord) : alpha <= phi0 alpha.
\end{Coqsrc}

Please note that the lemma \texttt{omega\_eq} above, is consistent with the interpretation of the ordering function $\varphi_0$ as the exponential of basis $\omega$. Indeed we could have written this lemma with our alternative notation:

\begin{Coqsrc}
 Lemma omega_eq : omega = omega^ 1.
\end{Coqsrc}

\section{More about \texorpdfstring{$\epsilon_0$}{\texttt{epsilon0}}}

Let us recall that the limit ordinal  $\epsilon_0$ cannot be written within the type \texttt{T1}. Since we are now considering the set of all countable ordinals, we can now prove some properties of this ordinal.


We prove the inequality  $\alpha<\omega^\alpha$ whenever $\alpha < \epsilon_0$.
\emph{Note that this condition was implicit in Module \href{../src/html/hydras.Epsilon0/T1.html\#lt_phi0}%
{\texttt{Epsilon0.T1}}.}

\begin{Coqsrc}
Lemma lt_phi0 (alpha : Ord):
  alpha < epsilon0 -> alpha < phi0 alpha.
\end{Coqsrc}

The proof is as follows:
\begin{enumerate}
\item Since $\alpha<\epsilon_0$, consider the least $i$ such that $\alpha$ is strictly less than the omega-tower of height $i$.
\item
  \begin{itemize}
  \item If $i=0$, then the result is trivial (because $\alpha=0$)
 \item  Otherwise let $i=j+1$; 
          $\alpha$ is greater than or equal to the omega-tower of height $j$.
         By monotonicity,  $\varphi_0(\alpha)$ is greater than or equal to 
        the omega-tower of height $j+1$, thus strictly greater than $\alpha$
  \end{itemize}
 \end{enumerate}

Moreover,  $\epsilon_0$ is the least ordinal $\alpha$ that verifies the equality 
$\alpha = \omega^\alpha$, in other words the least fixpoint of the function  $\varphi_0$.

\begin{Coqsrc}
Theorem epsilon0_lfp : least_fixpoint lt phi0 epsilon0.
\end{Coqsrc}


\section{Critical ordinals}

\index{Maths!Critical ordinals}

For any  (countable) ordinal $\alpha$, the set $\textit{Cr}(\alpha)$ is inductively defined 
as follows by Schütte (p.81 of~\cite{schutte}).

\begin{quote}
  \begin{itemize}
  \item $\textit{Cr}(0)$ is the set \textit{AP} of additive principal ordinals.
  \item If $0<\alpha$, then $\textit{Cr}(\alpha)$ is the intersection of all the sets of fixpoints of the $\textit{Cr}(\beta)$ for $\beta<\alpha$.
  \end{itemize}
\end{quote}

This definition is translated in \coq{} in 
Module \href{../src/html/hydras.Schutte.Critical.html}%
{\texttt{Schutte.Critical}}, as the least fixpoint of a functional. 


\begin{Coqsrc}
Definition Cr_fun : forall alpha : Ord,
       (forall beta : Ord, beta < alpha -> Ensemble Ord) ->
        Ensemble Ord 
:= 
   fun (alpha :Ord)
        (Cr : forall beta, 
                beta < alpha -> Ensemble Ord) 
        (x : Ord) => (
       (alpha = zero /\ AP x) \/
       (zero < alpha /\
        forall beta (H:beta < alpha),
          the_ordering_segment (Cr beta H) x /\ ord (Cr  beta H) x = x)).

Definition Cr (alpha : Ord) : Ensemble Ord := 
    (Fix  all_ord_acc (fun (_:Ord) => Ensemble Ord) Cr_fun) alpha.
\end{Coqsrc}


\begin{Coqsrc}
Definition phi (alpha : Ord) : Ord -> Ord 
    :=  ord (Cr alpha).

Definition A (alpha : Ord) : Ensemble Ord :=
  the_ordering_segment (Cr alpha).
\end{Coqsrc}

For instance,  we prove that $\textit{Cr}(0)$ is the set of additive principals and that $\epsilon_0$
belongs to $\textit{Cr}(1)$.

\begin{Coqsrc}
Lemma Cr_zero_AP :  Cr 0 = AP

Lemma epsilon0_Cr1 : In (Cr 1) epsilon0.
\end{Coqsrc}

\index{Exercises}

\begin{exercise}
 Prove that $\epsilon_0$ is the least element of $\textit{Cr}(1)$.
\end{exercise}


\subsection{A flavor of infinity}



The family of the $\textit{Cr}(\alpha)$s is made of infinitely many unbounded (hence infinite) sets.
Let us quote Lemma 5, p. 82  of~\cite{schutte}:
\begin{quote}
  For all $\alpha$, the set $\textit{Cr}(\alpha)$ is closed (for the least upper bound of non-empty countable sets) and unbounded.
\end{quote}

We prove this result by transfinite induction on $\alpha$ of both properties.

The proof is still quite long, by transfinite induction over $\alpha$.

\index{Coq!Techniques!Simultaneous induction}
\index{Coq!Techniques!Transfinite induction}

\begin{Coqsrc}
Section Proof_of_Lemma5.
  Let P (alpha:Ord) := Unbounded (Cr alpha) /\ Closed (Cr alpha).
 
 Lemma Lemma5 : forall alpha, P alpha.
(* ... *)
 End Proof_of_Lemma5.

Corollary Unbounded_Cr alpha : Unbounded (Cr alpha).
Proof.
  now destruct (Lemma5 alpha).
Qed.

Corollary Closed_Cr alpha : Closed (Cr alpha).
Proof.
  now destruct (Lemma5 alpha).
Qed.
\end{Coqsrc}

\section{Cantor normal form}

The notion of Cantor normal form is defined for all countable ordinals.
Nevertheless, note that, contrary to the implementation based on type \texttt{T1},
the Cantor normal form of an ordinal $\alpha$ may contain $\alpha$ as a 
sub-term\footnote{This would prevent us from trying to represent Cantor normal forms as finite trees (like in Sect.~\ref{sec:T1-inductive-def})}.


We represent  Cantor normal forms as lists of ordinals.
A  list $l$ is a Cantor normal form of a given ordinal $\alpha$ if it satisfies two conditions:



\begin{itemize}
\item The list  $l$ is sorted (in decreasing order) w.r.t. the order $\leq$
\item The sum of all the  $\omega^{\beta_i}$ where the $\beta_i$ are the terms of $l$ (in this order) is equal to $\alpha$.
\end{itemize}



\vspace{4pt}

\noindent\emph{From \href{../src/html/hydras.Schutte.CNF.html\#cnf_t}%
{\texttt{Schutte.CNF}}}


\begin{Coqsrc}
 Definition cnf_t := list Ord.

Fixpoint eval (l : cnf_t) : Ord :=
  match l with nil => zero
              | beta :: l' => phi0 beta + eval l'
  end.

Definition sorted (l: cnf_t) :=
  LocallySorted (fun alpha beta => beta <= alpha) l.

Definition is_cnf_of (alpha : Ord)(l : cnf_t) : Prop :=
  sorted l /\ alpha = eval l.
\end{Coqsrc}

\index{Maths!Transfinite induction}

By transfinite induction on $\alpha$, we prove that every countable ordinal $\alpha$ 
 has at least a Cantor normal form.

\begin{Coqsrc}
Theorem cnf_exists (alpha : Ord) :
  exists l: cnf_t, is_cnf_of alpha l.
\end{Coqsrc}

By structural induction on lists, we prove that this normal form is unique.

\begin{Coqsrc}
 Lemma cnf_unicity : forall l alpha, 
   is_cnf_of alpha l -> 
   forall l',  is_cnf_of alpha l' -> l=l'.
Proof.
 induction l.
 (*  ...  *)

Theorem cnf_exists_unique (alpha:Ord) :
  exists! l: cnf_t, is_cnf_of alpha l.
\end{Coqsrc}


Finally, the following two lemmas relate  $\epsilon_0$ with Cantor normal forms.

If $\alpha<\epsilon_0$, then the Cantor normal form of $\alpha$ is made of ordinals strictly less than $\alpha$.

\begin{Coqsrc}
Lemma cnf_lt_epsilon0 : 
 forall l alpha, 
   is_cnf_of alpha l ->  alpha < epsilon0 ->
   Forall (fun beta =>  beta < alpha) l.
\end{Coqsrc}

\index{Exercises}

\begin{exercise}
Please consider the following statement :

\begin{Coqsrc}
Lemma cnf_lt_epsilon0_iff : 
 forall l alpha, 
   is_cnf_of alpha l ->  
   (alpha < epsilon0 <->  Forall (fun beta =>  beta < alpha) l).
\end{Coqsrc}

Is it true ?
\end{exercise}

Finally, the Cantor normal form of $\epsilon_0$ is just $\omega^{\epsilon_0}$.

\begin{Coqsrc}
Lemma cnf_of_epsilon0 : is_cnf_of epsilon0 (epsilon0 :: nil).
Proof.
  split.
  - constructor.  
  - simpl;  now rewrite alpha_plus_zero, epsilon0_fxp.
Qed.
\end{Coqsrc}

\index{Projects}

\begin{project}
Implement pages 82 to 85 of~\cite{schutte} (critical, strongly critical, maximal critical ordinals, Feferman's ordinal $\Gamma_0$).
\end{project}

\begin{remark}
The sub-directory \href{../src/html/hydras.Gamma0.html}%
{\texttt{src/Gamma0}} contains an (incomplete, still undocumented) implementation of the set of ordinals below $\Gamma_0$, represented in Veblen normal form. 
\end{remark}

\section{An embedding of \texttt{T1} into \texttt{Ord}}


Our library 
\href{../src/html/hydras.Schutte.correctness_E0.html}%
{\texttt{Schutte.Correctness\_E0}} establishes the link between two very different modelizations of ordinal numbers. In other words, it ``validates'' a data structure in terms of
a classical mathematical discourse considered as a model. 
First, we define a function from \texttt{T1} into  \texttt{Ord} by structural recursion.

\begin{Coqsrc}
Fixpoint inject (t:T1) : Ord :=
 match t with 
     | T1.zero => zero
     | T1.ocons a n b =>  AP.phi0 (inject a) * S n + inject b
 end.  
\end{Coqsrc}

This function enjoys good commutation properties with respect to the main operations which
allow us to build Cantor normal form.

\begin{Coqsrc}
Theorem inject_of_zero : inject T1.zero = zero.

Theorem inject_of_finite (n : nat):
  inject (T1.fin n) =  n.

Theorem inject_of_phi0 (alpha : T1):
  inject (phi0 alpha) = AP.phi0 (inject alpha).

Theorem inject_plus (alpha beta : T1): nf alpha -> nf beta ->
  inject (alpha + beta)%t1 = inject alpha + inject beta.

Theorem inject_mult_n (alpha : T1)  :
  nf alpha -> forall n:nat , inject (alpha *  n)%t1 =  inject alpha * n.

Theorem inject_mono (beta gamma : T1) :
  T1.lt  beta gamma -> 
  T1.nf beta -> T1.nf gamma -> 
  inject beta < inject gamma.

Theorem inject_injective (beta gamma : T1) : nf beta -> nf gamma ->
  inject beta = inject gamma -> beta = gamma.
\end{Coqsrc}

Finally, we prove that \texttt{inject} is a bijection from the set of all terms of \texttt{T1} in normal form to the set 
\texttt{members epsilon0} of the elements of \texttt{Ord} strictly less than  $\epsilon_0$.

\begin{Coqsrc}
Theorem inject_lt_epsilon0 (alpha : T1):
      inject alpha < epsilon0.

Theorem embedding : 
     fun_bijection (nf: Ensemble T1)  (members epsilon0) inject.
 \end{Coqsrc}

\subsection{Remarks}
Let us recall that the library \href{../src/html/hydras.Schutte.Schutte.html}%
{\texttt{Schutte}} depends on five \emph{axioms} and lies explicitly in the  
framework of classical logic with a weak version of the axiom of choice
(please look at the documentation of
\href{https://coq.inria.fr/distrib/current/stdlib/Coq.Logic.ChoiceFacts.html}{\texttt{Coq.Logic.ChoiceFacts}}).
Nevertheless, the other modules:
\href{../src/html/hydras.Epsilon0.Epsilon0.html}%
{\texttt{Epsilon0}},
\href{../src/html/hydras.Hydra.Hydra.html}%
{\texttt{Hydra}}, et 
\href{../src/html/hydras.Gamma0.Gamma0.html}%
{\texttt{Gamma0}}
do not import any axioms and are really constructive.

\index{Projects}
\begin{project}
There is no construction of ordinal multiplication in~\cite{schutte}. 
It would be interesting to derive this operation from Schütte's axioms,
and prove its consistence with multiplication in ordinal notations for 
$\epsilon_0$ and $\Gamma_0$.
\end{project}

\section{Related work}

In~\cite{grimm:hal-00911710}, José Grimm establishes the consistency between or ordinal notations (\texttt{T1} and \texttt{T2} (Veblen normal form) and his implementation
of ordinal numbers after Bourbaki's set theory.


\chapter{The ordinal \texorpdfstring{$\Gamma_0$}{Gamma0}}

\textbf{To write!}



\begin{figure}[h]
  \centering
  \includegraphics[width=11cm]{epsilon0.jpg}
  \caption{Discrete mathematics and formal proofs}
  \label{fig:gamma0}
\end{figure}


\chapter{Appendices}

\bibliographystyle{alpha}
\bibliography{thebib}

\section{Future work (projects)}
\index{Projects}

This document and the proof scripts are far from being complete.

First, there must be a lot of typos to correct, references and index items to add. Many proofs are too complex and should be simplified, etc.

The following extensions are planned, but help is needed:

\begin{itemize}
\item Semi automatic tactics for proving inequalities $\alpha < \beta$, even when $\alpha$ and $\beta$ are not closed terms.
\item Extension to $\Gamma_0$ (in Veblen normal form)
\item More lemmas about hierarchies of rapidly growing functions, and their relationship 
    with primitive recursive functions and provability in Peano arithmetic 
(following~\cite{KS81, KP82}).
\item From \coq's point of view, this development could be used as an illustration of the evolution of the software, every time new libraries and sets of tactics could help to simplify the proofs.
\end{itemize}



\section{How to install the libraries}
\label{sec:orgheadline4}
\begin{itemize}
\item The present distribution has been checked with version 8.11.0 of the Coq proof assistant, with the plug-ins \texttt{coq-paramcoq} and \texttt{coq-equations}.

\item just go into the \texttt{src} directory, and type "\texttt{make}"
\end{itemize}

\section{Comments on exercises and projects}

Although we do not plan to include complete solutions to the exercises, 
we think it would be useful to include comments and hints, and questions/answers from the users. In constrast, ``projects'' are supposed, once completed, to be included in the repository.




\section{Index}






\begin{table}[h]
  \centering
  \begin{threeparttable}
    \caption{Main types and type classes}
\begin{tabular}{|c|c|c|l|}
\hline
Name & Description& Page \\\hline
Battle & Type class of battle types & \pageref{types:Battle} \\
E0 & (Well formed) ordinals below $\epsilon_0$ & \pageref{types:E0}\\
Hvariant & variants for proving termination of hydra battles & \pageref{sect:hvariant-def} \\
Hydra & tree-like representation of hydras & \pageref{types:Hydra} \\
Hydrae & finite sequences of hydras & \pageref{types:Hydrae} \\
ON & ordinal  notations & \pageref{types:ON} \\
ON\_for & comparison of an ordinal notation with Schütte's model &  \pageref{types:ON-for}\\
ON\_iso & isomorphism of ordinal notations  & \pageref{types:ON-iso}\\

Ord & countable ordinals (after Schütte) & \pageref{types:Ord} \\
ppT1 & pretty printed version of T1 & \pageref{types:ppT1} \\
SubON & comparison of ordinal notations & \pageref{types:SubON} \\
T1 & ordinal terms below $\epsilon_0$ & \pageref{types:T1} \\
T2 & ordinal terms below $\Gamma_0$ & \pageref{types:T2} \\
WO & well orders (Schütte's definition) & \pageref{types:WO} \\
\hline
\end{tabular}
\end{threeparttable}
\end{table}

\begin{table}[h]
  \centering
  \begin{threeparttable}
    \caption{Main functions and constants}
\begin{tabular}{|r | c|c|c|c|l|}
\hline
Name & Gallina & Math & Description & Page \\ \hline
canon & canon alpha i & $\canonseq{\alpha}{i}$& Canonical sequence & \pageref{Functions:canon}\\
canonS & canonS alpha i & $\canonseq{\alpha}{i+1}$& Helper for canon & \pageref{Functions:canonS}\\
F\_ & F\_ alpha n & $F_\alpha(n)$ & Wainer's \texttt{F} fast growing hierarchy & \pageref{Functions:F-alpha} \\
H\_ & H\_ alpha n & $H_\alpha(n)$ & Hardy's \texttt{H} fast growing hierarchy & \pageref{Functions:H-alpha} \\
iterate & iterate $f$ $n$ $x$ & $f^{(n)}(x)$ & Functional iteration & \pageref{Functions:iterate} \\
L\_ & L\_ alpha k & $L_\alpha(k)$ & final step of a standard path & \pageref{Functions:L-alpha} \\
succ & succ alpha & & Successor & \pageref{Functions:succ-T1}, \pageref{Functions:succ-sch} \dots \\
zero:Ord & & $0$ & The least ordinal (Schütte's model) & \pageref{Constants:zero:Ord} \\


\hline
\end{tabular}
\end{threeparttable}
\end{table}



\begin{table}[h]
  \centering
  \begin{threeparttable}
    \caption{Main predicates}
\begin{tabular}{|r | c|c|c|c|l|}
\hline
Name & Gallina&Math& Description& Page \\\hline
\texttt{lt : T1->T1->Prop}& lt alpha beta & $\alpha < \beta$& strict order on type \texttt{T1} \tnote{1} & \pageref{Predicates:lt-T1}\\
\texttt{LT: T1->T1->Prop}& alpha o< beta & $\alpha < \beta$& strict order on type \texttt{T1}   \tnote{2} & \pageref{Predicates:LT-T1}\\
\texttt{Lt : E0->E0->Prop} & alpha o< beta & $\alpha < \beta$& strict order on type \texttt{E0} \tnote{3} & \pageref{Predicates:Lt-E0} \\
\texttt{nf: T1->Prop} & \texttt{nf alpha} && alpha is in Cantor normal form & \pageref{Predicates:nf-T1}\\
\hline
\end{tabular}
\begin{tablenotes}
  \item[1] This order is total, but not well-founded, because of not well formed terms.
\item[2] Restriction of \texttt{lt} to terms in normal form; this order is partial, but well-founded.
\item[3] This order is total \emph{and} well-founded.
\end{tablenotes}
\end{threeparttable}
\end{table}



\vspace{4pt}
\begin{table}[h]
  \centering
  \begin{threeparttable}
    \caption{Infix notations}
\begin{tabular}{|c|c|c|c|l|}
\hline
Name & Gallina&Math& Description& Page \\\hline
 \texttt{on\_lt} & \texttt{alpha o< beta}&$\alpha<\beta$& ordinal inequality \tnote{1} & \pageref{sect:on-lt-notation}\\
 \texttt{on\_le} & \texttt{alpha o<= beta}&$\alpha\leq\beta$& ordinal inequality & \pageref{sect:on-lt-notation}\\
\texttt{plus} & \texttt{alpha + beta} & $\alpha + \beta$ & ordinal addition & \pageref{sect:infix-plus-T1} ,\dots\\
\texttt{oplus} & \texttt{alpha o+ beta} & $\alpha \oplus \beta$ & Hessenberg sum & \pageref{sect:infix-oplus} \\
\texttt{round} & \texttt{h -1-> h'} & & one round of a battle & \pageref{sect:infix-round} \\
\texttt{rounds} & \texttt{h -+-> h'} & & one or more  rounds of a battle & \pageref{sect:infix-rounds} \\
\texttt{round\_star} & \texttt{h -*-> h'} & & any number of rounds of a battle & \pageref{sect:infix-rounds} \\

\hline
\end{tabular}

\begin{tablenotes}
 \item [1]
Some notations may belong to several scopes. For instance, ``\texttt{o<}'' is
bound in \texttt{ON\_scope}, \texttt{E0\_scope}, \texttt{t1\_scope}, etc., and locally in several libraries.
\end{tablenotes}
\end{threeparttable}
\end{table}


\begin{table}[h]
  \centering
  \begin{threeparttable}
    \caption{Other notations}
\begin{tabular}{|c|c|c|c|l|}
\hline
Name & Gallina&Math& Description& Page \\\hline
F & \texttt{F $n$} & $n$ & The $n$-th finite ordinal &  
\pageref{sect:notation-F}, \pageref{sect:notation-F-sch}\\ 
FS & \texttt{FS $n$} & $n+1$ & The $n+1$-th finite ordinal  \tnote{2} &  
\pageref{sect:notation-FS}\\ 
omega & \texttt{omega} & $\omega$ &   the first infinite ordinal   & \pageref{sect:notation-omega}, \pageref{sect:omega-T1}, \pageref{sect:omega-notation2}, \dots\\
phi0     & \texttt{phi0 alpha} & $\varphi_0(\alpha),\; \omega^\alpha$&exponential of base $\omega$ & \pageref{sect:notation-phi0}\\

\hline
\end{tabular}

\begin{tablenotes}
  \item [2] Note that there exist also various coercions from \texttt{nat} to types of ordinal. Depending on the current scope and  \coq's syntactic analysis algorithm, \texttt{F} may be left implicit.
\end{tablenotes}
  \end{threeparttable}
 
\end{table}







\printindex

% \section{Main notations}

% \subsubsection*{Intervals}

% Let $\leq$ be a total order on a type $A$. The intervals $\{k\in A| i\leq k\leq j\}$ is written $[i,j]$. Likewise, $\{k\in A| i\leq k<j\}$ is written
% $[i,j[$.


\end{document} 
