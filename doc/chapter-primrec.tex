\chapter{Primitive Recursive Functions}

\section{Introduction}
The definition of primitive recursive functions we use come
from Russel O'Connors formalization in \coq{} of
G\"odel's incompleteness theorems~\cite{OConnor05}, now hosted in
the \texttt{theories/ordinals/Ackermann} directory.

This chapter contains some comments on Russel's library, as well as a few extensions.

\section{First look at the Ackermann library}

O'Connor's library on Gödel's incompleteness theorems contains a little more 
than 45K lines of scripts. The part dedicated to primitive recursive functions and Peano arithmetics is 32K lines long and is originally structured in 38 modules.

Thus, we propose a slow exploration of this library, through examples and exercises.

\subsection{Basic Libraries}

The formal definition of primitive recursive functions lies in the library
\href{../theories/html/hydras.Ackermann.primRec.html}{Ackermann.primRec},
with preliminary definitions in 
\href{../theories/html/hydras.Ackermann.extEqualNat.html}{Ackermann.extEqualNat}
and
\href{../theories/html/hydras.Ackermann.misc.html}{Ackermann.misc}.

\subsubsection{Functions of arbitrary arity}

The library allows us to consider primitive functions on \texttt{nat}, with any number of arguments, in 
curried form. This is made possible in 
\href{../theories/html/hydras.Ackermann.extEqualNat.html}{Ackermann.extEqualNat} by the following definition:

\begin{Coqsrc}
Fixpoint naryFunc (n : nat) : Set :=
  match n with
  | O => nat
  | S n => nat -> naryFunc n
  end.
\end{Coqsrc}

For instance (\texttt{naryFunc 1}) is convertible to \texttt{nat -> nat} and (\texttt{naryFunc 3})
to \texttt{nat -> nat -> nat -> nat}.

\vspace{4pt}
\noindent
\emph{From\href{../theories/html/hydras.Ackermann.PrimRecExamples.html}{Ackermann.PrimRecExamples}}.
\begin{Coqsrc}
Require Import primRec.
Import extEqualNat.

Compute naryFunc 3.
\end{Coqsrc}

\begin{Coqanswer}
= nat -> nat -> nat -> nat
  : Set  
\end{Coqanswer}

Thus, the function \texttt{plus} has also type (\texttt{naryFunc 2}).

Likewise, arbitrary boolean predicates may have an arbitrary number of arguments. The dependent type
(\texttt{naryRel $n$}), defined in the same way as \texttt{naryFunc}, is the type of $n$-ary functions from
\texttt{nat} into \texttt{bool}.

\begin{Coqsrc}
Compute naryRel 2.
\end{Coqsrc}

\begin{Coqanswer}
 = nat -> nat -> bool
     : Set
\end{Coqanswer}

The magic of dependent types makes it possible to define recursively extensionnal equality between functions of the same arity.

\vspace{4pt}
\noindent
\emph{From \href{../theories/html/hydras.Ackermann.extEqualNat.html}{Ackermann.extEqualNat}}
\begin{Coqsrc}
Fixpoint  extEqual (n : nat) : forall  (a b : naryFunc n), Prop :=
  match n with
    0 => fun a b => a = b
  | S p => fun a b => forall c, extEqual p (a c) (b c)
  end.
\end{Coqsrc}

\begin{Coqsrc}
Compute extEqual 2.
\end{Coqsrc}

\begin{Coqanswer}
     = fun a b : naryFunc 2 => forall x x0 : nat, a x x0 = b x x0
     : naryFunc 2 -> naryFunc 2 -> Prop
 \end{Coqanswer}
 
\begin{Coqsrc}
Example extEqual_ex1 : extEqual 2 mult (fun x y =>  y * x + x -x) .
Proof.
  intros x y.
\end{Coqsrc}

\begin{Coqanswer}
  x, y : nat
  ============================
  extEqual 0 (x * y) (y * x)
\end{Coqanswer}

\begin{Coqsrc}
  cbn.
\end{Coqsrc}

\begin{Coqanswer}
1 subgoal (ID 10)
  
  x, y : nat
  ============================
  x * y = y * x + x - x
\end{Coqanswer}

\begin{Coqsrc}
  rewrite <- Nat.add_sub_assoc, Nat.sub_diag.
  - ring.
  - apply le_n.  
Qed.
\end{Coqsrc}