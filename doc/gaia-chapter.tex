\chapter{Gaia and the hydras (draft)}

\section{Introduction}


\section{Data refinement}

\emph{The following definitions and lemmas are in
\href{../theories/html/gaia_hydras.T1Bridge.html}%
{\texttt{gaia\_hydras.T1Bridge}}}.

Since both libraries \texttt{gaia} and \texttt{hydra-battles} have their own type \texttt{T1}, we will work with the following notations.

\inputsnippets{T1Bridge/hT1gT1}

Most used constants and functions will also get standardized
non-ambiguous names.


\inputsnippets{T1Bridge/MoreNotations}



The bridge between both libraries is made of two functions.

\inputsnippets{T1Bridge/iotaPiDef}

\inputsnippets{T1Bridge/iotaPiRw, T1Bridge/piIotaRw}

The functions \texttt{iota} and \texttt{pi} allow us to define
a notion of refinement for constants, functions, predicates and relations.

\inputsnippets{T1Bridge/refineDefs}

\subsection{Examples of refinements}
Both libraries define constants like $0$, $1$, $\omega$, and arithmetic functions: successor, addition, multiplication, and exponential of base $\omega$ (function $\phi_0$). We prove consistency of these definitions. Finally, we prove that the boolean predicates `` to be in normal form'' are equivalent.

\subsubsection{A few constants}
For each constant: $0$, $1$, \dots, $n$ and $\omega$ we prove
that the \texttt{hydras}' definition refines \texttt{gaia}'s.
\inputsnippets{T1Bridge/constantRefs}

\subsubsection{Unary functions}

\inputsnippets{T1Bridge/unaryRefs}

\subsubsection{Binary operations}
The binary operations $+$ and $\times$ are defined the same way in both libraries. Please note that they require comparison of ordinal terms. So, the proof of the following lemmas applies
compatibility lemmas like \texttt{compare\_ref} (see Section~\vref{sect:gaia-compare-ref}).

\inputsnippets{T1Bridge/plusRef, T1Bridge/multRef}

\subsubsection{Porting a theorem}
The \texttt{Gaia} library containsa proof that the multiplication of ordinals less than $\epsilon_0$ is associative.

\emph{From~\href{https://github.com/coq-community/gaia/blob/master/theories/ssete9.v}{gaia.ssete9.v}}

\begin{Coqsrc}
Lemma mulA: associative T1mul.
\end{Coqsrc}

This lemma was missing from \texttt{hydra-battles}. Nevertheless, we could adapt this lemma to \texttt{hydra-battles}' ordinals, by a small sequence of rewritings.

\inputsnippets{T1Bridge/multA}

The module~\href{../theories/html/gaia_hydras.T1Bridge_use.html}%
{\texttt{gaia\_hydras.T1Bridge\_use}} shows a small
example of importation of \texttt{multA} into \texttt{hydra-battles}' world.

\inputsnippets{T1Bridge_use/T1BridgeUse}

\subsection{Order and comparison}
Leibniz equality and strict order on both libraries are
compatible:
\inputsnippets{T1Bridge/eqRef}

\label{sect:gaia-compare-ref}
\inputsnippets{T1Bridge/compareRef}

\inputsnippets{T1Bridge/ltRef}

\inputsnippets{T1Bridge/decideHLtRw}

Hence, both definitions of ``being in Cantor normal form'' are
compatible.

\inputsnippets{T1Bridge/nfRef}


\section{Gaia's proof of well-foundedness}
The library \texttt{gaia-hydras} already contains two proofs
of well-foundedness of the strict order on ordinal terms in
Cantor normal form: a direct proof~\vref{sec:T1Wf} and a proof based on the recursive path ordering~\vref{remark:a3pat}, by
\'Evelyne Contejean. We found it interesting to comment Gaia's proof, which uses tools already present in~\cite{CantorContrib}.

This proof, as it appears in \href{https://github.com/coq-community/gaia/blob/master/theories/ssete9.v}{gaia.ssete9.v}, is quite dense and qualified as ``a bit tricky''. We present here its main steps with the help of the alectryon tool.


\subsection{Restricted recursion}
As remarked in~\vref{sec:T1Wf}, the order \texttt{T1lt} on \texttt{T1} is not well-founded, because of terms not in normal form.
But the \emph{restriction} of \texttt{T1lt} to terms in normal form \emph{is} well-founded. The following section introduces a  vocabulary in order to reason about statements of the form  ``the restriction of the relation $R$ to a subset $P$ of $A$ is well founded''.

\inputsnippets{nfwfgaia/restrictedRecursion}

The following induction principle is just a variant of \texttt{well\_founded\_induction\_type} for relations defined with \texttt{restrict}.

\inputsnippets{nfwfgaia/restrictedRecursiona}

With $P=\texttt{T1nf}$, this principle will allow us to write proofs by transfinite induction.

\subsection{Main proof steps}

\emph{Note: the abreviation \texttt{LT}, and the \texttt{fold LT} and \texttt{unfold LT} tactics,are not in \texttt{gaia}, but  have been added to make the goal display more readable.}

\inputsnippets{nfwfgaia/nfWfProofa}
We start with classic structural induction on \texttt{a}.

\inputsnippets{nfwfgaia/nfWfProofaa}
The following \texttt{ssreflect} tactic  pushes a new hypothesis \texttt{Hb} which --- in the same terms as in Section~\vref{sec:strongly-accessible} --- tells that any ordinal less than $a$ is ``strongly accessible''.

\inputsnippets{nfwfgaia/nfWfProofb}

Now, let us consider any ordinal term which is (by \texttt{LT}) less than \texttt{(cons a n b)}.

There are three cases to consider.
\begin{enumerate}
\item It is of the form \texttt{(cons a' n' b')} where \texttt{a'} is strictly less than \texttt{a}.
  \inputsnippets{nfwfgaia/nfWfProofd}
  
  Then, it is accessible (by hypothesis \texttt{Hb}).
  
  \item The considered term  is equal to
    \texttt{(cons a n b'')}, where $b''<\omega^a$. 
    \inputsnippets{nfwfgaia/nfWfProofg}
    
    Then we can replace \texttt{a''} with \texttt{a}, \texttt{n''} with \texttt{n}, and apply the hypotheses \texttt{Hd} and \texttt{He}.
    
\item
  The term is equal to
  \texttt{(cons a n'' b'')}, where $n''<n$ and $b''<\omega^a$.
  This term is less than the accessible \texttt{(cons a n b'')}.
  
  
  \inputsnippets{nfwfgaia/nfWfProofe}
  
 Then we can apply the hypothesis \texttt{He}.
 
\end{enumerate}

In every case, the considered ordinal is accessible.
