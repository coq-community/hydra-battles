\chapter{General presentation (draft)}

\section{Introduction}
This part contains comments, examples  and exercises about Russel O'Connor's work on G\"{o}del's first incompleteness theorem  \cite{Godel1986-GDECW}.
O'Connor's work was published in 2005~\cite{OConnor05}, and released as a user-contribution of the \coq proof assistant.
This work is now maintained by Coq community~\cite{CoqCommunity} volunteers and split into two projects: Goedel~\cite{Goedel} and Hydra-battles~\cite{HydraBattles}.

The main reference to this work is  Russel O'Connor's article~\cite{OConnor05}, which we strongly encourage the reader to 
consult regularly. 

It was the first computer verified proof of
the essential incompleteness\footnote{Todo: explain ``essential''} of Peano arithmetic.
The main reference to this work is  Russel O'Connor's article~\cite{OConnor05}, which we strongly encourage the reader to 
consult regularly. 

Several reasons --- enumerated below --- led us to maintain and document this work in the framework of 
Coq community~\cite{CoqCommunity}.

\begin{todo}
  Cite~\cite{Dowek2023}.
\end{todo}

 \begin{itemize}
 \item Historical interest in G\"{o}del's proof and its mechanizations, as shown by the abundant litterature on this topic
   (for instance \cite{smullyan1992godel, Hofstadter1999Godel, GoedelCassou}).
 
   
  \item O'Connor's proof was written at the end of \coq V7, then rewritten at the beginning of \coq V8. Since then, \coq and its ecosystem evolved a lot (new styles, tactics, documentation tools, etc.).
          We think this evolution should benefit
          to the original proof-scripts and make their understanding easier.

        \item Finally, let us quote  Efim Zelmanov.
             \begin{quote}
      ``The purpose of a proof is \emph{understanding}'' ~\cite{mathproof}
       
    \end{quote}
    We hope that underlying some points of the proof will
     make easier to understand this large and technical work.

        \end{itemize}

     
        For technical reasons --- mainly in order to simplify the installation and use of its sub-libraries, we split the project
into two main parts: Goedel~\cite{Goedel} and Hydra-battles~\cite{HydraBattles}.



Some changes are made to the aforementionned libraries, mainly because of the recent evolution of Coq and its libraries. Nevertheless, the definitions, lemmas and theorems of the original contribution have been preserved in this new release.

\emph{\color{red!80}These maintenance and documentation 
jobs have just started, and will probably be long to complete. Help is welcome!}





\section{File contents}

All Russel O'Connor's files have been stored in two directories, in order to simplify packages maintenance. 

\begin{itemize}
\item \texttt{theories/goedel/}: Proofs which depend on 
\texttt{CoqPrime} package.
\item  \texttt{theories/ordinals/Ackermann/} : all the rest:
 definition of primitive recursive functions, first-order  logic,
Peano Arithmetic, G\"{o}del's encoding.
\item Some additions we made: examples, exercices, new notations, etc.,  are stored in a specific directory \texttt{theories/ordinal/MoreAck/}.
\end{itemize}

\subsection{The \texttt{Ackermann} sub-directory}

The following main topics are studied in  \texttt{theories/ordinals/Ackermann/}: 
The following list presents the main modules 
in a dependance-compatible order.

\begin{description}

 \item[Primitive Recursive Functions]
  \mbox{}
  
 This topic is discussed in Chapter~\vref{chapter:primrec}.
  
  \begin{itemize}
\item
\href{../theories/html/hydras.Ackermann.extEqualNat.html}{Ackermann.extEqualNat}
  \item 
  \href{../theories/html/hydras.Ackermann.primRec.html}{Ackermann.primRec}
 \item 
  \href{../theories/html/hydras.Ackermann.cPair.html}{Ackermann.cPair}

\item 
  \href{../theories/html/hydras.MoreAck.PrimRecExamples.html}{MoreAck.PrimRecExamples}  
 \item 
  \href{../theories/html/hydras.MoreAck.Ack.html}{MoreAck.Ack}  
 
\item
  \href{../theories/html/hydras.MoreAck.AckNotPR.html}{MoreAck.AckNotPR}  

Ackermann function is not primitive recursive

  \end{itemize}


  
  
\item[First Order Logic]
  \mbox{}
  
This part presents \coq definitions and basic properties of first-order languages and proofs.

  \begin{itemize}
  \item \href{../theories/html/hydras.Ackermann.fol.html}{Ackermann.fol}
 \item \href{../theories/html/hydras.Ackermann.folProp.html}{Ackermann.folProp}
\item \href{../theories/html/hydras.Ackermann.folProof.html}{Ackermann.folProof}
\item \href{../theories/html/hydras.Ackermann.model.html}{Ackermann.model}
\item \href{../theories/html/hydras.Ackermann.code.html}{Ackermann.code}
 \item 
  \href{../theories/html/hydras.Ackermann.prLogic.html}{Ackermann.prLogic} 

\item \href{../theories/html/hydras.Ackermann.codeList.html}{Ackermann.codeList}
\item \href{../theories/html/hydras.Ackermann.codeFreeVar.html}{Ackermann.codeFreeVar}

\item \href{../theories/html/hydras.Ackermann.checkPrf.html}{Ackermann.checkPrf}
\item \href{../theories/html/hydras.Ackermann.wellFormed.html}{Ackermann.wellFormed}
\item \href{../theories/html/hydras.Ackermann.codeSubTerm.html}{Ackermann.codeSubTerm}
\item \href{../theories/html/hydras.Ackermann.codeSubFormula.html}{Ackermann.codeSubFormula}


  \end{itemize}

\item[Natural Deduction]
 \mbox{}
Thanks to the \emph{Deduction Theorem}, we prove many 
lemmas about provability in first-order-logic, many of them can be considered as natural deduction rules.

\begin{itemize}
\item \href{../theories/html/hydras.Ackermann.Deduction.html}{Ackermann.Deduction}
\item \href{../theories/html/hydras.Ackermann.folLogic.html}{Ackermann.folLogic}
\item \href{../theories/html/hydras.Ackermann.folLogic2.html}{Ackermann.folLogic2}
\item \href{../theories/html/hydras.Ackermann.folLogic3.html}{Ackermann.folLogic3}
\item \href{../theories/html/hydras.Ackermann.folReplace.html}{Ackermann.folReplace}
\item \href{../theories/html/hydras.Ackermann.subProp.html}{Ackermann.subProp}
\item \href{../theories/html/hydras.Ackermann.subAll.html}{Ackermann.subAll}
 \item  \href{../theories/html/hydras.MoreAck.FolExamples.html}{MoreAck.FolExamples}
\end{itemize}

\item [Languages of Arithmetic]
\mbox{}

\begin{itemize}
\item \href{../theories/html/hydras.Ackermann.Languages.html}{Ackermann.Languages}
\item \href{../theories/html/hydras.Ackermann.LNN.html}{Ackermann.LNN}
\item \href{../theories/html/hydras.Ackermann.LNT.html}{Ackermann.LNT}
\item \href{../theories/html/hydras.Ackermann.NN.html}{Ackermann.NN} Axioms for Natural Numbers and basic properties.
\item \href{../theories/html/hydras.Ackermann.NNtheory.html}{Ackermann.NNtheory}
\item \href{../theories/html/hydras.Ackermann.PA.html}{Ackermann.PA} Peano Arithmetic: axioms  and first properties.
\item \href{../theories/html/hydras.Ackermann.LNN2LNT.html}{Ackermann.LNN2LNT}
\item \href{../theories/html/hydras.Ackermann.NN2PA.html}{Ackermann.NN2PA}

\item \href{../theories/html/hydras.Ackermann.PAtheory.html}{Ackermann.PAtheory}
\item \href{../theories/html/hydras.Ackermann.PAconsistent.html}{Ackermann.PAconsistent}

\item \href{../theories/html/hydras.Ackermann.codePA.html}{Ackermann.codePA}

\item \href{../theories/html/hydras.Ackermann.codeNatToTerm.html}{Ackermann.codeNat2Term}
\item \href{../theories/html/hydras.Ackermann.wConsistent.html}{Ackermann.wConsistent}
\item \href{../theories/html/hydras.Ackermann.expressible.html}{Ackermann.expressible}

\item \href{../theories/html/hydras.MoreAck.LNN_Examples.html}{MoreAck.LNN\_Examples}
\end{itemize}

\item[Modules dependent on \texttt{CoqPrime}]

\mbox{}

  \begin{itemize}
  \item \href{../theories/html/Goedel.PRrepresentable.html}{Goedel.PRrepresentable}
 \item \href{../theories/html/Goedel.fixPoint.html}{Goedel.fixPoint}
 \item \href{../theories/html/Goedel.codeSysPrf.html}{Goedel.codeSysPrf}
 \item \href{../theories/html/Goedel.rosser.html}{Goedel.rosser}
 \item \href{../theories/html/Goedel.goedel1.html}{Goedel.goedel1}
 \item \href{../theories/html/Goedel.rosserPA.html}{Goedel.rosserPA}
 \item \href{../theories/html/Goedel.goedel2.html}{Goedel.goedel2}
  \end{itemize}
\end{description}

\begin{todo}
Add information on recent developments on formal 
proofs of G\"odel  incompleteness theorems. Justify the decision of working on \emph{this} development.  
\end{todo}

\section{Warning}
Russel O'Connors contribution contains more than 42 KLoc.
Since its construction, \coq{},  its libraries and recommended style have evolved a lot. We have just started to ``modernize'' this code. We apologize for provisional inconsistencies of presentation (code and documentation).
